\documentclass[CJK]{beamer}
\usepackage{CJKutf8}
\usepackage{beamerthemesplit}
\usetheme{Malmoe}
\useoutertheme[footline=authortitle]{miniframes}
\usepackage{amsmath}
\usepackage{amssymb}
\usepackage{graphicx}
\usepackage{eufrak}
\usepackage{color}
\usepackage{slashed}
\usepackage{simplewick}
\usepackage{tikz}
\usepackage{tcolorbox}
\graphicspath{{../figures/}}
%%figures
\def\lfig#1#2{\includegraphics[width=#1 in]{#2}}
\def\addfig#1#2{\begin{center}\includegraphics[width=#1 in]{#2}\end{center}}
\def\wulian{\includegraphics[width=0.18in]{emoji_wulian.jpg}}
\def\bigwulian{\includegraphics[width=0.35in]{emoji_wulian.jpg}}
\def\bye{\includegraphics[width=0.18in]{emoji_bye.jpg}}
\def\bigbye{\includegraphics[width=0.35in]{emoji_bye.jpg}}
\def\huaixiao{\includegraphics[width=0.18in]{emoji_huaixiao.jpg}}
\def\bighuaixiao{\includegraphics[width=0.35in]{emoji_huaixiao.jpg}}
\def\jianxiao{\includegraphics[width=0.18in]{emoji_jianxiao.jpg}}
\def\bigjianxiao{\includegraphics[width=0.35in]{emoji_jianxiao.jpg}}
%% colors
\def\blacktext#1{{\color{black}#1}}
\def\bluetext#1{{\color{blue}#1}}
\def\redtext#1{{\color{red}#1}}
\def\darkbluetext#1{{\color[rgb]{0,0.2,0.6}#1}}
\def\skybluetext#1{{\color[rgb]{0.2,0.7,1.}#1}}
\def\cyantext#1{{\color[rgb]{0.,0.5,0.5}#1}}
\def\greentext#1{{\color[rgb]{0,0.7,0.1}#1}}
\def\darkgray{\color[rgb]{0.2,0.2,0.2}}
\def\lightgray{\color[rgb]{0.6,0.6,0.6}}
\def\gray{\color[rgb]{0.4,0.4,0.4}}
\def\blue{\color{blue}}
\def\red{\color{red}}
\def\green{\color{green}}
\def\darkgreen{\color[rgb]{0,0.4,0.1}}
\def\darkblue{\color[rgb]{0,0.2,0.6}}
\def\skyblue{\color[rgb]{0.2,0.7,1.}}
%%control
\def\be{\begin{equation}}
\def\ee{\nonumber\end{equation}}
\def\bea{\begin{eqnarray}}
\def\eea{\nonumber\end{eqnarray}}
\def\bch{\begin{CJK}{UTF8}{gbsn}}
\def\ech{\end{CJK}}
\def\bitem{\begin{itemize}}
\def\eitem{\end{itemize}}
\def\bcenter{\begin{center}}
\def\ecenter{\end{center}}
\def\bex{\begin{minipage}{0.2\textwidth}\includegraphics[width=0.6in]{jugelizi.png}\end{minipage}\begin{minipage}{0.76\textwidth}}
\def\eex{\end{minipage}}
\def\chtitle#1{\frametitle{\bch#1\ech}}
\def\bmat#1{\left(\begin{array}{#1}}
\def\emat{\end{array}\right)}
\def\bcase#1{\left\{\begin{array}{#1}}
\def\ecase{\end{array}\right.}
\def\bmini#1{\begin{minipage}{#1\textwidth}}
\def\emini{\end{minipage}}
\def\tbox#1{\begin{tcolorbox}#1\end{tcolorbox}}
\def\pfrac#1#2#3{\left(\frac{\partial #1}{\partial #2}\right)_{#3}}
%%symbols
\def\bropt{\,(\ \ \ )}
\def\sone{$\star$}
\def\stwo{$\star\star$}
\def\sthree{$\star\star\star$}
\def\sfour{$\star\star\star\star$}
\def\sfive{$\star\star\star\star\star$}
\def\rint{{\int_\leftrightarrow}}
\def\roint{{\oint_\leftrightarrow}}
\def\stdHf{{\textit{\r H}_f}}
\def\deltaH{{\Delta \textit{\r H}}}
\def\ii{{\dot{\imath}}}
\def\skipline{{\vskip0.1in}}
\def\skiplines{{\vskip0.2in}}
\def\lagr{{\mathcal{L}}}
\def\hamil{{\mathcal{H}}}
\def\vecv{{\mathbf{v}}}
\def\vecx{{\mathbf{x}}}
\def\vecy{{\mathbf{y}}}
\def\veck{{\mathbf{k}}}
\def\vecp{{\mathbf{p}}}
\def\vecn{{\mathbf{n}}}
\def\vecA{{\mathbf{A}}}
\def\vecP{{\mathbf{P}}}
\def\vecsigma{{\mathbf{\sigma}}}
\def\hatJn{{\hat{J_\vecn}}}
\def\hatJx{{\hat{J_x}}}
\def\hatJy{{\hat{J_y}}}
\def\hatJz{{\hat{J_z}}}
\def\hatj#1{\hat{J_{#1}}}
\def\hatphi{{\hat{\phi}}}
\def\hatq{{\hat{q}}}
\def\hatpi{{\hat{\pi}}}
\def\vel{\upsilon}
\def\Dint{{\mathcal{D}}}
\def\adag{{\hat{a}^\dagger}}
\def\bdag{{\hat{b}^\dagger}}
\def\cdag{{\hat{c}^\dagger}}
\def\ddag{{\hat{d}^\dagger}}
\def\hata{{\hat{a}}}
\def\hatb{{\hat{b}}}
\def\hatc{{\hat{c}}}
\def\hatd{{\hat{d}}}
\def\hatN{{\hat{N}}}
\def\hatH{{\hat{H}}}
\def\hatp{{\hat{p}}}
\def\Fup{{F^{\mu\nu}}}
\def\Fdown{{F_{\mu\nu}}}
\def\newl{\nonumber \\}
\def\vece{\mathrm{e}}
\def\calM{{\mathcal{M}}}
\def\calT{{\mathcal{T}}}
\def\calR{{\mathcal{R}}}
\def\barpsi{\bar{\psi}}
\def\baru{\bar{u}}
\def\barv{\bar{\upsilon}}
\def\qeq{\stackrel{?}{=}}
\def\torder#1{\mathcal{T}\left(#1\right)}
\def\rorder#1{\mathcal{R}\left(#1\right)}
\def\contr#1#2{\contraction{}{#1}{}{#2}#1#2}
\def\trof#1{\mathrm{Tr}\left(#1\right)}
\def\trace{\mathrm{Tr}}
\def\comm#1{\ \ \ \left(\mathrm{used}\ #1\right)}
\def\tcomm#1{\ \ \ (\text{#1})}
\def\slp{\slashed{p}}
\def\slk{\slashed{k}}
\def\calp{{\mathfrak{p}}}
\def\veccalp{\mathbf{\mathfrak{p}}}
\def\Tthree{T_{\tiny \textcircled{3}}}
\def\pthree{p_{\tiny \textcircled{3}}}
\def\dbar{{\,\mathchar'26\mkern-12mu d}}
\def\erf{\mathrm{erf}}
\def\const{\mathrm{constant}}
\def\pheat{\pfrac p{\ln T}V}
\def\vheat{\pfrac V{\ln T}p}
%%units
\def\fdeg{{^\circ \mathrm{F}}}
\def\cdeg{^\circ \mathrm{C}}
\def\atm{\,\mathrm{atm}}
\def\angstrom{\,\text{\AA}}
\def\SIL{\,\mathrm{L}}
\def\SIkm{\,\mathrm{km}}
\def\SIyr{\,\mathrm{yr}}
\def\SIGyr{\,\mathrm{Gyr}}
\def\SIV{\,\mathrm{V}}
\def\SImV{\,\mathrm{mV}}
\def\SIeV{\,\mathrm{eV}}
\def\SIkeV{\,\mathrm{keV}}
\def\SIMeV{\,\mathrm{MeV}}
\def\SIGeV{\,\mathrm{GeV}}
\def\SIcal{\,\mathrm{cal}}
\def\SIkcal{\,\mathrm{kcal}}
\def\SImol{\,\mathrm{mol}}
\def\SIN{\,\mathrm{N}}
\def\SIHz{\,\mathrm{Hz}}
\def\SIm{\,\mathrm{m}}
\def\SIcm{\,\mathrm{cm}}
\def\SIfm{\,\mathrm{fm}}
\def\SImm{\,\mathrm{mm}}
\def\SInm{\,\mathrm{nm}}
\def\SImum{\,\mathrm{\mu m}}
\def\SIJ{\,\mathrm{J}}
\def\SIW{\,\mathrm{W}}
\def\SIkJ{\,\mathrm{kJ}}
\def\SIs{\,\mathrm{s}}
\def\SIkg{\,\mathrm{kg}}
\def\SIg{\,\mathrm{g}}
\def\SIK{\,\mathrm{K}}
\def\SImmHg{\,\mathrm{mmHg}}
\def\SIPa{\,\mathrm{Pa}}

\def\courseurl{https://github.com/zqhuang/SYSU\_TD}

\def\tpage#1#2{
\begin{frame}
\begin{center}
\begin{Large}
\bch
热学 \\
第#1讲 #2

{\vskip 0.3in}

黄志琦

\ech
\end{Large}
\end{center}

\vskip 0.2in

\bch
教材:《热学》第二版,赵凯华,罗蔚茵,高等教育出版社
\ech

\bch
课件下载
\ech
\courseurl
\end{frame}
}

\def\bfr#1{
\begin{frame}
\chtitle{#1} 
\bch
}

\def\efr{
\ech 
\end{frame}
}

  \date{}
  \begin{document}
  \bch
\tpage{26}{级数方法解微分方程}




\begin{frame}
\frametitle{本讲内容}
\tableofcontents
\end{frame}

\section{How did we get Bessel functions.}

\begin{frame}
  \frametitle{回顾贝塞尔方程}
  贝塞尔方程
  $$\frac{d^2f}{dz^2} + \frac{1}{z} \frac{df}{dz} + \left(1-\frac{\nu^2}{z^2}\right) f = 0 $$
  的解是怎么算出来的?
\end{frame}


\begin{frame}
  \frametitle{一般套路}
  如果  $p(z)$和 $q(z)$ 在圆 $|z|<R$ 解析,则二阶微分方程
  $$\frac{d^2f}{dz^2} + \frac{p(z)}{z} \frac{df}{dz} + \frac{q(z)}{z^2} f = 0 $$
  在去心邻域$0<|z|<R$有两个线性独立的解具有如下形式:
  $$ f_1(z) = z^{\rho_1}\sum_{n=0}^\infty c_nz^n,$$ 
  $$ f_2(z) = \lambda f_1(z) \ln z + z^{\rho_2}\sum_{n=0}^\infty d_n z^n.$$
  其中 $\rho_1,\rho_2$ 是“指标方程”
    $$\rho^2 + \left[p(0)-1\right]\rho + q(0) = 0. $$
  的两个解;$c_0,d_0$均非零;仅当$\rho_1-\rho_2$为整数时,$\lambda$才有可能需要取非零值。
\end{frame}


\begin{frame}
  \frametitle{实战举例:贝塞尔方程}
  对贝塞尔方程,$p(z) = 1$ , $q(z) = z^2-\nu^2$,指标方程为
  $$ \rho^2 - \nu^2 = 0 .$$
  其解为
  $$\rho_{1,2} = \pm \nu. $$
  设第一个解为
  $$ f_1(z) = z^\nu \sum_{n=0}^\infty c_n z^n. $$
  代入贝塞尔方程,比较 $z^{n+\nu}$的系数,得到
  $$  c_n  = -\frac{1}{n(n+2\nu)} c_{n-2}. $$
\end{frame}


\begin{frame}
  \frametitle{实战举例:贝塞尔方程}
  由此可以递推得出
  $$c_{2n+1}=0;\ c_{2n} = (-1)^n\frac{\nu !}{2^{2n}(\nu+n)!}c_0.\ (n=0,1,2,\ldots).$$
  如果取 $c_0=\frac{1}{2^\nu \nu!}$: 恭喜你,得到了第一类贝塞尔函数 $J_\nu(z)$ 的级数表达式!


  \skipline
  
  如果取其他的非零 $c_0$ 值,结果无非是 $J_\nu(z)$ 的常数倍数而已。

  \skipline

  如果 $2\nu$ 不是整数,那么你对 $\rho_2=-\nu$ 如法炮制,可以得到线性独立的解 $J_{-\nu}(z)$ (或它的任意非零倍数)。

\end{frame}


\begin{frame}
  \frametitle{实战举例:贝塞尔方程}

  最后,还需要讨论 $2\nu$ 是整数的情形。为了更清晰地介绍方法而不是让你们迷失于纷繁复杂的计算中,我们仅以 $\nu=0$ 的情况为例讨论:

  这时$\rho_1=\rho_2=0$,第二个解为:

  $$ f_2(z) = \lambda J_0(z)\ln z + g(z)$$
  其中 $g(z)= \sum_{n=0}^\infty d_n z^n$。把$f_2$代入贝塞尔方程,得到:

  $$g''(z) + \frac{1}{z}g'(z) + g(z) +\frac{2\lambda J_0'(z)}{z} = 0.$$
  通过对比 $z^{2n-2}$的系数,得到:  
\end{frame}




\begin{frame}
  \frametitle{实战举例:贝塞尔方程}
  $$ d_{2n}= -\frac{1}{4n^2} d_{2n-2}-\frac{(-1)^n\lambda}{ 2^{2n}(n!)^2n }.$$
  从任意非零的$\lambda$, $d_0$ 出发,都可以得到$d_2,d_4,\ldots$

  \skipline
  
  我们熟知的 $N_0(z)$,则是取 $\lambda = \frac{2}{\pi}$, $d_0 = \frac{2}{\pi}\left(\gamma-\ln 2\right)$(这里的 $\gamma=0.5772\ldots$ 是欧拉常数)的结果。

    \skipline
    
   $N_0$ 为何对应如此古怪的 $\lambda$ 和 $d_0$?这为了凑 $N_0$的无穷远处渐近公式而所做的人为约定。只要你喜欢,完全可以取不同的$\lambda$和 $d_0$。
\end{frame}


\section{1-D harmonic oscillator in quantum mechanics}

\begin{frame}
  \frametitle{量子力学的谐振子问题}
  我们熟知的一维谐振子的定态束缚解问题对应的薛定谔方程:
  $$\psi'' - x^2\psi = -2E\psi $$
  其中 $E$为待定的常数(谐振子的能量)。束缚解要求波函数 $\psi(x)$ 在无穷远处趋向于零 $\lim_{|x|\rightarrow\infty}\psi(x) = 0.$
  
\end{frame}


\begin{frame}
  \frametitle{先把目光放长远……}
  解决这种问题的套路是:先分析无穷远处的渐近行为。

  当 $|x|$ 很大时,可以近似认为
  $$\psi''- x^2\psi \approx 0$$
  其粗略的解为
  $$ \psi \sim e^{\pm x^2/2}.$$
  (它满足的方程是$\psi''- (x^2-1)\psi = 0$,差别不大)。

  考虑到无穷远处的束缚条件,我们不太喜欢 $e^{x^2/2}$ 这个哥们,所以……
\end{frame}

\begin{frame}
  \frametitle{只留下喜欢的}
  令 $\psi = f(x) e^{-x^2/2}$ (我们希望 $f$是个比较有节制的家伙)。

  代入原方程,得到:
  $$f''-2xf'+(2E-1)f = 0. $$

  我们准备用级数解法。但在动手之前——
  
  我们应该意识到:这个方程和原方程等价。而且级数解法根本不能保证 $f$有节制。在一般情况下,我们会求得一个解$f\sim e^{x^2}$,令人讨厌的 $e^{x^2/2}$ 又会回来。
\end{frame}


\begin{frame}
  \frametitle{动手}
  $p(x)= -2x^2, q(x) = (2E-1)x^2$,其指标方程为
  $$\lambda^2-\lambda = 0.$$
  对数形式的解在 $x=0$ 处发散,所以无须考虑,只要考虑如下形式的解
  $$ f(x) = \sum_{n=0}^\infty c_n x^n.$$
  其中 $c_0$或者 $c_1$至少有一个非零。代入方程得到
  $$ c_{n+2} =  \frac{2\left(n+\frac{1}{2}-E\right)}{(n+2)(n+1)}c_n.$$
  一般情况下,这个递推关系就是会给出一个解 $\sim e^{x^2}$,除非——
\end{frame}


\begin{frame}
  \frametitle{量子力学的谐振子问题}
  $$ c_{n+2} =  \frac{2\left(n+\frac{1}{2}-E\right)}{(n+2)(n+1)}c_n.$$
   对某个 $n$,有
  $$ n+\frac{1}{2}-E=0.$$
  (好好一个无穷级数在$c_n$处断了香火)

    \skipline

  这就是为什么谐振子的能量只能是 $n+\frac{1}{2}$ (放回合适的物理单位后这个结果应该是 $(n+1/2)\hbar\omega$,$\omega$为固有圆频率)。
\end{frame}

\section{Hydrogen Atom}


\begin{frame}
  \frametitle{量子力学的氢原子问题}
  在研究氢原子的束缚能量本征态时,会遇到球坐标系的薛定谔方程:
  $$ \nabla^2\psi + \frac{2}{r}\psi = -2E\psi.$$

  考虑角动量和哈密顿量的共同本征态——好吧你不知道我在说什么……没关系,其实就是由于一些对称性的考虑我们只需考虑形如 $ \psi = f(r)Y_{\ell m}(\theta,\phi) $ 的解。

  代入上述方程得到:

  $$ f''+ \frac{2}{r}f' + \left(2E+\frac{2}{r}- \frac{\ell(\ell+1)}{r^2}\right)f = 0.$$
\end{frame}

\begin{frame}
  \frametitle{套路又来了}
  在$r$巨大时,方程近似成为

  $$ f'' + 2Ef = 0.$$
  如果 $E>=0$,$f$在无穷远处很不像会足够快地趋向于零的样子,所以只能$E<0$。记 $E=-k^2/2$,则 $f\sim e^{\pm kr}$。

  \skipline

  很明显,$f\sim e^{kr}$ 很不受待见——
\end{frame}

\begin{frame}
  \frametitle{只留下喜欢的}
  令 $f(r)  = e^{-kr} u(r) $(并祈祷$u(r)$能有所节制),代入原方程得到:
  $$ u'' +2\left(\frac{1}{r}-k\right)u' +\left(\frac{2(1-k)}{r} - \frac{\ell(\ell+1)}{r^2}\right)u = 0.$$
  同样道理,这个方程和原方程完全等价。当我们用级数解法求 $u(r)$ 时,在一般情况下,会得到 $u\sim e^{2kr}$的解,复活刚刚被我们鄙视掉的 $f\sim e^{kr}$。
\end{frame}

\begin{frame}
  \frametitle{氢原子的能级}
  由指标方程得出 $\rho_1=\ell$,我们来看 $u = r^\ell\sum_{n=0}^\infty c_nr^n$ 是否可行,代入方程得到:

  $$ c_{n+1}=2\frac{k(n+\ell+1)-1}{(\ell+n+1)(\ell+n+2)-\ell(\ell+1)}c_n $$

  一般情况下,这个递推公式会给出一个无穷级数 $\sim e^{2kr}$。除非——

  $k=\frac{1}{n+\ell+1}$.

  即 $$ E = -\frac{1}{2m^2}.$$
  ($m=n+\ell+1$为正整数) 
  
\end{frame}


\thinka{对应 $\rho_2=-\ell-1$的解需要考虑吗?}

\ech
\end{document}
