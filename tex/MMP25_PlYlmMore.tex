\documentclass[CJK]{beamer}
\usepackage{CJKutf8}
\usepackage{beamerthemesplit}
\usetheme{Malmoe}
\useoutertheme[footline=authortitle]{miniframes}
\usepackage{amsmath}
\usepackage{amssymb}
\usepackage{graphicx}
\usepackage{eufrak}
\usepackage{color}
\usepackage{slashed}
\usepackage{simplewick}
\usepackage{tikz}
\usepackage{tcolorbox}
\graphicspath{{../figures/}}
%%figures
\def\lfig#1#2{\includegraphics[width=#1 in]{#2}}
\def\addfig#1#2{\begin{center}\includegraphics[width=#1 in]{#2}\end{center}}
\def\wulian{\includegraphics[width=0.18in]{emoji_wulian.jpg}}
\def\bigwulian{\includegraphics[width=0.35in]{emoji_wulian.jpg}}
\def\bye{\includegraphics[width=0.18in]{emoji_bye.jpg}}
\def\bigbye{\includegraphics[width=0.35in]{emoji_bye.jpg}}
\def\huaixiao{\includegraphics[width=0.18in]{emoji_huaixiao.jpg}}
\def\bighuaixiao{\includegraphics[width=0.35in]{emoji_huaixiao.jpg}}
\def\jianxiao{\includegraphics[width=0.18in]{emoji_jianxiao.jpg}}
\def\bigjianxiao{\includegraphics[width=0.35in]{emoji_jianxiao.jpg}}
%% colors
\def\blacktext#1{{\color{black}#1}}
\def\bluetext#1{{\color{blue}#1}}
\def\redtext#1{{\color{red}#1}}
\def\darkbluetext#1{{\color[rgb]{0,0.2,0.6}#1}}
\def\skybluetext#1{{\color[rgb]{0.2,0.7,1.}#1}}
\def\cyantext#1{{\color[rgb]{0.,0.5,0.5}#1}}
\def\greentext#1{{\color[rgb]{0,0.7,0.1}#1}}
\def\darkgray{\color[rgb]{0.2,0.2,0.2}}
\def\lightgray{\color[rgb]{0.6,0.6,0.6}}
\def\gray{\color[rgb]{0.4,0.4,0.4}}
\def\blue{\color{blue}}
\def\red{\color{red}}
\def\green{\color{green}}
\def\darkgreen{\color[rgb]{0,0.4,0.1}}
\def\darkblue{\color[rgb]{0,0.2,0.6}}
\def\skyblue{\color[rgb]{0.2,0.7,1.}}
%%control
\def\be{\begin{equation}}
\def\ee{\nonumber\end{equation}}
\def\bea{\begin{eqnarray}}
\def\eea{\nonumber\end{eqnarray}}
\def\bch{\begin{CJK}{UTF8}{gbsn}}
\def\ech{\end{CJK}}
\def\bitem{\begin{itemize}}
\def\eitem{\end{itemize}}
\def\bcenter{\begin{center}}
\def\ecenter{\end{center}}
\def\bex{\begin{minipage}{0.2\textwidth}\includegraphics[width=0.6in]{jugelizi.png}\end{minipage}\begin{minipage}{0.76\textwidth}}
\def\eex{\end{minipage}}
\def\chtitle#1{\frametitle{\bch#1\ech}}
\def\bmat#1{\left(\begin{array}{#1}}
\def\emat{\end{array}\right)}
\def\bcase#1{\left\{\begin{array}{#1}}
\def\ecase{\end{array}\right.}
\def\bmini#1{\begin{minipage}{#1\textwidth}}
\def\emini{\end{minipage}}
\def\tbox#1{\begin{tcolorbox}#1\end{tcolorbox}}
\def\pfrac#1#2#3{\left(\frac{\partial #1}{\partial #2}\right)_{#3}}
%%symbols
\def\bropt{\,(\ \ \ )}
\def\sone{$\star$}
\def\stwo{$\star\star$}
\def\sthree{$\star\star\star$}
\def\sfour{$\star\star\star\star$}
\def\sfive{$\star\star\star\star\star$}
\def\rint{{\int_\leftrightarrow}}
\def\roint{{\oint_\leftrightarrow}}
\def\stdHf{{\textit{\r H}_f}}
\def\deltaH{{\Delta \textit{\r H}}}
\def\ii{{\dot{\imath}}}
\def\skipline{{\vskip0.1in}}
\def\skiplines{{\vskip0.2in}}
\def\lagr{{\mathcal{L}}}
\def\hamil{{\mathcal{H}}}
\def\vecv{{\mathbf{v}}}
\def\vecx{{\mathbf{x}}}
\def\vecy{{\mathbf{y}}}
\def\veck{{\mathbf{k}}}
\def\vecp{{\mathbf{p}}}
\def\vecn{{\mathbf{n}}}
\def\vecA{{\mathbf{A}}}
\def\vecP{{\mathbf{P}}}
\def\vecsigma{{\mathbf{\sigma}}}
\def\hatJn{{\hat{J_\vecn}}}
\def\hatJx{{\hat{J_x}}}
\def\hatJy{{\hat{J_y}}}
\def\hatJz{{\hat{J_z}}}
\def\hatj#1{\hat{J_{#1}}}
\def\hatphi{{\hat{\phi}}}
\def\hatq{{\hat{q}}}
\def\hatpi{{\hat{\pi}}}
\def\vel{\upsilon}
\def\Dint{{\mathcal{D}}}
\def\adag{{\hat{a}^\dagger}}
\def\bdag{{\hat{b}^\dagger}}
\def\cdag{{\hat{c}^\dagger}}
\def\ddag{{\hat{d}^\dagger}}
\def\hata{{\hat{a}}}
\def\hatb{{\hat{b}}}
\def\hatc{{\hat{c}}}
\def\hatd{{\hat{d}}}
\def\hatN{{\hat{N}}}
\def\hatH{{\hat{H}}}
\def\hatp{{\hat{p}}}
\def\Fup{{F^{\mu\nu}}}
\def\Fdown{{F_{\mu\nu}}}
\def\newl{\nonumber \\}
\def\vece{\mathrm{e}}
\def\calM{{\mathcal{M}}}
\def\calT{{\mathcal{T}}}
\def\calR{{\mathcal{R}}}
\def\barpsi{\bar{\psi}}
\def\baru{\bar{u}}
\def\barv{\bar{\upsilon}}
\def\qeq{\stackrel{?}{=}}
\def\torder#1{\mathcal{T}\left(#1\right)}
\def\rorder#1{\mathcal{R}\left(#1\right)}
\def\contr#1#2{\contraction{}{#1}{}{#2}#1#2}
\def\trof#1{\mathrm{Tr}\left(#1\right)}
\def\trace{\mathrm{Tr}}
\def\comm#1{\ \ \ \left(\mathrm{used}\ #1\right)}
\def\tcomm#1{\ \ \ (\text{#1})}
\def\slp{\slashed{p}}
\def\slk{\slashed{k}}
\def\calp{{\mathfrak{p}}}
\def\veccalp{\mathbf{\mathfrak{p}}}
\def\Tthree{T_{\tiny \textcircled{3}}}
\def\pthree{p_{\tiny \textcircled{3}}}
\def\dbar{{\,\mathchar'26\mkern-12mu d}}
\def\erf{\mathrm{erf}}
\def\const{\mathrm{constant}}
\def\pheat{\pfrac p{\ln T}V}
\def\vheat{\pfrac V{\ln T}p}
%%units
\def\fdeg{{^\circ \mathrm{F}}}
\def\cdeg{^\circ \mathrm{C}}
\def\atm{\,\mathrm{atm}}
\def\angstrom{\,\text{\AA}}
\def\SIL{\,\mathrm{L}}
\def\SIkm{\,\mathrm{km}}
\def\SIyr{\,\mathrm{yr}}
\def\SIGyr{\,\mathrm{Gyr}}
\def\SIV{\,\mathrm{V}}
\def\SImV{\,\mathrm{mV}}
\def\SIeV{\,\mathrm{eV}}
\def\SIkeV{\,\mathrm{keV}}
\def\SIMeV{\,\mathrm{MeV}}
\def\SIGeV{\,\mathrm{GeV}}
\def\SIcal{\,\mathrm{cal}}
\def\SIkcal{\,\mathrm{kcal}}
\def\SImol{\,\mathrm{mol}}
\def\SIN{\,\mathrm{N}}
\def\SIHz{\,\mathrm{Hz}}
\def\SIm{\,\mathrm{m}}
\def\SIcm{\,\mathrm{cm}}
\def\SIfm{\,\mathrm{fm}}
\def\SImm{\,\mathrm{mm}}
\def\SInm{\,\mathrm{nm}}
\def\SImum{\,\mathrm{\mu m}}
\def\SIJ{\,\mathrm{J}}
\def\SIW{\,\mathrm{W}}
\def\SIkJ{\,\mathrm{kJ}}
\def\SIs{\,\mathrm{s}}
\def\SIkg{\,\mathrm{kg}}
\def\SIg{\,\mathrm{g}}
\def\SIK{\,\mathrm{K}}
\def\SImmHg{\,\mathrm{mmHg}}
\def\SIPa{\,\mathrm{Pa}}

\def\courseurl{https://github.com/zqhuang/SYSU\_TD}

\def\tpage#1#2{
\begin{frame}
\begin{center}
\begin{Large}
\bch
热学 \\
第#1讲 #2

{\vskip 0.3in}

黄志琦

\ech
\end{Large}
\end{center}

\vskip 0.2in

\bch
教材:《热学》第二版,赵凯华,罗蔚茵,高等教育出版社
\ech

\bch
课件下载
\ech
\courseurl
\end{frame}
}

\def\bfr#1{
\begin{frame}
\chtitle{#1} 
\bch
}

\def\efr{
\ech 
\end{frame}
}

  \date{}
  \begin{document}
  \bch
\tpage{25}{勒让德多项式,球谐函数的更多性质}




\begin{frame}
\frametitle{本讲内容}

\tableofcontents
\end{frame}

\section{Legendre Polynomials}


\begin{frame}
\frametitle{金属球外的点电荷}

金属球外距离球心$a$处的点电荷$Q$造成的电势是一个熟悉的距离反比电势。
\bmini{0.5}
\lfig{2}{potential_inside.png}
\emini
\bmini{0.45}
如图建立球坐标系。设感应电荷产生的电势为$u(r, \theta)$ (由轴对称性很容易看出$u$不依赖于$\phi$)。则在星号位置处的总电势为:
\emini
$$U_{\rm total} = u(r, \theta) +\frac{Q}{4\pi\epsilon_0 \sqrt{a^2+r^2-2ar\cos\theta}}. $$

\end{frame}

\begin{frame}
\frametitle{勒让德函数的母函数定理}

套路当然是要把$\frac{Q}{4\pi\epsilon_0 \sqrt{a^2+r^2-2ar\cos\theta}}$写成一堆球坐标系谐函数的和。这要用到下列{\blue 勒让德多项式的母函数定理:
  $$ \frac{1}{\sqrt{1+t^2-2xt}} = \sum_{\ell =0}^\infty P_\ell (x) t^\ell,\ \ \ t<|x\pm \sqrt{x^2-1}|  .$$}
其中的勒让德多项式$P_\ell(x)$是一个$\ell$次多项式,它和$m=0$的球谐函数的关系为{\blue 
$$P_\ell(\cos\theta) = \sqrt{\frac{4\pi}{2\ell + 1}}Y_{\ell 0}(\theta,\phi).$$}

\skipline

(证明见附录)
\end{frame}


\begin{frame}
\frametitle{球外点电荷感应电势问题的解}

\bea
\frac{Q}{4\pi\epsilon_0 \sqrt{a^2+r^2-2ar\cos\theta}} &=& \frac{Q}{4\pi\epsilon_0 a \sqrt{1+ \left(\frac{r}{a}\right)^2-2\frac{r}{a}\cos\theta}} \newl
&=& \frac{Q}{4\pi\epsilon_0a}\sum_{\ell= 0}^\infty P_{\ell}(\cos\theta)\left(\frac{r}{a}\right)^\ell .
\eea
要求球内部总电势处处相等,则$\ell >0 $的项必须全部被$u$抵消。又根据球心处$u=0$,可以确定常数项为零:
$$\left. u(r, \theta)\right\vert_{r\le R} = - \frac{Q}{4\pi\epsilon_0a}\sum_{\ell= 1}^\infty P_{\ell}(\cos\theta)\left(\frac{r}{a}\right)^\ell . $$

\end{frame}


\begin{frame}
\frametitle{球外点电荷感应电势问题的解}

根据套路,在球外的解只要作个替换$r^\ell \rightarrow \frac{R^{2\ell +1}}{r^{\ell+1}}$:

$$\left. u(r, \theta)\right\vert_{r>R} = - \frac{Q}{4\pi\epsilon_0a}\sum_{\ell= 1}^\infty P_{\ell}(\cos\theta)\frac{R^{2\ell+1}}{a^\ell r^{\ell +1}}. $$

用镜像电荷(因为是电磁学内容,和数理方程关系不大,不再详细讲解)得到的表达式是有限的。这里的结果是级数展开。有兴趣的话可以尝试证明两者互相等价

\end{frame}




\begin{frame}
\frametitle{勒让德多项式的递推关系}

除了母函数定理之外,勒让德多项式最重要的性质是它的递推公式:
\tbox
    {$$ (2\ell+1)xP_\ell(x) = (\ell+1)P_{\ell+1}(x)+\ell P_{\ell-1}(x)$$ }

    (请先思考一下怎么证明)

\end{frame}

\begin{frame}
\frametitle{勒让德多项式的递推关系证明概要}

  $$ \frac{1}{\sqrt{1+t^2-2xt}} = \sum_{\ell =0}^\infty P_\ell (x) t^\ell,\ \ \ t<|x\pm \sqrt{x^2-1}|  .$$
两边对$t$求导,得到
$$ \frac{x-t}{(1-2xt+t^2)^{3/2}} =  \sum_{\ell = 1}^\infty \ell P_\ell(x)t^{\ell-1}$$
即
$$(x-t)\sum_{\ell=0}^\infty P_\ell(x)t^\ell  = (1-2xt+t^2)\sum_{\ell=1}^\infty \ell P_\ell(x)t^{\ell-1}$$
两边比较同次项系数即得证。

\end{frame}


\section{Rodrigues Formula}


\begin{frame}
  \frametitle{罗巨格公式}
  
  在球面谐函数的微分表示中令$m=0$,即得到
  $$ Y_{\ell 0}(\theta, \phi) = \frac{1}{2^\ell \ell!}\sqrt{\frac{2\ell+1}{4\pi}} \left(\frac{1}{\sin\theta}\frac{d}{d\theta}\right)^\ell \sin^{2\ell}\theta. $$
  利用$P_\ell(\cos\theta) = \sqrt{\frac{4\pi}{2\ell+1}}Y_{\ell 0}(\theta,\phi)$,上式成为
  \tbox
  {\blue  $$P_\ell(x) = \frac{1}{2^\ell \ell!}\frac{d^\ell}{dx^\ell}(x^2-1)^\ell.$$}
    这就是著名的{\blue 罗巨格公式 (Rodrigues' Formula)}.
  
  
\end{frame}

\begin{frame}
  \frametitle{罗巨格公式的直接证明}
  
  用球面谐函数的微分表示导出罗巨格公式有点杀鸡用牛刀的感觉。
  \skipline
  
  
  令$t = x-1$,也可以直截了当地用$P_\ell$的定义证明罗巨格公式
  \bea
  \frac{1}{2^\ell \ell!} \frac{d^\ell}{dx^\ell} \left[(x^2-1)^\ell\right] &=&  \frac{1}{\ell!} \frac{d^\ell}{dt^\ell} \left[t^\ell(1+\frac{t}{2})^\ell\right]  \newl
  &=&  \sum_{k=0}^\ell \frac{1}{2^kk!(\ell-k)!} \frac{d^\ell}{dt^\ell} \left[t^{\ell+k}\right]  \newl
  &=&  \sum_{k=0}^\ell \frac{(\ell+k)!}{(k!)^2(\ell-k)!} \left(\frac{t}{2}\right)^k  \newl
  &=& P_\ell(x)
  \eea
  
\end{frame}


\begin{frame}
  \frametitle{连带勒让德函数(Associated Legendre Functions)}
  
  很多MMP教材上的推导次序是相反的:从勒让德函数引出一个{\blue 连带勒让德函数:
  $$P_\ell^m(x) = (-1)^m(1-x^2)^{m/2}\frac{d^m}{dx^m}P_\ell(x), \ \ \ m = 0,1,2,\ldots, \ell. $$}
  然后再通过连带勒让德函数引出$m\ge 0$的球面谐函数,最后再定义$Y_{\ell,-m} = (-1)^mY_{\ell m}^*$来确定$m<0$的球面谐函数。

  \skipline
  
  $P_\ell^m$跟$\Psi_{\ell m}$之间的关系显而易见:
  $$P_\ell^m(\cos\theta) = \frac{1}{2^\ell\ell!}\Psi_{\ell m}(\theta). $$

   
\end{frame}

\begin{frame}
  \frametitle{例题1}
  
  把$x^n$ ($n\ge 0, n\in Z$, $-1\le x\le 1$) 展开成
  $$ x^n = \sum_{\ell = 0}^n c_\ell P_{\ell}(x) $$
  计算系数$c_\ell$。
  
\end{frame}

\begin{frame}
  \frametitle{解答}
  
  根据$P_\ell(x)$的正交归一性以及罗巨格公式,分部积分$\ell$次,得到:
  \bea
  c_\ell &=& \frac{2\ell+1}{2}\int_{-1}^1 P_\ell(x) x^n dx \newl
  &=& \frac{2\ell+1}{2^{\ell+1}\ell!} \int_{-1}^1 x^n \frac{d^\ell}{dx^\ell}(x^2-1)^\ell \newl
  &=& \frac{(2\ell+1)n!}{2^{\ell+1}\ell!(n-\ell)!} \int_{-1}^1  (1-x^2)^\ell  x^{n-\ell} dx
  \eea
  显然当$n-\ell$为奇数时,$c_\ell =0$。
  

  
  
\end{frame}


\begin{frame}
  \frametitle{解答(续)}
  
  当$n-\ell $为偶数时,设$n -\ell = 2k$,并做变量替换$t=x^2$:
  \bea
  c_\ell   &=& \frac{(2\ell+1)n!}{2^{\ell+1}\ell!(n-\ell)!} \int_0^1  (1-t)^\ell  t^{k-\frac{1}{2}} dt \newl
  &=&  \frac{(2\ell+1)(2k+\ell)!}{2^{\ell+1}\ell!(2k)!}\frac{\sfgamma{\ell+1}\sfgamma{k+\frac{1}{2}}}{\sfgamma{\ell +k+\frac{3}{2}}} \newl
  &=&  \frac{(2\ell+1) (2k+n)! \sfgamma{\frac{n-\ell+1}{2}} }{ 2^{\ell+1}(2k)! \sfgamma{\frac{n+\ell+3}{2} } } \newl
  &=&  \frac{2^\ell (2\ell+1)  (\ell+k)!(\ell+2k)! }{k! (2\ell+2k+1)! }  
  \eea
  因此最后得到:
  $$x^n = \sum_{k=0}^{[n/2]} \frac{2^{n-2k} (2n-4k+1)  (n-k)!n! }{k! (2n-2k+1)!} P_{n-2k}(x) .$$
  
\end{frame}

\begin{frame}
  \frametitle{例题2}
  
  把$e^{\ii \lambda x}$ ($-1\le x\le 1$为变量,$\lambda$为任意实数参量)展开成级数:
  $$ e^{\ii \lambda x} = \sum_{\ell = 0}^\infty c_\ell(\lambda) P_\ell(x). $$
  试计算展开系数$c_\ell(\lambda)$的显式表达式。
  
\end{frame}

\begin{frame}
  \frametitle{解答}
  
  \bea  
  e^{\ii \lambda x} &=& \sum_{n=0}^\infty \frac{\ii^n\lambda^n}{n!}x^n\newl
  &=& \sum_{n=0}^\infty \sum_{k=0}^{[n/2]}\frac{\ii^n\lambda^n}{n!}\frac{2^{n-2k} (2n-4k+1)  (n-k)!n! }{k! (2n-2k+1)!} P_{n-2k}(x) \newl
  &=& \sum_{\ell = 0}^\infty  2^\ell (2\ell+1)P_\ell(x) \sum_{k=0}^\infty \frac{(\ii\lambda)^{\ell+2k}(\ell+k)! }{k! (2\ell+2k+1)!} \newl
  &=& \sum_{\ell = 0}^\infty  (2\ell+1)\ii^\ell P_\ell(x)  \sqrt{\frac{\pi}{2\lambda}}\sum_{k=0}^\infty \frac{(-1)^k  }{k! \Gamma(\ell+k+\frac{3}{2})} \left(\frac{\lambda}{2}\right)^{\ell+2k+\frac{1}{2}} \newl
 &=&\sum_{\ell = 0}^\infty (2\ell+1)  \ii^\ell j_\ell(\lambda) P_\ell(x)   
  \eea

  
\end{frame}


\begin{frame}
  \frametitle{评论}
  
  在最后得到的结果
  \tbox{$$e^{\ii \lambda x} = \sum_{\ell = 0}^\infty (2\ell+1)  \ii^\ell j_\ell(\lambda) P_\ell(x)   $$}
  里,令$\lambda = kr$, $x = \cos\theta$,则得到
  \tbox{$$e^{\ii kr\cos\theta} = \sum_{\ell = 0}^\infty (2\ell+1)  \ii^\ell j_\ell(kr) P_\ell(\cos\theta)   $$}
  这相当于把波矢沿$z$轴方向的平面波写成了$k$相同的一堆球坐标系谐函数的线性组合(由轴对称性显然$m\ne 0$的项不会出现),再次验证了我们之前说过的{\blue “同一个$k$对应的不同坐标系下谐函数只是重新进行了线性组合而已“}。
  
\end{frame}


\section{Symmetries of $Y_{\ell m}$}


\begin{frame}
  \frametitle{球谐函数的宇称:$Y_{\ell m}(\vecn)$和$Y_{\ell,m}(-\vecn)$之间的关系}
  对相反的两个方向,有
  \tbox{$$ Y_{\ell m}(\vecn) = (-1)^\ell Y_{\ell m}(-\vecn)$$}
  
\end{frame}

\begin{frame}
  \frametitle{证明}
  
  如果$\vecn$对应$(\theta,\phi)$,则$-\vecn$对应$(\pi-\theta, \pi+\phi)$。{\small
  \bea
  && Y_{\ell m}(\pi - \theta,\pi+\phi) \newl
  &=&  N_{\ell m}\left\{ \sin^m(\pi - \theta) \left[\frac{1}{\sin(\pi-\theta)}\frac{d}{d(\pi-\theta)} \right]^{\ell + m}\sin^{2\ell}(\pi-\theta)\right\} e^{\ii m(\pi+\phi)} \newl
  &=&  N_{\ell m} \left\{\sin^m\theta \left(-\frac{1}{\sin \theta}\frac{d}{d \theta} \right)^{\ell + m}\sin^{2\ell}\theta\right\} (-1)^me^{\ii m\phi} \newl
    &=&  (-1)^{\ell+2m}N_{\ell m} \left[\sin^m\theta \left(\frac{1}{\sin \theta}\frac{d}{d \theta} \right)^{\ell + m}\sin^{2\ell}\theta\right]e^{\ii m\phi} \newl  
    &=& (-1)^\ell Y_{\ell m}(\theta,\phi)
  \eea}
  
  
\end{frame}

\begin{frame}
  \frametitle{$Y_{\ell m}$和$Y_{\ell,-m}$之间的关系}
  

  \tbox{$$Y_{\ell, -m}(\theta, \phi) = (-1)^m Y_{\ell, m}^*(\theta,\phi)$$}
  
\end{frame}


\begin{frame}
  \frametitle{证明大意}
  
  把$Y_{\ell m}^*(\theta,\phi) = N_{\ell, m}\Psi_{\ell m}(\theta)e^{-\ii m\phi}$在单位球面上展开:
  $$ Y_{\ell m}^*(\theta,\phi)  = \sum_{\ell',m'}c_{\ell' m'} Y_{\ell' m'}(\theta,\phi). $$
  两边同乘以$Y_{\ell'm'}^*$并在单位球面上积分,得到
  $$c_{\ell' m'} = \int_0^\pi \sin\theta d\theta \int_0^{2\pi}d\phi \,Y_{\ell' m'}^*(\theta,\phi) Y_{\ell m}^*(\theta,\phi).$$
  即
  $$c_{\ell' m'} = N_{\ell'm'}N_{\ell m}\int_0^\pi  \Psi _{\ell' m'}(\theta)\Psi_{\ell m}(\theta) \sin\theta d\theta \int_0^{2\pi}d\phi  e^{-\ii (m'+m)\phi}.$$
  显然,当$m'+m\ne 0$时上式为零。当$m'=-m$,但$\ell\ne \ell'$时,$\Psi_{\ell m}(\theta)\cos m\phi$和$\Psi'_{\ell' m'}(\theta)\cos m\phi$都是单位球面上的谐函数,且它们对应不同的$k^2_{2D}$ (即$\ell(\ell+1)$和$\ell'(\ell'+1)$),根据一般正交定理,上式的积分仍为零。



  
\end{frame}

\begin{frame}
  \frametitle{证明大意}
  

  
  因此我们只需要计算$c_{\ell ,-m}$。剩下的计算和证明$\Psi_{\ell m}$归一化积分时的过程几乎完全相同:
  {\scriptsize
  \bea
  c_{\ell ,-m} &=& 2\pi N_{\ell m}N_{\ell, -m} \int_0^\pi \sin\theta d\theta \Psi_{\ell m}(\theta) \Psi_{\ell ,-m}(\theta)  \newl
  &=& 2\pi N_{\ell m}N_{\ell, -m} \int_0^\pi \sin\theta d\theta \left(\frac{1}{\sin\theta} \frac{d}{d\theta}\right)^{\ell-m}\sin^{2\ell}\theta \left(\frac{1}{\sin\theta} \frac{d}{d\theta}\right)^{\ell+m}\sin^{2\ell}\theta \newl
  &=& 2\pi N_{\ell m}N_{\ell, -m} \int_{-1}^1  \frac{d^{\ell -m}}{dx^{\ell-m}}(x^2-1)^\ell   \frac{d^{\ell +m}}{dx^{\ell+m}}(x^2-1)^\ell dx \newl
  &=&  2\pi N_{\ell m}N_{\ell, -m} (-1)^{\ell+m}\int_{-1}^1  (x^2-1)^\ell    \frac{d^{2\ell}}{dx^{2\ell}}(x^2-1)^\ell dx \newl
  &=&  2\pi  N_{\ell m}N_{\ell, -m} (-1)^m (2\ell)!\int_{-1}^1  (1-x^2)^\ell dx  \newl
  &=&  2\pi  N_{\ell m}N_{\ell, -m}(-1)^m \frac{(\ell-m)!}{(\ell+m)!} \frac{1}{2\pi N_{\ell m}^2} \newl
  &=&  (-1)^m \frac{(\ell-m)!}{(\ell+m)!} \frac{N_{\ell, -m}}{N_{\ell m}} \newl
  &=& (-1)^m
  \eea
  }
  
\end{frame}

\begin{frame}
  \frametitle{三个$Y_{\ell m}$积分的对称性}
  
  定理:要使单位球面上的积分
  $$\int Y_{\ell_1m_1}(\vecn)Y_{\ell_2m_2}(\vecn)Y_{\ell_3m_3}(\vecn) d^2\vecn$$
  非零,则必须有:
  \bitem
  \item{$m_1+m_2+m_3 = 0;$}
  \item{以$\ell_1, \ell_2,\ell_3$为三条边可以构成三角形,即$|\ell_1-\ell_2|\le \ell_3\le \ell_1+\ell_2 ; $}
  \item{$\ell_1+\ell_2+\ell_3$是偶数。}
    \eitem

  
\end{frame}


\begin{frame}
  \frametitle{第一个条件的证明}
  
  $$ Y_{\ell_1m_1}(\vecn)Y_{\ell_2m_2}(\vecn)Y_{\ell_3m_3}(\vecn) \propto \Psi_{\ell_1m_1}(\theta)\Psi_{\ell_2m_2}(\theta)\Psi_{\ell_3m_3}(\theta) e^{\ii (m_1+m_2+m_3)\phi}. $$
  如果$m_1+m_2+m_3\ne 0$,则对$\phi$积分已经是零。

  
\end{frame}


\begin{frame}
  \frametitle{  第二个条件的证明}
  

  
  下面考虑$m_1+m_2+m_3 = 0$的情况,如果$\ell_1$, $\ell_2$, $\ell_3$不能构成三角形,不妨设$\ell_1>\ell_2+\ell_3$,则运用分部积分的技巧:
  \bea
  && \int_0^\pi \Psi_{\ell_1m_1}(\theta)\Psi_{\ell_2m_2}(\theta)\Psi_{\ell_3m_3}(\theta) \sin\theta d\theta  \newl
  &=& \int_{-1}^1 \frac{d^{\ell_1+m_1}}{dx^{\ell_1+m_1}}(x^2-1)^{\ell_1}  \frac{d^{\ell_2+m_2}}{dx^{\ell_2+m_2}}(x^2-1)^{\ell_2} \frac{d^{\ell_3+m_3}}{dx^{\ell_3+m_3}}(x^2-1)^{\ell_3} dx \newl
  &=& (-1)^{\ell_1+m_1}\int_{-1}^1 (x^2-1)^{\ell_1} \newl
  && \times \frac{d^{\ell_1+m_1}}{dx^{\ell_1+m_1}} \left[{\blue \frac{d^{\ell_2+m_2}}{dx^{\ell_2+m_2}}(x^2-1)^{\ell_2} \frac{d^{\ell_3+m_3}}{dx^{\ell_3+m_3}}(x^2-1)^{\ell_3}}\right]dx \newl
  &=&  0
  \eea

  最后一步的结果是根据蓝色部分是次数低于$\ell_1+m_1$次的多项式,求导$\ell_1+m_1$次后显然为零。

  
\end{frame}

\begin{frame}
  \frametitle{  第三个条件的证明}
  
  \bea
  && Y_{\ell_1m_1}(-\vecn)Y_{\ell_2m_2}(-\vecn)Y_{\ell_3m_3}(-\vecn) \newl
  &=& (-1)^{\ell_1+\ell_2+\ell_3}Y_{\ell_1m_1}(\vecn)Y_{\ell_2m_2}(\vecn)Y_{\ell_3m_3}(\vecn),
  \eea
  显然若$\ell_1+\ell_2+\ell_3$为奇数时,积分在$\vecn$和$-\vecn$处两两抵消,结果为零。
  
\end{frame}

\section{$\delta$ function expansion}
\secpage{$\delta$函数和正交归一完备函数组的关系}{$$\delta(\vecx-\vecx') = \sum_i Q_i(\vecx)Q_i(\vecx') $$}

\begin{frame}
  \frametitle{$\delta$函数和正交归一完备函数组的关系}
  
  设$n$维空间的某区域$\Omega$内函数组$Q_i(\vecx)$构成正交归一的完备函数组,即
  $$\int_\Omega Q_i(\vecx)Q_j(\vecx) d^n\vecx = \delta_{ij},$$
  由完备性,$\Omega$内的函数总是可以用$Q_i(\vecx)$进行展开,特别地,对固定的$\vecx'$,我们把函数$\delta(\vecx-\vecx')$进行展开:
  $$\delta(\vecx-\vecx') = \sum_i c_iQ_i(\vecx)$$
  由$Q_i$的正交归一性,系数
  $$c_i = \int_\Omega \delta(\vecx-\vecx')Q_i(\vecx) d^n\vecx = Q_i(\vecx')$$
  
  
\end{frame}


\begin{frame}
  \frametitle{$\delta$函数和正交归一完备函数组的关系}
  
  把$c_i$代回去得到:
  \tbox{
    $$\delta(\vecx-\vecx') = \sum_i Q_i(\vecx')Q_i(\vecx).$$}
  如果是用{\blue 复函数内积定义的正交归一性},则上式变为
  \tbox{\blue 
    $$\delta(\vecx-\vecx') = \sum_i Q_i^*(\vecx')Q_i(\vecx).$$}
  如果正交归一性里的内积定义带权重:$\int_\Omega Q_i(\vecx)Q_j(\vecx)\rho(\vecx)d^n\vecx = \delta_{ij} $,则

  \tbox{\blue 
    $$\delta(\vecx-\vecx') = \sum_i \rho(\vecx')Q_i(\vecx')Q_i(\vecx).$$}

  
\end{frame}

\begin{frame}
  \frametitle{单位球面上的$\delta$函数}
  
  单位球面上的坐标$\theta,\phi$对应了一个方向$\vecn$(从原点出发指向单位球面上$\theta,\phi$点的单位矢量)。单位球面上的函数可以抽象地写为$f(\vecn)$的形式。单位球面上的面积元$\sin\theta d\theta d\phi$可以抽象地写作$d^2\vecn$。
  

  \skipline
  
  在不致引起混淆的情况下,我们也{\blue 用$Y_{\ell m}(\vecn)$来表示$Y_{\ell m}(\theta,\phi)$}。根据前面的讨论,有
{\blue
  $$\delta(\vecn-\vecn') = \sum_{\ell m} Y_{\ell m}^*(\vecn') Y_{\ell m}(\vecn).$$}
这里的$\delta(\vecn-\vecn')$是单位球面上的二维$\delta$函数(在$\vecn'$处的大小为$\epsilon\rightarrow 0^+$的小面积上的函数值为$\frac{1}{\epsilon}$,其余处处为零)。
$\sum_{\ell m}$是双重求和$\sum_{\ell = 0}^\infty\sum_{m =-\ell}^\ell $的简写。
  
\end{frame}




\section{Addition Theorem}
\secpage{球谐函数的加法公式}{$$P_\ell (\vecn_1\cdot\vecn_2) = \frac{4\pi}{2\ell+1}\sum_{m=-\ell}^\ell Y_{\ell m}^*(\vecn_1)Y_{\ell m}(\vecn_2)$$ }

\begin{frame}
  \frametitle{球谐函数的加法公式}
  
  设$\vecn_1$, $\vecn_2$为任意两个方向,则
{\blue
  $$P_\ell (\vecn_1\cdot\vecn_2) = \frac{4\pi}{2\ell+1}\sum_{m=-\ell}^\ell Y_{\ell m}^*(\vecn_1)Y_{\ell m}(\vecn_2),$$}
这里的$\vecn_1\cdot\vecn_2$是$\vecn_1$和$\vecn_2$的夹角的余弦。它可以用$\vecn_1$的坐标$(\theta_1,\phi_1)$和$\vecn_2$的坐标$(\theta_2,\phi_2)$明确地表示出来:
$$\vecn_1\cdot\vecn_2 = \cos\theta_1\cos\theta_2+\sin\theta_1\sin\theta_2\cos(\phi_1-\phi_2) .$$

\end{frame}

\begin{frame}
  \frametitle{证明}
  
  考虑在单位球面上的电荷产生的电势。

  在单位球内:
  $$ u(r, \vecn) = \sum_{\ell,m}c_{\ell m}r^\ell Y_{\ell m}(\vecn),\ \ \ r<1 ;$$
  在单位球外:
  $$ u(r, \vecn) = \sum_{\ell,m}c_{\ell m}r^{-\ell-1} Y_{\ell m}(\vecn),\ \ \ r>1 .$$
  
  根据高斯定理算出单位球面上的电荷面密度为
  $$\sigma(\vecn) = \epsilon_0 \sum_{\ell,m}(2\ell+1)c_{\ell m}Y_{\ell m}(\vecn).$$
  
\end{frame}


\begin{frame}
  \frametitle{证明(续)}
  
  然后考虑球面上$\vecn_1$处有点电荷的情况:
  $$\sigma(\vecn) = Q \delta(\vecn-\vecn_1) = Q\sum_{\ell m}Y_{\ell m}^*(\vecn_1)Y_{\ell m}(\vecn).$$
  对比系数我们确定
  
  $$ c_{\ell m} = \frac{Q}{(2\ell+1)\epsilon_0}  Y_{\ell m}^*(\vecn_1). $$
  即在球内的电势为
  $$ u(r,\vecn_2)= \frac{Q}{(2\ell+1)\epsilon_0} \sum_{\ell m} Y_{\ell m}^*(\vecn_1)Y_{\ell m}(\vecn_2)r^\ell,  \ \ \ r<1. $$
  
\end{frame}

\begin{frame}
  \frametitle{证明(续)}
  
  另一方面,根据库仑定律直接可以算出$(r, n_2)$处的电势为
  $$u(r,\vecn_2) = \frac{Q}{4\pi\epsilon_0\sqrt{1+r^2-2r\vecn_1\cdot\vecn_2}} = \frac{Q}{4\pi\epsilon_0}\sum_{\ell}P_\ell(\vecn_1\cdot\vecn_2) r^\ell .$$
  比较$r^\ell$的系数,即得证。
  
\end{frame}


\section{Homework}

\begin{frame}
\frametitle{Homework}

\bitem
\item{计算积分
  $$\int_{-1}^1 P_\ell(x)\ln(1-x)dx, $$
其中$P_\ell$为勒让德多项式。
}
\item{证明$\cos\theta Y_{\ell m}(\theta,\phi)$可以写成$Y_{\ell+1,m}(\theta,\phi)$和$Y_{\ell-1,m}(\theta,\phi)$的线性组合:
  $$ \cos\theta Y_{\ell m}(\theta,\phi) = a Y_{\ell+1,m}(\theta,\phi)+ b Y_{\ell-1,m}(\theta,\phi), $$
  其中$a, b$为只依赖于$\ell, m$的常数。
  }
\item{我们曾经提到过同样的$k$对应的不同坐标系的谐函数可以互相线性表示出来。对直角坐标系和球坐标系而言,我们在课上证明了轴对称的情况。请利用球面谐函数的加法公式把这个结果推广到任意的平面波:
  $$ e^{\ii \veck\cdot\vecx} = 4\pi \sum_{\ell, m}\ii^\ell Y_{\ell m}^*(\hat{\veck})Y_{\ell m}(\hat{\vecx})j_{\ell}(kr),$$
其中$r=|\vecx|$, $\hat{\vecx}$和$\hat{\veck}$分别是$\vecx$和$\veck$方向的单位矢量。}
  \eitem

\end{frame}


\section{Appendix}

\begin{frame}
\frametitle{附录: 勒让德多项式的母函数定理的证明}
{\scriptsize
\bea
\frac{1}{\sqrt{1+t^2-2xt}} &=&\frac{1}{(1-t)\sqrt{1-\frac{2(x-1)t}{(1-t)^2}}} \newl
&=& \frac{1}{1-t}\sum_{k=0}^\infty \bral -\frac{1}{2}\\ k \brar \left(-\frac{2(x-1)t}{(1-t)^2}\right)^k \newl
&=& \sum_{k=0}^\infty \frac{(2k)!}{2^k(k!)^2}(x-1)^kt^k(1-t)^{-2k-1}\newl
&=& \sum_{k=0}^\infty \frac{(2k)!}{2^k(k!)^2}(x-1)^kt^k\sum_{n=0}^\infty \bral -2k-1 \\ n \brar (-t)^n\newl
&=& \sum_{k=0}^\infty\sum_{n=0}^\infty \frac{(2k+n)!}{2^k(k!)^2n!}(x-1)^k  t^{n+k}\newl
&=& \sum_{\ell=0}^\infty\sum_{k=0}^\ell \frac{(\ell + k)!}{(k!)^2(\ell-k)!}\left(\frac{x-1}{2}\right)^k  t^\ell
\eea
}

\end{frame}




\begin{frame}
\frametitle{附录: 勒让德多项式的母函数定理的证明(续)}

定义{\blue 勒让德多项式
  $$ P_{\ell}(x) = \sum_{k=0}^\ell \frac{(\ell + k)!}{(k!)^2(\ell-k)!}\left(\frac{x-1}{2}\right)^k. $$}
显然它是$\ell$次多项式。于是只要证明
\bitem
\item[1]{$P_\ell(\cos\theta)$满足$m=0$的单位球面谐函数方程}
\item[2]{$P_\ell$的归一化满足
$$ \frac{2\ell+1}{4\pi} \int_0^\pi P_\ell(\cos\theta)^2 \sin\theta d\theta\int_0^{2\pi}d\phi = 1 .$$
即{\blue
$$ \int_{-1}^1 \left[P_\ell(x)\right]^2 dx = \frac{2}{2\ell+1}.$$}
}
\eitem

\end{frame}


\begin{frame}
\frametitle{附录:勒让德多项式的母函数定理的证明(续)}

回忆$m=0$的谐函数方程为
$$ \frac{1}{\sin\theta}\frac{d}{d\theta}\left(\sin\theta \frac{d}{d\theta}\Psi\right) + \ell(\ell+1) \Psi= 0.$$
令$x = \cos\theta$,则该微分方程等价于:
$$ \frac{d}{dx}\left[(1-x^2)\frac{d}{dx} \Psi\right]+\ell(\ell+1)\Psi=0 .$$
验证$P_\ell(x)$满足该微分方程的过程并不困难,留为练习。

\skipline

剩下的归一化条件,只要用到不同$\ell$对应的勒让德多项式互相正交,就能很容易证明。也留为作业。

\end{frame}


\ech
\end{document}
