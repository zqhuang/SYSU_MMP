\documentclass[CJK]{beamer}
\usepackage{CJKutf8}
\usepackage{beamerthemesplit}
\usetheme{Malmoe}
\useoutertheme[footline=authortitle]{miniframes}
\usepackage{amsmath}
\usepackage{amssymb}
\usepackage{graphicx}
\usepackage{eufrak}
\usepackage{color}
\usepackage{slashed}
\usepackage{simplewick}
\usepackage{tikz}
\usepackage{tcolorbox}
\graphicspath{{../figures/}}
%%figures
\def\lfig#1#2{\includegraphics[width=#1 in]{#2}}
\def\addfig#1#2{\begin{center}\includegraphics[width=#1 in]{#2}\end{center}}
\def\wulian{\includegraphics[width=0.18in]{emoji_wulian.jpg}}
\def\bigwulian{\includegraphics[width=0.35in]{emoji_wulian.jpg}}
\def\bye{\includegraphics[width=0.18in]{emoji_bye.jpg}}
\def\bigbye{\includegraphics[width=0.35in]{emoji_bye.jpg}}
\def\huaixiao{\includegraphics[width=0.18in]{emoji_huaixiao.jpg}}
\def\bighuaixiao{\includegraphics[width=0.35in]{emoji_huaixiao.jpg}}
\def\jianxiao{\includegraphics[width=0.18in]{emoji_jianxiao.jpg}}
\def\bigjianxiao{\includegraphics[width=0.35in]{emoji_jianxiao.jpg}}
%% colors
\def\blacktext#1{{\color{black}#1}}
\def\bluetext#1{{\color{blue}#1}}
\def\redtext#1{{\color{red}#1}}
\def\darkbluetext#1{{\color[rgb]{0,0.2,0.6}#1}}
\def\skybluetext#1{{\color[rgb]{0.2,0.7,1.}#1}}
\def\cyantext#1{{\color[rgb]{0.,0.5,0.5}#1}}
\def\greentext#1{{\color[rgb]{0,0.7,0.1}#1}}
\def\darkgray{\color[rgb]{0.2,0.2,0.2}}
\def\lightgray{\color[rgb]{0.6,0.6,0.6}}
\def\gray{\color[rgb]{0.4,0.4,0.4}}
\def\blue{\color{blue}}
\def\red{\color{red}}
\def\green{\color{green}}
\def\darkgreen{\color[rgb]{0,0.4,0.1}}
\def\darkblue{\color[rgb]{0,0.2,0.6}}
\def\skyblue{\color[rgb]{0.2,0.7,1.}}
%%control
\def\be{\begin{equation}}
\def\ee{\nonumber\end{equation}}
\def\bea{\begin{eqnarray}}
\def\eea{\nonumber\end{eqnarray}}
\def\bch{\begin{CJK}{UTF8}{gbsn}}
\def\ech{\end{CJK}}
\def\bitem{\begin{itemize}}
\def\eitem{\end{itemize}}
\def\bcenter{\begin{center}}
\def\ecenter{\end{center}}
\def\bex{\begin{minipage}{0.2\textwidth}\includegraphics[width=0.6in]{jugelizi.png}\end{minipage}\begin{minipage}{0.76\textwidth}}
\def\eex{\end{minipage}}
\def\chtitle#1{\frametitle{\bch#1\ech}}
\def\bmat#1{\left(\begin{array}{#1}}
\def\emat{\end{array}\right)}
\def\bcase#1{\left\{\begin{array}{#1}}
\def\ecase{\end{array}\right.}
\def\bmini#1{\begin{minipage}{#1\textwidth}}
\def\emini{\end{minipage}}
\def\tbox#1{\begin{tcolorbox}#1\end{tcolorbox}}
\def\pfrac#1#2#3{\left(\frac{\partial #1}{\partial #2}\right)_{#3}}
%%symbols
\def\bropt{\,(\ \ \ )}
\def\sone{$\star$}
\def\stwo{$\star\star$}
\def\sthree{$\star\star\star$}
\def\sfour{$\star\star\star\star$}
\def\sfive{$\star\star\star\star\star$}
\def\rint{{\int_\leftrightarrow}}
\def\roint{{\oint_\leftrightarrow}}
\def\stdHf{{\textit{\r H}_f}}
\def\deltaH{{\Delta \textit{\r H}}}
\def\ii{{\dot{\imath}}}
\def\skipline{{\vskip0.1in}}
\def\skiplines{{\vskip0.2in}}
\def\lagr{{\mathcal{L}}}
\def\hamil{{\mathcal{H}}}
\def\vecv{{\mathbf{v}}}
\def\vecx{{\mathbf{x}}}
\def\vecy{{\mathbf{y}}}
\def\veck{{\mathbf{k}}}
\def\vecp{{\mathbf{p}}}
\def\vecn{{\mathbf{n}}}
\def\vecA{{\mathbf{A}}}
\def\vecP{{\mathbf{P}}}
\def\vecsigma{{\mathbf{\sigma}}}
\def\hatJn{{\hat{J_\vecn}}}
\def\hatJx{{\hat{J_x}}}
\def\hatJy{{\hat{J_y}}}
\def\hatJz{{\hat{J_z}}}
\def\hatj#1{\hat{J_{#1}}}
\def\hatphi{{\hat{\phi}}}
\def\hatq{{\hat{q}}}
\def\hatpi{{\hat{\pi}}}
\def\vel{\upsilon}
\def\Dint{{\mathcal{D}}}
\def\adag{{\hat{a}^\dagger}}
\def\bdag{{\hat{b}^\dagger}}
\def\cdag{{\hat{c}^\dagger}}
\def\ddag{{\hat{d}^\dagger}}
\def\hata{{\hat{a}}}
\def\hatb{{\hat{b}}}
\def\hatc{{\hat{c}}}
\def\hatd{{\hat{d}}}
\def\hatN{{\hat{N}}}
\def\hatH{{\hat{H}}}
\def\hatp{{\hat{p}}}
\def\Fup{{F^{\mu\nu}}}
\def\Fdown{{F_{\mu\nu}}}
\def\newl{\nonumber \\}
\def\vece{\mathrm{e}}
\def\calM{{\mathcal{M}}}
\def\calT{{\mathcal{T}}}
\def\calR{{\mathcal{R}}}
\def\barpsi{\bar{\psi}}
\def\baru{\bar{u}}
\def\barv{\bar{\upsilon}}
\def\qeq{\stackrel{?}{=}}
\def\torder#1{\mathcal{T}\left(#1\right)}
\def\rorder#1{\mathcal{R}\left(#1\right)}
\def\contr#1#2{\contraction{}{#1}{}{#2}#1#2}
\def\trof#1{\mathrm{Tr}\left(#1\right)}
\def\trace{\mathrm{Tr}}
\def\comm#1{\ \ \ \left(\mathrm{used}\ #1\right)}
\def\tcomm#1{\ \ \ (\text{#1})}
\def\slp{\slashed{p}}
\def\slk{\slashed{k}}
\def\calp{{\mathfrak{p}}}
\def\veccalp{\mathbf{\mathfrak{p}}}
\def\Tthree{T_{\tiny \textcircled{3}}}
\def\pthree{p_{\tiny \textcircled{3}}}
\def\dbar{{\,\mathchar'26\mkern-12mu d}}
\def\erf{\mathrm{erf}}
\def\const{\mathrm{constant}}
\def\pheat{\pfrac p{\ln T}V}
\def\vheat{\pfrac V{\ln T}p}
%%units
\def\fdeg{{^\circ \mathrm{F}}}
\def\cdeg{^\circ \mathrm{C}}
\def\atm{\,\mathrm{atm}}
\def\angstrom{\,\text{\AA}}
\def\SIL{\,\mathrm{L}}
\def\SIkm{\,\mathrm{km}}
\def\SIyr{\,\mathrm{yr}}
\def\SIGyr{\,\mathrm{Gyr}}
\def\SIV{\,\mathrm{V}}
\def\SImV{\,\mathrm{mV}}
\def\SIeV{\,\mathrm{eV}}
\def\SIkeV{\,\mathrm{keV}}
\def\SIMeV{\,\mathrm{MeV}}
\def\SIGeV{\,\mathrm{GeV}}
\def\SIcal{\,\mathrm{cal}}
\def\SIkcal{\,\mathrm{kcal}}
\def\SImol{\,\mathrm{mol}}
\def\SIN{\,\mathrm{N}}
\def\SIHz{\,\mathrm{Hz}}
\def\SIm{\,\mathrm{m}}
\def\SIcm{\,\mathrm{cm}}
\def\SIfm{\,\mathrm{fm}}
\def\SImm{\,\mathrm{mm}}
\def\SInm{\,\mathrm{nm}}
\def\SImum{\,\mathrm{\mu m}}
\def\SIJ{\,\mathrm{J}}
\def\SIW{\,\mathrm{W}}
\def\SIkJ{\,\mathrm{kJ}}
\def\SIs{\,\mathrm{s}}
\def\SIkg{\,\mathrm{kg}}
\def\SIg{\,\mathrm{g}}
\def\SIK{\,\mathrm{K}}
\def\SImmHg{\,\mathrm{mmHg}}
\def\SIPa{\,\mathrm{Pa}}

\def\courseurl{https://github.com/zqhuang/SYSU\_TD}

\def\tpage#1#2{
\begin{frame}
\begin{center}
\begin{Large}
\bch
热学 \\
第#1讲 #2

{\vskip 0.3in}

黄志琦

\ech
\end{Large}
\end{center}

\vskip 0.2in

\bch
教材:《热学》第二版,赵凯华,罗蔚茵,高等教育出版社
\ech

\bch
课件下载
\ech
\courseurl
\end{frame}
}

\def\bfr#1{
\begin{frame}
\chtitle{#1} 
\bch
}

\def\efr{
\ech 
\end{frame}
}

  \date{}
  \begin{document}
  \bch
\tpage{20}{第一类贝塞尔函数和应用举例}

\begin{frame}
\frametitle{本讲内容}
\tableofcontents
\end{frame}

\section{Bessel Functions of the First Kind}
\secpage{第一类贝塞尔函数的基本性质}{$$\frac{d}{dx}\left[x^mJ_m(x)\right] = x^mJ_{m-1}(x); \ \frac{d}{dx}\left[x^{-m}J_m(x)\right] = -x^{-m}J_{m+1}(x).$$}

\begin{frame}
  \frametitle{第一类贝塞尔函数的级数定义}
  
  第一类贝塞尔函数定义为
  \tbox{
    $$J_m(x) = \sum_{k=0}^\infty \frac{(-1)^k}{k!(k+m)!}\left(\frac{x}{2}\right)^{2k+m}$$}

  先讨论物理问题中最常用的$m=0,1,2,\ldots$的情形 ($m$不是整数和$m<0$的情况且听下回分解)。

  
\end{frame}


\begin{frame}
  \frametitle{$J_0$, $J_1$, 和 $J_2$}
  
  \addfig{3}{Bessel012.png}
  
\end{frame}

\begin{frame}
  \frametitle{$J_m(x)$的定性特点}
  
  \tbox{第一个峰,也就是最大值,大致在$x \approx m$处取到}
  \tbox{第一个峰左边近似以$x^m$增长}
  \tbox{第一个峰右边是振幅不断衰减的振荡型函数}
 注: 在$x$比较大时,振幅大致上$\sim \frac{1}{\sqrt{x}}$,周期趋向于$2\pi$。
  
\end{frame}


\begin{frame}
  \frametitle{思考题}
  
  \addfig{0.5}{think.jpg}
  请一位同学在黑板上画出$J_{100}(x)$的大致图像.
  
\end{frame}



\begin{frame}
  \frametitle{怎么计算$J_m(x)$}
  
  大多数编程语言(C, Fortran等)和带数学库的脚本语言 (Python, Mathematica等)都内置有贝塞尔函数,直接调用就能计算。

    \skiplines
    
  致手算强迫症患者\huaixiao:有一整套的近似公式可以满足你的需求,欢迎来电采购,量大优惠。
  
\end{frame}


\begin{frame}
  \frametitle{贝塞尔函数的递推关系}
  
  贝塞尔函数大概有几十条性质。养生MMP课里不可能全部介绍,因此,从最重要也最普遍的递推关系出发

  {\blue 贝塞尔函数的第一条递推性质}
  \tbox{$$\frac{d}{dx}\left[x^mJ_m(x)\right] = x^mJ_{m-1}(x).$$}

  {\blue 贝塞尔函数的第二条递推性质}
  \tbox{$$\frac{d}{dx}\left[x^{-m}J_m(x)\right] = -x^{-m}J_{m+1}(x).$$}

  (之所以说这是普遍关系,是因为第二类贝塞尔函数$N_m(x)$也满足完全一样的递推关系。)
  
\end{frame}


\begin{frame}
  \frametitle{贝塞尔函数的递推关系的简单应用}
  
  \addfig{0.6}{think2.jpg}
  证明:
  \bea
  J_{m-1}(x)-J_{m+1}(x) &=& 2 J_m'(x),\newl
  J_{m-1}(x)+J_{m+1}(x) &=& \frac{2m}{x} J_m(x).
  \eea
  
\end{frame}




\begin{frame}
  
  \addfig{1}{think1.jpg}
  
  剩下几十条性质,从哪儿开始入手呢?
  
\end{frame}



\section{Brane Oscillation}

\secpage{圆形薄膜的振动问题}{用谐函数$J_m(kr)\cos{m\theta}$和$J_m(kr)\sin{m\theta}$}

\begin{frame}
  \frametitle{圆形薄膜的振动问题}
  

  \addfig{1.2}{drum.jpg}
  
  考虑固定边界的,半径为$R$的圆形薄膜的横向小振动问题。在$t=0$时刻的初始位移为$A\left[1-\left(\frac{r}{R}\right)^2\right]$ (其中$A$为常量,$r$为距离圆盘中心的距离,$0\le r\le R$),初始速度为零。求解之后薄膜的振动。
  
\end{frame}


\begin{frame}
  \frametitle{写出方程和边界条件}
  
  取极坐标系,设位移为$u(r,\theta,t)$。

  \bea
  \frac{\partial^2u}{\partial t^2} - a^2\nabla^2 u = 0 , \newl
  \left.u\right\vert_{r=R} = 0,\newl
  \left.u\right\vert_{t = 0} = A\left[1-\left(\frac{r}{R}\right)^2\right] , \newl
  \left.\frac{\partial u}{\partial t}\right\vert_{t = 0} = 0.
  \eea
  
\end{frame}


\begin{frame}
  \frametitle{分离变量}
  
  我们知道圆盘内的谐函数为$J_m(kr)e^{\pm \ii m\theta}$。在这个问题里初始条件是旋转对称的(不依赖于$\theta$),所以解也不依赖于$\theta$。也就是说,只需要考虑$m=0$的情形,分离变量形式的解为
    $$J_0(kr)\cos (akt),\  \  J_0(kr)\sin(akt)$$
    


    然后考虑边界条件:符合$J_0(kR) = 0$的解为
    $$ k = \frac{\mu_i}{R}, i = 1,2,\ldots $$
    其中$\mu_i$是贝塞尔函数$J_0(x)$的第$i$个正实数根。

    
$\mu_1\approx 2.4048$, $\mu_2 \approx 5.5201$, $\mu_3 \approx 8.6537$, $\mu_4 \approx 11.7915$, $\mu_5\approx 14.9309$, \ldots	

    
  
\end{frame}




\begin{frame}
  \frametitle{展开}
  
  解的展开形式就是
  $$u = \sum_i c_i J_0\left(\frac{\mu_ir}{R}\right) \cos\left(\frac{\mu_iat}{R}\right) + s_i J_0\left(\frac{\mu_ir}{R}\right) \sin\left(\frac{\mu_iat}{R}\right). $$  
  利用速度为零的初始条件,立刻可以扔掉所有$\sin(akt)$项。
  $$u = \sum_i c_i J_0\left(\frac{\mu_ir}{R}\right) \cos\left(\frac{\mu_iat}{R}\right). $$
  再利用初始位移,得到
  $$\sum_i c_i J_0\left(\frac{\mu_ir}{R}\right) =  A\left[1-\left(\frac{r}{R}\right)^2\right]. $$  
  
\end{frame}


\begin{frame}
  \frametitle{求系数的套路}
  
  记得我们以前的套路是两边同乘上谐函数(正弦或余弦),然后在所求问题的物理范围内积分。利用谐函数的积分正交性就可以求出系数$c_i$。

  \skipline
  根据谐函数的正交定理,我们仍然可以这么做,区别就是要在圆盘内进行面积分。由谐函数正交定理,有:
  $$\int_0^R r dr\int_0^{2\pi} d\theta\, J_0\left(\frac{\mu_ir}{R}\right) J_0\left(\frac{\mu_jr}{R}\right) = N_i \delta_{ij}. $$
  这里
  $$N_i = 2\pi \int_0^R\left[ J_0\left(\frac{\mu_ir}{R}\right) \right]^2  r dr$$
\end{frame}


\begin{frame}
  把展开式
  $$\sum_i c_i J_0\left(\frac{\mu_ir}{R}\right) =  A\left[1-\left(\frac{r}{R}\right)^2\right]. $$  
  按照套路两边同乘以 $J_0\left(\frac{\mu_jr}{R}\right)$并在圆盘内进行面积分,得到
  $$ c_j N_j = \int_0^Rrdr\int_0^{2\pi}d\theta\, A\left[1-\left(\frac{r}{R}\right)^2\right]J_0\left(\frac{\mu_jr}{R}\right). $$
  稍作整理得到系数:
  $$ c_j = \frac{2\pi A}{N_j}\int_0^R r\left[1-\left(\frac{r}{R}\right)^2\right]J_0\left(\frac{\mu_jr}{R}\right)\, dr. $$  
\end{frame}


\begin{frame}
  再回过头来看结果中的两个积分
  $$N_i = 2\pi \int_0^R\left[ J_0\left(\frac{\mu_ir}{R}\right) \right]^2  r dr =  2\pi R^2 \int_0^1 x\left[ J_0\left(\mu_i x\right) \right]^2  \,dx$$
  和
  $$\int_0^R r\left[1-\left(\frac{r}{R}\right)^2\right]J_0\left(\frac{\mu_jr}{R}\right)\, dr = R^2\int_0^1 x(1-x^2)J_0(\mu_ix)dx.$$

  (我们作了替换$x=r/R$)

  \skiplines
  
  虽然这些积分会不会求并不影响大局,我们还是来仔细研究一下(前方高能请看戏就好)——
  
\end{frame}


\begin{frame}
  \frametitle{贝塞尔函数的正交定理}
  
  设$\mu_i$, $\mu_j$是$J_m(x)$第$i$个和第$j$个正实数根,则
  \tbox{
    $$\int_0^1 \,x\,J_m(\mu_ix)J_m(\mu_j x) \,dx = \delta_{ij} \frac{\left[J_{m+1}(\mu_i)\right]^2}{2} .$$}
  也就是说,只要定义内积时乘上权重$x$(这个权重来自于极坐标系的散度的”面积修正因子“),那么$J_m(\mu_ix)$ ($i=1,2,\ldots$)就是在$[0,1]$内的正交函数组了。
  
\end{frame}


\begin{frame}
  \frametitle{证明}
  
  回忆$J_m(\mu_ix)$满足的贝塞尔方程:

  \be
    \frac{1}{x}\frac{d}{dx}\left[x\frac{d}{dx} J_m(\mu_i x)\right] + \left(\mu_i^2-\frac{m^2}{x^2}\right)J_m(\mu_ix) = 0.
   \ee
  两边同乘以$xJ_m(\mu_j x)$,得到
   \begin{equation}
     J_m(\mu_jx)\frac{d}{dx}\left[x\frac{d}{dx} J_m(\mu_i x)\right] + \left(\mu_i^2-\frac{m^2}{x^2}\right)xJ_m(\mu_ix)J_m(\mu_jx) = 0. \label{eq1}
   \end{equation}
   交换$i$和$j$,得到
   \begin{equation}
     J_m(\mu_ix)\frac{d}{dx}\left[x\frac{d}{dx} J_m(\mu_j x)\right] + \left(\mu_j^2-\frac{m^2}{x^2}\right)xJ_m(\mu_ix)J_m(\mu_jx) = 0. \label{eq2}
   \end{equation}
  
\end{frame}


\begin{frame}
  \frametitle{证明(续)}
  
  \eqref{eq1}减去\eqref{eq2},得到
  \begin{eqnarray}
 && \frac{d}{dx}\left[ xJ_m(\mu_jx)\frac{d}{dx} J_m(\mu_i x) - xJ_m(\mu_ix)\frac{d}{dx} J_m(\mu_j x) \right] \newl
  && + (\mu_i^2-\mu_j^2)xJ_m(\mu_ix)J_m(\mu_jx) = 0. \label{eq3}
  \end{eqnarray}
  如果$i\ne j$,两边从$0$到$1$积分即得到
  $$\int_0^1 xJ_m(\mu_ix)J_m(\mu_jx) = 0. $$
  
\end{frame}


\begin{frame}
  \frametitle{证明(续)}
  
  事实上,在推导\eqref{eq3}时,$\mu_i$,$\mu_j$可以为任何正数,所以我们还可以取$\mu_i$为$J_m$的零点,而$\mu_j = \mu_i +\epsilon$。
    同样从$0$到$1$积分,得到
    $$ \mu_iJ_m(\mu_i+\epsilon) J_m'(\mu_i)  -(2\mu_i+\epsilon)\epsilon\int_0^1xJ_m(\mu_ix)J_m[(\mu_i+\epsilon)x]dx = 0 $$
    两边除以$\epsilon$并令 $\epsilon\rightarrow 0^+$
    $$ \int_0^1x\left[J_m(\mu_ix)\right]^2dx = \frac{[J_m'(\mu_i)]^2}{2} $$
    最后,在递推关系
    $$\frac{d}{dx}\left[x^{-m}J_m(x)\right] = -x^{-m} J_{m+1}(x)$$
  两边取$x = \mu_i$,即得到 $J_m'(\mu_i) = -J_{m+1}(\mu_i)$。于是命题在$i=j$时也得到了证明。
    
  
\end{frame}


\begin{frame}
  \frametitle{另一个积分}
  令$m=0$,上述定理给出了$N_j = \pi R^2\left[J_1(\mu_j)\right]^2$.

  然后考虑另一个积分
  $$I =   \int_0^1x\left(1-x^2\right)J_0(\mu_jx)dx. $$
\end{frame}

\begin{frame}
  利用
  $$ \mu_jxJ_0(\mu_j x) = \frac{d}{dx} \left[xJ_1(\mu_jx)\right]. $$
  分部积分,即得到
  $$I = \left.\frac{1-x^2}{\mu_j} xJ_1(\mu_jx) \right\vert_0^1 + \frac{2}{\mu_j} \int_0^1 x^2J_1(\mu_jx) dx. $$
  再利用
  $$\mu_jx^2J_1(\mu_j x) = \frac{d}{dx}\left[x^2J_2(\mu_j x)\right] $$
  得到
  $$I  =  \frac{2}{\mu_j} \int_0^1 x^2J_1(\mu_jx) dx = \frac{2}{\mu_j^2}J_2(\mu_j). $$
  
\end{frame}


\begin{frame}
  \frametitle{求出系数}
  
  最后,利用
  $$J_0(x)+J_2(x) = \frac{2}{x}J_1(x)$$
  得到
  $$J_2(\mu_j) = \frac{2}{\mu_j}J_1(\mu_j).$$
  综合前面所有结果,最终得到
  $$ c_j = \frac{8A}{\mu_j^3J_1(\mu_j)} $$
  
\end{frame}


\begin{frame}
  \frametitle{最终解}
  
  薄膜的振动解为
  $$ u = \sum_{j=0}^\infty \frac{8A}{\mu_j^3J_1(\mu_j)}J_0\left( \frac{\mu_jr}{R}\right)\cos\frac{\mu_jat}{R}. $$

  \skiplines

  (思考:会不会发生某个分母中的$J_1(\mu_j)=0$的情况?)
  
\end{frame}


\begin{frame}
  
  \bcenter
  \lfig{1}{blackq.jpg}

  刚才发生了什么.jpg
  \ecenter

\end{frame}


\section{Homework}

\begin{frame}
  \frametitle{Homework}
  
  \bitem
\item{通过逐项求导的办法证明贝塞尔函数的两个递推关系}
\item{如果$J_m(\mu) = 0$ ($\mu$为正实数,$m$为非负整数),计算积分
    $$\int_0^1x\left[J_{m+1}(\mu x)\right]^2 dx. $$
}
\item{ 考虑固定边界的,半径为$R$的圆形薄膜的横向小振动问题。在$t=0$时刻的初始速度为$A\left[1-\left(\frac{r}{R}\right)^2\right]$ (其中$A$为常量,$r$为距离圆盘中心的距离,$0\le r\le R$),初始位移为零。求解之后薄膜的振动。
 }
  \eitem
  
\end{frame}

\ech
\end{document}
