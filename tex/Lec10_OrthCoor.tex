\documentclass[CJK]{beamer}
\usepackage{CJKutf8}
\usepackage{beamerthemesplit}
\usetheme{Malmoe}
\useoutertheme[footline=authortitle]{miniframes}
\usepackage{amsmath}
\usepackage{amssymb}
\usepackage{graphicx}
\usepackage{eufrak}
\usepackage{color}
\usepackage{slashed}
\usepackage{simplewick}
\usepackage{tikz}
\usepackage{tcolorbox}
\graphicspath{{../figures/}}
%%figures
\def\lfig#1#2{\includegraphics[width=#1 in]{#2}}
\def\addfig#1#2{\begin{center}\includegraphics[width=#1 in]{#2}\end{center}}
\def\wulian{\includegraphics[width=0.18in]{emoji_wulian.jpg}}
\def\bigwulian{\includegraphics[width=0.35in]{emoji_wulian.jpg}}
\def\bye{\includegraphics[width=0.18in]{emoji_bye.jpg}}
\def\bigbye{\includegraphics[width=0.35in]{emoji_bye.jpg}}
\def\huaixiao{\includegraphics[width=0.18in]{emoji_huaixiao.jpg}}
\def\bighuaixiao{\includegraphics[width=0.35in]{emoji_huaixiao.jpg}}
\def\jianxiao{\includegraphics[width=0.18in]{emoji_jianxiao.jpg}}
\def\bigjianxiao{\includegraphics[width=0.35in]{emoji_jianxiao.jpg}}
%% colors
\def\blacktext#1{{\color{black}#1}}
\def\bluetext#1{{\color{blue}#1}}
\def\redtext#1{{\color{red}#1}}
\def\darkbluetext#1{{\color[rgb]{0,0.2,0.6}#1}}
\def\skybluetext#1{{\color[rgb]{0.2,0.7,1.}#1}}
\def\cyantext#1{{\color[rgb]{0.,0.5,0.5}#1}}
\def\greentext#1{{\color[rgb]{0,0.7,0.1}#1}}
\def\darkgray{\color[rgb]{0.2,0.2,0.2}}
\def\lightgray{\color[rgb]{0.6,0.6,0.6}}
\def\gray{\color[rgb]{0.4,0.4,0.4}}
\def\blue{\color{blue}}
\def\red{\color{red}}
\def\green{\color{green}}
\def\darkgreen{\color[rgb]{0,0.4,0.1}}
\def\darkblue{\color[rgb]{0,0.2,0.6}}
\def\skyblue{\color[rgb]{0.2,0.7,1.}}
%%control
\def\be{\begin{equation}}
\def\ee{\nonumber\end{equation}}
\def\bea{\begin{eqnarray}}
\def\eea{\nonumber\end{eqnarray}}
\def\bch{\begin{CJK}{UTF8}{gbsn}}
\def\ech{\end{CJK}}
\def\bitem{\begin{itemize}}
\def\eitem{\end{itemize}}
\def\bcenter{\begin{center}}
\def\ecenter{\end{center}}
\def\bex{\begin{minipage}{0.2\textwidth}\includegraphics[width=0.6in]{jugelizi.png}\end{minipage}\begin{minipage}{0.76\textwidth}}
\def\eex{\end{minipage}}
\def\chtitle#1{\frametitle{\bch#1\ech}}
\def\bmat#1{\left(\begin{array}{#1}}
\def\emat{\end{array}\right)}
\def\bcase#1{\left\{\begin{array}{#1}}
\def\ecase{\end{array}\right.}
\def\bmini#1{\begin{minipage}{#1\textwidth}}
\def\emini{\end{minipage}}
\def\tbox#1{\begin{tcolorbox}#1\end{tcolorbox}}
\def\pfrac#1#2#3{\left(\frac{\partial #1}{\partial #2}\right)_{#3}}
%%symbols
\def\bropt{\,(\ \ \ )}
\def\sone{$\star$}
\def\stwo{$\star\star$}
\def\sthree{$\star\star\star$}
\def\sfour{$\star\star\star\star$}
\def\sfive{$\star\star\star\star\star$}
\def\rint{{\int_\leftrightarrow}}
\def\roint{{\oint_\leftrightarrow}}
\def\stdHf{{\textit{\r H}_f}}
\def\deltaH{{\Delta \textit{\r H}}}
\def\ii{{\dot{\imath}}}
\def\skipline{{\vskip0.1in}}
\def\skiplines{{\vskip0.2in}}
\def\lagr{{\mathcal{L}}}
\def\hamil{{\mathcal{H}}}
\def\vecv{{\mathbf{v}}}
\def\vecx{{\mathbf{x}}}
\def\vecy{{\mathbf{y}}}
\def\veck{{\mathbf{k}}}
\def\vecp{{\mathbf{p}}}
\def\vecn{{\mathbf{n}}}
\def\vecA{{\mathbf{A}}}
\def\vecP{{\mathbf{P}}}
\def\vecsigma{{\mathbf{\sigma}}}
\def\hatJn{{\hat{J_\vecn}}}
\def\hatJx{{\hat{J_x}}}
\def\hatJy{{\hat{J_y}}}
\def\hatJz{{\hat{J_z}}}
\def\hatj#1{\hat{J_{#1}}}
\def\hatphi{{\hat{\phi}}}
\def\hatq{{\hat{q}}}
\def\hatpi{{\hat{\pi}}}
\def\vel{\upsilon}
\def\Dint{{\mathcal{D}}}
\def\adag{{\hat{a}^\dagger}}
\def\bdag{{\hat{b}^\dagger}}
\def\cdag{{\hat{c}^\dagger}}
\def\ddag{{\hat{d}^\dagger}}
\def\hata{{\hat{a}}}
\def\hatb{{\hat{b}}}
\def\hatc{{\hat{c}}}
\def\hatd{{\hat{d}}}
\def\hatN{{\hat{N}}}
\def\hatH{{\hat{H}}}
\def\hatp{{\hat{p}}}
\def\Fup{{F^{\mu\nu}}}
\def\Fdown{{F_{\mu\nu}}}
\def\newl{\nonumber \\}
\def\vece{\mathrm{e}}
\def\calM{{\mathcal{M}}}
\def\calT{{\mathcal{T}}}
\def\calR{{\mathcal{R}}}
\def\barpsi{\bar{\psi}}
\def\baru{\bar{u}}
\def\barv{\bar{\upsilon}}
\def\qeq{\stackrel{?}{=}}
\def\torder#1{\mathcal{T}\left(#1\right)}
\def\rorder#1{\mathcal{R}\left(#1\right)}
\def\contr#1#2{\contraction{}{#1}{}{#2}#1#2}
\def\trof#1{\mathrm{Tr}\left(#1\right)}
\def\trace{\mathrm{Tr}}
\def\comm#1{\ \ \ \left(\mathrm{used}\ #1\right)}
\def\tcomm#1{\ \ \ (\text{#1})}
\def\slp{\slashed{p}}
\def\slk{\slashed{k}}
\def\calp{{\mathfrak{p}}}
\def\veccalp{\mathbf{\mathfrak{p}}}
\def\Tthree{T_{\tiny \textcircled{3}}}
\def\pthree{p_{\tiny \textcircled{3}}}
\def\dbar{{\,\mathchar'26\mkern-12mu d}}
\def\erf{\mathrm{erf}}
\def\const{\mathrm{constant}}
\def\pheat{\pfrac p{\ln T}V}
\def\vheat{\pfrac V{\ln T}p}
%%units
\def\fdeg{{^\circ \mathrm{F}}}
\def\cdeg{^\circ \mathrm{C}}
\def\atm{\,\mathrm{atm}}
\def\angstrom{\,\text{\AA}}
\def\SIL{\,\mathrm{L}}
\def\SIkm{\,\mathrm{km}}
\def\SIyr{\,\mathrm{yr}}
\def\SIGyr{\,\mathrm{Gyr}}
\def\SIV{\,\mathrm{V}}
\def\SImV{\,\mathrm{mV}}
\def\SIeV{\,\mathrm{eV}}
\def\SIkeV{\,\mathrm{keV}}
\def\SIMeV{\,\mathrm{MeV}}
\def\SIGeV{\,\mathrm{GeV}}
\def\SIcal{\,\mathrm{cal}}
\def\SIkcal{\,\mathrm{kcal}}
\def\SImol{\,\mathrm{mol}}
\def\SIN{\,\mathrm{N}}
\def\SIHz{\,\mathrm{Hz}}
\def\SIm{\,\mathrm{m}}
\def\SIcm{\,\mathrm{cm}}
\def\SIfm{\,\mathrm{fm}}
\def\SImm{\,\mathrm{mm}}
\def\SInm{\,\mathrm{nm}}
\def\SImum{\,\mathrm{\mu m}}
\def\SIJ{\,\mathrm{J}}
\def\SIW{\,\mathrm{W}}
\def\SIkJ{\,\mathrm{kJ}}
\def\SIs{\,\mathrm{s}}
\def\SIkg{\,\mathrm{kg}}
\def\SIg{\,\mathrm{g}}
\def\SIK{\,\mathrm{K}}
\def\SImmHg{\,\mathrm{mmHg}}
\def\SIPa{\,\mathrm{Pa}}

\def\courseurl{https://github.com/zqhuang/SYSU\_TD}

\def\tpage#1#2{
\begin{frame}
\begin{center}
\begin{Large}
\bch
热学 \\
第#1讲 #2

{\vskip 0.3in}

黄志琦

\ech
\end{Large}
\end{center}

\vskip 0.2in

\bch
教材:《热学》第二版,赵凯华,罗蔚茵,高等教育出版社
\ech

\bch
课件下载
\ech
\courseurl
\end{frame}
}

\def\bfr#1{
\begin{frame}
\chtitle{#1} 
\bch
}

\def\efr{
\ech 
\end{frame}
}

  \date{}
  \begin{document}
  \bch
  
\tpage{10}{Orthogonal Coordinate System}

\begin{frame}
\frametitle{本讲内容}

\bitem
\item{正交曲面坐标系}
\eitem

\end{frame}


\section{Orthogonal Coordinate System}
\secpage{正交曲面坐标系}{先作平坦近似,再修正}



\begin{frame}
\frametitle{正交曲面坐标系}

如果任意点附近的坐标轴方向总是两两垂直,则称该坐标系为{\blue \bf 正交曲面坐标系}。

\skiplines
(所谓附近的坐标轴方向,是指仅变化一个坐标分量所得的曲线在这个点的切线方向。)

\skiplines
\bex
二维的直角坐标系 $x, y$;

二维的极坐标系 $r, \theta$;

三维的直角坐标系的 $x, y, z$;

三维的柱坐标系 $r, \theta, z$;

三维的球坐标系$ r, \theta, \varphi$。
\eex

\end{frame}


\begin{frame}
\frametitle{极坐标是正交坐标系的图示}

\addfig{2.2}{polarcoor.png}

变化$r$和变化$\theta$对应的小长度元分别为$dr$和$rd\theta$,且方向垂直。

\end{frame}


\begin{frame}
\frametitle{极坐标弧长公式}

\bmini{0.5}
根据勾股定理,显然有
$$ds^2 = dr^2 + (rd\theta)^2  = dr^2 + r^2 d\theta^2$$
\emini
\bmini{0.5}
\lfig{2.2}{lineelement.png}
\emini

如果曲线以$r(\theta)$的函数形式表示,则得到弧长公式
$$ ds = \sqrt{dr^2+r^2d\theta^2} = \sqrt{r'^2+r^2 }\,d\theta$$


\end{frame}


\begin{frame}
\frametitle{思考题}


\addfig{2.5}{cylindricalcoor.png}

写出柱坐标系$(r,\theta, z)$的正交长度元,由此推导柱坐标系弧长公式。


\end{frame}


\begin{frame}
\frametitle{思考题}


\addfig{3}{sphericalcoor.png}

写出球坐标系$(r,\theta, \phi)$的正交长度元,由此推导球坐标系弧长公式。


\end{frame}


\begin{frame}
  \frametitle{梯度 = 单位长度内标量函数的变化}
  
  以球面坐标系为例:正交长度元分别为$dr$, $rd\theta$, $r\sin\theta d\phi$。

  \skipline

  设$f$为某个标量函数(如温度,电势等不带方向的量)

  \bitem

\item{沿着$dr$方向(保持$\theta$, $\phi$不变,变化$r$),$f$的梯度分量为
    $$\lim_{\delta r\rightarrow 0}\frac{\delta f}{\delta r} = \frac{\partial f}{\partial r}.$$}

\item{沿着$rd\theta$方向(保持$r$, $\phi$不变,变化$\theta$),$f$的梯度分量为
  $$\lim_{\delta \theta \rightarrow 0}\frac{\delta f}{r\delta \theta} = \frac{1}{r}\frac{\partial f}{\partial \theta}$$}

\item{沿着$r\sin\theta d\phi$方向(保持$r$, $\theta$不变,变化$\phi$),$f$的梯度分量为
  $$\lim_{\delta \phi \rightarrow 0}\frac{\delta f}{r\sin\theta \delta \phi} = \frac{1}{r\sin\theta}\frac{\partial f}{\partial \phi}$$}
  
    \eitem

  
\end{frame}



\begin{frame}
  \frametitle{梯度 = 单位长度内标量函数的变化}
  

  因此,球坐标系下的梯度为

  $$\nabla f = \left(\frac{\partial f}{\partial r}, \frac{1}{r}\frac{\partial f}{\partial \theta}, \frac{1}{r\sin\theta}\frac{\partial f}{\partial \phi} \right).$$

  \skiplines
  
  (注意这个只是沿着局域的三个正交长度元方向进行分解,不同于线性代数里的用三个固定的基进行分解)
  
\end{frame}



\begin{frame}
\frametitle{思考题}


\addfig{1}{think2.jpg}

写出极坐标系和柱坐标系下的梯度。


\end{frame}



\begin{frame}
  \frametitle{流$j$的散度是单位体积的$j$净流出率}
  

仍以球坐标系为例:

  设$j$沿着长度元$dr$,$rd\theta$, $r\sin\theta d\phi$的分量为:
  $(j_r, j_\theta, j_\phi).$

  那么$j_r(rd\theta)(r\sin\theta d\phi)$代表了$j_r$流过垂直于$dr$方向的面积元的量。它沿$dr$方向的变化率代表了$dr$方向的流出流入不平衡,对净流出率的贡献为:
  $$\frac{\partial }{\partial r} \left[(rd\theta)(r\sin\theta d\phi)j_r\right]dr$$
  除以体积元$dr (rd\theta)(r\sin\theta d\phi)$,得到贡献:
  $$\frac{1}{r^2}\frac{\partial }{\partial r} \left(r^2j_r\right)$$  
  
\end{frame}


\begin{frame}
  \frametitle{流$j$的散度是单位体积的$j$净流出率}
  

  同理,沿$rd \theta$方向的贡献为
 $$\frac{1}{(dr) (rd\theta) (r\sin\theta d\phi)}\frac{\partial }{\partial \theta} \left[(dr)(r\sin\theta d\phi)j_\theta\right] d\theta  = \frac{1}{r\sin\theta}\frac{\partial  }{\partial \theta} \left( \sin\theta j_\theta\right) $$

  沿$r\sin\theta d\phi$方向的贡献为
  $$\frac{1}{(dr) (rd\theta) (r\sin\theta d\phi)}\frac{\partial }{\partial \phi} \left[(dr)(rd\theta)j_\phi\right] d\phi = \frac{1}{r\sin\theta}\frac{\partial  }{\partial \phi} j_\phi $$
  
  
\end{frame}




\begin{frame}
  \frametitle{总结快速写出散度的办法}
  
  \bitem
\item[1]{以三个长度元写出直角坐标系形式的“平坦近似散度”:
  $$\nabla\cdot \mathbf{j} \approx \frac{\partial}{\partial r} j_r + \frac{1}{r} \frac{\partial}{\partial \theta} j_\theta + \frac{1}{r\sin\theta}\frac{\partial}{\partial \phi} j_\phi $$
}
\item[2]{每一项都在微分号内乘以面积修正因子,在外面除掉。例如垂直$dr$方向的面积元为$rd\theta r\sin\theta d\phi$,含$r$的因子(即$r^2$,剩余部分$d\theta\sin\theta d\phi$因为可以直接提到偏微分号外面被除掉,所以不用考虑)是面积修正因子。于是 $\frac{\partial}{\partial r} j_r$被修正为$\frac{1}{r^2}\frac{\partial}{\partial r} (r^2j_r)$。
}
  \eitem
  
\end{frame}


\begin{frame}
  \frametitle{演算草稿}
  
  以球坐标系为例

  \skiplines

  \bmini{0.3}
  正交长度元:
  
  垂直面积元:
  
  面积修正因子:
  \emini
  \bmini{0.22}
  $dr$
  
  $rd\theta\, r\sin\theta d\phi$
  
  $r^2$
  \emini
  \bmini{0.22}
  $rd\theta$
  
  $dr\, r\sin\theta d\phi$
  
  $\sin\theta$
  \emini
  \bmini{0.22}
  $r\sin\theta d\phi$
  
  $dr\, rd\theta$
  
  $1$
  \emini

  \skipline
  
  平坦近似散度:
  $$\nabla\cdot \mathbf{j} \approx \frac{\partial}{\partial r} j_r + \frac{1}{r} \frac{\partial}{\partial \theta} j_\theta + \frac{1}{r\sin\theta}\frac{\partial}{\partial \phi} j_\phi $$

  修正后:
  $$\nabla\cdot \mathbf{j} =\frac{1}{r^2} \frac{\partial}{\partial r} (r^2j_r) + \frac{1}{r\sin\theta} \frac{\partial}{\partial \theta} (\sin\theta j_\theta) + \frac{1}{r\sin\theta}\frac{\partial}{\partial \phi} j_\phi $$
  
  
\end{frame}

\begin{frame}
  \frametitle{顺便提下:旋度算符的快速写法}
  
  对旋度算符,只要把面积修正换为长度元修正。仍以球坐标系为例

  \skiplines

  \bmini{0.3}
  正交长度元:
  \emini
  \bmini{0.22}
  $dr$
  \emini
  \bmini{0.22}
  $rd\theta$
  \emini
  \bmini{0.22}
  $r\sin\theta d\phi$
  \emini

  
  \skipline
  
  平坦近似旋度:
  $$\nabla\times \mathbf{j} \approx \left( \frac{1}{r}\frac{\partial}{\partial \theta} j_\phi -  \frac{1}{r\sin\theta}\frac{\partial}{\partial \phi} j_\theta ,  \frac{1}{r\sin\theta}\frac{\partial}{\partial \phi} j_r -  \frac{\partial}{\partial r} j_\phi, \frac{\partial}{\partial r} j_\theta-  \frac{1}{r}\frac{\partial}{\partial \theta} j_r\right). $$

  修正后右边为{\small 
  $$\left( \frac{1}{r\sin\theta}\frac{\partial}{\partial \theta} (\sin\theta j_\phi) -  \frac{1}{r\sin\theta}\frac{\partial}{\partial \phi} j_\theta ,  \frac{1}{r\sin\theta}\frac{\partial}{\partial \phi} j_r - \frac{1}{r} \frac{\partial}{\partial r} (rj_\phi), \frac{1}{r}\frac{\partial}{\partial r} (rj_\theta)-  \frac{1}{r}\frac{\partial}{\partial \theta} j_r\right). $$}
  
  
\end{frame}



\begin{frame}
  \frametitle{思考题}
  
  \addfig{1}{think3.jpg}
  
  写出柱坐标系下的散度和旋度
  
\end{frame}

\begin{frame}
  \frametitle{总结}
  
  先平坦近似, 然后
  \bitem
\item{梯度不需要修正}
\item{散度需要面积元修正}
\item{旋度需要长度元修正}
  \eitem
  
\end{frame}


\begin{frame}
  \frametitle{柱坐标系的拉普拉斯算符}
  
  以柱坐标系$r,\theta, z$为例,长度元为:$dr$, $rd\theta$, $dz$。

  对标量函数$f$,梯度
  $$\nabla f = \left(\frac{\partial f}{\partial r}, \frac{1}{r}\frac{\partial f}{\partial \theta} , \frac{\partial f}{\partial z}\right)$$

  再对梯度求散度,先作平坦近似:
  $$\nabla^2 f \approx \frac{\partial}{\partial r} \frac{\partial f}{\partial r} + \frac{1}{r}\frac{\partial }{\partial \theta} \left(\frac{1}{r}\frac{\partial f}{\partial \theta}\right) + \frac{\partial }{\partial z}\left(\frac{\partial f}{\partial z}\right)$$
  然后面积元修正:
  \bea
  \nabla^2 f &=&\frac{1}{r} \frac{\partial}{\partial r} \left(r\frac{\partial f}{\partial r}\right) + \frac{1}{r}\frac{\partial }{\partial \theta} \left(\frac{1}{r}\frac{\partial f}{\partial \theta}\right) + \frac{\partial }{\partial z}\left(\frac{\partial f}{\partial z}\right) \newl
  &=& \frac{1}{r} \frac{\partial}{\partial r} \left(r\frac{\partial f}{\partial r}\right) + \frac{1}{r^2}\frac{\partial^2 f}{\partial \theta^2}  + \frac{\partial^2f }{\partial z^2}.
  \eea
  
\end{frame}

\begin{frame}
  \frametitle{极坐标的拉普拉斯算符}
  
  极坐标的拉普拉斯算符,只要去掉$z$:
$$ \nabla^2 f =  \frac{1}{r} \frac{\partial}{\partial r} \left(r\frac{\partial f}{\partial r}\right) + \frac{1}{r^2}\frac{\partial^2 f}{\partial \theta^2}.  $$
  
\end{frame}


\section{Homework}

\begin{frame}
  \frametitle{Homework for quizphobia}
  
  \bitem
\item[28]{椭圆坐标$(\mu,\nu)$表示平面直角坐标系里的点$(\cosh \mu \cos\nu, \sinh\mu\sin\nu)$。说明椭圆坐标系是一种正交曲面坐标系,然后写出椭圆坐标系的拉普拉斯算符的表达式。}
\item[29]{设四维的正交曲面坐标系$(t,x,y,z)$的正交线元长度分别为$dt$, $a(t)dx$, $a(t)dy$, $a(t)dz$,其中$a(t)$为某个已知的函数。写出该坐标系的拉普拉斯算符的表达式。}
\item[30]{先写出三维球坐标的拉普拉斯算符。然后考虑用参数$(r, \theta, \phi, \alpha)$来表示的四维球坐标系;参数跟直角坐标$(x_1, x_2, x_3,x_4)$的关系为 $$x_1 = r\sin\theta\sin\phi\cos\alpha, x_2=r\sin\theta\sin\phi\sin\alpha,$$,
  $$x_3=r\sin\theta\cos\phi, x_4 = r\cos\theta.$$  写出四维球坐标系的拉普拉斯算符的表达式。}
  \eitem

  
\end{frame}

\ech
\end{document}
