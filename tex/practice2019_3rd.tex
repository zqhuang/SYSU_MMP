\item[(1)]{(为了检验你期中以前内容是否都量子波动速读了)请计算沿着逆时针方向的围道的积分\mark{10}:
    $$\int_{|z|=1}\, \frac{1}{\sin z - z\cos z} dz.$$ }
\item[(2)]{假设你MMP考试挂科,未能毕业,流落江湖摆摊卖糖葫芦为生。六一儿童节到了,糖葫芦摊抽奖送礼大酬宾。你在一个大袋子里装了很多红、黄、蓝色的小球,每次摸出每种颜色小球的概率都是$1/3$。你请光临摊位的小孩随机摸出 $12$ 个小球,不计颜色的次序,摸出的三种颜色小球的个数有$18$种可能性: $(0,0,12)$, $(0,1,11)$, $(0,2,10)$, $(0,3,9)$, $(0,4,8)$, $(0, 5, 7)$, $(0, 6, 6)$, $(1, 1, 10)$, $(1,2,9)$, $(1,3,8)$, $(1,4,7)$, $(1, 5, 6)$, $(2, 2, 8)$, $(2, 3, 7)$, $(2, 4, 6)$, $(2, 5, 5)$, $(3, 4, 5)$, $(4,4,4)$。只要小孩摸出的不是 $(3,4,5)$,就白送一根糖葫芦;否则需要按日常价 $10$ 元购买一根糖葫芦。已知平均一根糖葫芦的成本价是 $3$元。这样的促销活动亏本吗?为什么? \mark{10}}  
\item[(3)]{传古代印度与中国之间有一萨桑国,国王山鲁亚尔生性残暴嫉妒,因王后行为不端,将其杀死,此后每日娶一少女,翌日晨即杀掉,以示报复。宰相的女儿山鲁佐德为拯救无辜的女子,自愿嫁给国王,用让国王刷quiz的方法吸引国王,每夜刷到最精彩处,天刚好亮了,使国王爱不忍杀。quiz一直刷了一千零一夜,国王终于被感动,与她白首偕老。 \\
山鲁佐德给国王传授了自然对数的算法。国王深深迷醉于这个超越时代近千年的技术。每天刷quiz前,国王都要练习自然对数的计算。\\
为了让国王每天有新的挑战,山鲁佐德使用了满足下列递推式的数列:
$$a_n = \frac{4}{3+a_{n-1}^2}.$$
数列的起始点设为$a_0=0$,也就是第一天给国王的数字是$a_1=\frac{4}{3}$,之后依次递推。\\
国王在第$n$天($n=1,2,...,1001$)刷quiz前的热身运动就是计算$a_n$的对数的绝对值的对数,也就是$\ln|\ln a_n|$。在当时可没有计算机,全靠手算,山鲁佐德只要求国王算到一位有效数就可以了。\\
那么问题来了:在最后第一千零一夜,国王算出来的$\ln|\ln a_{1001}|$等于多少? \mark{20} }
\item[(4)]{从前有一个神秘的星球叫浮士德星,这个星球的每年有一万天。星球上的人出生时都会和魔鬼签署了一份协议:每个人每天起床都要投掷 $600$ 次骰子。如果 $1,2,3,4,5,6$ 每个数字均恰好出现 $100$ 次,魔鬼就要带走这个人的灵魂,否则魔鬼会保护这个人一天平安无事。\\
    有一天赌王来到了浮士德星,传授给了浮士德星人一个秘术——666大法。使用秘术之后,每次掷骰子出现 $6$ 的概率会稍稍增加,但增加的幅度非常小以致于魔鬼无法察觉。从此浮士德村人的平均寿命延长了5年。请由此推断: 秘术能使单次掷骰子出现 $6$ 的概率增加多少? (结果请保留至少2位有效数) \mark{25}}
\item[(5)]{一个质量为$m$,固有圆频率为 $\omega$ 的谐振子,其三个空间坐标 $x,y,z$ 和三个动量分量 $p_x, p_y, p_z$ 构成了六维的 ``相空间''。它在任意时刻的状态(即位置和运动情况)可以用相空间的一个点来表示。从经典物理的角度看,谐振子的状态允许是相空间中任意一个点。但如果要求谐振子的能量不超过某个阈值 $E_0$ (常量$E_0>0$),即
    $$\frac{1}{2}m\omega^2 \left(x^2+y^2+z^2\right) + \frac{1}{2m}\left(p_x^2+p_y^2+p_z^2\right) \le E_0,$$      
    谐振子就被限制在相空间一个有限大小的区域内。这个区域的(六维)体积是多少?\mark{15}}
