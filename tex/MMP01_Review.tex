\documentclass[CJK,13pt]{beamer}
\usepackage{CJKutf8}
\usepackage{beamerthemesplit}
\usetheme{Malmoe}
\useoutertheme[footline=authortitle]{miniframes}
\usepackage{amsmath}
\usepackage{amssymb}
\usepackage{graphicx}
\usepackage{eufrak}
\usepackage{color}
\usepackage{slashed}
\usepackage{simplewick}
\usepackage{tikz}
\usepackage{tcolorbox}
\graphicspath{{../figures/}}
%%figures
\def\lfig#1#2{\includegraphics[width=#1 in]{#2}}
\def\addfig#1#2{\begin{center}\includegraphics[width=#1 in]{#2}\end{center}}
\def\wulian{\includegraphics[width=0.18in]{emoji_wulian.jpg}}
\def\bigwulian{\includegraphics[width=0.35in]{emoji_wulian.jpg}}
\def\bye{\includegraphics[width=0.18in]{emoji_bye.jpg}}
\def\bigbye{\includegraphics[width=0.35in]{emoji_bye.jpg}}
\def\huaixiao{\includegraphics[width=0.18in]{emoji_huaixiao.jpg}}
\def\bighuaixiao{\includegraphics[width=0.35in]{emoji_huaixiao.jpg}}
\def\jianxiao{\includegraphics[width=0.18in]{emoji_jianxiao.jpg}}
\def\bigjianxiao{\includegraphics[width=0.35in]{emoji_jianxiao.jpg}}
%% colors
\def\blacktext#1{{\color{black}#1}}
\def\bluetext#1{{\color{blue}#1}}
\def\redtext#1{{\color{red}#1}}
\def\darkbluetext#1{{\color[rgb]{0,0.2,0.6}#1}}
\def\skybluetext#1{{\color[rgb]{0.2,0.7,1.}#1}}
\def\cyantext#1{{\color[rgb]{0.,0.5,0.5}#1}}
\def\greentext#1{{\color[rgb]{0,0.7,0.1}#1}}
\def\darkgray{\color[rgb]{0.2,0.2,0.2}}
\def\lightgray{\color[rgb]{0.6,0.6,0.6}}
\def\gray{\color[rgb]{0.4,0.4,0.4}}
\def\blue{\color{blue}}
\def\red{\color{red}}
\def\green{\color{green}}
\def\darkgreen{\color[rgb]{0,0.4,0.1}}
\def\darkblue{\color[rgb]{0,0.2,0.6}}
\def\skyblue{\color[rgb]{0.2,0.7,1.}}
%%control
\def\be{\begin{equation}}
\def\ee{\nonumber\end{equation}}
\def\bea{\begin{eqnarray}}
\def\eea{\nonumber\end{eqnarray}}
\def\bch{\begin{CJK}{UTF8}{gbsn}}
\def\ech{\end{CJK}}
\def\bitem{\begin{itemize}}
\def\eitem{\end{itemize}}
\def\bcenter{\begin{center}}
\def\ecenter{\end{center}}
\def\bex{\begin{minipage}{0.2\textwidth}\includegraphics[width=0.6in]{jugelizi.png}\end{minipage}\begin{minipage}{0.76\textwidth}}
\def\eex{\end{minipage}}
\def\chtitle#1{\frametitle{\bch#1\ech}}
\def\bmat#1{\left(\begin{array}{#1}}
\def\emat{\end{array}\right)}
\def\bcase#1{\left\{\begin{array}{#1}}
\def\ecase{\end{array}\right.}
\def\bmini#1{\begin{minipage}{#1\textwidth}}
\def\emini{\end{minipage}}
\def\tbox#1{\begin{tcolorbox}#1\end{tcolorbox}}
\def\pfrac#1#2#3{\left(\frac{\partial #1}{\partial #2}\right)_{#3}}
%%symbols
\def\bropt{\,(\ \ \ )}
\def\sone{$\star$}
\def\stwo{$\star\star$}
\def\sthree{$\star\star\star$}
\def\sfour{$\star\star\star\star$}
\def\sfive{$\star\star\star\star\star$}
\def\rint{{\int_\leftrightarrow}}
\def\roint{{\oint_\leftrightarrow}}
\def\stdHf{{\textit{\r H}_f}}
\def\deltaH{{\Delta \textit{\r H}}}
\def\ii{{\dot{\imath}}}
\def\skipline{{\vskip0.1in}}
\def\skiplines{{\vskip0.2in}}
\def\lagr{{\mathcal{L}}}
\def\hamil{{\mathcal{H}}}
\def\vecv{{\mathbf{v}}}
\def\vecx{{\mathbf{x}}}
\def\vecy{{\mathbf{y}}}
\def\veck{{\mathbf{k}}}
\def\vecp{{\mathbf{p}}}
\def\vecn{{\mathbf{n}}}
\def\vecA{{\mathbf{A}}}
\def\vecP{{\mathbf{P}}}
\def\vecsigma{{\mathbf{\sigma}}}
\def\hatJn{{\hat{J_\vecn}}}
\def\hatJx{{\hat{J_x}}}
\def\hatJy{{\hat{J_y}}}
\def\hatJz{{\hat{J_z}}}
\def\hatj#1{\hat{J_{#1}}}
\def\hatphi{{\hat{\phi}}}
\def\hatq{{\hat{q}}}
\def\hatpi{{\hat{\pi}}}
\def\vel{\upsilon}
\def\Dint{{\mathcal{D}}}
\def\adag{{\hat{a}^\dagger}}
\def\bdag{{\hat{b}^\dagger}}
\def\cdag{{\hat{c}^\dagger}}
\def\ddag{{\hat{d}^\dagger}}
\def\hata{{\hat{a}}}
\def\hatb{{\hat{b}}}
\def\hatc{{\hat{c}}}
\def\hatd{{\hat{d}}}
\def\hatN{{\hat{N}}}
\def\hatH{{\hat{H}}}
\def\hatp{{\hat{p}}}
\def\Fup{{F^{\mu\nu}}}
\def\Fdown{{F_{\mu\nu}}}
\def\newl{\nonumber \\}
\def\vece{\mathrm{e}}
\def\calM{{\mathcal{M}}}
\def\calT{{\mathcal{T}}}
\def\calR{{\mathcal{R}}}
\def\barpsi{\bar{\psi}}
\def\baru{\bar{u}}
\def\barv{\bar{\upsilon}}
\def\qeq{\stackrel{?}{=}}
\def\torder#1{\mathcal{T}\left(#1\right)}
\def\rorder#1{\mathcal{R}\left(#1\right)}
\def\contr#1#2{\contraction{}{#1}{}{#2}#1#2}
\def\trof#1{\mathrm{Tr}\left(#1\right)}
\def\trace{\mathrm{Tr}}
\def\comm#1{\ \ \ \left(\mathrm{used}\ #1\right)}
\def\tcomm#1{\ \ \ (\text{#1})}
\def\slp{\slashed{p}}
\def\slk{\slashed{k}}
\def\calp{{\mathfrak{p}}}
\def\veccalp{\mathbf{\mathfrak{p}}}
\def\Tthree{T_{\tiny \textcircled{3}}}
\def\pthree{p_{\tiny \textcircled{3}}}
\def\dbar{{\,\mathchar'26\mkern-12mu d}}
\def\erf{\mathrm{erf}}
\def\const{\mathrm{constant}}
\def\pheat{\pfrac p{\ln T}V}
\def\vheat{\pfrac V{\ln T}p}
%%units
\def\fdeg{{^\circ \mathrm{F}}}
\def\cdeg{^\circ \mathrm{C}}
\def\atm{\,\mathrm{atm}}
\def\angstrom{\,\text{\AA}}
\def\SIL{\,\mathrm{L}}
\def\SIkm{\,\mathrm{km}}
\def\SIyr{\,\mathrm{yr}}
\def\SIGyr{\,\mathrm{Gyr}}
\def\SIV{\,\mathrm{V}}
\def\SImV{\,\mathrm{mV}}
\def\SIeV{\,\mathrm{eV}}
\def\SIkeV{\,\mathrm{keV}}
\def\SIMeV{\,\mathrm{MeV}}
\def\SIGeV{\,\mathrm{GeV}}
\def\SIcal{\,\mathrm{cal}}
\def\SIkcal{\,\mathrm{kcal}}
\def\SImol{\,\mathrm{mol}}
\def\SIN{\,\mathrm{N}}
\def\SIHz{\,\mathrm{Hz}}
\def\SIm{\,\mathrm{m}}
\def\SIcm{\,\mathrm{cm}}
\def\SIfm{\,\mathrm{fm}}
\def\SImm{\,\mathrm{mm}}
\def\SInm{\,\mathrm{nm}}
\def\SImum{\,\mathrm{\mu m}}
\def\SIJ{\,\mathrm{J}}
\def\SIW{\,\mathrm{W}}
\def\SIkJ{\,\mathrm{kJ}}
\def\SIs{\,\mathrm{s}}
\def\SIkg{\,\mathrm{kg}}
\def\SIg{\,\mathrm{g}}
\def\SIK{\,\mathrm{K}}
\def\SImmHg{\,\mathrm{mmHg}}
\def\SIPa{\,\mathrm{Pa}}

\def\courseurl{https://github.com/zqhuang/SYSU\_TD}

\def\tpage#1#2{
\begin{frame}
\begin{center}
\begin{Large}
\bch
热学 \\
第#1讲 #2

{\vskip 0.3in}

黄志琦

\ech
\end{Large}
\end{center}

\vskip 0.2in

\bch
教材:《热学》第二版,赵凯华,罗蔚茵,高等教育出版社
\ech

\bch
课件下载
\ech
\courseurl
\end{frame}
}

\def\bfr#1{
\begin{frame}
\chtitle{#1} 
\bch
}

\def\efr{
\ech 
\end{frame}
}

  \date{}
  \begin{document}
  \bch
\tpage{1}{Review}

\begin{frame}
\frametitle{Outline}
\tableofcontents
\end{frame}

\section{Calculus}

\secpage{微积分回顾}{$$(1+x)^\alpha = 1+\alpha x+\frac{\alpha(\alpha-1)}{2!}x^2 +\frac{\alpha(\alpha-1)(\alpha-2)}{3!}x^3 + \ldots $$
}

\begin{frame}
  \frametitle{五个展开公式(前三个)}
  \tbox{$$e^x = 1+x+\frac{x^2}{2!} + \frac{x^3}{3!} + \ldots$$}
  \tbox{$$\sin x = x-\frac{x^3}{3!} + \frac{x^5}{5!} - \frac{x^7}{7!} + \ldots$$}
  \tbox{$$\cos x = 1-\frac{x^2}{2!} + \frac{x^4}{4!} - \frac{x^6}{6!} + \ldots$$}
  这三个公式无条件成立
\end{frame}


\begin{frame}
  \frametitle{五个展开公式(后两个)}
  \tbox{$$(1+x)^\alpha = 1+\alpha x+\frac{\alpha(\alpha-1)}{2!}x^2 +\frac{\alpha(\alpha-1)(\alpha-2)}{3!}x^3 + \ldots $$}
  \tbox{$$\ln (1+x) =  x-\frac{x^2}{2} +\frac{x^3}{3} -\frac{x^4}{4} + \ldots $$}
  这两个公式{\blue 适用于$|x|< 1$}; 你也可以在$|x|=1$时浪一下,一般不会出问题。但是,{\blue 在$|x|>1$的情况下一定不能浪,在$|x|>1$的情况下一定不能浪,在$|x|>1$的情况下一定不能浪}——重要的事情说三遍。
\end{frame}



\thinka{$e^\pi$和$\pi^e$哪个大?(2019年大学生夏令营面试题)}


\begin{frame}
  \frametitle{含参量的积分:固定上下界}
  设$f$是已知的二元函数,$a,b$是常数,可以用含参量的积分来定义函数
  $$F(x) =\int_a^b f(x, t) dt.$$
  当{\bf 积分收敛性和光滑性很好时,可以交换积分和求导的次序}
  $$F'(x) =\int_a^b \frac{\partial f(x, t)}{\partial x} dt.$$  
\end{frame}


\begin{frame}
  \frametitle{含参量的积分:变上下界}
  设$\alpha,\beta$是已知的一元函数,$f$是已知的二元函数,定义函数
  $$F(x) =\int_{\alpha(x)}^{\beta(x)} f(x, t) dt.$$
  当积分收敛性和光滑性很好时,
  \tbox{$$F'(x) =\int_{\alpha(x)}^{\beta(x)} \frac{\partial f(x, t)}{\partial x} dt + \beta'(x) f\left(x,\beta(x)\right) - \alpha'(x)f\left(x,\alpha(x)\right) .$$ }
  {\scriptsize (上面的结果有非常简单的几何解释,你能给出来吗?)}
  
\end{frame}

\thinka{在$x\in(0,\infty)$上定义$\Gamma$函数$$\Gamma(x) = \int_0^\infty t^{x-1}e^{-t}dt.$$
  试证明$-1<\Gamma'(1)<0.$}

\section{Linear Algebra}

\secpage{线性代数回顾}{厄米矩阵本征值都是实数;本征矢可以取成正交归一化。}

\begin{frame}
  \frametitle{共轭与共轭转置(conjugate transpose)}

  复数取共轭是指令其虚部反号:
  
  \tbox{$$(a+b\ii)^* \equiv a - b\ii,  \  a, b\in \Re $$}
  
  把一个复数矩阵$A$每个元素都取共轭,然后再把整个矩阵取转置,就得到$A$的{\bf 共轭转置矩阵},简单记作$A^\dagger$。当然,你也可以先转置,再取共轭。

  \tbox{$$A^\dagger\equiv (A^*)^T= (A^T)^*$$}

  显然,对实数矩阵而言,共轭转置就是转置。

\end{frame}


\begin{frame}
    \frametitle{例子}

  \be
    A = \left(
    \begin{array}{lll}
      1+\ii &  2+\ii & \ii \\
      3-\ii &  -4-2\ii & 0 
    \end{array}
    \right)
    \ee
取共轭转置后为

  \be
    A^\dagger = \left(
    \begin{array}{ll}
      1-\ii & 3+\ii \\
      2-\ii & -4+2\ii \\
      -\ii &   0 
    \end{array}
    \right)
    \ee

\end{frame}


\begin{frame}
    \frametitle{共轭转置的基本性质}
    容易证明:
    \bitem
    \item{乘积的共轭转置等于共轭转置的倒序乘积
    $$  (AB)^\dagger = B^\dagger A^\dagger, $$
    $$ (ABC)^\dagger  = C^\dagger B^\dagger A^\dagger, $$
      \ldots}
    \item{共轭转置矩阵的行列式是原矩阵的行列式的共轭
      $$\det\left(A^\dagger\right)  = (\det A)^*$$
    }
      \eitem
\end{frame}


\begin{frame}
  \frametitle{内积}

  \bitem
  \item{实向量(看成$n\times 1$实元素矩阵)$\vecu,\vecv$的内积可以用矩阵乘法简洁地写成$\vecu^T\vecv$.
  
  {\scriptsize (它恒等于$\vecv^T \vecu$,所以实向量内积满足交换律)}}

\item{复向量(看成$n\times 1$复元素矩阵)$\vecu,\vecv$的内积则定义为 $\vecu^\dagger\vecv$.

  {\scriptsize (它一般不等于$\vecv^\dagger \vecu$,复向量内积只有当内积为实数时才满足交换律)}}
\item{  复向量的模定义为它和自己的内积的平方根 $\lVert \vecu\rVert \equiv \sqrt{\vecu^\dagger\vecu } $

  {\scriptsize(即它的每个分量的模的平方和再开平方根)}}
\item{如果两个向量内积为零,则说它们正交。}
\item{如果一个向量的所有元素为零,则称它为零向量}
  \eitem
  
\end{frame}


\begin{frame}
  \frametitle{酉矩阵(Unitary Matrix)}
  如果一个方阵和自己的共轭转置互逆($AA^\dagger = A^\dagger A = I$),就称它为{\bf \blue 酉(Unitary)矩阵}。

  \skipline

  $n\times n$的酉矩阵的所有列(或行)构成$n$维复空间的一组完备的正交归一化的基。反过来,如果有$n$维复空间的一组完备的正交归一化的基,以它们为列(或行)可以得到一个酉矩阵。

  \skipline
  
  当然,实正交矩阵是酉矩阵的特例。
\end{frame}

\begin{frame}
  \frametitle{酉矩阵(Unitary Matrix)的行列式的模为$1$}
  对酉矩阵$U$,
  $$  |\det U|^2 = \det(U^\dagger)\det(U) =  \det(U^\dagger U)  = 1$$
  所以$\det U$是模为$1$的复数。
\end{frame}

\begin{frame}
  \frametitle{复向量的``旋转''}
  酉矩阵可以看成复向量空间的旋转操作,具有保内积不变的特点:
  $$ (U\vecx)^\dagger (U\vecy) = \vecx^\dagger U^\dagger U \vecy = \vecx^\dagger \vecy. $$
  特别地,当酉矩阵$U$作用到一个复向量上$\vecx$时,复向量的模不变。这符合我们对“旋转”的直观理解。
  
\end{frame}



\begin{frame}
  \frametitle{厄米矩阵(Hermitian Matrix)}
  如果一个方阵的共轭转置等于自身($A^\dagger = A$),就称它为{\bf \blue 厄米(Hermitian)矩阵}。
  \skipline

  例如
  \be
    A = \left(
    \begin{array}{ll}
      1 &  2+\ii \\
      2-\ii &  -4
    \end{array}
    \right)
  \ee
  就是厄米矩阵。

  \skipline
  
  当然,实对称矩阵是厄米矩阵的特例。
\end{frame}

\begin{frame}
  \frametitle{厄米矩阵的本征值和本征矢}
  \bitem
      \item{设$A$是$n\times n$厄米矩阵,通过求解方程
  $$ \det\left( \lambda I - A\right) = 0 $$
  可以得到$n$个本征值($m$重根视为$m$个),这些本征值一定是实数。}

  \item{只出现$m=1$次的本征值对应的本征矢方向确定(但允许乘$-1$),且和其他本征值的本征矢都正交。}

  \item{重复$m>1$次的本征值的所有本征矢构成(和其他本征值的本征矢都正交的)$m$维线性子空间。可以取一组正交基,人为地使这些本征矢也两两正交;这样的取法当然有无穷多种。}
    \eitem


  {\scriptsize 证明参见附录B }
\end{frame}


\begin{frame}
  \frametitle{厄米矩阵的酉对角化}
  
  根据前述讨论,对厄米矩阵$A$总是可以取一组正交归一化的本征矢,令它们为列向量就得到一个酉矩阵$U$。容易根据本征矢的定义直接验证
  $U^\dagger A U$是一个以$A$的所有本征值为对角元的对角矩阵。

  \skipline

 {\scriptsize 题外话:一般地,对一个方阵$S$,如果存在一个酉矩阵$U$使得$U^\dagger SU$是对角矩阵,则称$S$可以酉对角化。在矩阵论中有一个深奥的定理:一个矩阵$S$可以酉对角化的充分必要条件是它和自己的共轭矩阵对易(也就是$S S^\dagger = S^\dagger S$) —— 这样的矩阵叫正规矩阵(Normal Matrix)。显然,酉矩阵和厄米矩阵都是正规矩阵,所以都能酉对角化。}
\end{frame}

\thinka{计算厄米矩阵$$\left(
  \begin{array}{lll}
    1 & 0 & 0 \\
    0 &   0 & -\ii \\
    0 &  \ii & 0 
    \end{array}
    \right)$$的所有本征值,并写出一组正交归一化的本征矢。
}

\begin{frame}
  \frametitle{厄米矩阵和二次型}
  设$n\times n$的厄米矩阵本征值分别为$\lambda_1\le\lambda_2\le\ldots\le\lambda_n$.把对应的本征矢取为正交归一化的基$\vecu_1,\vecu_2,\ldots, \vecu_n$。考虑任意非零复矢量
  $$ \vecx = x_1\vecu_1 + x_2\vecu_2+\ldots+x_n\vecu_n.$$
  那么,$\vecx$的二次型
  $$\frac{\vecx^\dagger A\vecx}{\vecx^\dagger \vecx} = \frac{\lambda_1|x_1|^2 + \lambda_2|x_2|^2+\ldots + \lambda_n|x_n|^2}{|x_1|^2+|x_2|^2+\ldots+|x_n|^2}$$
  也就是说二次型$\frac{\vecx^\dagger A\vecx}{\vecx^\dagger \vecx}$是$A$的本征值的加权平均(权重均为非负实数)。显然,当$\vecx$取遍所有非零复矢量时,$\frac{\vecx^\dagger A\vecx}{\vecx^\dagger \vecx}$可以取遍$[\lambda_1, \lambda_n]$区间内的所有实数。
\end{frame}

\thinka{设$x,y$为复变量,满足$|x|^2+|y|^2=1$。证明下述表达式
  $$ 2|x|^2 + 3|y|^2 + (1+\ii)x^*y + (1-\ii)xy^*$$
  一定是实数。并计算这个表达式的最大可能取值。}

\section{Hyperbolic Function}

\secpage{双曲函数}{$$\cosh^2x - \sinh^2x=1$$}


\begin{frame}
\frametitle{双曲正弦和双曲余弦}

双曲正弦函数$\sinh$和双曲余弦函数$\cosh$分别定义为:

\tbox{$$ \sinh x \equiv \frac{e^x-e^{-x}}{2} $$}

\tbox{$$ \cosh x \equiv \frac{e^x+e^{-x}}{2} $$}

此外还有双曲正切$\tanh x \equiv \frac{\sinh x }{ \cosh x}$,双曲余切$\coth\equiv  \frac{ \cosh x}{\sinh x }$,双曲正割$\sech x\equiv \frac{1}{\cosh x}$,双曲余割$\csch x\equiv \frac{1}{\sinh x}$等。
\end{frame}

\begin{frame}
\frametitle{双曲函数公式和三角函数公式对比}

\bmini{0.48}
\begin{eqnarray}
  \cos^2x + \sin^2x   &=& 1 \newl
  \frac{d}{dx}\sin x   &=& \cos x \newl
  \frac{d}{dx}\cos x   &=& -\sin x \newl
  \frac{d}{dx}\tan x   &=& \sec^2 x \newl  
  \frac{d}{dx}\cot x   &=& -\csc^2 x \newl  
  \sin(2x) &=& 2\sin x \cos x \newl
  \cos(2x) &=& 2\cos^2 x - 1 \nonumber
\end{eqnarray}
\emini
\bmini{0.48}
\begin{eqnarray}
 \cosh^2x - \sinh^2x   &=& 1 \newl
  \frac{d}{dx}\sinh x   &=& \cosh x \newl
  \frac{d}{dx}\cosh x   &=& \sinh x \newl
  \frac{d}{dx}\tanh x   &=& \sech^2 x \newl    
  \frac{d}{dx}\coth x   &=& -\csch^2 x \newl    
  \sinh(2x) &=& 2\sinh x \cosh x \newl
  \cosh(2x) &=& 2\cosh^2x - 1 \nonumber
\end{eqnarray}
\emini
\end{frame}

\thinka{计算积分$\int_0^1 \sqrt{1+x^2} dx$}


\section{Cycloid}

\secpage{滚轮线(Cycloid)}{$$x=a(\theta-\sin\theta)
  ,\ y=a(1-\cos\theta) $$}

\begin{frame}
  \frametitle{滚轮线(Cycloid)}
  当一个轮子在平地上滚动时,轮子边上的固定点走过的轨迹就是滚轮线 (cycloid)。滚轮线又叫摆线。
  \addfig{3.2}{cycloid.jpg}
  设轮子半径为$a$,$y$对$x$的依赖关系可以通过中间变量(滚过的角度)$\theta$间接确定。
  $$x=a(\theta-\sin\theta) ,\ y=a(1-\cos\theta). $$


\end{frame}

\thinka{半径为$1$的轮子在平地上滚一圈,轮子边缘上的固定点的轨迹的长度是多少?}

\section{Homework}

\begin{frame}
  \frametitle{Homework}
  \bitem
\item{求极限$$\lim_{x\rightarrow 0}\frac{1}{x^2} \ln\frac{\pi x}{\sin (\pi x)}.$$}
\item{思考题中介绍了$\Gamma$函数并估算了$-1<\Gamma'(1)<0$。请证明更强的结果: $-\frac{29}{36}<\Gamma'(1)<\frac{1}{e}-\frac{3}{4}$。}  
\item{已知$A$为厄米矩阵。如果对任意非零矢量$\vecu$,$\vecu^\dagger A\vecu $都为正实数,则称$A$正定。证明:$A$正定的充分必要条件是它的本征值全部为正实数。}
\item{计算积分$$\int_0^1\frac{1}{(1+x^2)^{3/2}}dx.$$}
\item{有滚轮线$$x=\theta-\sin\theta ,\ y=1-\cos\theta.$$
  在$0\le x\le 2\pi$范围内,曲线下方围的面积$\int_0^{2\pi} y dx$是多少?}
  \eitem
\end{frame}


\section{Appendices}

\append{A}{线性变换的本质}

\begin{frame}
  \frametitle{线性变换}

  设$A$为线性算符;$\vecx$为矢量;$c$为普通的数,则
  
  $$ A (c\vecx) = c (A\vecx) $$  

  $$ A(\vecx +\vecy)= A\vecx + A\vecy $$

\end{frame}


\begin{frame}
  \bmini{0.3}
  \lfig{1.}{xiongbenxiong.jpg}
  \emini
  \bmini{0.32}

  线性变换$\Rightarrow$

  $$(\vecx \Rightarrow A\vecx)$$

  \skiplines

  \skiplines

  \skiplines  

  非线性变换$\Rightarrow$

  $$(\vecx \Rightarrow \#\$\%\&)$$
  \emini
  \bmini{0.25}
  \lfig{0.8}{xiongbenxiong_rotate.jpg}

  \skipline
  
  \lfig{0.8}{xiongbenxiong_curved.jpg}  
  \emini
\end{frame}


\begin{frame}

  \question 什么是线性变换?
  
  \answer {\bf 拉升+旋转}

  \question 什么是拉升?
  
  \answer 缩放坐标

  \question 什么是旋转?

  \answer 换一组基
\end{frame}
  


\begin{frame}
  \frametitle{奇异值分解(SVD)定理(证明略)}

  如果一个给定的线性算符$A$把$n$维“原始空间”的矢量映射到$m$维“目标空间”,那么一定存在原始空间的一组正交归一化的基$\vecu_1, \vecu_2,\ldots, \vecu_n$,在目标空间的一组正交归一化的基$\vecv_1,\vecv_2,\ldots, \vecv_m$,以及一组缩放系数$\lambda_1,\lambda_2,\ldots,\lambda_p$ (这里$p=\min(m,n)$);使得
    $$A(x_1\vecu_1+x_2\vecu_2+\ldots+x_n\vecu_n) = (\lambda_1x_1)\vecv_1+(\lambda_2x_2)\vecv_2+\ldots+(\lambda_px_p)\vecv_p.$$
    对任意$x_1,x_2,\ldots, x_n$成立.

  \skipline

找出正交归一化基以及缩放系数的操作称为A的{\bf 奇异值分解(Single Value Decomposition,简称SVD)}。

    \skipline
  
 {\scriptsize SVD存在符号可交换性(例如可以把$\vecu_1$换成$-\vecu_1$,同时把$\lambda_1$换成$-\lambda_1$)、次序可交换性(例如把下标1和2互换)、以及简并可重组性(当两个$\lambda_i=\lambda_j$,可以把$\vecu_i, \vecu_j$以及$\vecv_i,\vecv_j$进行同样方式的线性重组);从这些意义上讲,SVD不是唯一的。}

\end{frame}



\append{B}{厄米矩阵的本征值和本征矢}

\begin{frame}
  \frametitle{厄米矩阵的本征值都是实数}
  设$A$是厄米矩阵,$\lambda$是$A$的本征值,则存在非零向量$x$(本征矢)使得

  \begin{equation}
    A \vecx = \lambda \vecx. \label{eq:1}
  \end{equation}
  左边乘以$\vecx^\dagger$,得到:
  \begin{equation}
      \vecx^\dagger A \vecx  = \lambda \vecx^\dagger \vecx. \label{eq:2}
  \end{equation}
  然后两边取共轭转置,并利用$A^\dagger = A$,得到:
  \begin{equation}
      \vecx^\dagger A \vecx  = \lambda^* \vecx^\dagger \vecx. \label{eq:3}
  \end{equation}
  比较\eqref{eq:2}和\eqref{eq:3}立刻得到$\lambda = \lambda^*$。  
\end{frame}


\begin{frame}
  \frametitle{不同本征值对应的本征向量正交}
  设$A$是厄米矩阵,$\lambda_1 \ne \lambda_2$是$A$的不同本征值(根据前面讨论,它们都是实数)。 分别取它们各自的一个本征矢量$\vecx_1$, $\vecx_2$

    \begin{equation}
      A\vecx_1 = \lambda_1 \vecx_1. \label{eq:4}
    \end{equation}
    \begin{equation}
      A\vecx_2 = \lambda_2 \vecx_2. \label{eq:5}
    \end{equation}
    \eqref{eq:4}取共轭转置,得到
    \begin{equation}
      \vecx_1^\dagger A = \lambda_1 \vecx_1^\dagger. \label{eq:6}
    \end{equation}
    \eqref{eq:6}右乘$\vecx_2$,减去\eqref{eq:5}左乘$\vecx_1^\dagger$,得到
    $$ 0 = (\lambda_1-\lambda_2)\vecx_1^\dagger \vecx_2 . $$
    由于$\lambda_1\ne \lambda_2$,上式就说明  $\vecx_1^\dagger \vecx_2 = 0.$
\end{frame}

\begin{frame}
  \frametitle{同一个本征值对应的本征向量构成线性子空间}
  设$A$是厄米矩阵,$\vecx_1$, $\vecx_2$都是本征值$\lambda$的本征向量
  $$A\vecx_1 =\lambda \vecx_1,\  A\vecx_2 = \lambda \vecx_2 .$$
  显然$\vecx_1$,$\vecx_2$的任意非零线性组合都是$A$对应于$\lambda$的本征向量。
  $$A(c_1\vecx_1 +c_2\vecx_2) = \lambda(c_1\vecx_1+c_2\vecx_2).$$
  也就是说对应$\lambda$的所有本征向量构成一个线性子空间,这个子空间的维数称为本征值$\lambda$的简并度。简并度实际上就是方程
  $$ \det\left( \lambda I - A\right) = 0 $$
  的根的重数。  
  在$\lambda$对应的本征向量空间可以取一组正交基,只要简并度大于$1$,正交基的取法就有无穷多种。
\end{frame}

\begin{frame}
  \frametitle{从SVD定理的角度理解本征矢和本征值}
  对$n\times n$的厄米矩阵A,设其本征值$\lambda_1,\lambda_2, \ldots, \lambda_n$对应的正交归一化本征矢分别为$\vecv_1,\vecv_2,\ldots,\vecv_n$。根据本征矢的定义,容易看出,对任意$x_1,x_2,\ldots, x_n$均有

  $$A(x_1\vecv_1+x_2\vecv_2+\ldots+x_n\vecv_n) = (\lambda_1x_1)\vecv_1+(\lambda_2x_2)\vecv_2+\ldots+(\lambda_nx_n)\vecv_n.$$

  
  和SVD定理对比可以发现,这时原始空间和目标空间重合,原始空间里“合适的基”也和目标空间里“合适的基”重合:它们就是本征矢$\vecv_1,\vecv_2,\ldots,\vecv_n$。坐标分量的缩放比例就是本征值$\lambda_1,\lambda_2, \ldots, \lambda_n$。矩阵的厄米性导致了这些特殊的性质。
 
\end{frame}


\ech
\end{document}
