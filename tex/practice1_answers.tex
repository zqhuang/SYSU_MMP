\documentclass[12pt,CJK]{article}
\usepackage{geometry}
\input{reduced_macros.tex}
\geometry{tmargin=0.3in, bmargin=0.5in, lmargin=0.5in, rmargin=0.9in, nohead, nofoot}
\def\mark#1{{\color{blue} (#1分)}}
\renewcommand{\thepage}{}
\begin{document}
\bch
{\large 数理方法 课堂小测I}

{\vskip 0.2in}

姓名 ....................... {\hskip 0.5in}    学号 .......................{\hskip 0.5in}  分数 ...................

\bitem
\item[(一)]{选择题,每题3分,共45分。

  \bitem
\item[(1)]{ $e^{\ii \pi}$的值为  \brans{A}
  
  \optlist{$-1$}{$0$}{$1$}{$\ii$} }
\item[(2)]{方程$e^z = 1 +\ii$ 的全部复数解为 $z=$ \brans{C}


\foptlist{$\ln 2 + \frac{\pi}{4}\ii$}{$\ln 2 + \left(2n+\frac{1}{4}\right)\pi \ii,\,n\in Z$}{$\frac{1}{2}\ln 2 + \left(2n+\frac{1}{4}\right)\pi \ii,\,n\in Z$}{$\frac{1}{2}\ln 2 + \left(n+\frac{1}{4}\right)\pi \ii,\,n\in Z$}}


\item[(3)]{复变函数 $z\cos z$ 在 $z=0$ 处的导数为 \brans{C}
  
  \optlist{$-1$}{$0$}{$1$}{$2\pi\ii$}}

\item[(4)]{复变函数$f(z) = (1+z)e^z$的原函数为(忽略不写积分常数): \brans{B}

  \optlist{$e^z$}{$ze^z$}{$\left(z+\frac{z^2}{2}\right)e^z$}{$\frac{e^z}{1+z}$}}
\item[(5)]{方程 $z^5+ z^4+z^3+2=0$ 的所有复数解的平方和为 \brans{A} 

  \optlist{$-1$}{$0$}{$1$}{$2$}

{\red \small 设$z^5+z^4+z^3+2 = (z-z_1)(z-z_2)\ldots(z-z_5)$,展开比较$z^4$和$z^3$系数}
}  

\item[(6)]{已知$f$和$g$在全复平面上解析,且在某个复数集$S$上$f$和$g$恒等。当$S$为下述哪个集合时,我们{\bf 不能}断言$f$和$g$在全复平面上恒等? \brans{D}

  \optlist{单位圆$|z|=1$内部}{区间$(0,1)$}{$(0,1)$上的所有有理数}{整数集}}
  
\item[(7)]{$\frac{1}{(1+e^z)\sin z}$ 在区域$|z|< 5 $内有多少个孤立奇点? \brans{C}
  
  \optlist{$3$}{$4$}{$5$}{$6$}}
\item[(8)]{$\frac{1}{z^2-3z+2}$ 在 $z=1$ 处的留数等于 \brans{D}

    \optlist{$2$}{$1$}{$0$}{$-1$}}  

\item[(9)]{下列哪个多值函数在区域 $1<|z|<2$ 内可以规定适当的幅角范围成为解析函数? \brans{D}

  \optlist{$\ln(z-1)$}{$\ln (z+1)$}{$\ln [(z-1)(z+1)]$}{$\ln \frac{z-1}{z+1}$}}


\item[(10)]{$\sqrt{2\pi}\,\delta (x^2-1)$的傅立叶变换为 \brans{B}

  \optlist{$1$}{$\cos k$}{$\sin k$}{$\frac{\sin k}{k}$}}

\item[(11)]{$e^{-t}\sin{2t}$的拉普拉斯变换为 \brans{A}

  \optlist{$\frac{2}{(p+1)^2+4}$}{$\frac{2}{(p-1)^2+4}$}{$\frac{2p}{p^2+4}$}{$\frac{2e^{-p}}{p^2+4}$}}

\item[(12)]{$ \left\vert\oint_{|z|=1}\, \frac{\cos z}{z^{2719}}\,dz\right\vert$ 和下列哪个数量级最接近? \brans{C}

  \optlist{$10^{-900}$}{$10^{-2700}$}{$10^{-5100}$}{$10^{-15300}$}

{\small \red 用Stirling公式算出结果$\sim 10^{-8100}$,数量级接近指取对数后接近。}
}

\item[(13)]{ 在单位球内的积分 $\iiint_{x^2+y^2+z^2\le 1} (|x|+|y|+|z|)dxdydz$ 等于 \brans{A}

  \optlist{$\frac{3}{2}\pi$}{$\sqrt{3}\pi$}{$2\pi$}{$2\sqrt{3}\pi$}

{\small \red 根据对称性把积分写成 $24\int_{x,y,z\ge 0, x^2+y^2+z^2\le 1} x dxdydz$,然后用变量替换以及限和积分公式}
}
  
\item[(14)]{用$z^*$表示$z$的共轭复数,按逆时针沿着曲线$|z-3|+|z+3|=10$的围道积分$\oint_{|z-3|+|z+3|=10}\, (z^*dz - zdz^*)$  等于 \brans{C}
  
  \optlist{$48\pi\ii$}{$60\pi\ii$}{$80\pi\ii$}{$96\pi\ii$}

  {\small \red 转化为极坐标下的实数积分}
}

\item[(15)]{下列哪一个复变函数在复平面上处处不可导? \brans{D}

  \optlist{$\frac{1}{z}$}{$|z|^2$}{$\sin{|z|}$}{$e^{|z|}$}

  {\small \red  $|z|^2$在$z=0$可导,$\sin |z|$在$z=\left(n+\frac{1}{2}\right)\pi$可导。}
}  
    
  \eitem
  }
\item[(二)]{ 填空题,每题5分,共35分。

  \bitem  
\item[(1)]{复变函数 $e^{\frac{\cos z}{1+z^2+z^4}}\cos{(z^3)}$ 在 $z=0$ 处的导数为  \ans{0}。

  {\small \red 偶函数在$z=0$导数为零。}

}  
\item[(2)]{函数 $\frac{1}{z^{5}+z+1}$ 的所有孤立奇点处的留数之和为 \ans{0}。


  {\small \red 考虑半径趋向无穷大的圆上的围道积分。} 
}  
\item[(3)]{积分 $\int_0^{\frac{\pi}{2}}\, \cos^5\theta \, \sin^4\theta\, d\theta$ 等于 \ans{$\frac{8}{315}$} 。


  {\small \red 作变量替换$x = \sin^2\theta$,然后用$B$函数和$\Gamma$函数的关系。}
}
  
\item[(4)]{实积分 $\int_0^\infty e^{-x^2}\cos{2x}\,dx$ 等于 \ans{$\frac{\sqrt{\pi}}{2e}$} 。

  {\small \red 取适当围道把路径 $(-\infty+\ii, \infty +\ii)$上的积分转化为$(-\infty,\infty)$上的积分 }
}
\item[(5)]{逆时针方向沿着上半个单位圆的积分$\int_{|z|=1, \mathrm{Im}(z)\ge 0}\, z^{\frac{1}{3}}\,dz$ 等于 \ans{$\frac{3}{4}e^{\ii\theta}\left(e^{\frac{4\pi\ii}{3}}-1\right)$,$\theta = 0, \pm\frac{2\pi}{3}.$}  {\small \red 或者按在正实轴上幅角为零的默认约定把答案写成$\frac{3}{4}\left(e^{\frac{4\pi\ii}{3}}-1\right)$也算对。}}
\item[(6)]{复变函数$f(z) = |1-z|^4 + (1+|z|)^4$的最小可能值等于 \ans{2}。

{\small \red 利用 $|1-z|^4\ge 1 - 4|z|-6|z|^2 - \ldots$}
}   
\item[(7)]{定义函数  $f(x) = \frac{1}{\pi}\int_0^\pi \cos{\left( x \cos t\right)} \, dt. $,则积分 $\int_0^{\frac{\pi}{2}} f(x)f(\frac{\pi}{2}-x)\,dx$ 等于 \ans{1}。

  {\small \red 把$f(x)$进行拉普拉斯变换,然后用卷积定理。 }
}
\eitem

  }
  

\item[(三)]{长度为$L$的不良导体棒一端和温度为$T_0$的热库接触,并在$t=0$时刻和热库处于热平衡。从$t=0$时刻开始,在不良导体棒的另一端注入恒定大小为$j$的热流。设不良导体棒的导热系数$\lambda$,单位质量的比热$c$和质量密度$\rho$均已知。写出不良导体棒上温度$T(x, t)$ ($0\le x\le L, t\ge 0$)满足的方程和边界条件。(10分) 并简要分析当$t$很大时的解的渐近行为。(10分)


  {\blue
    \skipline
    
    方程为
    \be
    \frac{\partial T}{\partial t} -a\frac{\partial^2T}{\partial x^2} = 0 \, .
    \ee
    其中$a= \frac{\lambda}{\rho c}$。

    边界条件为
    \bea
    \left.T\right\vert_{t=0} &=& T_0 \newl
    \left.T\right\vert_{x=0} &=& T_0 \newl
    \left.\frac{\partial T}{\partial x}\right\vert_{x=L} &=& \frac{j}{\lambda}
    \eea

    假设当$t$远大于典型的变化时间$L^2/a$时,系统处于稳恒状态(温度梯度不再变化)。因为一端温度是固定的,要得到稳恒状态的必要条件是热量不在不良导体棒上积累,也就是说进来的热流$j$必须保持不变地通过整个不良导体棒,最后从另一端进入热库。这说明稳恒状态下$\frac{\partial T}{\partial x}$处处等于$\frac{j}{\lambda}$。由此得出:
    $$T(x,t) \rightarrow T_0+\frac{j}{\lambda} x$$
    
    }
}
\eitem    


\ech
\end{document}
