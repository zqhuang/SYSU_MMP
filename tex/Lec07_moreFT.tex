\documentclass[CJK]{beamer}
\usepackage{CJKutf8}
\usepackage{beamerthemesplit}
\usetheme{Malmoe}
\useoutertheme[footline=authortitle]{miniframes}
\usepackage{amsmath}
\usepackage{amssymb}
\usepackage{graphicx}
\usepackage{eufrak}
\usepackage{color}
\usepackage{slashed}
\usepackage{simplewick}
\usepackage{tikz}
\usepackage{tcolorbox}
\graphicspath{{../figures/}}
%%figures
\def\lfig#1#2{\includegraphics[width=#1 in]{#2}}
\def\addfig#1#2{\begin{center}\includegraphics[width=#1 in]{#2}\end{center}}
\def\wulian{\includegraphics[width=0.18in]{emoji_wulian.jpg}}
\def\bigwulian{\includegraphics[width=0.35in]{emoji_wulian.jpg}}
\def\bye{\includegraphics[width=0.18in]{emoji_bye.jpg}}
\def\bigbye{\includegraphics[width=0.35in]{emoji_bye.jpg}}
\def\huaixiao{\includegraphics[width=0.18in]{emoji_huaixiao.jpg}}
\def\bighuaixiao{\includegraphics[width=0.35in]{emoji_huaixiao.jpg}}
\def\jianxiao{\includegraphics[width=0.18in]{emoji_jianxiao.jpg}}
\def\bigjianxiao{\includegraphics[width=0.35in]{emoji_jianxiao.jpg}}
%% colors
\def\blacktext#1{{\color{black}#1}}
\def\bluetext#1{{\color{blue}#1}}
\def\redtext#1{{\color{red}#1}}
\def\darkbluetext#1{{\color[rgb]{0,0.2,0.6}#1}}
\def\skybluetext#1{{\color[rgb]{0.2,0.7,1.}#1}}
\def\cyantext#1{{\color[rgb]{0.,0.5,0.5}#1}}
\def\greentext#1{{\color[rgb]{0,0.7,0.1}#1}}
\def\darkgray{\color[rgb]{0.2,0.2,0.2}}
\def\lightgray{\color[rgb]{0.6,0.6,0.6}}
\def\gray{\color[rgb]{0.4,0.4,0.4}}
\def\blue{\color{blue}}
\def\red{\color{red}}
\def\green{\color{green}}
\def\darkgreen{\color[rgb]{0,0.4,0.1}}
\def\darkblue{\color[rgb]{0,0.2,0.6}}
\def\skyblue{\color[rgb]{0.2,0.7,1.}}
%%control
\def\be{\begin{equation}}
\def\ee{\nonumber\end{equation}}
\def\bea{\begin{eqnarray}}
\def\eea{\nonumber\end{eqnarray}}
\def\bch{\begin{CJK}{UTF8}{gbsn}}
\def\ech{\end{CJK}}
\def\bitem{\begin{itemize}}
\def\eitem{\end{itemize}}
\def\bcenter{\begin{center}}
\def\ecenter{\end{center}}
\def\bex{\begin{minipage}{0.2\textwidth}\includegraphics[width=0.6in]{jugelizi.png}\end{minipage}\begin{minipage}{0.76\textwidth}}
\def\eex{\end{minipage}}
\def\chtitle#1{\frametitle{\bch#1\ech}}
\def\bmat#1{\left(\begin{array}{#1}}
\def\emat{\end{array}\right)}
\def\bcase#1{\left\{\begin{array}{#1}}
\def\ecase{\end{array}\right.}
\def\bmini#1{\begin{minipage}{#1\textwidth}}
\def\emini{\end{minipage}}
\def\tbox#1{\begin{tcolorbox}#1\end{tcolorbox}}
\def\pfrac#1#2#3{\left(\frac{\partial #1}{\partial #2}\right)_{#3}}
%%symbols
\def\bropt{\,(\ \ \ )}
\def\sone{$\star$}
\def\stwo{$\star\star$}
\def\sthree{$\star\star\star$}
\def\sfour{$\star\star\star\star$}
\def\sfive{$\star\star\star\star\star$}
\def\rint{{\int_\leftrightarrow}}
\def\roint{{\oint_\leftrightarrow}}
\def\stdHf{{\textit{\r H}_f}}
\def\deltaH{{\Delta \textit{\r H}}}
\def\ii{{\dot{\imath}}}
\def\skipline{{\vskip0.1in}}
\def\skiplines{{\vskip0.2in}}
\def\lagr{{\mathcal{L}}}
\def\hamil{{\mathcal{H}}}
\def\vecv{{\mathbf{v}}}
\def\vecx{{\mathbf{x}}}
\def\vecy{{\mathbf{y}}}
\def\veck{{\mathbf{k}}}
\def\vecp{{\mathbf{p}}}
\def\vecn{{\mathbf{n}}}
\def\vecA{{\mathbf{A}}}
\def\vecP{{\mathbf{P}}}
\def\vecsigma{{\mathbf{\sigma}}}
\def\hatJn{{\hat{J_\vecn}}}
\def\hatJx{{\hat{J_x}}}
\def\hatJy{{\hat{J_y}}}
\def\hatJz{{\hat{J_z}}}
\def\hatj#1{\hat{J_{#1}}}
\def\hatphi{{\hat{\phi}}}
\def\hatq{{\hat{q}}}
\def\hatpi{{\hat{\pi}}}
\def\vel{\upsilon}
\def\Dint{{\mathcal{D}}}
\def\adag{{\hat{a}^\dagger}}
\def\bdag{{\hat{b}^\dagger}}
\def\cdag{{\hat{c}^\dagger}}
\def\ddag{{\hat{d}^\dagger}}
\def\hata{{\hat{a}}}
\def\hatb{{\hat{b}}}
\def\hatc{{\hat{c}}}
\def\hatd{{\hat{d}}}
\def\hatN{{\hat{N}}}
\def\hatH{{\hat{H}}}
\def\hatp{{\hat{p}}}
\def\Fup{{F^{\mu\nu}}}
\def\Fdown{{F_{\mu\nu}}}
\def\newl{\nonumber \\}
\def\vece{\mathrm{e}}
\def\calM{{\mathcal{M}}}
\def\calT{{\mathcal{T}}}
\def\calR{{\mathcal{R}}}
\def\barpsi{\bar{\psi}}
\def\baru{\bar{u}}
\def\barv{\bar{\upsilon}}
\def\qeq{\stackrel{?}{=}}
\def\torder#1{\mathcal{T}\left(#1\right)}
\def\rorder#1{\mathcal{R}\left(#1\right)}
\def\contr#1#2{\contraction{}{#1}{}{#2}#1#2}
\def\trof#1{\mathrm{Tr}\left(#1\right)}
\def\trace{\mathrm{Tr}}
\def\comm#1{\ \ \ \left(\mathrm{used}\ #1\right)}
\def\tcomm#1{\ \ \ (\text{#1})}
\def\slp{\slashed{p}}
\def\slk{\slashed{k}}
\def\calp{{\mathfrak{p}}}
\def\veccalp{\mathbf{\mathfrak{p}}}
\def\Tthree{T_{\tiny \textcircled{3}}}
\def\pthree{p_{\tiny \textcircled{3}}}
\def\dbar{{\,\mathchar'26\mkern-12mu d}}
\def\erf{\mathrm{erf}}
\def\const{\mathrm{constant}}
\def\pheat{\pfrac p{\ln T}V}
\def\vheat{\pfrac V{\ln T}p}
%%units
\def\fdeg{{^\circ \mathrm{F}}}
\def\cdeg{^\circ \mathrm{C}}
\def\atm{\,\mathrm{atm}}
\def\angstrom{\,\text{\AA}}
\def\SIL{\,\mathrm{L}}
\def\SIkm{\,\mathrm{km}}
\def\SIyr{\,\mathrm{yr}}
\def\SIGyr{\,\mathrm{Gyr}}
\def\SIV{\,\mathrm{V}}
\def\SImV{\,\mathrm{mV}}
\def\SIeV{\,\mathrm{eV}}
\def\SIkeV{\,\mathrm{keV}}
\def\SIMeV{\,\mathrm{MeV}}
\def\SIGeV{\,\mathrm{GeV}}
\def\SIcal{\,\mathrm{cal}}
\def\SIkcal{\,\mathrm{kcal}}
\def\SImol{\,\mathrm{mol}}
\def\SIN{\,\mathrm{N}}
\def\SIHz{\,\mathrm{Hz}}
\def\SIm{\,\mathrm{m}}
\def\SIcm{\,\mathrm{cm}}
\def\SIfm{\,\mathrm{fm}}
\def\SImm{\,\mathrm{mm}}
\def\SInm{\,\mathrm{nm}}
\def\SImum{\,\mathrm{\mu m}}
\def\SIJ{\,\mathrm{J}}
\def\SIW{\,\mathrm{W}}
\def\SIkJ{\,\mathrm{kJ}}
\def\SIs{\,\mathrm{s}}
\def\SIkg{\,\mathrm{kg}}
\def\SIg{\,\mathrm{g}}
\def\SIK{\,\mathrm{K}}
\def\SImmHg{\,\mathrm{mmHg}}
\def\SIPa{\,\mathrm{Pa}}

\def\courseurl{https://github.com/zqhuang/SYSU\_TD}

\def\tpage#1#2{
\begin{frame}
\begin{center}
\begin{Large}
\bch
热学 \\
第#1讲 #2

{\vskip 0.3in}

黄志琦

\ech
\end{Large}
\end{center}

\vskip 0.2in

\bch
教材:《热学》第二版,赵凯华,罗蔚茵,高等教育出版社
\ech

\bch
课件下载
\ech
\courseurl
\end{frame}
}

\def\bfr#1{
\begin{frame}
\chtitle{#1} 
\bch
}

\def\efr{
\ech 
\end{frame}
}

  \date{}
\begin{document}
\tpage{7}{More about Fourier Transform}




\begin{frame}
  \chtitle{本讲内容}
\bch
\bitem
\item{高维空间的$\delta$函数和傅立叶变换}
\item{共轭关系}
\item{梯度算符和拉普拉斯算符}
\item{卷积定理}    
\eitem
\ech
\end{frame}




\section{$n$-D Fourier transform}

\secpage{高维空间的$\delta$函数和傅立叶变换}{就是每个维度来一下}

\begin{frame}
  \chtitle{$n$维内积空间}
  \bch
  $n$维内积空间的点可以记作$\vecx = (x_1, x_2, \ldots, x_n)$,多元函数$f(x_1, x_2, \ldots, x_n)$可以简单写成$f(\vecx)$。

  \skipline
  
  两个矢量$\vecx, \vecy$的内积为
 {\blue $$ \vecx\cdot\vecy = x_1y_1+x_2y_2+\ldots + x_n y_n .$$}

  \skipline

  函数$f(\vecx)$在全空间的积分可以简单写作
 {\blue $$\int f(\vecx)\, d^n\vecx $$}
  用你们熟悉的语言写出来就是
  $$\int_{-\infty}^\infty d x_1 \int_{-\infty}^\infty dx_2 \ldots \int_{-\infty}^\infty d x_n\, f(x_1, x_2, \ldots, x_n). $$

  \ech
  \end{frame}

\begin{frame}
  \chtitle{$n$维空间的$\delta$函数}
  \bch
  $n$维$\delta$函数标记为{\blue $\delta^{(n)}(\vecx)$},既可以理解为
      {\blue
        $$\delta^{(n)}(\vecx) = \delta(x_1)\delta(x_2)\ldots\delta(x_n), $$
      }
      也可以直接抽象地理解为{\blue 在原点附近体积为$\epsilon \rightarrow 0^+$的邻域内,函数值为$\frac{1}{\epsilon}\rightarrow \infty $,而在其余位置函数值均为零的函数。}

      \skiplines
      
\bex
三维空间某点$\vecx‘$处的点电荷的电荷密度可以写成
$$\rho(\vecx) = Q\, \delta^{(3)}(\vecx - \vecx')$$
\eex
  \ech
\end{frame}

\begin{frame}
  \chtitle{$n$维空间的$\delta$函数的性质}
  \bch
  和一维空间类似,
  \tbox{
   $$ \int \delta^{(n)}(\vecx - \vecx_0) f(\vecx)\, d^n\vecx = f(\vecx_0)$$
   }
  \ech
\end{frame}


\begin{frame}
  \chtitle{$n$维空间的傅立叶变换}
  \bch
  $n$维内积空间的函数$f(\vecx)$的傅立叶变换就是对每个维度都进行傅立叶变换:
  $$ F(\veck) = \frac{1}{(2\pi)^{n/2}}\int f(\vecx) e^{-\ii \veck \cdot \vecx} d^n\vecx $$

  {\scriptsize (如果对上式感到困惑,请自行写出2维空间和3维空间的“你们熟悉的表达式”)}

  显然,其逆变换为
  $$ f(\vecx) = \frac{1}{(2\pi)^{n/2}}\int F(\veck) e^{\ii \veck \cdot \vecx} d^n\veck. $$
  
  \ech
\end{frame}

\begin{frame}
  \chtitle{非常重要的积分式}
  \bch
  反复应用一维空间的傅立叶变换的性质,很容易得到
  \tbox{\blue $$\frac{1}{(2\pi)^n}\int e^{\ii \veck\cdot\vecx} d^n\vecx = \delta^{(n)}(\veck).$$}
  
  \ech
\end{frame}


\begin{frame}
  \chtitle{思考题}
  \bch
  \addfig{1}{think3.jpg}
  
  试证明:  高维空间傅立叶变换保持内积不变的结论也依然成立,设$\tilde{f}, \tilde{g}$分别为$f,g$的傅立叶变换,则
{\blue  $$\int  f^*(\vecx)g(\vecx)\,d^n\vecx = \int \tilde{f}^*(\veck) \tilde{g}(\veck)\, d^n\veck . $$}

  \ech
\end{frame}




\section{Conjugate Relation}
\secpage{共轭关系}{源像互换取共轭}


\begin{frame}
  \chtitle{先精简一下符号}
  \bch
  我们用
  $$f \ftf F $$
  表示$f(\vecx)$的傅立叶变换为$F(\veck)$。
  \ech
\end{frame}


\begin{frame}
  \chtitle{共轭关系}
  \bch
  设$f \ftf F $。  对逆变换  
  $$ f(\vecx) = \frac{1}{(2\pi)^{n/2}}\int F(\veck) e^{\ii \veck \cdot \vecx} d^n\veck. $$
  两边取共轭得到
  $$ f^*(\vecx) = \frac{1}{(2\pi)^{n/2}}\int F^*(\veck) e^{-\ii \veck \cdot \vecx} d^n\veck. $$
  上式等价于$ F^*\ftf f^*$,这说明
  \tbox{ 傅立叶变换的源和像,可以取共轭并对换位置}。
  \ech
\end{frame}


\begin{frame}
  \chtitle{实的偶函数}
  \bch
  显然,实的偶函数(满足$f(-\vecx) = f(\vecx)$的实函数)的傅立叶变换仍然是实的偶函数。

  {\scriptsize(可以通过对傅立叶变换式两边取共轭并对积分变量作$\vecx \rightarrow -\vecx$的替换得到)。}

  \skipline

  在这种情况下,无所谓哪个是源,哪个是像,可以使用下述符号:
  $$ f \ftfb F $$
  \ech
\end{frame}

\begin{frame}
  \chtitle{思考题}
  \bch
  \addfig{1}{think3.jpg}
  
  求$f(x) = \frac{\sin x}{x}$的傅立叶变换。
  \ech
\end{frame}

\section{Gradient Operator}
\secpage{梯度算符和拉普拉斯算符}{引而不发才是高境界}

\begin{frame}
  \chtitle{算符}
  \bch
      {\blue 算符是把一种对象映射到另一种对象的操作}。


      \bcenter
      \lfig{1}{blackq.jpg}
      
      感觉等于什么都没说.jpg
      \ecenter
      \ech
\end{frame}


\begin{frame}
  \chtitle{梯度算符}
  \bch
  \bex
  $n$维空间梯度算符$\nabla$把$n$维空间的函数$f(\vecx)$映射为该函数的梯度(矢量)
    $$ \nabla f \equiv (\partial_1 f, \partial_2 f,\ldots, \partial_n f) .$$
    
    (为了简单起见,我们把第$i$个方向的偏导算符$\frac{\partial}{\partial_{x_i}}$简写为了$\partial_i$。)
  \eex

  \ech
\end{frame}


\begin{frame}
  \chtitle{广义矢量}
  \bch
  $n$维空间的矢量$\vecx = (x_1, x_2, \ldots, x_n)$, 可以看成$n$个基的线性组合:

  $$\vecx = x_1\vece_1 + x_2\vece_2 + \ldots + x_n\vece_n. $$

  对广义矢量,{\blue 基的线性组合可以不仅仅用数字作为系数,还能用任何对象。}
  例如,在二维内积空间,

  \skipline
  
  (苹果,橘子) = 苹果$\times \vece_1$ + 橘子$\times\vece_2$.

  \skipline
  
  (震小羊,桂小荣) = 震小羊$\times \vece_1$ + 桂小荣$\times\vece_2$.  
  
  \ech
\end{frame}


\begin{frame}
  \chtitle{思考题}
  \bch
  \addfig{1}{think3.jpg}
  
  计算矢量$(3,2)$和(苹果,橘子) 的内积。
  \ech
\end{frame}

\begin{frame}
  \chtitle{梯度算符}
  \bch
  梯度算符$\nabla$又可以看成一个广义矢量,它的每个分量是偏导算符:
  $$\nabla \equiv (\partial_1, \partial_2,\ldots, \partial_n). $$

  这就是数学中省略作用对象的“引而不发”的写法。如果在两边补充写个$f$,则又回到熟悉的小学生可以理解的形式:
  $$\nabla f \equiv (\partial_1f, \partial_2f,\ldots, \partial_nf). $$
  \ech
\end{frame}

\begin{frame}
  \chtitle{散度}
  \bch
  梯度算符$\nabla$和普通矢量$\mathbf{E} = (E_1, E_2, \ldots, E_n)$的内积为
  $$\nabla\cdot \mathbf{E} \equiv \partial_1 E_1 + \partial_2 E_2 + \ldots + \partial_n E_n $$
  这通常称为$\mathbf{E}$的{\blue 散度}。
  \skipline
  
  (在三维空间的情形可以勾起你们对电磁学的美好回忆\bye)
  \ech
\end{frame}

\begin{frame}
  \chtitle{拉普拉斯算符}
  \bch
  梯度算符作用到一个普通函数$f(\vecx)$上,就得到一个矢量函数$\nabla f = (\partial_1 f, \partial_2 f, \ldots, \partial_n f)$。然后$\nabla$和$\nabla f$的内积就是
  
  $$ \nabla \cdot \nabla f \equiv \partial_1^2 f + \partial_2^2 f + \ldots + \partial_n^2 f.$$
  上式左边的$\nabla\cdot \nabla$称为{\blue 拉普拉斯算符},通常简写为{\blue $\nabla^2$},重写一遍就是:\tbox{
    $$ \nabla^2 f \equiv \partial_1^2 f + \partial_2^2 f + \ldots + \partial_n^2 f.$$}
  
  {
    \small
  在静电磁学里,电场强度$\mathbf{E}$正比于电势$\varphi$的梯度: $\mathbf{E} = -\nabla \varphi$。高斯定律(电场的散度正比于电荷密度)就可以写成
  $$ \nabla^2 \varphi = -\frac{\rho}{\epsilon_0} $$}
  
  \ech
\end{frame}


\begin{frame}
  \chtitle{$\nabla \rightarrow \ii \veck$}
  \bch
  设$f \ftf F $,对逆变换式
  $$ f(\vecx) = \frac{1}{(2\pi)^{n/2}}\int F(\veck) e^{\ii \veck \cdot \vecx} d^n\veck. $$
  两边作用$\partial_j$
  $$ \partial_j f(\vecx) =  \frac{1}{(2\pi)^{n/2}}\int (\ii k_j F(\veck)) e^{\ii \veck \cdot \vecx} d^n\veck. $$
  当$j$取遍$1,2,\ldots, n$,上式可以写成矢量形式:
{\blue  $$ \nabla f(\vecx) \ftf \ii \veck F(\veck) $$}
  它的具体含义就是: $\partial_jf(\vecx)$的傅立叶变换为$ \ii k_j F(\veck)$ ($j=1,2,\ldots, n$)。
  \ech
\end{frame}


\begin{frame}
  \chtitle{思考题}
  \bch
  \addfig{1}{think3.jpg}
  设$f(\vecx) \ftf F(\veck) $,  试证明{\blue
    $$\nabla^2f(\vecx) \ftf -k^2F(\veck)$$}

  其中$k \equiv |\veck|$。
  \ech
\end{frame}


\begin{frame}
  \chtitle{第一个数理方程的例子}
  \bch
  用高斯定律来计算在原点的点电荷$Q$造成的电势$\varphi$
  $$\nabla^2 \varphi(\vecx) = -\frac{Q}{\epsilon_0} \,\delta^{(3)}(\vecx)$$
  
  对这个方程两边进行傅立叶变换,
  $$ -k^2\tilde{\varphi}(\veck) = -\frac{1}{(2\pi)^{3/2}} \frac{Q}{\epsilon_0} ,$$
  即
  $$\tilde{\varphi}(\veck) = \frac{1}{(2\pi)^{3/2}\epsilon_0} \frac{Q}{k^2}.$$
  再进行逆变换
 $$\varphi(\vecx) =\frac{Q}{(2\pi)^3\epsilon_0} \int \frac{e^{\ii \veck\cdot\vecx}}{k^2}d^3\veck$$
  \ech
\end{frame}

\begin{frame}
  \chtitle{第一个数理方程的例子}
  \bch
  取$\vecx$方向为北极方向建立球坐标$(k, \theta,\phi)$,记$r=|\vecx|$
  \bea
  \int \frac{e^{\ii \veck\cdot\vecx}}{k^2}d^3\veck &=& \int_0^\infty k^2 dk \int_{-1}^1 d(\cos\theta)\int_0^{2\pi} \frac{e^{\ii kr cos\theta}}{k^2} d\phi \newl
  &=& 2\pi \int_0^\infty dk \int_{-1}^1 d(\cos\theta) e^{\ii kr cos\theta} \newl
  &=& 2\pi  \int_0^\infty dk  \frac{2\sin(kr)}{kr} \newl
  &=& \frac{2\pi^2}{r}
  \eea
  由此我们求出
  $$\varphi(\vecx) = \frac{Q}{4\pi \epsilon_0r}$$
  \ech
\end{frame}

\begin{frame}
  \chtitle{思考题}
  \bch
  \addfig{0.5}{think.jpg}
  
  电势可以随便取零点,为什么解出来的电势不带积分常数?
  \ech
\end{frame}

\section{Convolution}
\secpage{卷积定理}{在同一个频道上的两个…}

\begin{frame}
  \chtitle{时间序列的频谱分析}
  \bch
  让我们再次回到一维的情况: 假设{\blue $f(t)$是依赖于时间的信号,其傅立叶变换$\tilde{f}(k)$代表了以$k$为频率的信号}。

  如果要研究两个信号$f(t)$和$g(t)$是否显著地包含相同频率的信号,则有两种方法。一种是直接计算 $\tilde{f}^*(k) \tilde{g}(k)$并找出使乘积(的模)比较大的$k$,另一种是计算延时乘积
  $$C(\Delta t) = \frac{1}{\sqrt{2\pi}}\int_{-\infty}^{\infty} f^*(t) g(t+\Delta t ) dt $$
  的傅立叶变换$\tilde{C}(k)$,找出使$\tilde{C}(k)$(的模)比较大的$k$。

  {\scriptsize (如果$f$和$g$显著具有相同频率$k$的信号,则当$\Delta t$为调制相位加上$\frac{2\pi}{k}$的整数倍时$C(\Delta t)$的信号比较强,也就是说$\tilde{C}(k)$包含有较强的频率为$k$的信号。)}

  这两种方法之间有什么内在联系呢?

  \ech
\end{frame}


\begin{frame}
  \chtitle{卷积定理的延时表述}
  \bch
  设$f$和$g$的延时关联函数定义为:
  $$C(t) =\frac{1}{\sqrt{2\pi}} \int_{-\infty}^\infty f^*(t')g(t'+t)\, dt',$$
  则$C, f, g$的傅立叶变换$\tilde{C}, \tilde{f},\tilde{g}$满足:
  $$\tilde{C}(k) = \tilde{f}^*(k) \tilde{g}(k)$$
  \ech
\end{frame}

\begin{frame}
  \chtitle{卷积定理的延时表述的证明}
  \bch
  \bea
  \tilde{f}^*(k)\tilde{g}(k) &=& \frac{1}{2\pi}\int_{-\infty}^\infty e^{\ii k u} f^*(u) du  \int_{-\infty}^\infty e^{-\ii k\upsilon} g(\upsilon) d\upsilon  \newl
  &=& \frac{1}{2\pi}\int_{-\infty}^\infty e^{\ii k (u-\upsilon)} f^*(u) du  \int_{-\infty}^\infty  g(\upsilon) d\upsilon
  \eea
  做变量替换$t = \upsilon - u$, $t' =u$得到
  \bea
  \tilde{f}^*(k)\tilde{g}(k) &=& \frac{1}{2\pi}\int_{-\infty}^\infty e^{-\ii k t} dt  \int_{-\infty}^\infty  f^*(t')  g(t'+t) dt' \newl
  &=& \frac{1}{\sqrt{2\pi}}\int_{-\infty}^\infty e^{-\ii k t}  C(t) dt \newl
  &=& \tilde{C}(k)
  \eea

  \ech
\end{frame}



\begin{frame}
  \chtitle{卷积定理的对称表述}
  \bch
  上述表述虽然有很强的物理背景,但并不容易记忆,一般书中介绍的是下面的对称形式:

  \skipline
  
  两个函数的卷积定义为
  \tbox{$$ (f*g)(t) \equiv \frac{1}{\sqrt{2\pi}}\int_{-\infty}^\infty f(t')g(t-t') dt'  $$}

  显然,卷积满足交换律: {\blue $f*g = g*f$}。更重要的是卷积定理: 设$f,g$的傅立叶变换依次为$\tilde{f},\tilde{g}$,则$f*g$的傅立叶变换为$\tilde{f}\tilde{g}$,即
  \tbox{卷积的傅立叶变换等于傅立叶变换的乘积}
  
  {\scriptsize  注:在有些文献中的傅立叶变换定义不带$\frac{1}{\sqrt{2\pi}}$因子,则卷积的定义中也不含$\frac{1}{\sqrt{2\pi}}$因子。}
  
  
  \ech
\end{frame}


\begin{frame}
  \chtitle{卷积定理的对称表述的证明}
  \bch
  令$h(t) = f^*(-t)$,则
  $$ \tilde{h}^*(k) =\frac{1}{\sqrt{2\pi}} \int_{-\infty}^{\infty} f(-t)e^{\ii kt}dt$$
  做积分变量的替换$t'=-t$,得到
  $$\tilde{h}^*(k)= \frac{1}{\sqrt{2\pi}} \int_{-\infty}^{\infty} f(t')e^{-\ii kt'}dt' = \tilde{f}(k)$$ 
  $h$和$g$的延时关联函数为
  $$C(t) = \frac{1}{\sqrt{2\pi}}\int_{-\infty}^\infty h^*(u)g(t+u) du = \frac{1}{\sqrt{2\pi}} \int_{-\infty}^\infty f(-u)g(t+u) du $$
  做变量替换$t' = -u$,得到
  $$C(t) =\frac{1}{\sqrt{2\pi}} \int_{-\infty}^\infty f(t')g(t-t') dt' = (f*g)(t) $$  

  最后,利用卷积定理的延时表述即得结论。
  \ech
\end{frame}

\section{Homework}

\begin{frame}
  \chtitle{课后作业}
  \bch
  {\small
  \bitem
\item[19]{三维空间的top-hat函数在单位球内取值为$1$,否则为零:
  \be
  f(\vecx) = \left\{\begin{array}{ll} 1, & \text{ if } |\vecx|<1; \\
  0, & \text{ else.} \end{array}\right. 
  \ee
  求$f(\vecx)$的傅立叶变换$F(\veck)$。
}

\item[20]{设$\phi(\vecx)$是三维空间中的场,其傅立叶变换为$\tilde{\phi}(\veck)$,试证明$$\int |\nabla\phi(\vecx)|^2 d^3\vecx = \int k^2|\tilde{\phi}(\veck)|^2\, d^3\veck, $$
其中$k =|\veck|$。}

\item[21]{设$f(x) = \frac{ e^{-x^2/2} \sin x}{x}$的傅立叶变换为$F(k)$,试计算
  $$ \sum_{n=-\infty}^\infty F(n) $$
的值。}  
  \eitem

  }
  \ech
\end{frame}

\end{document}
