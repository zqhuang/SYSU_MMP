\documentclass[CJK]{beamer}
\usepackage{CJKutf8}
\usepackage{beamerthemesplit}
\usetheme{Malmoe}
\useoutertheme[footline=authortitle]{miniframes}
\usepackage{amsmath}
\usepackage{amssymb}
\usepackage{graphicx}
\usepackage{eufrak}
\usepackage{color}
\usepackage{slashed}
\usepackage{simplewick}
\usepackage{tikz}
\usepackage{tcolorbox}
\graphicspath{{../figures/}}
%%figures
\def\lfig#1#2{\includegraphics[width=#1 in]{#2}}
\def\addfig#1#2{\begin{center}\includegraphics[width=#1 in]{#2}\end{center}}
\def\wulian{\includegraphics[width=0.18in]{emoji_wulian.jpg}}
\def\bigwulian{\includegraphics[width=0.35in]{emoji_wulian.jpg}}
\def\bye{\includegraphics[width=0.18in]{emoji_bye.jpg}}
\def\bigbye{\includegraphics[width=0.35in]{emoji_bye.jpg}}
\def\huaixiao{\includegraphics[width=0.18in]{emoji_huaixiao.jpg}}
\def\bighuaixiao{\includegraphics[width=0.35in]{emoji_huaixiao.jpg}}
\def\jianxiao{\includegraphics[width=0.18in]{emoji_jianxiao.jpg}}
\def\bigjianxiao{\includegraphics[width=0.35in]{emoji_jianxiao.jpg}}
%% colors
\def\blacktext#1{{\color{black}#1}}
\def\bluetext#1{{\color{blue}#1}}
\def\redtext#1{{\color{red}#1}}
\def\darkbluetext#1{{\color[rgb]{0,0.2,0.6}#1}}
\def\skybluetext#1{{\color[rgb]{0.2,0.7,1.}#1}}
\def\cyantext#1{{\color[rgb]{0.,0.5,0.5}#1}}
\def\greentext#1{{\color[rgb]{0,0.7,0.1}#1}}
\def\darkgray{\color[rgb]{0.2,0.2,0.2}}
\def\lightgray{\color[rgb]{0.6,0.6,0.6}}
\def\gray{\color[rgb]{0.4,0.4,0.4}}
\def\blue{\color{blue}}
\def\red{\color{red}}
\def\green{\color{green}}
\def\darkgreen{\color[rgb]{0,0.4,0.1}}
\def\darkblue{\color[rgb]{0,0.2,0.6}}
\def\skyblue{\color[rgb]{0.2,0.7,1.}}
%%control
\def\be{\begin{equation}}
\def\ee{\nonumber\end{equation}}
\def\bea{\begin{eqnarray}}
\def\eea{\nonumber\end{eqnarray}}
\def\bch{\begin{CJK}{UTF8}{gbsn}}
\def\ech{\end{CJK}}
\def\bitem{\begin{itemize}}
\def\eitem{\end{itemize}}
\def\bcenter{\begin{center}}
\def\ecenter{\end{center}}
\def\bex{\begin{minipage}{0.2\textwidth}\includegraphics[width=0.6in]{jugelizi.png}\end{minipage}\begin{minipage}{0.76\textwidth}}
\def\eex{\end{minipage}}
\def\chtitle#1{\frametitle{\bch#1\ech}}
\def\bmat#1{\left(\begin{array}{#1}}
\def\emat{\end{array}\right)}
\def\bcase#1{\left\{\begin{array}{#1}}
\def\ecase{\end{array}\right.}
\def\bmini#1{\begin{minipage}{#1\textwidth}}
\def\emini{\end{minipage}}
\def\tbox#1{\begin{tcolorbox}#1\end{tcolorbox}}
\def\pfrac#1#2#3{\left(\frac{\partial #1}{\partial #2}\right)_{#3}}
%%symbols
\def\bropt{\,(\ \ \ )}
\def\sone{$\star$}
\def\stwo{$\star\star$}
\def\sthree{$\star\star\star$}
\def\sfour{$\star\star\star\star$}
\def\sfive{$\star\star\star\star\star$}
\def\rint{{\int_\leftrightarrow}}
\def\roint{{\oint_\leftrightarrow}}
\def\stdHf{{\textit{\r H}_f}}
\def\deltaH{{\Delta \textit{\r H}}}
\def\ii{{\dot{\imath}}}
\def\skipline{{\vskip0.1in}}
\def\skiplines{{\vskip0.2in}}
\def\lagr{{\mathcal{L}}}
\def\hamil{{\mathcal{H}}}
\def\vecv{{\mathbf{v}}}
\def\vecx{{\mathbf{x}}}
\def\vecy{{\mathbf{y}}}
\def\veck{{\mathbf{k}}}
\def\vecp{{\mathbf{p}}}
\def\vecn{{\mathbf{n}}}
\def\vecA{{\mathbf{A}}}
\def\vecP{{\mathbf{P}}}
\def\vecsigma{{\mathbf{\sigma}}}
\def\hatJn{{\hat{J_\vecn}}}
\def\hatJx{{\hat{J_x}}}
\def\hatJy{{\hat{J_y}}}
\def\hatJz{{\hat{J_z}}}
\def\hatj#1{\hat{J_{#1}}}
\def\hatphi{{\hat{\phi}}}
\def\hatq{{\hat{q}}}
\def\hatpi{{\hat{\pi}}}
\def\vel{\upsilon}
\def\Dint{{\mathcal{D}}}
\def\adag{{\hat{a}^\dagger}}
\def\bdag{{\hat{b}^\dagger}}
\def\cdag{{\hat{c}^\dagger}}
\def\ddag{{\hat{d}^\dagger}}
\def\hata{{\hat{a}}}
\def\hatb{{\hat{b}}}
\def\hatc{{\hat{c}}}
\def\hatd{{\hat{d}}}
\def\hatN{{\hat{N}}}
\def\hatH{{\hat{H}}}
\def\hatp{{\hat{p}}}
\def\Fup{{F^{\mu\nu}}}
\def\Fdown{{F_{\mu\nu}}}
\def\newl{\nonumber \\}
\def\vece{\mathrm{e}}
\def\calM{{\mathcal{M}}}
\def\calT{{\mathcal{T}}}
\def\calR{{\mathcal{R}}}
\def\barpsi{\bar{\psi}}
\def\baru{\bar{u}}
\def\barv{\bar{\upsilon}}
\def\qeq{\stackrel{?}{=}}
\def\torder#1{\mathcal{T}\left(#1\right)}
\def\rorder#1{\mathcal{R}\left(#1\right)}
\def\contr#1#2{\contraction{}{#1}{}{#2}#1#2}
\def\trof#1{\mathrm{Tr}\left(#1\right)}
\def\trace{\mathrm{Tr}}
\def\comm#1{\ \ \ \left(\mathrm{used}\ #1\right)}
\def\tcomm#1{\ \ \ (\text{#1})}
\def\slp{\slashed{p}}
\def\slk{\slashed{k}}
\def\calp{{\mathfrak{p}}}
\def\veccalp{\mathbf{\mathfrak{p}}}
\def\Tthree{T_{\tiny \textcircled{3}}}
\def\pthree{p_{\tiny \textcircled{3}}}
\def\dbar{{\,\mathchar'26\mkern-12mu d}}
\def\erf{\mathrm{erf}}
\def\const{\mathrm{constant}}
\def\pheat{\pfrac p{\ln T}V}
\def\vheat{\pfrac V{\ln T}p}
%%units
\def\fdeg{{^\circ \mathrm{F}}}
\def\cdeg{^\circ \mathrm{C}}
\def\atm{\,\mathrm{atm}}
\def\angstrom{\,\text{\AA}}
\def\SIL{\,\mathrm{L}}
\def\SIkm{\,\mathrm{km}}
\def\SIyr{\,\mathrm{yr}}
\def\SIGyr{\,\mathrm{Gyr}}
\def\SIV{\,\mathrm{V}}
\def\SImV{\,\mathrm{mV}}
\def\SIeV{\,\mathrm{eV}}
\def\SIkeV{\,\mathrm{keV}}
\def\SIMeV{\,\mathrm{MeV}}
\def\SIGeV{\,\mathrm{GeV}}
\def\SIcal{\,\mathrm{cal}}
\def\SIkcal{\,\mathrm{kcal}}
\def\SImol{\,\mathrm{mol}}
\def\SIN{\,\mathrm{N}}
\def\SIHz{\,\mathrm{Hz}}
\def\SIm{\,\mathrm{m}}
\def\SIcm{\,\mathrm{cm}}
\def\SIfm{\,\mathrm{fm}}
\def\SImm{\,\mathrm{mm}}
\def\SInm{\,\mathrm{nm}}
\def\SImum{\,\mathrm{\mu m}}
\def\SIJ{\,\mathrm{J}}
\def\SIW{\,\mathrm{W}}
\def\SIkJ{\,\mathrm{kJ}}
\def\SIs{\,\mathrm{s}}
\def\SIkg{\,\mathrm{kg}}
\def\SIg{\,\mathrm{g}}
\def\SIK{\,\mathrm{K}}
\def\SImmHg{\,\mathrm{mmHg}}
\def\SIPa{\,\mathrm{Pa}}

\def\courseurl{https://github.com/zqhuang/SYSU\_TD}

\def\tpage#1#2{
\begin{frame}
\begin{center}
\begin{Large}
\bch
热学 \\
第#1讲 #2

{\vskip 0.3in}

黄志琦

\ech
\end{Large}
\end{center}

\vskip 0.2in

\bch
教材:《热学》第二版,赵凯华,罗蔚茵,高等教育出版社
\ech

\bch
课件下载
\ech
\courseurl
\end{frame}
}

\def\bfr#1{
\begin{frame}
\chtitle{#1} 
\bch
}

\def\efr{
\ech 
\end{frame}
}

  \date{}
\begin{document}
\tpage{21}{Problems in Spherical Coordinates}

\begin{frame}
\chtitle{本讲内容}
\bch
\bitem
\item{回顾}
\item{球坐标系的数理方程举例}
\eitem
\ech
\end{frame}

\section{Review}
\secpage{回顾}{解 = $\sum$ 待定系数 $\times $ 谐函数 $\times$ 时间依赖因子\\
  用一般正交定理可以确定系数
}

\begin{frame}
\chtitle{回顾}
\bch
三类问题在无源和满足齐次边界条件时的分离变量形式解:
\bitem
\item[1]{热传导方程:$Q_k(\vecx)e^{-ak^2t}$.}
\item[2]{波动方程:$Q_k(\vecx)\cos{akt}$和$Q_k(\vecx)\sin{akt}$.}
\item[3]{静电问题,静引力势问题等:$Q_0(\vecx)$ ($k=0$,没有时间依赖性)}  
  \eitem
  其中谐函数$Q_k(\vecx)$描述的是对空间坐标的依赖性,它满足方程:
  $$ \nabla^2 Q = -k^2Q.$$
  这里$k\ge 0$具有波矢长度的物理意义。
\bitem
\item{在有限区域内,波矢往往只能取一些离散的分布的矢量才能保证有满足边界条件的解(于是$k$也只能取一些离散的值)。}
\item{在无限区域内,波矢一般可以连续取值。这时分离变量法转变为积分变换的方法。}
  \eitem
  \ech
\end{frame}


\secpage{直角坐标系}{直角坐标系的谐函数为平面波$ e^{\ii \veck \cdot \vecx} =e^{\ii (k_1x+k_2y+k_3z)}$。根据问题的具体形式,可以灵活地把指数函数$e^{\pm \ii k_1x}$替换为三角函数$\sin k_1x$和$\cos k_1x$的线性组合等。}

\begin{frame}
\chtitle{直角坐标系}
\bch


例如,在一个$x,y,z$方向长度分别为$L_1,L_2, L_3$的箱子里的数理方程,如果取零边界条件,则谐函数的形式只能是:
$$ Q(x,y,z) = \sin{\frac{n_1\pi x}{L_1}}\sin{\frac{n_2\pi y}{L_2}}\sin{\frac{n_3\pi z}{L_3}},  $$
$$ 0\le x\le L_1, 0\le y\le L_2, 0\le z\le L_3; n_1, n_2, n_3\in Z.$$
可以看出三维的波矢为$\veck = \left(\frac{n_1\pi}{L_1}, \frac{n_2\pi}{L_2}, \frac{n_3\pi}{L_3}\right)$,允许的$k$只能取
$$k=|\veck|=\pi\sqrt{\frac{n_1^2}{L_1^2}+\frac{n_2^2}{L_2^2}+\frac{n_3^2}{L_3^2}}.$$
\ech
\end{frame}



\begin{frame}
\chtitle{谐函数按空间坐标的分解}
\bch
三维的谐函数可以看成考虑三个一维坐标$x$,$y$,$z$的谐函数 $\sin{\frac{n_1\pi x}{L_1}}$, $\sin{\frac{n_2\pi y}{L_2}}$, $\sin{\frac{n_3\pi z}{L_3}}$的乘积。
每个谐函数分别对应波矢$k_1 = \frac{n_1\pi}{L_1}$, $k_2=\frac{n_2\pi}{L_2}$, $k_3=\frac{n_3\pi}{L_3}$。

显然有:
$$k_1^2+k_2^2+k_3^2 = k^2.$$

另一种观点是把三维的谐函数看成二维的谐函数$\sin{\frac{n_1\pi x}{L_1}}\sin{\frac{n_2\pi y}{L_2}}$和一维$z$坐标的谐函数$\sin{\frac{n_3\pi z}{L_3}}$的乘积,前者对应一个二维的波矢$\veck_{2D} = (k_1,k_2) = \left(\frac{n_1\pi }{L_1}, \frac{n_2\pi }{L_2}\right)$,后者对应一个波矢$k_3 = \frac{n_3\pi }{L_3}$,显然有
$$ |\veck_{2D}|^2 + k_3^2 = k^2 .$$
\ech
\end{frame}


\secpage{柱坐标系}{柱坐标系的谐函数为$J_m(k_{2D}r)e^{\ii m\theta}e^{\ii k_3 z}, \ m\in Z$。
 {\small 视具体情况可以把$e^{\pm \ii m\theta}$换成$\cos{m\theta}$和$\sin{m\theta}$的线性组合;把$e^{\pm \ii k_3 z}$换成$\cos{k_3z}$和$\sin{k_3z}$的线性组合。}
  对空心柱坐标系,$J_m$要替换为$J_m$和$Y_m$的任意线性组合。}


\begin{frame}
\chtitle{柱坐标谐函数按空间坐标的分解}
\bch
三维柱坐标的谐函数看成二维的谐函数$J_m(k_{2D}r)e^{\ii m\theta}e^{\ii k_3 z}$和一维$z$坐标的谐函数$e^{\pm \ii k_3 z}$的乘积,前者对应一个大小为$k_{2D}$的二维波矢,后者对应一个波矢$k_3$,显然有
$$ k_{2D}^2 + k_3^2 = k^2 .$$

\skiplines

极坐标的波矢能否继续分解呢?它的分解是动态的:
$$k_{2D}^2 = \frac{m^2}{r^2}+k_r^2,$$
其中$k_r$是径向的波矢,$\frac{m}{r}$是切向(垂直半径方向)的波矢。

{\small \darkgreen (正因为波矢的分解依赖于$r$,径向的谐函数$J_m(k_{2D}r)$同时依赖于$k_{2D}$和$m$,而不是简单地只依赖于径向的波矢$k_r$。)}
\ech
\end{frame}


\secpage{球坐标系}{球坐标系的谐函数为$Y_{\ell m}(\theta,\phi)j_\ell(kr)$。\\
  如果是空心球,则要把$j_\ell(kr)$换成$j_\ell(kr)$和$y_\ell(kr)$的任意线性组合。\\
  球坐标系$k=0$极限要特别注意,对应的两个解是$r^{\ell}Y_{\ell m}(\theta,\phi)$(在无穷远处发散)和$r^{-\ell-1}Y_{\ell m}(\theta,\phi)$ (在球心发散).}


\begin{frame}
\chtitle{球坐标谐函数按空间坐标的分解}
\bch
球坐标系的波矢分解形式为
$$ k^2 = k_r^2 + \frac{\ell(\ell+1)}{r^2}$$
其中$k_r$是径向的波矢,$\frac{\ell(\ell+1)}{r^2}$是固定半径为$r$的球面上的波矢大小。

\skiplines

球面上的波矢可以进一步进行分解为:
$$ \frac{\ell(\ell+1)}{r^2} = \frac{m^2}{r^2\sin^2\theta} + k_\theta^2. $$
其中$k_\theta$是经线方向的波矢,$\frac{m}{r\sin\theta}$是纬线方向的波矢。

{\small \darkgreen $\phi$的谐函数$e^{\ii m\phi}$只依赖于$m$,$\theta$谐函数同时依赖于$\ell, m$,径向的谐函数同时依赖于$k$和$\ell$:这些复杂性同样是由于波矢分解的坐标依赖性。}
\ech
\end{frame}


\begin{frame}
\chtitle{球坐标谐函数按空间坐标的分解}
\bch
常用的函数有必要熟悉一下:
\bea
j_0(x) &=& \frac{\sin x}{x}, \newl
j_1(x) &=& \frac{\sin x - x\cos x}{x^2}; \newl
y_0(x) &=& -\frac{\cos x}{x}, \newl
y_1(x) &=& -\frac{\cos x + x\sin x}{x^2}; \newl
Y_{00}(\theta,\phi)&=& \frac{1}{\sqrt{4\pi}}, \newl
Y_{10}(\theta,\phi) &=& \sqrt{\frac{3}{8\pi}}\cos\theta, \newl
Y_{1,\pm 1}(\theta,\phi) &=& \mp \sqrt{\frac{3}{8\pi}}\sin\theta e^{\pm \ii\phi}.
\eea
\ech
\end{frame}

\section{Equations in Spherical Coordinates}
\secpage{球坐标中的数理方程}{先考虑对称性很重要}

\begin{frame}
\chtitle{例题1}
\bch
\addfig{1.5}{sphereheat.png}

把一个半径为$R$,温度为$T_0$的均匀实心金属球放到温度为$T_1>T_0$的热库中。已知金属球的导热系数为$\lambda$, 质量密度为$\rho$,单位质量比热为$c$。问金属球的球心多久之后温度可以达到$\frac{T_0+T_1}{2}$。
\ech
\end{frame}


\begin{frame}
\chtitle{例题1解答}
\bch
显然稳恒态温度为$T_1$。令$u(r)=T(r)-T_1,\ 0\le r\le R$,则
\bea
\frac{\partial u}{\partial t} - a\nabla^2 u &=& 0, \newl
\left. u\right\vert_{r=R} &=& 0 , \newl
\left. u\right\vert_{t=0} &=& T_0-T_1. \newl
\eea
其中$a=\frac{\lambda}{\rho c}$。 根据问题对称性只能取$\ell = m = 0$的谐函数:$j_0(kr) = \frac{\sin kr}{kr}$。又根据零边界条件,可以得到
$$k = \frac{n\pi}{R}, n = 1,2,\ldots $$
所以设
$$ u(r,t) = \sum_{n=1}^\infty c_n\frac{\sin \frac{n\pi r}{R}}{\frac{n\pi r}{R}} e^{-\frac{n^2\pi^2 at}{R^2}}.$$ 
\ech
\end{frame}


\begin{frame}
\chtitle{例题1解答}
\bch
令$t=0$,得到

$$ \sum_{n=1}^\infty c_n\frac{\sin \frac{n\pi r}{R}}{\frac{n\pi r}{R}} = T_0-T_1,\ \   0\le r\le R$$

利用一般正交定理(或者直接观察)可以写出

$$ c_n =  \frac{2n\pi(T_0-T_1)}{R^2}\int_0^R  \left(\sin \frac{n\pi r}{R}\right) r dr = 2(-1)^{n+1}(T_0-T_1) . $$
当$r=0$时,
$$ u(0,t) = 2(T_0-T_1)\sum_{n=1}^\infty (-1)^{n+1} e^{-\frac{n^2\pi^2 at}{R^2}} $$
\ech
\end{frame}


\begin{frame}
\chtitle{例题1解答}
\bch
当$T(0,t)$达到$\frac{T_0+T_1}{2}$,也就是当$u(0,t)$达到$\frac{T_0-T_1}{2}$时,
$$\sum_{n=1}^\infty (-1)^{n+1} e^{-\frac{n^2\pi^2 at}{R^2}} = \frac{1}{4}.$$
适当采用近似的方法容易算出:
$$ t \approx \frac{R^2}{\pi^2a}\left(2\ln 2 - \frac{1}{4^3}\right) = 0.1389 \frac{R^2}{a} $$
\ech
\end{frame}



\begin{frame}
\chtitle{例题2}
\bch

\addfig{1}{think3.jpg}

把一个不带电的半径为$R$的金属球放进场强为$E$的匀强电场中,求金属球表面的电荷密度分布。
\ech
\end{frame}


\begin{frame}
\chtitle{解答}
\bch

感应电荷在球外和球内产生的电势$u$必须满足拉普拉斯方程
$$\nabla^2 u = 0,\ \ \ r\ne R$$
因为总感应电荷为零,显然球心处$u=0$。由此容易看出感应电荷在球内产生的电势为
$$ u(r, \theta) = E r \cos\theta,\ \ \ r<R $$
这恰好是$\ell =1, m = 0, k=0$的球坐标系谐函数。

由电势的连续性以及感应电荷产生的电势在无穷远处为零的条件,得出感应电荷在球外产生的电势为
$$ u(r, \theta) = E \frac{R^3}{r^2}\cos\theta, \ \ \ r>R $$
于是电荷面密度为
$$\sigma =  \epsilon_0\left(\left.\frac{\partial u}{\partial r}\right\vert_{r=R-0}-\left.\frac{\partial u}{\partial r}\right\vert_{r=R+0}\right) = 3\epsilon_0E\cos\theta $$ 
\ech
\end{frame}



\begin{frame}
\chtitle{思考题}
\bch

\addfig{1}{think1.jpg}

例题2中匀强电场造成的电势恰好是$\ell = 1, m = 0, k=0$的谐函数,这是一个巧合。一般情况下怎样求解这类问题?
\ech
\end{frame}




\section{Homework}

\begin{frame}
\chtitle{课后作业(题号49-50)}
\bch
\bitem
\item[49]{一个半径为$R$的孤立均匀金属球,导热系数为$\lambda$, 质量密度为$\rho$,单位质量比热为$c$。记$r$为到球心的距离,一开始$t=0$时刻球内各处的温度为$T(r) = T_0\left(1+\frac{r}{R}\right)$。计算球内各处温度随时间的变化。}
\item[50]{已知某真空区域的电势在直角坐标系下可以写成$ u(x,y,z) = \lambda(x^2-y^2)$, $\lambda$为常量。把一个不带电的半径为$R$的金属球放入该区域,球心和原点重合。计算金属球表面的电荷密度分布。}
  
  \eitem
\ech
\end{frame}

\end{document}
