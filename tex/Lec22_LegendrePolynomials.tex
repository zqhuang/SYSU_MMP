\documentclass[CJK]{beamer}
\usepackage{CJKutf8}
\usepackage{beamerthemesplit}
\usetheme{Malmoe}
\useoutertheme[footline=authortitle]{miniframes}
\usepackage{amsmath}
\usepackage{amssymb}
\usepackage{graphicx}
\usepackage{eufrak}
\usepackage{color}
\usepackage{slashed}
\usepackage{simplewick}
\usepackage{tikz}
\usepackage{tcolorbox}
\graphicspath{{../figures/}}
%%figures
\def\lfig#1#2{\includegraphics[width=#1 in]{#2}}
\def\addfig#1#2{\begin{center}\includegraphics[width=#1 in]{#2}\end{center}}
\def\wulian{\includegraphics[width=0.18in]{emoji_wulian.jpg}}
\def\bigwulian{\includegraphics[width=0.35in]{emoji_wulian.jpg}}
\def\bye{\includegraphics[width=0.18in]{emoji_bye.jpg}}
\def\bigbye{\includegraphics[width=0.35in]{emoji_bye.jpg}}
\def\huaixiao{\includegraphics[width=0.18in]{emoji_huaixiao.jpg}}
\def\bighuaixiao{\includegraphics[width=0.35in]{emoji_huaixiao.jpg}}
\def\jianxiao{\includegraphics[width=0.18in]{emoji_jianxiao.jpg}}
\def\bigjianxiao{\includegraphics[width=0.35in]{emoji_jianxiao.jpg}}
%% colors
\def\blacktext#1{{\color{black}#1}}
\def\bluetext#1{{\color{blue}#1}}
\def\redtext#1{{\color{red}#1}}
\def\darkbluetext#1{{\color[rgb]{0,0.2,0.6}#1}}
\def\skybluetext#1{{\color[rgb]{0.2,0.7,1.}#1}}
\def\cyantext#1{{\color[rgb]{0.,0.5,0.5}#1}}
\def\greentext#1{{\color[rgb]{0,0.7,0.1}#1}}
\def\darkgray{\color[rgb]{0.2,0.2,0.2}}
\def\lightgray{\color[rgb]{0.6,0.6,0.6}}
\def\gray{\color[rgb]{0.4,0.4,0.4}}
\def\blue{\color{blue}}
\def\red{\color{red}}
\def\green{\color{green}}
\def\darkgreen{\color[rgb]{0,0.4,0.1}}
\def\darkblue{\color[rgb]{0,0.2,0.6}}
\def\skyblue{\color[rgb]{0.2,0.7,1.}}
%%control
\def\be{\begin{equation}}
\def\ee{\nonumber\end{equation}}
\def\bea{\begin{eqnarray}}
\def\eea{\nonumber\end{eqnarray}}
\def\bch{\begin{CJK}{UTF8}{gbsn}}
\def\ech{\end{CJK}}
\def\bitem{\begin{itemize}}
\def\eitem{\end{itemize}}
\def\bcenter{\begin{center}}
\def\ecenter{\end{center}}
\def\bex{\begin{minipage}{0.2\textwidth}\includegraphics[width=0.6in]{jugelizi.png}\end{minipage}\begin{minipage}{0.76\textwidth}}
\def\eex{\end{minipage}}
\def\chtitle#1{\frametitle{\bch#1\ech}}
\def\bmat#1{\left(\begin{array}{#1}}
\def\emat{\end{array}\right)}
\def\bcase#1{\left\{\begin{array}{#1}}
\def\ecase{\end{array}\right.}
\def\bmini#1{\begin{minipage}{#1\textwidth}}
\def\emini{\end{minipage}}
\def\tbox#1{\begin{tcolorbox}#1\end{tcolorbox}}
\def\pfrac#1#2#3{\left(\frac{\partial #1}{\partial #2}\right)_{#3}}
%%symbols
\def\bropt{\,(\ \ \ )}
\def\sone{$\star$}
\def\stwo{$\star\star$}
\def\sthree{$\star\star\star$}
\def\sfour{$\star\star\star\star$}
\def\sfive{$\star\star\star\star\star$}
\def\rint{{\int_\leftrightarrow}}
\def\roint{{\oint_\leftrightarrow}}
\def\stdHf{{\textit{\r H}_f}}
\def\deltaH{{\Delta \textit{\r H}}}
\def\ii{{\dot{\imath}}}
\def\skipline{{\vskip0.1in}}
\def\skiplines{{\vskip0.2in}}
\def\lagr{{\mathcal{L}}}
\def\hamil{{\mathcal{H}}}
\def\vecv{{\mathbf{v}}}
\def\vecx{{\mathbf{x}}}
\def\vecy{{\mathbf{y}}}
\def\veck{{\mathbf{k}}}
\def\vecp{{\mathbf{p}}}
\def\vecn{{\mathbf{n}}}
\def\vecA{{\mathbf{A}}}
\def\vecP{{\mathbf{P}}}
\def\vecsigma{{\mathbf{\sigma}}}
\def\hatJn{{\hat{J_\vecn}}}
\def\hatJx{{\hat{J_x}}}
\def\hatJy{{\hat{J_y}}}
\def\hatJz{{\hat{J_z}}}
\def\hatj#1{\hat{J_{#1}}}
\def\hatphi{{\hat{\phi}}}
\def\hatq{{\hat{q}}}
\def\hatpi{{\hat{\pi}}}
\def\vel{\upsilon}
\def\Dint{{\mathcal{D}}}
\def\adag{{\hat{a}^\dagger}}
\def\bdag{{\hat{b}^\dagger}}
\def\cdag{{\hat{c}^\dagger}}
\def\ddag{{\hat{d}^\dagger}}
\def\hata{{\hat{a}}}
\def\hatb{{\hat{b}}}
\def\hatc{{\hat{c}}}
\def\hatd{{\hat{d}}}
\def\hatN{{\hat{N}}}
\def\hatH{{\hat{H}}}
\def\hatp{{\hat{p}}}
\def\Fup{{F^{\mu\nu}}}
\def\Fdown{{F_{\mu\nu}}}
\def\newl{\nonumber \\}
\def\vece{\mathrm{e}}
\def\calM{{\mathcal{M}}}
\def\calT{{\mathcal{T}}}
\def\calR{{\mathcal{R}}}
\def\barpsi{\bar{\psi}}
\def\baru{\bar{u}}
\def\barv{\bar{\upsilon}}
\def\qeq{\stackrel{?}{=}}
\def\torder#1{\mathcal{T}\left(#1\right)}
\def\rorder#1{\mathcal{R}\left(#1\right)}
\def\contr#1#2{\contraction{}{#1}{}{#2}#1#2}
\def\trof#1{\mathrm{Tr}\left(#1\right)}
\def\trace{\mathrm{Tr}}
\def\comm#1{\ \ \ \left(\mathrm{used}\ #1\right)}
\def\tcomm#1{\ \ \ (\text{#1})}
\def\slp{\slashed{p}}
\def\slk{\slashed{k}}
\def\calp{{\mathfrak{p}}}
\def\veccalp{\mathbf{\mathfrak{p}}}
\def\Tthree{T_{\tiny \textcircled{3}}}
\def\pthree{p_{\tiny \textcircled{3}}}
\def\dbar{{\,\mathchar'26\mkern-12mu d}}
\def\erf{\mathrm{erf}}
\def\const{\mathrm{constant}}
\def\pheat{\pfrac p{\ln T}V}
\def\vheat{\pfrac V{\ln T}p}
%%units
\def\fdeg{{^\circ \mathrm{F}}}
\def\cdeg{^\circ \mathrm{C}}
\def\atm{\,\mathrm{atm}}
\def\angstrom{\,\text{\AA}}
\def\SIL{\,\mathrm{L}}
\def\SIkm{\,\mathrm{km}}
\def\SIyr{\,\mathrm{yr}}
\def\SIGyr{\,\mathrm{Gyr}}
\def\SIV{\,\mathrm{V}}
\def\SImV{\,\mathrm{mV}}
\def\SIeV{\,\mathrm{eV}}
\def\SIkeV{\,\mathrm{keV}}
\def\SIMeV{\,\mathrm{MeV}}
\def\SIGeV{\,\mathrm{GeV}}
\def\SIcal{\,\mathrm{cal}}
\def\SIkcal{\,\mathrm{kcal}}
\def\SImol{\,\mathrm{mol}}
\def\SIN{\,\mathrm{N}}
\def\SIHz{\,\mathrm{Hz}}
\def\SIm{\,\mathrm{m}}
\def\SIcm{\,\mathrm{cm}}
\def\SIfm{\,\mathrm{fm}}
\def\SImm{\,\mathrm{mm}}
\def\SInm{\,\mathrm{nm}}
\def\SImum{\,\mathrm{\mu m}}
\def\SIJ{\,\mathrm{J}}
\def\SIW{\,\mathrm{W}}
\def\SIkJ{\,\mathrm{kJ}}
\def\SIs{\,\mathrm{s}}
\def\SIkg{\,\mathrm{kg}}
\def\SIg{\,\mathrm{g}}
\def\SIK{\,\mathrm{K}}
\def\SImmHg{\,\mathrm{mmHg}}
\def\SIPa{\,\mathrm{Pa}}

\def\courseurl{https://github.com/zqhuang/SYSU\_TD}

\def\tpage#1#2{
\begin{frame}
\begin{center}
\begin{Large}
\bch
热学 \\
第#1讲 #2

{\vskip 0.3in}

黄志琦

\ech
\end{Large}
\end{center}

\vskip 0.2in

\bch
教材:《热学》第二版,赵凯华,罗蔚茵,高等教育出版社
\ech

\bch
课件下载
\ech
\courseurl
\end{frame}
}

\def\bfr#1{
\begin{frame}
\chtitle{#1} 
\bch
}

\def\efr{
\ech 
\end{frame}
}

  \date{}
\begin{document}
\tpage{22}{Legendre Polynomials}

\begin{frame}
\chtitle{本讲内容}
\bch
\bitem
\item{回顾:不同坐标系的谐函数之间的关系}
\item{回顾:金属球外点电荷的感应电荷问题}
\item{勒让德多项式}
\eitem
\ech
\end{frame}

\section{Harmonic Functions in different coordinates}
\secpage{不同坐标系的谐函数之间的关系}{对同一$k$,不同坐标系的谐函数无非只是重新线性组合一下}

\begin{frame}
  \chtitle{简并的本征值}
  \bch
  实对称矩阵$A$总是能通过一个正交变换转化为对角矩阵,从而得到它的全部本征值。

  \skiplines
  
  如果某个本征值$\lambda$在结果的对角元里出现了$m$次,则称$\lambda$是矩阵$A$的$m$重简并的本征值,或者说$\lambda$具有$m$重简并度。它对应$m$个线性无关本征矢可以通过标准的正交化手段化为全部互相正交。

  \skiplines
  
  在谐函数理论里,简并是司空见惯的事情。例如下列无穷多个在无边界的二维平面上正交的谐函数
  $$J_m(kr)e^{\ii m \theta}, m=0,\pm 1, \pm 2, \ldots $$
  对应的都是同一个$k$ (这里$k>0$)。也就是说,在(无边界的)二维平面上,算符$-\nabla^2$的本征值$k^2$具有无穷多重简并度。
  
  \ech
\end{frame}



\begin{frame}
  \chtitle{不同的正交方式}
  \bch
  如果$\lambda$是实对称矩阵$A$的$m$重本征值,则$\lambda$的所有本征矢量构成一个$m$维子空间。如果$m>1$,在这个$m$维子空间选取一组正交基的方式是相当随意的:显然有无穷多种方式可供选择。

  \skiplines

  在谐函数理论里也是如此,例如在无边界的二维平面上,我们既可以选取直角坐标系的
  $$e^{\ii \veck\cdot\vecx}$$
  作为一组正交基,其中$\veck$满足$|\veck|=k$;

  也可以选取极坐标系的
  $$J_m(kr)e^{\ii m \theta}, m=0,\pm 1, \pm 2, \ldots $$
  作为一组正交基。
  \ech
\end{frame}

\begin{frame}
  \chtitle{集合论爱好者的疑惑}
  \bch
  这件事从集合论的观点看比较奇怪: 对固定的$k$,满足$|\veck|=k$的$\veck$有不可数个。所以
  $$e^{\ii \veck\cdot\vecx}$$
  这组基包含了不可数个谐函数。

  {\small \darkgreen (如果你从未听说过可数与不可数,请直接忽略本页\wulian)}
  
  而极坐标里的谐函数
  $$J_m(kr)e^{\ii m \theta}, m=0,\pm 1, \pm 2, \ldots $$
  显然是可数个。

    \skipline

  
    不可数个矢量构成的线性空间能和可数个矢量构成的线性空间相同吗?

  \skipline
  \ech
\end{frame}


\begin{frame}
  \chtitle{贝塞尔函数的母函数}
  \bch
  要证明两组谐函数构成的函数空间相同,我们只要证明两组函数可以互相线性表示出来。

  \skipline

  为此,我们先介绍一个贝塞尔函数的母函数公式:

  \tbox{
  $$ e^{\frac{x}{2}(t-\frac{1}{t})} = \sum_{n=-\infty}^\infty J_n(x) t^n.   $$}
  \ech
\end{frame}


\begin{frame}
  \chtitle{贝塞尔函数的母函数}
  \bch
  证明:
  \bea
  e^{\frac{x}{2}(t-\frac{1}{t})} &=& e^{\frac{xt}{2}} e^{-\frac{x}{2t}} \newl
  &=&\sum_{m=0}^\infty \frac{1}{m!}\left(\frac{xt}{2}\right)^m\sum_{k=0}^\infty \frac{1}{k!}\left(-\frac{x}{2t}\right)^k \newl
  &=& \sum_{n=-\infty}^\infty \left(\sum_{k=0}\frac{(-1)^k}{k!(n+k)!}\left(\frac{x}{2}\right)^{2k+n}\right) t^n \newl
  &=& \sum_{n=-\infty}^\infty J_n(x) t^n
  \eea
  \ech
\end{frame}


\begin{frame}
  \chtitle{母函数和贝塞尔函数的积分表达的关系}
  \bch
  在母函数公式中令$t = e^{\ii \theta}$,即得到
    $$ e^{\ii x\sin\theta} = \sum_{n=-\infty}^\infty J_n(x) e^{\ii n\theta}$$
  两边乘以$e^{-\ii m\theta}$并从$-\pi$积分至$\pi$,得到我们曾经学习过的贝塞尔函数的积分表达
  \tbox{
  $$ J_m(x) =  \frac{1}{2\pi}\int_{-\pi}^\pi e^{\ii(x\sin\theta - m\theta)} d\theta .$$}
  \ech
\end{frame}



\begin{frame}
  \chtitle{把平面波分解成极坐标的谐函数}
  \bch
  在母函数公式中令$t = \ii e^{\ii \theta} $,$x=kr$即得到
$$ e^{\ii kr\cos\theta} = \sum_{n=-\infty}^\infty \ii^nJ_n(kr) e^{\ii n\theta}$$
把等式左边的$\theta$理解为$\veck$和$\vecx$的幅角差,$k$理解为$|\veck|$, $r$理解为$|\vecx|$,那么等式左边就是一个平面波$e^{\ii \veck\cdot\vecx}$。等式右边,显然是极坐标下的谐函数的线性组合。

\skiplines

但是,这样的写法还有些小小的bug:好像我们必须把$\theta=0$的方向取在$\veck$的方向?

  \ech
\end{frame}



\begin{frame}
  \chtitle{把平面波分解成极坐标的谐函数}
  \bch
  这个bug很容易修好:令$\veck$的幅角为$\theta_k$,$\vecx$的幅角为$\theta_x$,并令$\theta=\theta_x-\theta_k$。
  于是
  $$ e^{\ii \veck\cdot\vecx} = e^{\ii k r\cos(\theta_x-\theta_k)} = \sum_{n=-\infty}^\infty (\ii^ne^{-\ii n\theta_k}) J_n(kr)e^{\ii n\theta_x}$$
  终于,任意的平面波都被分解为极坐标的谐函数的线性组合。

  
  \ech
\end{frame}



\begin{frame}
  \chtitle{反过来的分解可以用傅立叶变换的知识简单搞定}
  \bch
  反过来,假设给你的函数是$J_m(\mu r)e^{\ii m\theta}$ ($\mu >0$为给定常数),如何把它分解为平面波的线性叠加呢?

  \skiplines
  
  ``分解为平面波的线性叠加''其实是傅立叶变换的另一种说法。因为如果你假设了如下的线性分解
  $$J_m(\mu r)e^{\ii m\theta} = \frac{1}{2\pi} \int e^{\ii \veck \cdot\vecx}f(\veck) d^2\veck $$
  那么$f(\veck)$是$J_m(\mu r)e^{\ii m\theta}$的二维傅立叶变换。它的存在性毋庸置疑。把它化简的工作也并不难,留为作业。

  \ech
\end{frame}




\section{Point Charge Problem}

\begin{frame}
  \chtitle{回顾把金属球放进匀强电场的问题}
  \bch
  我们设金属球表面的感应电荷产生的电势为$u$,{\bf 在球内部},要满足静电平衡:
  $$ u - Er\cos\theta =  \const $$
  其中$-Er\cos\theta$是匀强电场产生的电势。

  \skiplines
  
  等式右边的常数($\const$)怎么确定呢?
  

  \bitem
\item{如果金属球是接地的,则等式右边直接为零。}    
\item{如果金属球是孤立的,则因感应电荷的总量为零,所以在球心处的$u=0$。由此可以确定等式右边常数。}
  \eitem

  在这个具体的例子里,在球内部
  $$ u - Er\cos\theta =  0. $$  
  
  \ech
\end{frame}



\begin{frame}
\chtitle{球外的解}
\bch
  内部解$u=Er\cos\theta$就是$\ell = 1, m=0, k=0$的谐函数。

  \skipline
  
  在球外,$u$一般性地可以设为
  $$u(r, \theta,\phi)= \sum_{\ell,m} c_{\ell m}\left(\frac{r}{R}\right)^{-\ell-1} Y_{\ell m}(\theta,\phi).$$

{\darkgreen \small (类似$r^\ell Y_{\ell m}$的谐函数因在无穷远处发散而被抛弃)}


然后根据电势$u$的连续性(请思考它为什么连续),把球内部解和外部解衔接起来:
$$ \sum_{\ell,m} c_{\ell m} Y_{\ell m}(\theta,\phi)  = ER\cos\theta $$
等式右边正比于$Y_{10}$,由$Y_{\ell m}$的正交性知道等式左边也只能是$Y_{10}$项的系数非零。
\ech
\end{frame}


\begin{frame}
\chtitle{最后的解}
\bch
在$r>R$时,
$$ u = c_{10}\left(\frac{r}{R}\right)^{-2}Y_{10}(\theta,\phi) = ER \left(\frac{r}{R}\right)^{-2}\cos\theta. $$

最后,利用内外的法向感应电场差可以求出感应电荷的面密度,这属于电磁学最基本的高斯定理运用技巧。

\ech
\end{frame}





\begin{frame}
\chtitle{金属球感应电势总结套路}
\bch
\addfig{1}{taolu.jpg}
\bitem
\item{把外电场造成的电势在金属球内部写成一个或多个$k=0$的谐函数。}
\item{利用金属球静电平衡后内部等势的条件,得到感应电荷在球内部产生的电势。}
\item{把感应电荷在球内部产生的电势中每一个谐函数的$\left(\frac{r}{R}\right)^\ell$换成$\left(\frac{r}{R}\right)^{-\ell-1}$,就得到球外部解。}  
\eitem
\ech
\end{frame}

\begin{frame}
\chtitle{金属球外的点电荷}
\bch
金属球外距离球心$a$处的点电荷$Q$造成的电势是一个熟悉的距离反比电势。
\bmini{0.5}
\lfig{2}{potential_inside.png}
\emini
\bmini{0.45}
如图建立球坐标系。设感应电荷产生的电势为$u(r, \theta)$ (由轴对称性很容易看出$u$不依赖于$\phi$)。则在星号位置处的总电势为:
\emini
$$U_{\rm total} = u(r, \theta) +\frac{Q}{4\pi\epsilon_0 \sqrt{a^2+r^2-2ar\cos\theta}}. $$
\ech
\end{frame}

\begin{frame}
\chtitle{勒让德函数的母函数定理}
\bch
套路当然是要把$\frac{Q}{4\pi\epsilon_0 \sqrt{a^2+r^2-2ar\cos\theta}}$写成一堆球坐标系谐函数的和。这要用到下列{\blue 勒让德多项式的母函数定理:
  $$ \frac{1}{\sqrt{1+t^2-2xt}} = \sum_{\ell =0}^\infty P_\ell (x) t^\ell,\ \ \ t<|x\pm \sqrt{x^2-1}|  .$$}
其中的勒让德多项式$P_\ell(x)$是一个$\ell$次多项式,它和$m=0$的球谐函数的关系为{\blue 
$$P_\ell(\cos\theta) = \sqrt{\frac{4\pi}{2\ell + 1}}Y_{\ell 0}(\theta,\phi).$$}
\ech
\end{frame}


\begin{frame}
\chtitle{球外点电荷感应电势问题的解}
\bch
\bea
\frac{Q}{4\pi\epsilon_0 \sqrt{a^2+r^2-2ar\cos\theta}} &=& \frac{Q}{4\pi\epsilon_0 a \sqrt{1+ \left(\frac{r}{a}\right)^2-2\frac{r}{a}\cos\theta}} \newl
&=& \frac{Q}{4\pi\epsilon_0a}\sum_{\ell= 0}^\infty P_{\ell}(\cos\theta)\left(\frac{r}{a}\right)^\ell .
\eea
要求球内部总电势处处相等,则$\ell >0 $的项必须全部被$u$抵消。又根据球心处$u=0$,可以确定常数项为零:
$$\left. u(r, \theta)\right\vert_{r\le R} = - \frac{Q}{4\pi\epsilon_0a}\sum_{\ell= 1}^\infty P_{\ell}(\cos\theta)\left(\frac{r}{a}\right)^\ell . $$
\ech
\end{frame}


\begin{frame}
\chtitle{球外点电荷感应电势问题的解}
\bch
根据套路,在球外的解只要作个替换$r^\ell \rightarrow \frac{R^{2\ell +1}}{r^{\ell+1}}$:

$$\left. u(r, \theta)\right\vert_{r>R} = - \frac{Q}{4\pi\epsilon_0a}\sum_{\ell= 1}^\infty P_{\ell}(\cos\theta)\frac{R^{2\ell+1}}{a^\ell r^{\ell +1}}. $$

用镜像电荷(因为是电磁学内容,和数理方程关系不大,不再详细讲解\bye)得到的表达式是有限的。这里的结果是级数展开。有兴趣的话可以尝试证明两者互相等价(不难\bye)。
\ech
\end{frame}


\section{Legendre Polynomials}

\begin{frame}
\chtitle{勒让德多项式的母函数定理的证明}
\bch
{\small
\bea
\frac{1}{\sqrt{1+t^2-2xt}} &=&\frac{1}{(1-t)\sqrt{1-\frac{2(x-1)t}{(1-t)^2}}} \newl
&=& \frac{1}{1-t}\sum_{k=0}^\infty \bral -\frac{1}{2}\\ k \brar \left(-\frac{2(x-1)t}{(1-t)^2}\right)^k \newl
&=& \sum_{k=0}^\infty \frac{(2k)!}{2^k(k!)^2}(x-1)^kt^k(1-t)^{-2k-1}\newl
&=& \sum_{k=0}^\infty \frac{(2k)!}{2^k(k!)^2}(x-1)^kt^k\sum_{n=0}^\infty \bral -2k-1 \\ n \brar (-t)^n\newl
&=& \sum_{k=0}^\infty\sum_{n=0}^\infty \frac{(2k+n)!}{2^k(k!)^2n!}(x-1)^k  t^{n+k}\newl
&=& \sum_{\ell=0}^\infty\sum_{k=0}^\ell \frac{(\ell + k)!}{(k!)^2(\ell-k)!}\left(\frac{x-1}{2}\right)^k  t^\ell
\eea
}
\ech
\end{frame}



\begin{frame}
\chtitle{勒让德多项式的母函数定理的证明(续)}
\bch
定义{\blue 勒让德多项式
  $$ P_{\ell} = \sum_{k=0}^\ell \frac{(\ell + k)!}{(k!)^2(\ell-k)!}\left(\frac{x-1}{2}\right)^k. $$}
显然它是$\ell$次多项式。于是只要证明
\bitem
\item[1]{$P_\ell(\cos\theta)$满足$m=0$的单位球面谐函数方程}
\item[2]{$P_\ell$的归一化满足
$$ \frac{2\ell+1}{4\pi} \int_0^\pi P_\ell(\cos\theta)^2 \sin\theta d\theta\int_0^{2\pi}d\phi = 1 .$$
即{\blue
$$ \int_{-1}^1 \left[P_\ell(x)\right]^2 dx = \frac{2}{2\ell+1}.$$}
}
\eitem
\ech
\end{frame}


\begin{frame}
\chtitle{勒让德多项式的母函数定理的证明(续)}
\bch
回忆$m=0$的谐函数方程为
$$ \frac{1}{\sin\theta}\frac{d}{d\theta}\left(\sin\theta \frac{d}{d\theta}\Psi\right) + \ell(\ell+1) \Psi= 0.$$
令$x = \cos\theta$,则该微分方程等价于:
$$ \frac{d}{dx}\left[(1-x^2)\frac{d}{dx} \Psi\right]+\ell(\ell+1)\Psi=0 .$$
验证$P_\ell(x)$满足该微分方程的过程并不困难,留为作业\bye。

\skipline

剩下的归一化条件,只要用到不同$\ell$对应的勒让德多项式互相正交,就能很容易证明。也留为作业\bye。
\ech
\end{frame}

\begin{frame}
\chtitle{勒让德多项式的递推关系}
\bch
除了母函数定理之外,勒让德多项式最重要的性质是它的递推公式:
\tbox
    {$$ (2\ell+1)xP_\ell(x) = (\ell+1)P_{\ell+1}(x)+\ell P_{\ell-1}(x)$$ }

    (请先思考一下怎么证明)
\ech
\end{frame}

\begin{frame}
\chtitle{勒让德多项式的递推关系证明概要}
\bch
  $$ \frac{1}{\sqrt{1+t^2-2xt}} = \sum_{\ell =0}^\infty P_\ell (x) t^\ell,\ \ \ t<|x\pm \sqrt{x^2-1}|  .$$
两边对$t$求导,得到
$$ \frac{t-x}{(1-2xt+t^2)^{3/2}} =  \sum_{\ell = 1}^\infty \ell P_\ell(x)t^{\ell-1}$$
即
$$(x-t)\sum_{\ell=0}^\infty P_\ell(x)t^\ell  = (1-2xt+t^2)\sum_{\ell=1}^\infty \ell P_\ell(x)t^{\ell-1}$$
两边比较同次项系数即得证。
\ech
\end{frame}

\section{Homework}

\begin{frame}
\chtitle{课后作业(题号51-53)}
\bch
\bitem
\item[51]{在二维平面上,按照课上所讲的$m\in Z, \mu>0$的情形下,计算
  $$J_m(\mu r)e^{\ii m\theta} = \frac{1}{2\pi} \int e^{\ii \veck \cdot\vecx}f(\veck) d^2\veck $$
  所对应的$f(\veck)$。
}
\item[52]{完成课上的证明:
  $$ \frac{d}{dx}\left[(1-x^2)\frac{d}{dx} P_{\ell}(x)\right]+\ell(\ell+1) P_\ell(x)=0 .$$}
\item[53]{利用勒让德多项式的正交性直接证明它的归一化公式:
  $$ \int_{-1}^1 \left[P_\ell(x)\right]^2 dx = \frac{2}{2\ell+1}.$$
  }
  \eitem
\ech
\end{frame}

\end{document}
