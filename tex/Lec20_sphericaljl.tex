\documentclass[CJK]{beamer}
\usepackage{CJKutf8}
\usepackage{beamerthemesplit}
\usetheme{Malmoe}
\useoutertheme[footline=authortitle]{miniframes}
\usepackage{amsmath}
\usepackage{amssymb}
\usepackage{graphicx}
\usepackage{eufrak}
\usepackage{color}
\usepackage{slashed}
\usepackage{simplewick}
\usepackage{tikz}
\usepackage{tcolorbox}
\graphicspath{{../figures/}}
%%figures
\def\lfig#1#2{\includegraphics[width=#1 in]{#2}}
\def\addfig#1#2{\begin{center}\includegraphics[width=#1 in]{#2}\end{center}}
\def\wulian{\includegraphics[width=0.18in]{emoji_wulian.jpg}}
\def\bigwulian{\includegraphics[width=0.35in]{emoji_wulian.jpg}}
\def\bye{\includegraphics[width=0.18in]{emoji_bye.jpg}}
\def\bigbye{\includegraphics[width=0.35in]{emoji_bye.jpg}}
\def\huaixiao{\includegraphics[width=0.18in]{emoji_huaixiao.jpg}}
\def\bighuaixiao{\includegraphics[width=0.35in]{emoji_huaixiao.jpg}}
\def\jianxiao{\includegraphics[width=0.18in]{emoji_jianxiao.jpg}}
\def\bigjianxiao{\includegraphics[width=0.35in]{emoji_jianxiao.jpg}}
%% colors
\def\blacktext#1{{\color{black}#1}}
\def\bluetext#1{{\color{blue}#1}}
\def\redtext#1{{\color{red}#1}}
\def\darkbluetext#1{{\color[rgb]{0,0.2,0.6}#1}}
\def\skybluetext#1{{\color[rgb]{0.2,0.7,1.}#1}}
\def\cyantext#1{{\color[rgb]{0.,0.5,0.5}#1}}
\def\greentext#1{{\color[rgb]{0,0.7,0.1}#1}}
\def\darkgray{\color[rgb]{0.2,0.2,0.2}}
\def\lightgray{\color[rgb]{0.6,0.6,0.6}}
\def\gray{\color[rgb]{0.4,0.4,0.4}}
\def\blue{\color{blue}}
\def\red{\color{red}}
\def\green{\color{green}}
\def\darkgreen{\color[rgb]{0,0.4,0.1}}
\def\darkblue{\color[rgb]{0,0.2,0.6}}
\def\skyblue{\color[rgb]{0.2,0.7,1.}}
%%control
\def\be{\begin{equation}}
\def\ee{\nonumber\end{equation}}
\def\bea{\begin{eqnarray}}
\def\eea{\nonumber\end{eqnarray}}
\def\bch{\begin{CJK}{UTF8}{gbsn}}
\def\ech{\end{CJK}}
\def\bitem{\begin{itemize}}
\def\eitem{\end{itemize}}
\def\bcenter{\begin{center}}
\def\ecenter{\end{center}}
\def\bex{\begin{minipage}{0.2\textwidth}\includegraphics[width=0.6in]{jugelizi.png}\end{minipage}\begin{minipage}{0.76\textwidth}}
\def\eex{\end{minipage}}
\def\chtitle#1{\frametitle{\bch#1\ech}}
\def\bmat#1{\left(\begin{array}{#1}}
\def\emat{\end{array}\right)}
\def\bcase#1{\left\{\begin{array}{#1}}
\def\ecase{\end{array}\right.}
\def\bmini#1{\begin{minipage}{#1\textwidth}}
\def\emini{\end{minipage}}
\def\tbox#1{\begin{tcolorbox}#1\end{tcolorbox}}
\def\pfrac#1#2#3{\left(\frac{\partial #1}{\partial #2}\right)_{#3}}
%%symbols
\def\bropt{\,(\ \ \ )}
\def\sone{$\star$}
\def\stwo{$\star\star$}
\def\sthree{$\star\star\star$}
\def\sfour{$\star\star\star\star$}
\def\sfive{$\star\star\star\star\star$}
\def\rint{{\int_\leftrightarrow}}
\def\roint{{\oint_\leftrightarrow}}
\def\stdHf{{\textit{\r H}_f}}
\def\deltaH{{\Delta \textit{\r H}}}
\def\ii{{\dot{\imath}}}
\def\skipline{{\vskip0.1in}}
\def\skiplines{{\vskip0.2in}}
\def\lagr{{\mathcal{L}}}
\def\hamil{{\mathcal{H}}}
\def\vecv{{\mathbf{v}}}
\def\vecx{{\mathbf{x}}}
\def\vecy{{\mathbf{y}}}
\def\veck{{\mathbf{k}}}
\def\vecp{{\mathbf{p}}}
\def\vecn{{\mathbf{n}}}
\def\vecA{{\mathbf{A}}}
\def\vecP{{\mathbf{P}}}
\def\vecsigma{{\mathbf{\sigma}}}
\def\hatJn{{\hat{J_\vecn}}}
\def\hatJx{{\hat{J_x}}}
\def\hatJy{{\hat{J_y}}}
\def\hatJz{{\hat{J_z}}}
\def\hatj#1{\hat{J_{#1}}}
\def\hatphi{{\hat{\phi}}}
\def\hatq{{\hat{q}}}
\def\hatpi{{\hat{\pi}}}
\def\vel{\upsilon}
\def\Dint{{\mathcal{D}}}
\def\adag{{\hat{a}^\dagger}}
\def\bdag{{\hat{b}^\dagger}}
\def\cdag{{\hat{c}^\dagger}}
\def\ddag{{\hat{d}^\dagger}}
\def\hata{{\hat{a}}}
\def\hatb{{\hat{b}}}
\def\hatc{{\hat{c}}}
\def\hatd{{\hat{d}}}
\def\hatN{{\hat{N}}}
\def\hatH{{\hat{H}}}
\def\hatp{{\hat{p}}}
\def\Fup{{F^{\mu\nu}}}
\def\Fdown{{F_{\mu\nu}}}
\def\newl{\nonumber \\}
\def\vece{\mathrm{e}}
\def\calM{{\mathcal{M}}}
\def\calT{{\mathcal{T}}}
\def\calR{{\mathcal{R}}}
\def\barpsi{\bar{\psi}}
\def\baru{\bar{u}}
\def\barv{\bar{\upsilon}}
\def\qeq{\stackrel{?}{=}}
\def\torder#1{\mathcal{T}\left(#1\right)}
\def\rorder#1{\mathcal{R}\left(#1\right)}
\def\contr#1#2{\contraction{}{#1}{}{#2}#1#2}
\def\trof#1{\mathrm{Tr}\left(#1\right)}
\def\trace{\mathrm{Tr}}
\def\comm#1{\ \ \ \left(\mathrm{used}\ #1\right)}
\def\tcomm#1{\ \ \ (\text{#1})}
\def\slp{\slashed{p}}
\def\slk{\slashed{k}}
\def\calp{{\mathfrak{p}}}
\def\veccalp{\mathbf{\mathfrak{p}}}
\def\Tthree{T_{\tiny \textcircled{3}}}
\def\pthree{p_{\tiny \textcircled{3}}}
\def\dbar{{\,\mathchar'26\mkern-12mu d}}
\def\erf{\mathrm{erf}}
\def\const{\mathrm{constant}}
\def\pheat{\pfrac p{\ln T}V}
\def\vheat{\pfrac V{\ln T}p}
%%units
\def\fdeg{{^\circ \mathrm{F}}}
\def\cdeg{^\circ \mathrm{C}}
\def\atm{\,\mathrm{atm}}
\def\angstrom{\,\text{\AA}}
\def\SIL{\,\mathrm{L}}
\def\SIkm{\,\mathrm{km}}
\def\SIyr{\,\mathrm{yr}}
\def\SIGyr{\,\mathrm{Gyr}}
\def\SIV{\,\mathrm{V}}
\def\SImV{\,\mathrm{mV}}
\def\SIeV{\,\mathrm{eV}}
\def\SIkeV{\,\mathrm{keV}}
\def\SIMeV{\,\mathrm{MeV}}
\def\SIGeV{\,\mathrm{GeV}}
\def\SIcal{\,\mathrm{cal}}
\def\SIkcal{\,\mathrm{kcal}}
\def\SImol{\,\mathrm{mol}}
\def\SIN{\,\mathrm{N}}
\def\SIHz{\,\mathrm{Hz}}
\def\SIm{\,\mathrm{m}}
\def\SIcm{\,\mathrm{cm}}
\def\SIfm{\,\mathrm{fm}}
\def\SImm{\,\mathrm{mm}}
\def\SInm{\,\mathrm{nm}}
\def\SImum{\,\mathrm{\mu m}}
\def\SIJ{\,\mathrm{J}}
\def\SIW{\,\mathrm{W}}
\def\SIkJ{\,\mathrm{kJ}}
\def\SIs{\,\mathrm{s}}
\def\SIkg{\,\mathrm{kg}}
\def\SIg{\,\mathrm{g}}
\def\SIK{\,\mathrm{K}}
\def\SImmHg{\,\mathrm{mmHg}}
\def\SIPa{\,\mathrm{Pa}}

\def\courseurl{https://github.com/zqhuang/SYSU\_TD}

\def\tpage#1#2{
\begin{frame}
\begin{center}
\begin{Large}
\bch
热学 \\
第#1讲 #2

{\vskip 0.3in}

黄志琦

\ech
\end{Large}
\end{center}

\vskip 0.2in

\bch
教材:《热学》第二版,赵凯华,罗蔚茵,高等教育出版社
\ech

\bch
课件下载
\ech
\courseurl
\end{frame}
}

\def\bfr#1{
\begin{frame}
\chtitle{#1} 
\bch
}

\def\efr{
\ech 
\end{frame}
}

  \date{}
\begin{document}
\tpage{20}{Harmonics in Spherical Coordinates}

\begin{frame}
\chtitle{本讲内容}
\bch
\bitem
\item{球谐函数的推导回顾}
\item{球坐标系的谐函数}
\eitem
\ech
\end{frame}


\section{Derivation}
\secpage{球谐函数的推导回顾}{上一讲好像有一个严重bug……}

\begin{frame}
  \chtitle{回顾}
  \bch
  回顾:我们上一讲把单位球面上的谐函数分解为$\Psi(\theta)\Lambda(\phi)$,并按照分离变量法的套路很快得出$\Lambda(\phi) = e^{\ii m\phi}, \ m\in Z$。剩下$\Psi(\theta)$的方程由微分方程
  $$ \frac{1}{\sin\theta}\frac{d}{d\theta}\left(\Psi'\sin\theta \right) +\left(k^2-\frac{m^2}{\sin^2\theta}\right) \Psi = 0. $$
  或者等价的
  $$ \Psi''+\cot\theta \,\Psi' +\left(k^2-\frac{m^2}{\sin^2\theta}\right) \Psi = 0. $$
  确定。

  \skipline
  
  求解$\Psi(\theta)$是一个非常典型的数学方法(虽然从物理实用角度讲并不十分重要),很多特殊函数都能用类似的方法确定。
  
  \ech
\end{frame}


\begin{frame}
  \chtitle{令人困惑的分解}
  \bch
  由$[0,\pi]$上$\cos{(n\theta)}, \sin{(n\theta)}$的完备性可以把$\Psi$分解为这些三角函数的线性组合。

  \skipline
  
  但是在上一讲的操作:{\bf 仅仅凭边界条件$\Psi' = 0$就排除$\sin{(n\theta)}$类的项是有bug的}。

  \skipline
  
  (在分离变量法确定谐函数的$k$时我们曾不止一次这样做,思考和现在的情况有什么区别。)

  
  \ech
\end{frame}


\begin{frame}
  \chtitle{最安全的做法}
  \bch
  令
  $$ \Psi = \sum_{n=0}^\infty c_n\cos^n\theta + \sin\theta \sum_{n=0}^\infty d_n\cos^n\theta. $$
  (思考:这样为什么就安全了?)
  在后面的推导中,我不再详细地追踪$n$的下界。(只需要默认$c_{-1}=c_{-2}=\ldots =0 $, $d_{-1}=d_{-2}=\ldots = 0$就不会有任何理解困难)。
  \ech
\end{frame}


\begin{frame}
  \chtitle{开始推土}
  \bch
      {\scriptsize
        \bea
        \Psi &=& \sum_{n=0}^\infty c_n\cos^n\theta + \sin\theta \sum_{n=0}^\infty d_n\cos^n\theta; \newl
  \Psi' &=& \sum_n (-nc_n)\cos^{n-1}\theta\, \sin\theta + \sum_n d_n\cos^{n+1}\theta - \sum_n nd_n \cos^{n-1}\theta \sin^2\theta \newl
  &=& -\sum_n (n+1)c_{n+1}\cos^n\theta\, \sin\theta + \sum_n d_{n-1}\cos^n\theta - \sum_n nd_n\left(\cos^{n-1}\theta -\cos^{n+1}\theta\right) \newl
  &=& -\sum_n (n+1)c_{n+1}\cos^n\theta\, \sin\theta + \sum_n\left[ nd_{n-1} - (n+1)d_{n+1}\right]\cos^n\theta ; \newl
  \Psi'' &=& \sum_n\left\{ n(n+1)c_{n+1}\cos^{n-1}\theta\sin^2\theta -(n+1)c_{n+1}\cos^{n+1}\theta \right. \newl
  && \left. - n\sin\theta\left[ nd_{n-1} - (n+1)d_{n+1}\right]\cos^{n-1}\theta \right\}\newl
  &=& \sum_n\left\{(n+1)(n+2)c_{n+2} - n^2c_n + \sin\theta\left[ (n+1)(n+2)d_{n+2}-(n+1)^2d_n\right]\right\}\cos^n\theta
  \eea
  }
  \ech
\end{frame}

\begin{frame}
  \chtitle{继续推土}
  \bch
      {\scriptsize
$$ \Psi'\cos\theta = -\sum_n nc_n\sin\theta\cos^n\theta + \sum_n\left[ (n-1)d_{n-2} - nd_n\right]\cos^n\theta $$ 

如果$m=0$,则由
$$\Psi'\cos\theta+\sin\theta(\Psi''+k^2\Psi) = 0.$$
可以得到上式左边$\sin\theta\cos^n\theta$项的系数:
$$ (n+1)(n+2)c_{n+2}+\left[k^2-n(n+1)\right]c_n = 0$$
这和我们之前得到的递推式是一样的,用同样的方式可以证明$k^2=\ell(\ell+1), \ell = 0,1,2\ldots$

  }
  \ech
\end{frame}


\begin{frame}
  \chtitle{思考题}
  \bch
  \addfig{1}{think1.jpg}
  
  我们只验证了$\sin\theta \cos^n\theta$的系数为零,还需要考虑$\cos^n\theta$的系数,这会给出什么结果?
  \ech
\end{frame}


\section{3D-Sphere Harmonic}

\secpage{球坐标下的谐函数}{解球坐标系数理方程的必备工具}

\begin{frame}
  \chtitle{球坐标系的谐函数}
  \bch
  把自由度$r$重新纳入之后,令谐函数为$Q = f(r)Y_{\ell m}(\theta,\phi)$,则由$\nabla^2Q = -k^2Q$得到:
  $$ f''+\frac{2}{r}f'+\left(k^2-\frac{\ell(\ell+1)}{r^2}\right) f = 0 $$
  这个方程看着“几乎”就是贝塞尔方程,只是$f'$前面的系数变成了$\frac{2}{r}$。怎么去求解呢?
  \ech
\end{frame}

\begin{frame}
  \chtitle{思考题1}
  \bch
  \addfig{1}{think1.jpg}
  
  方程
  $$ f''+\frac{2}{r}f'+\left(k^2-\frac{\ell(\ell+1)}{r^2}\right) f = 0, $$
  当$k=0$时,你能在$r>0$范围内猜出两个线性独立的解吗?

  \ech
\end{frame}


\begin{frame}
  \chtitle{思考题2}
  \bch
  \addfig{1}{think3.jpg}
  
  证明:类贝塞尔方程
  $$ y''+\frac{1-2\alpha}{x} y' + \left[\beta^2\gamma^2x^{2\gamma-2}+\frac{\alpha^2-\nu^2\gamma^2)}{x^2}\right]y=0 $$
  在$x>0$范围内有两个线性无关解: $x^{\alpha}J_\nu(\beta x^\gamma)$和$x^{\alpha}Y_\nu(\beta x^\gamma)$.
  \ech
\end{frame}


\begin{frame}
  \chtitle{思考题3}
  \bch
  \addfig{1}{think3.jpg}

  根据上题结论,在$k>0$时求解方程
  $$ f''+\frac{2}{r}f'+\left(k^2-\frac{\ell(\ell+1)}{r^2}\right) f = 0. $$
  \ech
\end{frame}

\begin{frame}
  \chtitle{球贝塞尔函数}
  \bch
  $$ f''+\frac{2}{r}f'+\left(k^2-\frac{\ell(\ell+1)}{r^2}\right) f = 0. $$  
  两个线性无关解$f =  (kr)^{-1/2}J_{\ell+\frac{1}{2}}(kr)$和$f=(kr)^{-1/2}Y_{\ell+\frac{1}{2}}(kr)$可以重新写为$ j_\ell(kr)$和$y_{\ell}(kr)$。
  其中第一类球贝塞尔函数$j_\ell$和第二类球贝塞尔函数$y_\ell$分别定义为:{\blue
  $$ j_\ell(x) :=\sqrt{ \frac{\pi}{2x}} J_{\ell+1/2}(x), $$
  $$ y_\ell(x) :=\sqrt{ \frac{\pi}{2x}} Y_{\ell+1/2}(x). $$  }

  
  \ech
\end{frame}


\begin{frame}
  \chtitle{前几个$j_\ell$}
  \bch
  \addfig{4}{jell.png}
  \ech
\end{frame}


\begin{frame}
  \chtitle{前几个$y_\ell$}
  \bch
  \addfig{4}{yell.png}
  \ech
\end{frame}

\begin{frame}
  \chtitle{思考题}
  \bch
  \addfig{1}{think1.jpg}
  
  证明:
  $$y_\ell(x) = (-1)^{\ell+1}\sqrt{\frac{\pi}{2x}}J_{-\ell-1/2}(x).$$
  \ech
\end{frame}


\begin{frame}
  \chtitle{球坐标系谐函数的总结}
  \bch

  综合前面的思考题结果,容易总结出:
  \bitem
\item{\blue 对$k>0$,球坐标系的包含球心的谐函数为$j_\ell(kr)Y_{\ell m}(\theta, \phi)$; 不包含球心的谐函数则多了另一个线性独立解$y_\ell(kr)Y_{\ell m}(\theta, \phi)$.}
\item{\blue 对$k=0$,球坐标系包含球心的谐函数为$r^\ell Y_{\ell m}(\theta,\phi)$;不包含球心的谐函数则多了另一个线性独立解$r^{-\ell -1}Y_{\ell m}(\theta,\phi)$.}
  \eitem

  (思考:在$k>0$情况的解里令$k\rightarrow 0^+$,是否得到和$k=0$情况一样的结果。
  \ech
\end{frame}



\section{Homework}

\begin{frame}
\chtitle{课后作业(题号46-48)}
\bch
\bitem
\item[46]{求微分方程
  $$ \frac{d^2y}{dx^2} -\frac{1}{x}\frac{dy}{dx} + \left(4x^2-\frac{15}{x^2}\right)y = 0$$
  在$x>0$范围内的两个线性无关解。}
\item[47]{证明球贝塞尔函数$j_n$和$y_n$满足
  $$ j_n(x) = (-x)^n\left(\frac{1}{x}\frac{d}{dx}\right)^n\frac{\sin x}{x}; $$
  $$ y_n(x) = -(-x)^n\left(\frac{1}{x}\frac{d}{dx}\right)^n\frac{\cos x}{x}; $$}
\item[48]{利用之前学过的贝塞尔函数的正交定理证明球贝塞尔函数的正交定理:设$k_1, k_2>0$,则
  $$\int_0^\infty j_{\ell}(k_1r)j_{\ell}(k_2r) r^2 dr = \frac{\pi}{2k_1^2} \delta(k_1-k_2).$$}
 
  \eitem
\ech
\end{frame}

\end{document}
