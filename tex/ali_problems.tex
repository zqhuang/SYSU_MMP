\documentclass[12pt,CJK]{article}
\usepackage{geometry}
\input{reduced_macros.tex}
%\geometry{tmargin=0.3in, bmargin=0.5in, lmargin=0.5in, rmargin=0.9in, nohead, nofoot}
%\def\mark#1{{\color{blue} (#1分)}}
\renewcommand{\thepage}{}
\begin{document}
\bch

{\large 阿里巴巴全球数学竞赛复赛推荐题}

\skipline

{命题人:黄志琦}

\skiplines

\bitem
\item[P1.]{
用$A^{(N)}$表示这样的$N\times N$实对称矩阵: 其对角元素为
$$A^{(N)}_{k,k}= \ln (1+k), \ k=1,2,\ldots, N,$$
次对角元素为
$$A^{(N)}_{k+1,k}=A_{k,k+1}^{(N)}=-\frac{1}{2}\sqrt{A^{(N)}_{k,k}A^{(N)}_{k+1,k+1}},\ k=1,2,\ldots, N-1.$$
其余元素均为零。

例如
$$ A^{(4)} = 
\begin{pmatrix}
\ln 2 & -\frac{\sqrt{\ln 2 \ln 3}}{2} & 0 & 0 \\
-\frac{\sqrt{\ln 2 \ln 3}}{2} & \ln 3 & -\frac{\sqrt{\ln 3 \ln 4}}{2} & 0 \\
0 & -\frac{\sqrt{\ln 3 \ln 4}}{2} & \ln 4 & -\frac{\sqrt{\ln 4 \ln 5}}{2} \\
0 & 0 &  -\frac{\sqrt{\ln 4 \ln 5}}{2} & \ln 5
\end{pmatrix}
$$
\bitem
\item[(1)]{ 证明: 对任意正整数$N$,$A^{(N)}$的所有本征值均为非负实数。}
\item[(2)]{ 把$A^{(N)}$的所有本征值的平方根之和记作$s_N$,计算:
    $$\lim_{N\rightarrow \infty} \frac{s_N}{N\sqrt{\ln N}}.$$}
\eitem
}

\item[P2]{
    

  }

\eitem




\ech
\end{document}
