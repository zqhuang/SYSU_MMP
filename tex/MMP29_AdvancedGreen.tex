\documentclass[CJK]{beamer}
\usepackage{CJKutf8}
\usepackage{beamerthemesplit}
\usetheme{Malmoe}
\useoutertheme[footline=authortitle]{miniframes}
\usepackage{amsmath}
\usepackage{amssymb}
\usepackage{graphicx}
\usepackage{eufrak}
\usepackage{color}
\usepackage{slashed}
\usepackage{simplewick}
\usepackage{tikz}
\usepackage{tcolorbox}
\graphicspath{{../figures/}}
%%figures
\def\lfig#1#2{\includegraphics[width=#1 in]{#2}}
\def\addfig#1#2{\begin{center}\includegraphics[width=#1 in]{#2}\end{center}}
\def\wulian{\includegraphics[width=0.18in]{emoji_wulian.jpg}}
\def\bigwulian{\includegraphics[width=0.35in]{emoji_wulian.jpg}}
\def\bye{\includegraphics[width=0.18in]{emoji_bye.jpg}}
\def\bigbye{\includegraphics[width=0.35in]{emoji_bye.jpg}}
\def\huaixiao{\includegraphics[width=0.18in]{emoji_huaixiao.jpg}}
\def\bighuaixiao{\includegraphics[width=0.35in]{emoji_huaixiao.jpg}}
\def\jianxiao{\includegraphics[width=0.18in]{emoji_jianxiao.jpg}}
\def\bigjianxiao{\includegraphics[width=0.35in]{emoji_jianxiao.jpg}}
%% colors
\def\blacktext#1{{\color{black}#1}}
\def\bluetext#1{{\color{blue}#1}}
\def\redtext#1{{\color{red}#1}}
\def\darkbluetext#1{{\color[rgb]{0,0.2,0.6}#1}}
\def\skybluetext#1{{\color[rgb]{0.2,0.7,1.}#1}}
\def\cyantext#1{{\color[rgb]{0.,0.5,0.5}#1}}
\def\greentext#1{{\color[rgb]{0,0.7,0.1}#1}}
\def\darkgray{\color[rgb]{0.2,0.2,0.2}}
\def\lightgray{\color[rgb]{0.6,0.6,0.6}}
\def\gray{\color[rgb]{0.4,0.4,0.4}}
\def\blue{\color{blue}}
\def\red{\color{red}}
\def\green{\color{green}}
\def\darkgreen{\color[rgb]{0,0.4,0.1}}
\def\darkblue{\color[rgb]{0,0.2,0.6}}
\def\skyblue{\color[rgb]{0.2,0.7,1.}}
%%control
\def\be{\begin{equation}}
\def\ee{\nonumber\end{equation}}
\def\bea{\begin{eqnarray}}
\def\eea{\nonumber\end{eqnarray}}
\def\bch{\begin{CJK}{UTF8}{gbsn}}
\def\ech{\end{CJK}}
\def\bitem{\begin{itemize}}
\def\eitem{\end{itemize}}
\def\bcenter{\begin{center}}
\def\ecenter{\end{center}}
\def\bex{\begin{minipage}{0.2\textwidth}\includegraphics[width=0.6in]{jugelizi.png}\end{minipage}\begin{minipage}{0.76\textwidth}}
\def\eex{\end{minipage}}
\def\chtitle#1{\frametitle{\bch#1\ech}}
\def\bmat#1{\left(\begin{array}{#1}}
\def\emat{\end{array}\right)}
\def\bcase#1{\left\{\begin{array}{#1}}
\def\ecase{\end{array}\right.}
\def\bmini#1{\begin{minipage}{#1\textwidth}}
\def\emini{\end{minipage}}
\def\tbox#1{\begin{tcolorbox}#1\end{tcolorbox}}
\def\pfrac#1#2#3{\left(\frac{\partial #1}{\partial #2}\right)_{#3}}
%%symbols
\def\bropt{\,(\ \ \ )}
\def\sone{$\star$}
\def\stwo{$\star\star$}
\def\sthree{$\star\star\star$}
\def\sfour{$\star\star\star\star$}
\def\sfive{$\star\star\star\star\star$}
\def\rint{{\int_\leftrightarrow}}
\def\roint{{\oint_\leftrightarrow}}
\def\stdHf{{\textit{\r H}_f}}
\def\deltaH{{\Delta \textit{\r H}}}
\def\ii{{\dot{\imath}}}
\def\skipline{{\vskip0.1in}}
\def\skiplines{{\vskip0.2in}}
\def\lagr{{\mathcal{L}}}
\def\hamil{{\mathcal{H}}}
\def\vecv{{\mathbf{v}}}
\def\vecx{{\mathbf{x}}}
\def\vecy{{\mathbf{y}}}
\def\veck{{\mathbf{k}}}
\def\vecp{{\mathbf{p}}}
\def\vecn{{\mathbf{n}}}
\def\vecA{{\mathbf{A}}}
\def\vecP{{\mathbf{P}}}
\def\vecsigma{{\mathbf{\sigma}}}
\def\hatJn{{\hat{J_\vecn}}}
\def\hatJx{{\hat{J_x}}}
\def\hatJy{{\hat{J_y}}}
\def\hatJz{{\hat{J_z}}}
\def\hatj#1{\hat{J_{#1}}}
\def\hatphi{{\hat{\phi}}}
\def\hatq{{\hat{q}}}
\def\hatpi{{\hat{\pi}}}
\def\vel{\upsilon}
\def\Dint{{\mathcal{D}}}
\def\adag{{\hat{a}^\dagger}}
\def\bdag{{\hat{b}^\dagger}}
\def\cdag{{\hat{c}^\dagger}}
\def\ddag{{\hat{d}^\dagger}}
\def\hata{{\hat{a}}}
\def\hatb{{\hat{b}}}
\def\hatc{{\hat{c}}}
\def\hatd{{\hat{d}}}
\def\hatN{{\hat{N}}}
\def\hatH{{\hat{H}}}
\def\hatp{{\hat{p}}}
\def\Fup{{F^{\mu\nu}}}
\def\Fdown{{F_{\mu\nu}}}
\def\newl{\nonumber \\}
\def\vece{\mathrm{e}}
\def\calM{{\mathcal{M}}}
\def\calT{{\mathcal{T}}}
\def\calR{{\mathcal{R}}}
\def\barpsi{\bar{\psi}}
\def\baru{\bar{u}}
\def\barv{\bar{\upsilon}}
\def\qeq{\stackrel{?}{=}}
\def\torder#1{\mathcal{T}\left(#1\right)}
\def\rorder#1{\mathcal{R}\left(#1\right)}
\def\contr#1#2{\contraction{}{#1}{}{#2}#1#2}
\def\trof#1{\mathrm{Tr}\left(#1\right)}
\def\trace{\mathrm{Tr}}
\def\comm#1{\ \ \ \left(\mathrm{used}\ #1\right)}
\def\tcomm#1{\ \ \ (\text{#1})}
\def\slp{\slashed{p}}
\def\slk{\slashed{k}}
\def\calp{{\mathfrak{p}}}
\def\veccalp{\mathbf{\mathfrak{p}}}
\def\Tthree{T_{\tiny \textcircled{3}}}
\def\pthree{p_{\tiny \textcircled{3}}}
\def\dbar{{\,\mathchar'26\mkern-12mu d}}
\def\erf{\mathrm{erf}}
\def\const{\mathrm{constant}}
\def\pheat{\pfrac p{\ln T}V}
\def\vheat{\pfrac V{\ln T}p}
%%units
\def\fdeg{{^\circ \mathrm{F}}}
\def\cdeg{^\circ \mathrm{C}}
\def\atm{\,\mathrm{atm}}
\def\angstrom{\,\text{\AA}}
\def\SIL{\,\mathrm{L}}
\def\SIkm{\,\mathrm{km}}
\def\SIyr{\,\mathrm{yr}}
\def\SIGyr{\,\mathrm{Gyr}}
\def\SIV{\,\mathrm{V}}
\def\SImV{\,\mathrm{mV}}
\def\SIeV{\,\mathrm{eV}}
\def\SIkeV{\,\mathrm{keV}}
\def\SIMeV{\,\mathrm{MeV}}
\def\SIGeV{\,\mathrm{GeV}}
\def\SIcal{\,\mathrm{cal}}
\def\SIkcal{\,\mathrm{kcal}}
\def\SImol{\,\mathrm{mol}}
\def\SIN{\,\mathrm{N}}
\def\SIHz{\,\mathrm{Hz}}
\def\SIm{\,\mathrm{m}}
\def\SIcm{\,\mathrm{cm}}
\def\SIfm{\,\mathrm{fm}}
\def\SImm{\,\mathrm{mm}}
\def\SInm{\,\mathrm{nm}}
\def\SImum{\,\mathrm{\mu m}}
\def\SIJ{\,\mathrm{J}}
\def\SIW{\,\mathrm{W}}
\def\SIkJ{\,\mathrm{kJ}}
\def\SIs{\,\mathrm{s}}
\def\SIkg{\,\mathrm{kg}}
\def\SIg{\,\mathrm{g}}
\def\SIK{\,\mathrm{K}}
\def\SImmHg{\,\mathrm{mmHg}}
\def\SIPa{\,\mathrm{Pa}}

\def\courseurl{https://github.com/zqhuang/SYSU\_TD}

\def\tpage#1#2{
\begin{frame}
\begin{center}
\begin{Large}
\bch
热学 \\
第#1讲 #2

{\vskip 0.3in}

黄志琦

\ech
\end{Large}
\end{center}

\vskip 0.2in

\bch
教材:《热学》第二版,赵凯华,罗蔚茵,高等教育出版社
\ech

\bch
课件下载
\ech
\courseurl
\end{frame}
}

\def\bfr#1{
\begin{frame}
\chtitle{#1} 
\bch
}

\def\efr{
\ech 
\end{frame}
}

  \date{}
  \begin{document}
  \bch
\tpage{29}{格林函数进阶知识}

\begin{frame}
\frametitle{本讲内容}

\tableofcontents

\end{frame}

\section{Review and Practices}

\thinka{
  已知正交曲面坐标系$(X,Y,Z)$中的长度元的表达式:
  $$ dL =\sqrt{\frac{dX^2+dY^2+ X^2Y^2 dZ^2}{1+X^2+Y^2}}.$$
  那么体积元的表达式是什么?}

\thinka{计算积分 $$\int_{-\infty}^\infty \delta(x^4-1)x^2dx.$$}

\thinkb{长度为$0.1\mathrm{m}$的均匀铝棒一端和温度为$300\mathrm{K}$的热库接触(棒的其余各处都孤立绝热),并在 $t < 0$ 时和热库处于热平衡。从 $t=0$ 时刻开始,在棒的另一端持续地沿着棒地方向注入大小为 $j = 2.3\times 10^4\mathrm{W/m^2}$ 的热流。经过10分钟后,棒的中点的温度为多少K?已知铝的导热系数为 $230\mathrm{W\,m^{-1}K^{-1}}$,比热为 $900 \mathrm{J\,kg^{-1}K^{-1}}$, 密度为 $2700\mathrm{kg\,m^{-3}}$。}

\thinkc{证明 Wallis 公式 $$\pi= \lim_{n\rightarrow\infty}\frac{1}{n}\left(\frac{(2n)!!}{(2n-1)!!}\right)^2 $$
  其中 $m!! \equiv m(m-2)(m-4)\ldots$ 即从 $m$ 开始,递减量为 $2$ 的连乘积,直到乘到 $2$ 或 $1$ 为止。例如 $5!!=5\times 3\times 1 = 15$, $6!! = 6\times 4\times 2 = 48$.}


\thinkb{计算不定积分$$ \int x^4 J_1(x) dx$$}

\begin{frame}
  \frametitle{活跃一下气氛($7\star$)}
  对正整数$n$定义
$$f(n) = \sum_{k=1}^n\tan^2\frac{k\pi}{2n+1}$$
计算无穷级数和
$$\sum_{n=1}^\infty \frac{1}{f(n)}. $$

\end{frame}

\section{Green's Function: Heat Equation}

\secpage{格林函数进阶}{持续热源}

\begin{frame}
  \frametitle{瞬时点热源}
  
  我们讨论过一维无界区域的瞬时点热源的扩散问题:
  
  $$\frac{\partial u}{\partial t} - a\frac{\partial^2 u}{\partial x^2}=0.$$
  $$\left. u \right\vert_{t=0}= \delta(x-x_0).$$

  这是个格林函数问题,容易看出其解为:
  $$  G_{\rm 1D}(t, x; x_0) = \frac{1}{2\pi}\int_{-\infty}^\infty e^{ik(x-x_0)-ak^2t}dk = \frac{1}{\sqrt{4\pi at}} e^{-\frac{(x-x_0)^2}{4at}} . $$

  (它很容易推广到高维空间: $$G_{\rm 2D}(t, x,y;x_0,y_0) = G_{\rm 1D}(t, x; x_0)G_{\rm 1D}(t, y; y_0)$$
  $$G_{\rm 3D}(t, x,y,z;x_0,y_0,z_0) = G_{\rm 1D}(t, x; x_0)G_{\rm 1D}(t, y; y_0)G_{\rm 1D}(t, z; z_0)$$
  等)
  
\end{frame}

\begin{frame}
  \frametitle{时间平移}
  如果瞬时点热源是$t=\tau$时刻放的,
  $$\frac{\partial u}{\partial t} - a\frac{\partial^2 u}{\partial x^2}=0.$$
  $$\left. u \right\vert_{t=\tau}= \delta(x-x_0).$$
  则只需要做一下时间平移:
  $$ u  = \frac{1}{\sqrt{4\pi a(t-\tau)}} e^{-\frac{(x-x_0)^2}{4a(t-\tau)}},\ \  (t>\tau).$$
\end{frame}


\begin{frame}
  \frametitle{持续点热源}
  如果有持续的点热源,
  $$\frac{\partial u}{\partial t} - a\frac{\partial^2 u}{\partial x^2}=\delta(x-x_0).$$
  $$\left. u \right\vert_{t=0}=0.$$
  这可以看成 $\tau=0$ 时刻开始每隔 $d\tau$ 时间就放一个点热源 $\delta(x-x_0)d\tau$。在$t$时刻,只有$0\le \tau\le t$时刻的热源才有贡献。因此问题的解为:
  $$ u(x,t) = \int_0^t d\tau  \frac{1}{\sqrt{4\pi a(t-\tau)}} e^{-\frac{(x-x_0)^2}{4a(t-\tau)}}.$$
  把$t-\tau$换为$\tau$,
  $$ u(x,t) = \int_0^t d\tau  \frac{1}{\sqrt{4\pi a\tau}} e^{-\frac{(x-x_0)^2}{4a\tau}}.$$
\end{frame}


\begin{frame}
  \frametitle{检验解}
  我们来检验一下这个解——
  
  积分号下求导
  $$\frac{\partial u}{\partial t}= \frac{1}{\sqrt{4\pi at}} e^{-\frac{(x-x_0)^2}{4at}}.$$
  $$a\frac{\partial^2 u}{\partial x^2} = \int_0^t d\tau  \left[\frac{(x-x_0)^2}{4a\tau^2}-\frac{1}{2\tau} \right]\frac{1}{\sqrt{4\pi a\tau}} e^{-\frac{(x-x_0)^2}{4a\tau}} = \left.\frac{1}{\sqrt{4\pi a\tau}} e^{-\frac{(x-x_0)^2}{4a\tau}} \right\vert_0^t.$$
  于是有
  $$\frac{\partial u}{\partial t} - a\frac{\partial^2 u}{\partial x^2}= \lim_{\tau\rightarrow 0^+}\frac{1}{\sqrt{4\pi a\tau}} e^{-\frac{(x-x_0)^2}{4a\tau}} = \delta(x-x_0).$$
\end{frame}



\begin{frame}
  \frametitle{持续点热源的高维情况}
  很容易看出,二维和三维空间的解为
  $$ u(x,t) = \int_0^t d\tau  \frac{1}{\sqrt{4\pi a(t-\tau)}} e^{-\frac{(x-x_0)^2-(y-y_0)^2}{4a(t-\tau)}}.$$
  $$ u(x,t) = \int_0^t d\tau  \frac{1}{\sqrt{4\pi a(t-\tau)}} e^{-\frac{(x-x_0)^2-(y-y_0)^2-(z-z_0)^2}{4a(t-\tau)}}.$$
  
\end{frame}


\section{Green's Function: Helmholtz Equation}
\secpage{格林函数进阶}{三维无界空间的Helmholtz方程}


\begin{frame}
 \frametitle{自由Helmholtz方程}
  谐函数满足的方程
  $$ \nabla^2f + k^2 f = 0. $$
  也叫自由Helmholtz方程。当 $k=0$,就是无源的泊松方程。
\end{frame}

\begin{frame}
 \frametitle{有点源的Helmholtz方程}
  $$ \nabla^2f + k^2 f = \delta(\mathbf{x}-\mathbf{x_0}) $$
 是有点源的Helmholtz方程。当 $k=0$,就是有点源的泊松方程。

 \skiplines
 
 我们重点讨论物理中常见的三维情况。
\end{frame}


\begin{frame}
  \frametitle{三维空间$k=0$的情况}
  当 $k=0$ 时,
  $$ \nabla^2f  = \delta(\mathbf{x}-\mathbf{x_0}) $$
  (在静电学的点电荷问题里,右边还要多乘个 $-\frac{q}{\epsilon_0}$.)

    默认无穷远处的$f$趋向于零,就可以通过和静电学的结果比较,得到 $f$ 等于
    $$ G(\mathbf{x};\mathbf{x}_0) = -\frac{1}{4\pi |\mathbf{x}-\mathbf{x}_0|} .$$
    这就是三维无界空间的泊松方程的格林函数。
\end{frame}


\begin{frame}
  \frametitle{$k=0$的情况}
  对有任意静态源的情况
    $$ \nabla^2f  = s(\mathbf{x})$$
  显然只需要把静态源分解为一系列的 $\delta$ 函数(相当于把连续电荷分布看成很多点电荷的集合),
  $$ s(\mathbf{x}) = \sum_{x_0} [d^3x_0 f(x_0)] \delta(x-x_0). $$
  好吧还是写成你喜欢的积分形式
  $$ s(\mathbf{x}) = \int d^3x_0 f(x_0) \delta(x-x_0). $$  
\end{frame}

\begin{frame}
  \frametitle{$k=0$的情况}
  问题的解就是把这些$\delta$源的解做完全相同的线性组合:
  $$ f(\mathbf{x}) = \int d^3x_0f(x_0) G(\mathbf{x};\mathbf{x}_0) = -\int d^3x_0\frac{s(x_0)}{4\pi |\mathbf{x}-\mathbf{x}_0|}. $$
  好吧,你已经认出来了,把 $s(\mathbf{x})$换成 $-\frac{\rho(\mathbf{x})}{\epsilon_0}$,这就是电磁学的静电势积分公式。
\end{frame}


\begin{frame}
  \frametitle{$k>0$的情况}
  下面我们来讨论更有意思的 $k>0$ 的情况,在波动问题中,分离变量后常常出现这样的情况
  $$ \nabla^2f + k^2 f = \delta(\mathbf{x}-\mathbf{x_0}) $$
  我们以 $\mathbf{x_0}$ 为球心建立球坐标系,那么问题的解必然可以写成(挖掉球心的空间内的)谐函数的线性组合。又问题是各向同性的,我们可以直接写出
  $$ f = a j_0(kr) + b n_0(kr) $$
  的形式。
\end{frame}

\begin{frame}
  \frametitle{$k>0$的情况}
  利用球贝塞尔函数的具体形式
  $$ j_0(kr) = \frac{\sin(kr)}{(kr)},\ n_0(kr) = \frac{\cos(kr)}{(kr)}$$
  我们也可以把解写成

  $$ f = C_1\frac{e^{ikr}}{kr} + C_2\frac{e^{-ikr}}{kr}. $$
  的形式。在原点附近很小体积内对原方程进行积分,物理上通常可以假设$f$的体积分趋向于零,于是和泊松方程的情况一样,可以确定出
   $$ C_1+C_2=-\frac{1}{4\pi}$$
\end{frame}

\begin{frame}
  \frametitle{$k>0$的情况}
  那么具体 $C_1$ 和 $C_2$ 的比例该取多少呢?这要由具体问题的无限远处边界条件决定。如果是个发散球面波问题,要求越远相位就越大,则通常只能取 $C_1=1, C_2=0$。即格林函数为
  $$ G(\mathbf{x};\mathbf{x}_0) = -\frac{e^{ik|\mathbf{x}-\mathbf{x}_0|}}{4\pi k|\mathbf{x}-\mathbf{x}_0|}.$$
  如果是会聚球面波问题,则
  $$ G(\mathbf{x};\mathbf{x}_0) = -\frac{e^{-ik|\mathbf{x}-\mathbf{x}_0|}}{4\pi k|\mathbf{x}-\mathbf{x}_0|}.$$  
  上述任意一个情况令$k=0$,都能得到泊松方程的格林函数。
    
\end{frame}


\ech
\end{document}
