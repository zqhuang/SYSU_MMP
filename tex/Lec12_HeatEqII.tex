\documentclass[CJK]{beamer}
\usepackage{CJKutf8}
\usepackage{beamerthemesplit}
\usetheme{Malmoe}
\useoutertheme[footline=authortitle]{miniframes}
\usepackage{amsmath}
\usepackage{amssymb}
\usepackage{graphicx}
\usepackage{eufrak}
\usepackage{color}
\usepackage{slashed}
\usepackage{simplewick}
\usepackage{tikz}
\usepackage{tcolorbox}
\graphicspath{{../figures/}}
%%figures
\def\lfig#1#2{\includegraphics[width=#1 in]{#2}}
\def\addfig#1#2{\begin{center}\includegraphics[width=#1 in]{#2}\end{center}}
\def\wulian{\includegraphics[width=0.18in]{emoji_wulian.jpg}}
\def\bigwulian{\includegraphics[width=0.35in]{emoji_wulian.jpg}}
\def\bye{\includegraphics[width=0.18in]{emoji_bye.jpg}}
\def\bigbye{\includegraphics[width=0.35in]{emoji_bye.jpg}}
\def\huaixiao{\includegraphics[width=0.18in]{emoji_huaixiao.jpg}}
\def\bighuaixiao{\includegraphics[width=0.35in]{emoji_huaixiao.jpg}}
\def\jianxiao{\includegraphics[width=0.18in]{emoji_jianxiao.jpg}}
\def\bigjianxiao{\includegraphics[width=0.35in]{emoji_jianxiao.jpg}}
%% colors
\def\blacktext#1{{\color{black}#1}}
\def\bluetext#1{{\color{blue}#1}}
\def\redtext#1{{\color{red}#1}}
\def\darkbluetext#1{{\color[rgb]{0,0.2,0.6}#1}}
\def\skybluetext#1{{\color[rgb]{0.2,0.7,1.}#1}}
\def\cyantext#1{{\color[rgb]{0.,0.5,0.5}#1}}
\def\greentext#1{{\color[rgb]{0,0.7,0.1}#1}}
\def\darkgray{\color[rgb]{0.2,0.2,0.2}}
\def\lightgray{\color[rgb]{0.6,0.6,0.6}}
\def\gray{\color[rgb]{0.4,0.4,0.4}}
\def\blue{\color{blue}}
\def\red{\color{red}}
\def\green{\color{green}}
\def\darkgreen{\color[rgb]{0,0.4,0.1}}
\def\darkblue{\color[rgb]{0,0.2,0.6}}
\def\skyblue{\color[rgb]{0.2,0.7,1.}}
%%control
\def\be{\begin{equation}}
\def\ee{\nonumber\end{equation}}
\def\bea{\begin{eqnarray}}
\def\eea{\nonumber\end{eqnarray}}
\def\bch{\begin{CJK}{UTF8}{gbsn}}
\def\ech{\end{CJK}}
\def\bitem{\begin{itemize}}
\def\eitem{\end{itemize}}
\def\bcenter{\begin{center}}
\def\ecenter{\end{center}}
\def\bex{\begin{minipage}{0.2\textwidth}\includegraphics[width=0.6in]{jugelizi.png}\end{minipage}\begin{minipage}{0.76\textwidth}}
\def\eex{\end{minipage}}
\def\chtitle#1{\frametitle{\bch#1\ech}}
\def\bmat#1{\left(\begin{array}{#1}}
\def\emat{\end{array}\right)}
\def\bcase#1{\left\{\begin{array}{#1}}
\def\ecase{\end{array}\right.}
\def\bmini#1{\begin{minipage}{#1\textwidth}}
\def\emini{\end{minipage}}
\def\tbox#1{\begin{tcolorbox}#1\end{tcolorbox}}
\def\pfrac#1#2#3{\left(\frac{\partial #1}{\partial #2}\right)_{#3}}
%%symbols
\def\bropt{\,(\ \ \ )}
\def\sone{$\star$}
\def\stwo{$\star\star$}
\def\sthree{$\star\star\star$}
\def\sfour{$\star\star\star\star$}
\def\sfive{$\star\star\star\star\star$}
\def\rint{{\int_\leftrightarrow}}
\def\roint{{\oint_\leftrightarrow}}
\def\stdHf{{\textit{\r H}_f}}
\def\deltaH{{\Delta \textit{\r H}}}
\def\ii{{\dot{\imath}}}
\def\skipline{{\vskip0.1in}}
\def\skiplines{{\vskip0.2in}}
\def\lagr{{\mathcal{L}}}
\def\hamil{{\mathcal{H}}}
\def\vecv{{\mathbf{v}}}
\def\vecx{{\mathbf{x}}}
\def\vecy{{\mathbf{y}}}
\def\veck{{\mathbf{k}}}
\def\vecp{{\mathbf{p}}}
\def\vecn{{\mathbf{n}}}
\def\vecA{{\mathbf{A}}}
\def\vecP{{\mathbf{P}}}
\def\vecsigma{{\mathbf{\sigma}}}
\def\hatJn{{\hat{J_\vecn}}}
\def\hatJx{{\hat{J_x}}}
\def\hatJy{{\hat{J_y}}}
\def\hatJz{{\hat{J_z}}}
\def\hatj#1{\hat{J_{#1}}}
\def\hatphi{{\hat{\phi}}}
\def\hatq{{\hat{q}}}
\def\hatpi{{\hat{\pi}}}
\def\vel{\upsilon}
\def\Dint{{\mathcal{D}}}
\def\adag{{\hat{a}^\dagger}}
\def\bdag{{\hat{b}^\dagger}}
\def\cdag{{\hat{c}^\dagger}}
\def\ddag{{\hat{d}^\dagger}}
\def\hata{{\hat{a}}}
\def\hatb{{\hat{b}}}
\def\hatc{{\hat{c}}}
\def\hatd{{\hat{d}}}
\def\hatN{{\hat{N}}}
\def\hatH{{\hat{H}}}
\def\hatp{{\hat{p}}}
\def\Fup{{F^{\mu\nu}}}
\def\Fdown{{F_{\mu\nu}}}
\def\newl{\nonumber \\}
\def\vece{\mathrm{e}}
\def\calM{{\mathcal{M}}}
\def\calT{{\mathcal{T}}}
\def\calR{{\mathcal{R}}}
\def\barpsi{\bar{\psi}}
\def\baru{\bar{u}}
\def\barv{\bar{\upsilon}}
\def\qeq{\stackrel{?}{=}}
\def\torder#1{\mathcal{T}\left(#1\right)}
\def\rorder#1{\mathcal{R}\left(#1\right)}
\def\contr#1#2{\contraction{}{#1}{}{#2}#1#2}
\def\trof#1{\mathrm{Tr}\left(#1\right)}
\def\trace{\mathrm{Tr}}
\def\comm#1{\ \ \ \left(\mathrm{used}\ #1\right)}
\def\tcomm#1{\ \ \ (\text{#1})}
\def\slp{\slashed{p}}
\def\slk{\slashed{k}}
\def\calp{{\mathfrak{p}}}
\def\veccalp{\mathbf{\mathfrak{p}}}
\def\Tthree{T_{\tiny \textcircled{3}}}
\def\pthree{p_{\tiny \textcircled{3}}}
\def\dbar{{\,\mathchar'26\mkern-12mu d}}
\def\erf{\mathrm{erf}}
\def\const{\mathrm{constant}}
\def\pheat{\pfrac p{\ln T}V}
\def\vheat{\pfrac V{\ln T}p}
%%units
\def\fdeg{{^\circ \mathrm{F}}}
\def\cdeg{^\circ \mathrm{C}}
\def\atm{\,\mathrm{atm}}
\def\angstrom{\,\text{\AA}}
\def\SIL{\,\mathrm{L}}
\def\SIkm{\,\mathrm{km}}
\def\SIyr{\,\mathrm{yr}}
\def\SIGyr{\,\mathrm{Gyr}}
\def\SIV{\,\mathrm{V}}
\def\SImV{\,\mathrm{mV}}
\def\SIeV{\,\mathrm{eV}}
\def\SIkeV{\,\mathrm{keV}}
\def\SIMeV{\,\mathrm{MeV}}
\def\SIGeV{\,\mathrm{GeV}}
\def\SIcal{\,\mathrm{cal}}
\def\SIkcal{\,\mathrm{kcal}}
\def\SImol{\,\mathrm{mol}}
\def\SIN{\,\mathrm{N}}
\def\SIHz{\,\mathrm{Hz}}
\def\SIm{\,\mathrm{m}}
\def\SIcm{\,\mathrm{cm}}
\def\SIfm{\,\mathrm{fm}}
\def\SImm{\,\mathrm{mm}}
\def\SInm{\,\mathrm{nm}}
\def\SImum{\,\mathrm{\mu m}}
\def\SIJ{\,\mathrm{J}}
\def\SIW{\,\mathrm{W}}
\def\SIkJ{\,\mathrm{kJ}}
\def\SIs{\,\mathrm{s}}
\def\SIkg{\,\mathrm{kg}}
\def\SIg{\,\mathrm{g}}
\def\SIK{\,\mathrm{K}}
\def\SImmHg{\,\mathrm{mmHg}}
\def\SIPa{\,\mathrm{Pa}}

\def\courseurl{https://github.com/zqhuang/SYSU\_TD}

\def\tpage#1#2{
\begin{frame}
\begin{center}
\begin{Large}
\bch
热学 \\
第#1讲 #2

{\vskip 0.3in}

黄志琦

\ech
\end{Large}
\end{center}

\vskip 0.2in

\bch
教材:《热学》第二版,赵凯华,罗蔚茵,高等教育出版社
\ech

\bch
课件下载
\ech
\courseurl
\end{frame}
}

\def\bfr#1{
\begin{frame}
\chtitle{#1} 
\bch
}

\def\efr{
\ech 
\end{frame}
}

  \date{}
\begin{document}
\tpage{12}{Heat Equation (II)}

\begin{frame}
  \chtitle{积分变换和格林函数方法}
  \bch
  \bitem
\item{积分变换方法}
\item{格林函数方法}
  \eitem
  \ech
\end{frame}

\section{BD at infinity}
\secpage{积分变换方法}{其实就是分离变量法的连续极限}

\begin{frame}
  \chtitle{有限到无限,离散到连续}
  \bch
  热传导方程的解通常是
  \tbox{渐近解 + 一堆$e^{-ak^2t+\ii kx}$的线性叠加}

  \skiplines
  
  长为$L$的不良导体棒上满足齐次边界条件的解对$k$有限制,这些{\blue 允许的$k$之间的间隔通常为$\sim\frac{1}{L}$的量级}。

  \skiplines

  无界或者半无界问题可以看成$L\rightarrow \infty$的极限,允许的$k$之间的间隔$\sim\frac{1}{L}$趋向于零。也就是说,$k$可以连续取值。

  
  \ech
\end{frame}


\begin{frame}
  \chtitle{瞬时点注入的热}
  \bch

  \addfig{3}{heatinject.png}
  
在一根均匀的无限长不良导体棒上的某点注入热量$Q$,不良导体棒上的温度分布将如何变化?设初始温度$T_0$,截面积$S$,质量密度$\rho$, 单位质量比热$c$,导热系数$\lambda$均已知。


  \ech
\end{frame}



\begin{frame}
  \chtitle{写出方程和边界条件}
  \bch
  设注入点为$x=x_0$,记注入时间为$t=0$。假设刚注入时热量分布在小区间$\left[x_0-\frac{\epsilon}{2},x_0+\frac{\epsilon}{2}\right]$上,则在这个小区间上温度升高了$\frac{Q}{c\rho S\epsilon}$,这可以用$\delta$函数来描述。


  于是写出如下的方程和初始条件:
  
  \bea
  \frac{\partial T}{\partial t} - a\frac{\partial^2 T}{\partial x^2} &=& 0, \newl
  T_{t=0} &=& T_0 + b \,\delta(x-x_0) , \newl
  T_{x=\pm \infty} &=& T_0.  
  \eea
  其中$a=\frac{\lambda}{\rho c}$, $b=\frac{Q}{c \rho  S}$。

  {\scriptsize 为了看得更清楚,我特地补上了最后一条“齐次边界条件”,这是根据简单的物理分析得出的(热在有限时间内不可能传播到无穷远处)。}
  
  \ech
\end{frame}


\begin{frame}
  \chtitle{按套路出牌}
  \bch
  
  按套路第一步,根据物理分析求渐近解。显然,热量均匀分配到无穷长的不良导体棒上后对温度不会有什么影响。所以$t\rightarrow \infty$时的渐近解是$T\rightarrow T_0$。

  \ech
\end{frame}

\begin{frame}
  \chtitle{按套路出牌}
  \bch
  
  按套路第二步,把$T(x,t)$分解为:渐近解 + 衰减模式
  $$ T(x,t) = T_0 +  u(x,t).$$
  为了不和$\delta$函数的符号发生混淆,我们用$u(x,t)$来表示对稳恒渐近解的偏离。
  
  \bea
  \frac{\partial u}{\partial t} - a\frac{\partial^2 u}{\partial x^2} &=& 0, \newl
   u_{t=0} &=&  b \,\delta(x-x_0) ,\newl
   u_{x=\pm \infty} &=& 0.  
  \eea

  \ech
\end{frame}


\begin{frame}
  \chtitle{对套路的迷惑}
  \bch
  因为$u(x,t)$满足齐次边界条件,所以令
  $$ u(x,t) = \sum_{k} c_k e^{-ak^2t}e^{\ii kx}. $$
  考虑边界条件对$k$的限制时出了一点小小的bug:似乎任何一个$e^{\ii kx}$或$\sin(kx)$或$\cos(kx)$都不在无穷远处趋向于零……
  \addfig{1}{ganga.jpg}
  \ech
\end{frame}


\begin{frame}
  \chtitle{傅立叶变换的默认设置}
  \bch
  记得某叫兽曾经讲过:{\blue $f(x)$可以进行傅立叶变换的默认设置是$f$在无穷远处为零。}

  \skiplines
  
  当然这句话并非绝对正确,数学上很容易构造出一堆反例。

  \skiplines
  
  把喜欢搞事情的数学家都踢飞之后,写出
  $$ u(x,t) = \frac{1}{\sqrt{2\pi}}\int_{-\infty}^\infty c(k) e^{-ak^2t}e^{\ii k x} dk $$
  也就是说,之前的$c_k$被写成了$\frac{c(k)dk}{\sqrt{2\pi}}$,在$k$连续取值的极限下,求和变成了积分。
  \ech
\end{frame}


\begin{frame}
  \chtitle{利用初始条件求系数}
  \bch
  仍然按照套路,为了求出系数$c(k)$,令$t=0$并利用初始条件,
  $$ b\delta(x-x_0) = \frac{1}{\sqrt{2\pi}}\int_{-\infty}^\infty c(k) e^{\ii k x} dk $$
  把上式看成傅立叶逆变换,显然$c(k)$可以用傅立叶变换求得:
  $$ c(k) = \frac{1}{\sqrt{2\pi}}\int_{-\infty}^\infty b \delta(x-x_0)e^{-\ii kx}dx = \frac{b e^{-\ii kx_0}}{\sqrt{2\pi}} $$
  于是得到
  $$ u(x,t) = \frac{b}{2\pi}\int_{-\infty}^\infty e^{-ak^2t}e^{\ii k (x-x_0)} dk $$
  
  \ech
\end{frame}


\begin{frame}
  \chtitle{大功告成}
  \bch
  再进行简单的积分(回忆用长方形围道计算  $e^{-x^2}\cos x$的积分时的技巧):
    \bea
    u(x,t) &=& \frac{b}{2\pi}\int_{-\infty}^\infty e^{-ak^2t}e^{\ii k (x-x_0)} dk \newl
    &=& \frac{b}{2\pi}e^{-\frac{(x-x_0)^2}{4at}}\int_{-\infty}^\infty e^{-at(k+\ii\frac{x-x_0}{2at})^2} dk \newl
    &=& \frac{b}{2\pi}e^{-\frac{(x-x_0)^2}{4at}} \sqrt{\frac{\pi}{at}}\newl
    &=& b \frac{1}{\sqrt{4\pi at}} e^{-\frac{(x-x_0)^2}{4at}}    
    \eea
    即最后的解为
    $$T = T_0 + b \frac{1}{\sqrt{4\pi at}} e^{-\frac{(x-x_0)^2}{4at}}  $$
  \ech
\end{frame}


\begin{frame}
  \chtitle{物理直觉大补汤:随机热运动的时间积累}
  \bch
  $$T = T_0 + b \frac{1}{\sqrt{4\pi at}} e^{-\frac{(x-x_0)^2}{4at}}  $$
  这个解说明一开始集中在一点的热,经过时间$t$后大致分布在宽度为$\sqrt{at}$的范围内。这是随机热运动的一般特性:

  \tbox{\bf \blue 随机走动的探索范围和$\sqrt{t}$成正比.}

  {\scriptsize 可以设想你(譬如,失恋后)漫无目的地在城市中行走,在每个路口随机地变换方向。那么,你的探索范围(走过的地方之间的最大直线距离)会和$\sqrt{t}$近似成正比。走的步数越多,这种近似就越精确。}
  \ech
\end{frame}



\section{Green's function}
\secpage{格林函数}{线性系统对单位脉冲的响应}


\begin{frame}
  \chtitle{Green's function}
  \bch
  在$b=1$的情况下的$u$的解有个特殊的名字:格林函数。

  \tbox{\bf \blue 满足线性齐次的微分方程和边界条件,初始条件为任意一点上的$\delta$函数的解称为格林函数(Green's function)。}

  无边界的热传导方程的格林函数是
  \tbox{$$ G(x,t;x_0) = \frac{1}{\sqrt{4\pi at}} e^{-\frac{(x-x_0)^2}{4at}} $$}


  \ech
\end{frame}


\begin{frame}
  \chtitle{格林函数的物理意义}
  \bch
  \tbox{\bf \blue 格林函数是线性系统对初始脉冲输入的响应。}

  \bex
  无边界的热传导方程的格林函数
  $$ G(x,t;x_0) = \frac{1}{\sqrt{4\pi at}} e^{-\frac{(x-x_0)^2}{4at}} $$
  是无边界的一维均匀热传导系统对初始$x_0$处脉冲输入的响应。
  \eex
  \ech
\end{frame}


\begin{frame}
  \chtitle{格林函数方法}
  \bch

  容易直接验证,如果$f$是在无穷远处趋向于零的函数,无边界热传导问题
  \bea
  \frac{\partial T}{\partial t} - a\frac{\partial^2 T}{\partial x^2} &=& 0, \newl
  T_{t=0} &=& f(x) .
  \eea
  的解为
  \tbox{\blue $$T(x,t) = \int_{-\infty}^\infty \,f(x_0)G(x,t; x_0)\,dx_0 $$}

  
  它的物理意义非常简单,就是{\blue 线性系统对初始各处的输入分别发生响应,这些响应的线性叠加就是所求的解}。
  \ech
\end{frame}


\begin{frame}
  \chtitle{例题}
  \bch
  \addfig{3}{problem12-1.png}
  
  有长为$2L$,温度为$T_0$的均匀不良导体棒。导热系数$\lambda$,质量密度$\rho$,单位质量比热$c$均已知。在$t=0$时刻,在它的两端$x=\pm L$处分别接上温度为$T_1$的相同材质相同截面形状的非常长的均匀不良导体棒。求之后不良导体棒上的温度变化。
  \ech
\end{frame}



\begin{frame}
  \chtitle{写出方程}
  \bch
  \addfig{3}{problem12-1.png}

  令$T(x,t) = T_1 + u(x,t)$,则$u$满足
  \bea
  \frac{\partial u}{\partial t} - a\frac{\partial^2 u}{\partial x^2} &=& 0, \newl
  u_{t=0} &=&  (T_0-T_1) \theta_L(x) .
  \eea
  其中$\theta_L(x)$当且仅当$|x|<L$时为$1$,否则为零。
  
  \ech
\end{frame}


\begin{frame}
  \chtitle{直接求解}
  \bch
  \bea
  u(x,t) &=& \int_{-L}^L \frac{1}{\sqrt{4\pi at}}e^{-\frac{(x-x_0)^2}{4at}} (T_0-T_1)dx_0 \newl
  &=& \frac{T_0-T_1}{\sqrt{4\pi at}} \int_{-L}^L e^{-\frac{(x-x_0)^2}{4at}} dx_0
  \eea
  当然,如果你喜欢,可以把上述积分写成误差函数。

  最后结果为
  $$T(x,t) = T_1 +  \frac{T_0-T_1}{\sqrt{4\pi at}} \int_{-L}^L e^{-\frac{(x-x_0)^2}{4at}} dx_0 $$
  \ech
\end{frame}


\begin{frame}
  \chtitle{思考题}
  \bch

  \addfig{1}{think3.jpg}

  在一根很长的不良导体棒一端输入热量$Q$,之后温度将如何变化?设初始温度$T_0$,截面积$S$,质量密度$\rho$, 单位质量比热$c$,导热系数$\lambda$均已知。
  
  \ech
\end{frame}

\section{Homework}

\begin{frame}
  \chtitle{课后作业(题号26)}
  \bch
  \bitem
\item[26]{  有长为$L$,温度为$T_0$的均匀不良导体棒。导热系数$\lambda$,质量密度$\rho$,单位质量比热$c$均已知。在$t=0$时刻,在它的一端$x= L$处接上温度为$T_1$的相同材质相同截面形状的非常长的均匀不良导体棒。求之后不良导体棒上的温度变化。

  \addfig{3}{problem12-2.png}
}
  \eitem
  \ech
\end{frame}

\end{document}
