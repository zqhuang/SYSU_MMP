\def\version{00000000}
\def\opts#1#2#3#4{({\bf A})\,{#1}\ \ ({\bf B})\,{#2}\ \ ({\bf C})\,{#3}\ \ ({\bf D})\,{#4}}
\documentclass[CJK]{article}
\usepackage{geometry,amssymb, amsmath}
\input{reduced_macros.tex}
\geometry{tmargin=0.3in, bmargin=0.4in, lmargin=0.7in, rmargin=0.8in, nohead, nofoot}
\begin{document}
\bch
\bcenter
第二届华南地区大学生天文竞赛 总赛 (共4页,满分100分) 

{\vskip 0.05in}

姓名 \uline{0.8} {\hskip 0.2in}  学校 \uline{1.5}{\hskip 0.2in} 学号 \uline{0.8}{\hskip 0.2in} 得分 \uline{0.5}

\ecenter

{\bf \noindent(一)选择题,每小题2分,共40分。请集中把答案写在下面的答案区:}

{\vskip 0.04in}

{(1) \uline{0.3}\ (2) \uline{0.3}\ (3) \uline{0.3}\ (4) \uline{0.3}\ (5) \uline{0.3}\ (6) \uline{0.3}\ (7) \uline{0.3}\ (8) \uline{0.3}\ (9) \uline{0.3}\ (10) \uline{0.23}}

{\vskip 0.04in}

{(11) \uline{0.23}\  (12) \uline{0.23}\ (13) \uline{0.23}\ (14) \uline{0.23}\ (15) \uline{0.23}\ (16) \uline{0.23}\ (17) \uline{0.23}\ (18) \uline{0.23}\ (19) \uline{0.23}\ (20) \uline{0.23}}

\bitem
\item[(1)]{日、月食只可能发生在哪些月相发生的时候

    \opts{望和朔}{上弦月和下弦月}{上弦月和望}{下弦月和望}
}    
\item[(2)]{下列哪个不属于类地行星?
  
  \opts{地球}{土星}{水星}{火星}
}
\item[(3)]{北极星属于哪个星座?

  \opts{大熊座}{小熊座}{双子座}{仙后座}
}
\item[(4)]{太阳黑子活动的周期大约是
  
  \opts{7年}{11年}{19年}{25年}
}
\item[(5)]{《春秋》记载“秋七月,有星孛入北斗”是世界上最早的关于\uline{0.5}的最早记录?

    \opts{哈雷彗星}{超新星爆发}{$\gamma$射线暴}{类星体}
}  
\item[(6)]{《周髀算经》是中国最古老的天文学和数学著作,一般认为它约成书于

  \opts{夏朝}{西周时期}{战国前后}{初唐}
}
\item[(7)]{我国古历中干支纪年的周期是

  \opts{十年}{十二年}{六十年}{一百二十年}
}
\item[(8)]{下列哪一颗恒星目前{\bf 不在}主序阶段?

    \opts{太阳}{牛郎星}{织女星}{北极星}
}    
\item[(9)]{根据现有恒星演化理论,太阳大概在多久以后变成红巨星

    \opts{1亿年}{20亿年}{100亿年}{500亿年}
}  
\item[(10)]{天狼星A的伴星,也就是天狼星B,亮度比天狼星A差10个星等,表面温度比太阳还高。它是一颗

    \opts{白矮星}{中子星}{黑洞}{红巨星}
}
\item[(11)]{著名的蟹状星云(NGC1952)是

    \opts{超新星爆发遗迹}{遥远的河外星系}{黑洞吸积盘}{恒星周围电离的气体和尘埃}
}  
\item[(12)]{一般认为,脉冲星是

    \opts{高速旋转的白矮星}{高速旋转的中子星}{高速旋转的黑洞}{表面不断膨胀收缩的恒星}
}
\item[(13)]{大麦哲伦云和小麦哲伦云绕我们银河系旋转。按照哈勃星系分类法,它们属于

    \opts{椭圆星系}{漩涡星系}{棒旋星系}{不规则星系}
}  
\item[(14)]{在宇宙学标准模型里,宇宙晚期的加速膨胀由什么造成?

    \opts{暗物质}{暗能量}{引力透镜效应}{宇宙微波背景辐射}
}  
\item[(15)]{可观测宇宙中,氦元素占所有元素的质量百分比大约为

    \opts{5\%}{25\%}{75\%}{95\%}
}
\item[(16)]{宇宙学标准模型预言了,但是目前还没有测量到的是

    \opts{宇宙微波背景辐射}{宇宙中微子背景辐射}{弱引力透镜}{重子声波振荡}
}  
\item[(17)]{下列哪一个望远镜是光学望远镜?

    \opts{郭守敬望远镜(LAMOST)}{平方千米阵(SKA)}{慧眼(HXMT)}{天眼(FAST)}
}  
\item[(18)]{一个直径6.5米的光学望远镜的衍射极限分辨率大约为

    \opts{0.02角秒}{0.2角秒}{2角秒}{20角秒}
}      
\item[(19)]{天文观测中的“蒙气差”,是指

  \opts{大气折射使天体看起来在天空中位置更高}{大气折射使天体看起来在天空中位置更低 \\}{大气散射使天体视位置发生随机移动}{大气散射使天体看起来模糊}
}
\item[(20)]{下列哪个天文学的成就获得了诺贝尔奖?
  
  \opts{爱因斯坦的广义相对论}{哈勃发现宇宙膨胀 \\}{霍金的黑洞辐射理论}{福勒的恒星化学元素形成理论}
}
  
\eitem

{\noindent {\bf (二)}
《周髀算经》中记载了用髀(直立竿)晷(在正午时的影子长度)标定地理位置的办法:“周髀长{\bf 八尺}……故冬至日晷{\bf 丈三尺五寸},夏至日晷{\bf 尺六寸}。”
 请由此估算当时的黄赤交角和测量地的纬度。{\blue (10分)}

{\vskip 4in}

\noindent {\bf(三)}
某行星的地质构造和平均密度都和地球相近。表面具有稳定的、温度和地球的大气相近的大气层,其主要成分是氦气。请根据这些信息(必要的话加上一些合理的假设)粗略地估算该行星的半径。 {\blue (10分)}

\newpage
}
姓名 \uline{0.8} {\hskip 0.2in}  学校 \uline{1.5}{\hskip 0.2in} 学号 \uline{0.8}{\hskip 0.2in}

{\noindent {\bf (四)}
把地球理想化为一个密度均匀的,半长轴(赤道半径)为 $6380\SIkm$,半短轴(南北极距离的一半)为 $6360\SIkm$ 的扁平状旋转椭球体 (由椭圆绕短轴旋转而成)。忽略潮汐和其他天体的影响,估算在赤道上的重力常数和在北极(或南极)的重力常数的相对差异。{\blue (20分)}
\newpage
}
{\noindent {\bf (五)}
量子力学的海森堡测不准原理告诉我们:无法同时完全确定粒子的位置$x$和动量$p$,位置和动量的不确定性的乘积有个下限:$\Delta x \Delta p \gtrsim h$,其中$h=6.63\times 10^{-34}\mathrm{J\cdot s}$是普朗克常量。

现在考虑一个质量为$M$的史瓦西黑洞。它的视界半径$R_H$是多大?{\blue (5分)}把该黑洞当成一个半径为$R_H$的球形封闭空间,假想该封闭空间区域内有一些测试光子,用海森堡测不准原理估算每个光子的最小可能动量的数量级。{\blue (5分)} 如果认为这些测试光子处于某种温度为$T$的热平衡状态,估算$T$的下限。 {\blue (5分)} 把黑洞的辐射当成黑体辐射,估算现在能观测到的来自早期宇宙的黑洞的质量下限。{\blue (5分)}

{\vskip 7.8in}
}


常数表:

万有引力常数 $G = 6.7\times 10^{-11}\mathrm{N\cdot m^2/kg^2}$

太阳质量 $M_\odot = 2.0\times 10^{30}\SIkg$

天文单位(日地平均距离)$1\mathrm{AU} = 1.5\times 10^{11}\SIm$

玻尔兹曼常数 $k_B=1.38\times 10^{-23}\mathrm{J/K}$
\bcenter
    {\scriptsize 试卷印刷号: \version}
\ecenter
\ech
\end{document}
