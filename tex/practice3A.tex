\documentclass[12pt,CJK]{article}
\usepackage{geometry}
\input{reduced_macros.tex}
\geometry{tmargin=0.3in, bmargin=0.5in, lmargin=0.5in, rmargin=0.9in, nohead, nofoot}
\def\mark#1{{\color{blue} (#1分)}}
\renewcommand{\thepage}{}
\begin{document}
\bch
{\large 数理方法 课堂小测III 初入江湖版}

{\vskip 0.2in}

特殊函数定义:
\bea
J_\nu(x) &=& \sum_{k=0}^\infty \frac{(-1)^k}{k!(k+\nu)!}\left(\frac{x}{2}\right)^{2k+\nu}; \newl
Y_\nu(x) &=& \lim_{\mu\rightarrow \nu}\frac{J_\mu(x)\cos\mu\pi - J_{-\mu}(x)}{\sin (\mu\pi)}; \newl
j_\ell(x) &=& \sqrt{\frac{\pi}{2x}} J_{\ell+1/2}(x),\ \ \ell\in Z; \newl
y_\ell(x) &=& \sqrt{\frac{\pi}{2x}} Y_{\ell+1/2}(x), \ \ \ell\in Z; \newl
P_{\ell}(x) &=& \sum_{k=0}^\ell \frac{(\ell + k)!}{(k!)^2(\ell-k)!}\left(\frac{x-1}{2}\right)^k,\ \ \ell\in Z, \ell\ge 0; \newl
Y_{\ell, m}(\theta,\phi) &=&\frac{1}{2^\ell \ell !}\sqrt{\frac{(2\ell+1)}{4\pi} \frac{(\ell-m)!}{(\ell+m)!}}\left[\sin^m\theta \left(\frac{1}{\sin\theta}\frac{d}{d\theta}\right)^{\ell+m}\sin^{2\ell}\theta \right] e^{\ii m\phi},\ \ \ell,m\in Z, \ell\ge|m|.
\eea

\bitem
\item[(一)]{选择题,每题3分,共30分。

  \bitem

\item[(1)]{下列哪个物理问题对应的方程{\bf 不是}波动方程 \bropt
  
  \foptlist{均匀弦的横向小振动}{均匀弹性杆的纵向小振动}{孤立均匀球的温度变化}{引力波的传播}}


\item[(2)]{下列哪个函数具有分离变量的形式 \bropt
  
  \optlist{$e^{-xt}$}{$\frac{x^2}{1+e^t}$}{$\sqrt{t^2+x^2}$}{$e^{\frac{x}{t}}$}}
  

\item[(3)]{下列关于格林函数的说法{\bf 错误}的是 \bropt
  
  \optlist{是线性系统对脉冲输入的响应}{依赖于脉冲输入的位置}{依赖于边界条件}{只适用于无限大空间的问题}}


\item[(4)]{球坐标系$(r,\theta,\phi)$的拉普拉斯算符为: \bropt
  
  \goptlist{$\frac{1}{r^2}\frac{\partial}{\partial r}\left(r^2\frac{\partial}{\partial r}\right) + \frac{1}{r^2\sin\theta}\frac{\partial}{\partial \theta}\left(\sin\theta\frac{\partial}{\partial\theta}\right) + \frac{1}{r^2\sin^2\theta}\frac{\partial^2}{\partial \phi^2} $}
           {$\frac{1}{r}\frac{\partial}{\partial r}\left(r\frac{\partial}{\partial r}\right) + \frac{1}{r^2}\frac{\partial^2}{\partial \theta^2} + \frac{1}{r^2\sin^2\theta}\frac{\partial^2}{\partial \phi^2} $}
           {$\frac{\partial^2}{\partial r^2} + \frac{\partial^2}{\partial \theta^2} +\frac{\partial^2}{\partial \phi^2}$}
           {$\frac{\partial^2}{\partial r^2} + \frac{1}{r^2}\frac{\partial^2}{\partial \theta^2} + \frac{1}{r^2\sin^2\theta}\frac{\partial^2}{\partial \phi^2}$}
}  

\item[(5)]{贝塞尔函数的{\bf 导函数} $J'_{100}(x)$ 的最小正实数零点大约为 \bropt

  \optlist{$1$}{$10$}{$100$}{$10^4$}}
  
\item[(6)]{ 下列哪个贝塞尔函数在 $x=0$ 处发散? \bropt
  
  \optlist{$J_1(x)$}{$J_{-1}(x)$}{$J_{1/2}(x)$}{$J_{-1/2}(x)$} }

\item[(7)]{ 积分$\int_0^a J_1(x) dx= $ \bropt
  
  \optlist{$J_0(a)$}{$1-J_0(a)$}{$aJ_0(a)$}{$a J_{1}(a)$} }

\item[(8)]{两类贝塞尔函数的平方和 $[J_{5}(100)]^2+[Y_{5}(100)]^2$ 大约为 \bropt

  \optlist{$0.006$}{$0.01$}{$0.06$}{$1$}}  

\item[(9)]{设$\mu, \nu$是球贝塞尔函数$j_\ell(x)$ ($\ell\ge 0$)的两个不同的零点,则下列哪个等式一定正确? \bropt
  
  \goptlist{$\int_0^1j_\ell(\mu x)j_\ell(\nu x)dx = 0 $}{ $\int_0^1j_\ell(\mu x)j_\ell(\nu x)xdx = 0 $}{$\int_0^1j_\ell(\mu x)j_\ell(\nu x)x^2dx = 0 $}{$\int_0^{1}(1-x^2)j_\ell(\mu x)j_\ell(\nu x)dx = 0 $}}
  
\item[(10)]{$\cos\theta\, Y_{4,2}(\theta,\phi)$ 可写成哪两个球面谐函数的线性组合? \bropt
  
  \foptlist{$ Y_{3,2}(\theta,\phi)$ 和 $Y_{5,2}(\theta,\phi)$}{$Y_{4,2}(\theta,\phi)$ 和 $Y_{5,2}(\theta,\phi)$}{$Y_{3,2}(\theta,\phi)$ 和 $Y_{4,2}(\theta,\phi)$}{$Y_{4,1}(\theta,\phi)$ 和 $Y_{4,3}(\theta,\phi)$}}  

\eitem  
}
\item[(二)]{填空题(每题5分,共30分)
  \bitem
\item[(1)]{设二维正交曲面坐标系$(x,y)$的正交线元长度依次为 $dx$ 和 $e^{-x^2}dy$,写出该坐标系的拉普拉斯算符的显式表达: $\nabla^2 f= $\uline{3}。}
\item[(2)]{积分$\int_0^1 x^4 P_6(x) dx = $ \uline{1}。}
\item[(3)]{函数 $f_1(x,y)$, $f_2(x,y)$ 在椭圆 $2x^2+ y^2 = 1$ 上的取值处处为零,在该椭圆内部分别满足

  $ \nabla^2 f_1 =-k_1^2f_1$ 以及 $\nabla^2 f_2 = -k_2^2f_2$,其中$k_1>k_2>0$。

  则在该椭圆内的积分$\iint f_1(x,y)f_2(x,y) dx dy = $ \uline{1}。}
\item[(4)]{设 $\mu>0$,$J_2(\mu)=0$,则积分$\int_0^1\left[J_2(\mu x)\right]^2 x dx =$ \uline{1}。}  
\item[(5)]{一维无边界的空间内满足方程 $\frac{\partial^2u}{\partial t^2}-a^2\frac{\partial^2u}{\partial x^2} = 0$ 和初始条件  $\left.u\right\vert_{t=0} = e^{-x^2},\ \ \left.\frac{\partial u}{\partial t}\right\vert_{t=0} = 0$ 的解是: $u(x,t)=$ \uline{3}。 }
\item[(6)]{一维无边界的空间内满足方程 $\frac{\partial u}{\partial t}-a\frac{\partial^2u}{\partial x^2} = 0$ 和初始条件  $\left.u\right\vert_{t=0} = e^{-x^2}$ 的解是: $u(x,t)=$ \uline{3}。 }

  \eitem
}
\item[(三)]{两端固定的,长度为$L$的均匀弦的横向振动$u(x,t)$ ($0\le x\le L$)满足初始条件:
  \bea
  \left.u\right\vert_{t=0} &=& A\sin \frac{\pi x}{L} \newl
  \left. \frac{\partial u}{\partial t}\right\vert_{t=0} &=& 0.
  \eea
  求解$u(x,t)$。设弦的线密度$\lambda$,张力$f$均已知。(20分)
}

\item[(四)]{有半径为 $R$,导热系数为 $\lambda$,单位质量比热为 $c$,质量密度为 $\rho$ 的孤立均匀薄圆盘。以圆盘中心为原点建立极坐标 $(r,\theta)$。初始时刻各点的温度为 $\left.T\right\vert_{t=0}= T_0\left(2 + \frac{r^2}{R^2}\cos\theta\right)$。计算之后圆盘上各点的温度变化。
(20分)}

  
\eitem  





\ech
\end{document}
