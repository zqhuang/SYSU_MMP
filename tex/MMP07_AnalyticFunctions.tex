\documentclass[CJK]{beamer}
\usepackage{CJKutf8}
\usepackage{beamerthemesplit}
\usetheme{Malmoe}
\useoutertheme[footline=authortitle]{miniframes}
\usepackage{amsmath}
\usepackage{amssymb}
\usepackage{graphicx}
\usepackage{eufrak}
\usepackage{color}
\usepackage{slashed}
\usepackage{simplewick}
\usepackage{tikz}
\usepackage{tcolorbox}
\graphicspath{{../figures/}}
%%figures
\def\lfig#1#2{\includegraphics[width=#1 in]{#2}}
\def\addfig#1#2{\begin{center}\includegraphics[width=#1 in]{#2}\end{center}}
\def\wulian{\includegraphics[width=0.18in]{emoji_wulian.jpg}}
\def\bigwulian{\includegraphics[width=0.35in]{emoji_wulian.jpg}}
\def\bye{\includegraphics[width=0.18in]{emoji_bye.jpg}}
\def\bigbye{\includegraphics[width=0.35in]{emoji_bye.jpg}}
\def\huaixiao{\includegraphics[width=0.18in]{emoji_huaixiao.jpg}}
\def\bighuaixiao{\includegraphics[width=0.35in]{emoji_huaixiao.jpg}}
\def\jianxiao{\includegraphics[width=0.18in]{emoji_jianxiao.jpg}}
\def\bigjianxiao{\includegraphics[width=0.35in]{emoji_jianxiao.jpg}}
%% colors
\def\blacktext#1{{\color{black}#1}}
\def\bluetext#1{{\color{blue}#1}}
\def\redtext#1{{\color{red}#1}}
\def\darkbluetext#1{{\color[rgb]{0,0.2,0.6}#1}}
\def\skybluetext#1{{\color[rgb]{0.2,0.7,1.}#1}}
\def\cyantext#1{{\color[rgb]{0.,0.5,0.5}#1}}
\def\greentext#1{{\color[rgb]{0,0.7,0.1}#1}}
\def\darkgray{\color[rgb]{0.2,0.2,0.2}}
\def\lightgray{\color[rgb]{0.6,0.6,0.6}}
\def\gray{\color[rgb]{0.4,0.4,0.4}}
\def\blue{\color{blue}}
\def\red{\color{red}}
\def\green{\color{green}}
\def\darkgreen{\color[rgb]{0,0.4,0.1}}
\def\darkblue{\color[rgb]{0,0.2,0.6}}
\def\skyblue{\color[rgb]{0.2,0.7,1.}}
%%control
\def\be{\begin{equation}}
\def\ee{\nonumber\end{equation}}
\def\bea{\begin{eqnarray}}
\def\eea{\nonumber\end{eqnarray}}
\def\bch{\begin{CJK}{UTF8}{gbsn}}
\def\ech{\end{CJK}}
\def\bitem{\begin{itemize}}
\def\eitem{\end{itemize}}
\def\bcenter{\begin{center}}
\def\ecenter{\end{center}}
\def\bex{\begin{minipage}{0.2\textwidth}\includegraphics[width=0.6in]{jugelizi.png}\end{minipage}\begin{minipage}{0.76\textwidth}}
\def\eex{\end{minipage}}
\def\chtitle#1{\frametitle{\bch#1\ech}}
\def\bmat#1{\left(\begin{array}{#1}}
\def\emat{\end{array}\right)}
\def\bcase#1{\left\{\begin{array}{#1}}
\def\ecase{\end{array}\right.}
\def\bmini#1{\begin{minipage}{#1\textwidth}}
\def\emini{\end{minipage}}
\def\tbox#1{\begin{tcolorbox}#1\end{tcolorbox}}
\def\pfrac#1#2#3{\left(\frac{\partial #1}{\partial #2}\right)_{#3}}
%%symbols
\def\bropt{\,(\ \ \ )}
\def\sone{$\star$}
\def\stwo{$\star\star$}
\def\sthree{$\star\star\star$}
\def\sfour{$\star\star\star\star$}
\def\sfive{$\star\star\star\star\star$}
\def\rint{{\int_\leftrightarrow}}
\def\roint{{\oint_\leftrightarrow}}
\def\stdHf{{\textit{\r H}_f}}
\def\deltaH{{\Delta \textit{\r H}}}
\def\ii{{\dot{\imath}}}
\def\skipline{{\vskip0.1in}}
\def\skiplines{{\vskip0.2in}}
\def\lagr{{\mathcal{L}}}
\def\hamil{{\mathcal{H}}}
\def\vecv{{\mathbf{v}}}
\def\vecx{{\mathbf{x}}}
\def\vecy{{\mathbf{y}}}
\def\veck{{\mathbf{k}}}
\def\vecp{{\mathbf{p}}}
\def\vecn{{\mathbf{n}}}
\def\vecA{{\mathbf{A}}}
\def\vecP{{\mathbf{P}}}
\def\vecsigma{{\mathbf{\sigma}}}
\def\hatJn{{\hat{J_\vecn}}}
\def\hatJx{{\hat{J_x}}}
\def\hatJy{{\hat{J_y}}}
\def\hatJz{{\hat{J_z}}}
\def\hatj#1{\hat{J_{#1}}}
\def\hatphi{{\hat{\phi}}}
\def\hatq{{\hat{q}}}
\def\hatpi{{\hat{\pi}}}
\def\vel{\upsilon}
\def\Dint{{\mathcal{D}}}
\def\adag{{\hat{a}^\dagger}}
\def\bdag{{\hat{b}^\dagger}}
\def\cdag{{\hat{c}^\dagger}}
\def\ddag{{\hat{d}^\dagger}}
\def\hata{{\hat{a}}}
\def\hatb{{\hat{b}}}
\def\hatc{{\hat{c}}}
\def\hatd{{\hat{d}}}
\def\hatN{{\hat{N}}}
\def\hatH{{\hat{H}}}
\def\hatp{{\hat{p}}}
\def\Fup{{F^{\mu\nu}}}
\def\Fdown{{F_{\mu\nu}}}
\def\newl{\nonumber \\}
\def\vece{\mathrm{e}}
\def\calM{{\mathcal{M}}}
\def\calT{{\mathcal{T}}}
\def\calR{{\mathcal{R}}}
\def\barpsi{\bar{\psi}}
\def\baru{\bar{u}}
\def\barv{\bar{\upsilon}}
\def\qeq{\stackrel{?}{=}}
\def\torder#1{\mathcal{T}\left(#1\right)}
\def\rorder#1{\mathcal{R}\left(#1\right)}
\def\contr#1#2{\contraction{}{#1}{}{#2}#1#2}
\def\trof#1{\mathrm{Tr}\left(#1\right)}
\def\trace{\mathrm{Tr}}
\def\comm#1{\ \ \ \left(\mathrm{used}\ #1\right)}
\def\tcomm#1{\ \ \ (\text{#1})}
\def\slp{\slashed{p}}
\def\slk{\slashed{k}}
\def\calp{{\mathfrak{p}}}
\def\veccalp{\mathbf{\mathfrak{p}}}
\def\Tthree{T_{\tiny \textcircled{3}}}
\def\pthree{p_{\tiny \textcircled{3}}}
\def\dbar{{\,\mathchar'26\mkern-12mu d}}
\def\erf{\mathrm{erf}}
\def\const{\mathrm{constant}}
\def\pheat{\pfrac p{\ln T}V}
\def\vheat{\pfrac V{\ln T}p}
%%units
\def\fdeg{{^\circ \mathrm{F}}}
\def\cdeg{^\circ \mathrm{C}}
\def\atm{\,\mathrm{atm}}
\def\angstrom{\,\text{\AA}}
\def\SIL{\,\mathrm{L}}
\def\SIkm{\,\mathrm{km}}
\def\SIyr{\,\mathrm{yr}}
\def\SIGyr{\,\mathrm{Gyr}}
\def\SIV{\,\mathrm{V}}
\def\SImV{\,\mathrm{mV}}
\def\SIeV{\,\mathrm{eV}}
\def\SIkeV{\,\mathrm{keV}}
\def\SIMeV{\,\mathrm{MeV}}
\def\SIGeV{\,\mathrm{GeV}}
\def\SIcal{\,\mathrm{cal}}
\def\SIkcal{\,\mathrm{kcal}}
\def\SImol{\,\mathrm{mol}}
\def\SIN{\,\mathrm{N}}
\def\SIHz{\,\mathrm{Hz}}
\def\SIm{\,\mathrm{m}}
\def\SIcm{\,\mathrm{cm}}
\def\SIfm{\,\mathrm{fm}}
\def\SImm{\,\mathrm{mm}}
\def\SInm{\,\mathrm{nm}}
\def\SImum{\,\mathrm{\mu m}}
\def\SIJ{\,\mathrm{J}}
\def\SIW{\,\mathrm{W}}
\def\SIkJ{\,\mathrm{kJ}}
\def\SIs{\,\mathrm{s}}
\def\SIkg{\,\mathrm{kg}}
\def\SIg{\,\mathrm{g}}
\def\SIK{\,\mathrm{K}}
\def\SImmHg{\,\mathrm{mmHg}}
\def\SIPa{\,\mathrm{Pa}}

\def\courseurl{https://github.com/zqhuang/SYSU\_TD}

\def\tpage#1#2{
\begin{frame}
\begin{center}
\begin{Large}
\bch
热学 \\
第#1讲 #2

{\vskip 0.3in}

黄志琦

\ech
\end{Large}
\end{center}

\vskip 0.2in

\bch
教材:《热学》第二版,赵凯华,罗蔚茵,高等教育出版社
\ech

\bch
课件下载
\ech
\courseurl
\end{frame}
}

\def\bfr#1{
\begin{frame}
\chtitle{#1} 
\bch
}

\def\efr{
\ech 
\end{frame}
}

  \date{}
  \begin{document}
  \bch
\tpage{7}{Analytic Functions}


\begin{frame}
\frametitle{Outline}
\tableofcontents
\end{frame}


\section{Differentiation}

\secpage{复变函数的导数}{跟实变函数没多大区别,但要小心多值函数}

\begin{frame}
  \frametitle{复变函数的导数}
  
  复变函数$f(z)$的导数还是定义为$f$的微小改变量和$z$的微小改变量之比:
  $$f'(z) = \frac{df}{dz} \equiv \lim_{\Delta z\rightarrow 0}\frac{f(z+\Delta z)-f(z)}{\Delta z}$$

  因为$\Delta z$可以从任何方向趋向于零,所以复变函数的可导性(即上述极限的存在性)是一个比较强的条件。
  
\end{frame}

\begin{frame}
  \frametitle{复变函数的导数}
  
  对$f(z)=z^2$,根据定义:
  $$f'(z) = \lim_{\Delta z\rightarrow 0}\frac{(z+\Delta z)^2-z^2}{\Delta z} = \lim_{\Delta z\rightarrow 0}(2z +\Delta z) = 2z $$
  
\end{frame}

\begin{frame}
  \frametitle{复变函数的导数}
  
  对$f(z)=\frac{1}{1+z}$,根据定义:

  \bea
  f'(z) &=& \lim_{\Delta z\rightarrow 0}\frac{\frac{1}{1+z+\Delta z}-\frac{1}{1+z}}{\Delta z} \nonumber \\
  &=& -\lim_{\Delta z\rightarrow 0}\frac{1}{(1+z)(1+z+\Delta z)} \nonumber \\
  &=& -\frac{1}{(1+z)^2} 
  \eea
  
\end{frame}

\begin{frame}
  
  {\Large {\blue 只涉及加减乘除}的复变函数的求导运算和实变函数完全相同。}
  
\end{frame}

\begin{frame}
  \frametitle{指数函数的导数}
  
  对$f(z) = e^z$按定义式逐项求导:
  $$f'(z) = \sum_{n=0}^\infty \frac{1}{n!} \frac{d(z^n)}{dz} = \sum_{n=1}^\infty \frac{nz^{n-1}}{n!} = \sum_{n=1}^\infty \frac{z^{n-1}}{(n-1)!} =\sum_{m=0}^\infty \frac{z^m}{m!}  = e^z$$
  
\end{frame}

\begin{frame}
  \frametitle{$\ln z$的连续变化观点和导数}
  
  当$z=re^{\ii \theta}$沿复平面上的一条(不经过原点的)曲线连续变化时,我们让幅角$\theta$也连续地变化(即不允许突然心血来潮加个$2\pi$什么的),这样$\ln z=\ln r+\ii\theta $也连续地发生变化。在这个意义下可以定义导数。

    \skiplines
    
   不出所料地:

 $$d\ln z = d\ln r + \ii d\theta = \frac{dr}{r} + \ii d\theta = \frac{e^{\ii\theta}dr + \ii re^{\ii\theta}d\theta}{re^{\ii\theta}} = \frac{dz}{z}$$
    
  
\end{frame}

\section{Integration}
\secpage{复变函数的积分}{原函数如果是多值函数就要小心了}

\begin{frame}
  \frametitle{复变函数的积分}
  
当$z$沿着一条(带方向的)分段光滑曲线$C$连续变化时,$z$的变化量$dz$和$f(z)$的乘积之和(当$dz$趋向于零时)的极限可以写成积分:
$$\int_C f(z) dz \equiv \lim_{dz\rightarrow 0}\sum_{z\in C} f(z) dz.$$

\skiplines

显然,如果把沿着曲线积分的方向改成相反,则积分结果相差个负号。
  
\end{frame}

\begin{frame}
  \frametitle{原函数}
  
  模仿实变函数的定积分技巧:

  \skipline
  
  如果能找到一个函数$F(z)$使得它的导数为$f(z)$,即
  $$f(z)dz = d F(z)$$
  那么$f(z)$沿曲线$C$的积分就只要计算从起点到终点$F(z)$的总改变量就可以了。

  $$\int f(z) dz = \Delta F$$

  $F(z)$称为$f(z)$的{\bf 原函数}。

  (要注意,原函数未必是存在的\bye)
  
\end{frame}

\begin{frame}
  \frametitle{把我们之前讨论的一切反过来写,就有}
  
  \bea
  \int z^n dz  &=& \frac{z^{n+1}}{n+1} + c,\ \ \  (n\in Z, n\ne -1) \newl
  \int e^z dz &=& e^z + c, \newl
  \int \frac{1}{z} dz  &=& \ln z + c,
  \eea
  当然,最后一个等式仍然需要遵循幅角连续变化的约定。
  
  
\end{frame}


\section{Analytic function}
\secpage{解析函数}{在任何圆内都是幂级数}


\begin{frame}
  \frametitle{内部,外部和边界}
  
  设复平面上有个点集$S$。
  
  \bmini{0.3}
  \lfig{1.3}{region.png}
  \emini
  \bmini{0.66}
  \bitem
  \item{
  内部的点$a$:可取足够小的$\delta$,使邻域$\{z: |z-a|<\delta\}$完全在$S$之内。}
  \item{外部的点$b$:可取足够小的$\delta$, 使邻域$\{z: |z-b|<\delta\}$完全在$S$之外。}
  \item{边界的点$c$:无论取多小的$\delta$,邻域$\{z: |z-c|<\delta\}$总是部分在$S$内,部分在$S$外。}
    \eitem
  \emini
  
\end{frame}


\begin{frame}
  \frametitle{边界的方向:左内法则}
  
  我们可以对复平面上的点集$S$的边界规定一个方向:当沿着边界的{\bf 正方向}移动,{内部的点}总在你的{\bf 左手边}。
  \addfig{2}{boundary_direction.png}
  
\end{frame}

\begin{frame}
  \frametitle{开区域和闭区域}
  
  \bitem
  \item
 { 如果点集$S$只包含内部的点,则称$S$为开区域。
  \bex
  ${z: 1<|z|<2}$
  是开区域
  \eex
 }
 \item{
  如果点集$S$包含所有边界上的点,则称$S$为闭区域。
  \bex
  ${z: 1\le |z|\le 2}$
  是闭区域
  \eex
 }
   \eitem
  
\end{frame}

\begin{frame}
  \frametitle{解析函数的定义}
  
  如果复变函数$f(z)$在某个开区域$S$内处处可导,则称$f(z)$为$S$上的{\bf 解析函数}。

  \skiplines
  
  (可导已经比较苛刻了,“处处可导”是更加苛刻的条件。)
  
\end{frame}


\begin{frame}
  \frametitle{思考题}
  
  \addfig{1}{think.jpg}
  
  在解析函数的定义里,为什么要限定处处可导的范围是开区域?换成闭区域行吗?
  
\end{frame}


\begin{frame}
  \frametitle{解析函数的例子}
  
  \bex
  $$f(z) = z^2$$
  是全复平面上的解析函数。
  \eex
  
\end{frame}

\begin{frame}
  \frametitle{解析函数的例子}
  
  \bex
  $$f(z) = \frac{z^2}{z+1}$$
  是全复平面挖去$z=-1$后的区域上的解析函数。
  \eex
  
\end{frame}

\begin{frame}
  \frametitle{解析函数的例子}
  
  \bex
  $$f(z) = e^z$$
  是全复平面上的解析函数。
  \eex
  
\end{frame}


\begin{frame}
  \frametitle{解析函数的例子}
  
  \bex
  $$f(z) = \ln z$$
  在任何可以限定幅角范围且不会导致幅角不连续变化的区域内,我们都可以通过限定幅角范围的方法来限定$\ln z$为连续的单值函数,在这个意义下$\ln z$是解析函数。

  \skiplines
  
  例如挖掉负实轴和原点的复平面满足上述条件。

  开圆$|z-1|<1$也满足上述条件。

  \eex
  
\end{frame}


\begin{frame}
  \frametitle{解析函数的反例}
  
  \bex
  $$f(z) = |z|$$
  处处不可导,所以在任何开区域内{\bf 不是}解析函数。
  \eex
  
\end{frame}


\begin{frame}
  \frametitle{解析函数的反例}
  
  \bex
  $$f(z) = \Arg z$$
  处处不可导,所以在任何开区域内{\bf 不是}解析函数。
  \eex

  \skiplines

{\scriptsize  注:幅角主值$\Arg z$一般约定取值范围为$(-\pi,\pi]$。}
  
\end{frame}

\begin{frame}
  \frametitle{解析函数的反例}
  
  \bex
  $$f(z) = z^2 \Arg z$$
  仅在$z=0$可导,无法在一个开区域内处处可导,所以在任何开区域内{\bf 不是}解析函数。
  \eex
  
\end{frame}


\begin{frame}
  \frametitle{dalao泰勒}
  
  \bcenter
  \lfig{1.5}{Taylor.jpg}

  Brook Taylor (1685-1731)
  \ecenter
  
\end{frame}

\begin{frame}
  \frametitle{泰勒展开: 一级近似}
  
  假设函数$f$在$z_0$无限次可导且各阶导数已知,我们想求$f$在$z_0$附近的一点$z$处的函数值。


  \skiplines
  
  按照物理学家的思路,先求个一阶近似:
  \begin{equation}
    f(z) \approx f(z_0) + f'(z_0)(z-z_0). \nonumber
  \end{equation}
  
  
\end{frame}

\begin{frame}
  \frametitle{泰勒级数: 二级近似}
  
  当然这样的一阶近似对导函数$f'$也成立,且对$z_0$附近的任何一点$\zeta$成立:
$$    f'(\zeta) \approx f'(z_0) + f''(z_0)(\zeta-z_0).  $$

  把上式对$\zeta$从$z_0$到$z$积分,得到
  $$     f(z) - f(z_0) \approx f'(z_0)(z-z_0) + \frac{f''(z_0)}{2} (z-z_0)^2 ,$$
  移项即得$f(z)$的二阶近似:
  \begin{equation}
    f(z) \approx f(z_0)  + f'(z_0)(z-z_0) + \frac{f''(z_0)}{2} (z-z_0)^2 \nonumber
  \end{equation}
  
  
\end{frame}


\begin{frame}
  \frametitle{泰勒级数: 三级近似}
  
  同理,这样的二阶近似对导函数$f'$也成立,且对$z_0$附近的任何一点$\zeta$成立:
$$    f'(\zeta) \approx f'(z_0) + f''(z_0)(\zeta-z_0) + \frac{f^{(3)}(z_0)}{2} (\zeta - z_0)^2.  $$

  把上式对$\zeta$从$z_0$到$z$积分,得到
  $$     f(z) - f(z_0)  \approx  f'(z_0)(z-z_0) + \frac{f''(z_0)}{2} (z-z_0)^2 + \frac{f^{(3)}(z_0)}{3!}(z-z_0)^3. $$
  移项即得$f(z)$的三级近似:
  \begin{equation}
    f(z)  \approx f(z_0) + f'(z_0)(z-z_0) + \frac{f''(z_0)}{2} (z-z_0)^2 + \frac{f^{(3)}(z_0)}{3!}(z-z_0)^3. \nonumber
  \end{equation}
  
  
\end{frame}


\begin{frame}
  \frametitle{泰勒级数}
  
  把这样的过程进行无限多次,就得到{\blue 泰勒展开公式:
  $$f(z) = \sum_{n=0}^\infty a_n(z-z_0)^n,$$
  其中$a_n = \frac{f^{(n)}(z_0)}{n!}$.}

  \skiplines

  (在数学严谨性上毫无节操的)物理学家对上述一系列操作表现得十分自信,但仍默默担心一件事情:这个级数有可能是发散的。

  例如:按上述操作得到的$\frac{1}{1-z} = 1+z+z^2+\ldots$就在$|z|>1$时明显发散。

  \skiplines
  
  好在对在整个圆内解析的函数而言,这个问题并不存在——
  
\end{frame}


\begin{frame}
  \frametitle{解析函数的本质: 任何圆内的泰勒级数收敛}
  
  {\blue \bf
  如果$f(z)$在圆$C_R$: $|z-z_0|< R$内解析,则$f(z)$在$C_R$内无限次可导,且泰勒级数收敛
  $$f(z) = \sum_{n=0}^\infty a_n(z-z_0)^n, \  \ \ \ z\in C_R.$$
  其中$a_n = \frac{f^{(n)}(z_0)}{n!}$.
  }
  
\end{frame}

\begin{frame}
  \frametitle{证明第一步:物理学家的直觉} 
  
      {\blue 设$f$是很小的简单闭合围道内的解析函数,且在围道上连续,则$f$沿这个围道的积分是围道包围的面积的高阶无穷小量。
        \lfig{0.8}{dSint.png}$\oint_C f(z)dz = o(dS)$
        }

  \bitem
  \item{  先估算主要部分:因为解析函数可导,在很小区域内可用线性函数来近似,任何线性函数($a+bz$)都存在单值的原函数($az+\frac{b}{2}z^2$),起点到终点的变化为零,所以积分严格为零。}

  \item {再估算误差:函数值的误差为围道的尺度的高阶无穷小量,再乘以$dz$积分后,误差为围道尺度平方的高阶无穷小量。对简单闭合围道而言,面积为尺度平方的量级,故得证。}
  \eitem
  
\end{frame}


\begin{frame}
  \frametitle{证明第二步:柯西定理} 
  
 {\bf \blue 如果$f$在开区域$T$上解析,在$T$的边界上连续,则$f$沿$T$的边界正方向的积分为零。}

 把子区域$T$划分成很多简单小区域$T_1$, $T_2$, $\ldots$, $T_N$,并在每个小区域的边界上沿正方向进行积分。相邻小区域在公共边界上(按左内原则)规定的边界方向相反,积分互相抵消,所以最后只剩下那些非公共边界(即$T$的边界)上的积分。也就是说,沿$T$的边界的积分等于沿所有小区域的边界的积分之和。

   \bmini{0.35}
   \lfig{1.5}{ctproof.png}
   \emini
   \bmini{0.61}
   另一方面,沿任何小区域的边界积分是小区域面积($\frac{1}{N}$量级)的高阶无穷小量。故当划分块数$N$趋向于无穷时,所有沿小区域边界的积分之和(即$N$个$\frac{1}{N}$的高阶无穷小量之和)趋向于零。故得证。
   \emini
  
\end{frame}

\begin{frame}
  \frametitle{dalao柯西}
  
  \bcenter
  \lfig{1.2}{Cauchy.jpg}

  Augustin-Louis Cauchy
  \ecenter
  
\end{frame}

\begin{frame}
  \frametitle{证明第三步:柯西积分公式} 
  
      {\bf \blue 如果$a$是区域$T$内一点,$f(z)$是$T$内的解析函数并在$T$边界上连续,则
        $$ f(a) = \frac{1}{2\pi \ii}\oint_{\partial T} \frac{f(z)}{z-a} dz $$
        其中$\oint_{\partial T}$表示沿$T$的边界的正方向积分。}

      \bmini{0.35}
      \lfig{1.5}{ciproof.png}
      \emini
      \bmini{0.61}
      如图,在$T$内以$a$为中心挖掉一个小圆孔。$\frac{f(z)}{z-a}$在剩下的区域内解析,故沿$T$的边界以及(小圆孔的边界)$C$的积分之和为零。

      而在$C$上积分时,可以把$\frac{f(z)}{z-a}$近似为$\frac{f(a)}{z-a}$ (请思考这样做带来的误差是否致命),积分一周原函数变化为
      $$f(a)\Delta\ln(z-a)  = -2\pi \ii f(a)$$
      故得证。
      \emini
  
\end{frame}

\begin{frame}
  \frametitle{证明第四步:推广的柯西积分公式} 
  
  柯西积分公式右边可以对$a$进行$n$次求导(请思考这样做的合法性),得到解析函数的$n$阶导数表达式:

  {\blue
        $$f^{(n)}(a) = \frac{n!}{2\pi \ii}\oint_{\partial T} \frac{f(z)}{(z-a)^{n+1}} dz$$    
  }
  这说明解析函数不仅可导,而且无限次可导。

  于是得到泰勒展开:
  $$f(z) = \sum_{n=0}^\infty \frac{f^{(n)}(z_0)}{n!} (z-z_0)^n$$
  
\end{frame}


\begin{frame}
  \frametitle{证明第五步:泰勒级数展开在圆内收敛} 
  
 对圆内任意一点$z$,设$|z-z_0|=s<r$,取介于$s$和$r$之间的正数$q$ ($s<q<r$),以$q$为半径取圆区域$C_q$: $|z-z_0|<q$,则$f(z)$在$C_q$内解析,边界上连续。于是
  $$ a_n = \frac{f^{(n)}(z_0)}{n!} = \frac{1}{2\pi \ii}\oint_{\partial C_q} \frac{f(z)}{(z-z_0)^{n+1}}dz $$
  设$|f(z)|$在圆周$\partial C_q$上的上界为$A$ (有限闭集上的连续实函数一定有上限,请思考为什么),则
  $$ |a_n| \le \frac{1}{2\pi} \int_0^{2\pi} \frac{A}{q^{n+1}} qd\theta = \frac{A}{q^n} $$
  因此把幂级数截断后的余项
  $$ \left\vert \sum_{n=N}^\infty a_n (z-z_0)^n \right\vert \le \sum_{n=N}^\infty \frac{A}{q^n}s^n =\frac{A}{1-\frac{s}{q}} \left(\frac{s}{q}\right)^N$$
  当$N\rightarrow$上式右边趋向于零,故得证。
  
\end{frame}


\begin{frame}
  \frametitle{思考题 (数学灵魂复苏……)}
  
  \addfig{1}{think2.jpg}
  如果在一个点$a$的任意小邻域内都有点集$S$里的无数个点,则称$a$为$S$的聚点。例如$0$是点集$\{ \frac{1}{n}, n\in Z^+\}$的聚点。


  试证明
  \bitem
\item[1]{\bf 在有限大小的闭区域内的无穷点集必然有聚点。}
\item[2]{\bf 有限大小的闭区域上的连续实函数必然有上下限。}
  \eitem
  
\end{frame}


\section{Homework}

\begin{frame}
\frametitle{Homework}

    {\small
      
\bitem
\item{仿照复数域上指数函数$e^z$的定义方法,用全复平面收敛的幂级数把正弦和余弦函数推广到复数域:
  $$\sin z \equiv \sum_{n=0}^\infty \frac{(-1)^nz^{2n+1}}{(2n+1)!};\ \ \ \cos z \equiv \sum_{n=0}^\infty \frac{(-1)^nz^{2n}}{(2n)!}.$$
  \bitem
  \item[(1)]{ 用指数函数来表示正弦和余弦函数;}
  \item[(2)]{ 求出满足$\sin z = 0$的全部复数解$z$。}
    \eitem
    }
\item{在复平面上画出同时满足$2<|z|<4$和$\reof{z}>-3$的区域$T$的边界$\partial T$,并标明边界的正方向。} 
\item{沿着上题的边界$\partial T$的正方向计算积分:
  $$\oint_{\partial T} \frac{\sin z}{(z-\pi)^4}dz .$$
}
\eitem
 }

\end{frame}

\ech

\end{document}
