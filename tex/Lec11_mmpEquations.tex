\documentclass[CJK]{beamer}
\usepackage{CJKutf8}
\usepackage{beamerthemesplit}
\usetheme{Malmoe}
\useoutertheme[footline=authortitle]{miniframes}
\usepackage{amsmath}
\usepackage{amssymb}
\usepackage{graphicx}
\usepackage{eufrak}
\usepackage{color}
\usepackage{slashed}
\usepackage{simplewick}
\usepackage{tikz}
\usepackage{tcolorbox}
\graphicspath{{../figures/}}
%%figures
\def\lfig#1#2{\includegraphics[width=#1 in]{#2}}
\def\addfig#1#2{\begin{center}\includegraphics[width=#1 in]{#2}\end{center}}
\def\wulian{\includegraphics[width=0.18in]{emoji_wulian.jpg}}
\def\bigwulian{\includegraphics[width=0.35in]{emoji_wulian.jpg}}
\def\bye{\includegraphics[width=0.18in]{emoji_bye.jpg}}
\def\bigbye{\includegraphics[width=0.35in]{emoji_bye.jpg}}
\def\huaixiao{\includegraphics[width=0.18in]{emoji_huaixiao.jpg}}
\def\bighuaixiao{\includegraphics[width=0.35in]{emoji_huaixiao.jpg}}
\def\jianxiao{\includegraphics[width=0.18in]{emoji_jianxiao.jpg}}
\def\bigjianxiao{\includegraphics[width=0.35in]{emoji_jianxiao.jpg}}
%% colors
\def\blacktext#1{{\color{black}#1}}
\def\bluetext#1{{\color{blue}#1}}
\def\redtext#1{{\color{red}#1}}
\def\darkbluetext#1{{\color[rgb]{0,0.2,0.6}#1}}
\def\skybluetext#1{{\color[rgb]{0.2,0.7,1.}#1}}
\def\cyantext#1{{\color[rgb]{0.,0.5,0.5}#1}}
\def\greentext#1{{\color[rgb]{0,0.7,0.1}#1}}
\def\darkgray{\color[rgb]{0.2,0.2,0.2}}
\def\lightgray{\color[rgb]{0.6,0.6,0.6}}
\def\gray{\color[rgb]{0.4,0.4,0.4}}
\def\blue{\color{blue}}
\def\red{\color{red}}
\def\green{\color{green}}
\def\darkgreen{\color[rgb]{0,0.4,0.1}}
\def\darkblue{\color[rgb]{0,0.2,0.6}}
\def\skyblue{\color[rgb]{0.2,0.7,1.}}
%%control
\def\be{\begin{equation}}
\def\ee{\nonumber\end{equation}}
\def\bea{\begin{eqnarray}}
\def\eea{\nonumber\end{eqnarray}}
\def\bch{\begin{CJK}{UTF8}{gbsn}}
\def\ech{\end{CJK}}
\def\bitem{\begin{itemize}}
\def\eitem{\end{itemize}}
\def\bcenter{\begin{center}}
\def\ecenter{\end{center}}
\def\bex{\begin{minipage}{0.2\textwidth}\includegraphics[width=0.6in]{jugelizi.png}\end{minipage}\begin{minipage}{0.76\textwidth}}
\def\eex{\end{minipage}}
\def\chtitle#1{\frametitle{\bch#1\ech}}
\def\bmat#1{\left(\begin{array}{#1}}
\def\emat{\end{array}\right)}
\def\bcase#1{\left\{\begin{array}{#1}}
\def\ecase{\end{array}\right.}
\def\bmini#1{\begin{minipage}{#1\textwidth}}
\def\emini{\end{minipage}}
\def\tbox#1{\begin{tcolorbox}#1\end{tcolorbox}}
\def\pfrac#1#2#3{\left(\frac{\partial #1}{\partial #2}\right)_{#3}}
%%symbols
\def\bropt{\,(\ \ \ )}
\def\sone{$\star$}
\def\stwo{$\star\star$}
\def\sthree{$\star\star\star$}
\def\sfour{$\star\star\star\star$}
\def\sfive{$\star\star\star\star\star$}
\def\rint{{\int_\leftrightarrow}}
\def\roint{{\oint_\leftrightarrow}}
\def\stdHf{{\textit{\r H}_f}}
\def\deltaH{{\Delta \textit{\r H}}}
\def\ii{{\dot{\imath}}}
\def\skipline{{\vskip0.1in}}
\def\skiplines{{\vskip0.2in}}
\def\lagr{{\mathcal{L}}}
\def\hamil{{\mathcal{H}}}
\def\vecv{{\mathbf{v}}}
\def\vecx{{\mathbf{x}}}
\def\vecy{{\mathbf{y}}}
\def\veck{{\mathbf{k}}}
\def\vecp{{\mathbf{p}}}
\def\vecn{{\mathbf{n}}}
\def\vecA{{\mathbf{A}}}
\def\vecP{{\mathbf{P}}}
\def\vecsigma{{\mathbf{\sigma}}}
\def\hatJn{{\hat{J_\vecn}}}
\def\hatJx{{\hat{J_x}}}
\def\hatJy{{\hat{J_y}}}
\def\hatJz{{\hat{J_z}}}
\def\hatj#1{\hat{J_{#1}}}
\def\hatphi{{\hat{\phi}}}
\def\hatq{{\hat{q}}}
\def\hatpi{{\hat{\pi}}}
\def\vel{\upsilon}
\def\Dint{{\mathcal{D}}}
\def\adag{{\hat{a}^\dagger}}
\def\bdag{{\hat{b}^\dagger}}
\def\cdag{{\hat{c}^\dagger}}
\def\ddag{{\hat{d}^\dagger}}
\def\hata{{\hat{a}}}
\def\hatb{{\hat{b}}}
\def\hatc{{\hat{c}}}
\def\hatd{{\hat{d}}}
\def\hatN{{\hat{N}}}
\def\hatH{{\hat{H}}}
\def\hatp{{\hat{p}}}
\def\Fup{{F^{\mu\nu}}}
\def\Fdown{{F_{\mu\nu}}}
\def\newl{\nonumber \\}
\def\vece{\mathrm{e}}
\def\calM{{\mathcal{M}}}
\def\calT{{\mathcal{T}}}
\def\calR{{\mathcal{R}}}
\def\barpsi{\bar{\psi}}
\def\baru{\bar{u}}
\def\barv{\bar{\upsilon}}
\def\qeq{\stackrel{?}{=}}
\def\torder#1{\mathcal{T}\left(#1\right)}
\def\rorder#1{\mathcal{R}\left(#1\right)}
\def\contr#1#2{\contraction{}{#1}{}{#2}#1#2}
\def\trof#1{\mathrm{Tr}\left(#1\right)}
\def\trace{\mathrm{Tr}}
\def\comm#1{\ \ \ \left(\mathrm{used}\ #1\right)}
\def\tcomm#1{\ \ \ (\text{#1})}
\def\slp{\slashed{p}}
\def\slk{\slashed{k}}
\def\calp{{\mathfrak{p}}}
\def\veccalp{\mathbf{\mathfrak{p}}}
\def\Tthree{T_{\tiny \textcircled{3}}}
\def\pthree{p_{\tiny \textcircled{3}}}
\def\dbar{{\,\mathchar'26\mkern-12mu d}}
\def\erf{\mathrm{erf}}
\def\const{\mathrm{constant}}
\def\pheat{\pfrac p{\ln T}V}
\def\vheat{\pfrac V{\ln T}p}
%%units
\def\fdeg{{^\circ \mathrm{F}}}
\def\cdeg{^\circ \mathrm{C}}
\def\atm{\,\mathrm{atm}}
\def\angstrom{\,\text{\AA}}
\def\SIL{\,\mathrm{L}}
\def\SIkm{\,\mathrm{km}}
\def\SIyr{\,\mathrm{yr}}
\def\SIGyr{\,\mathrm{Gyr}}
\def\SIV{\,\mathrm{V}}
\def\SImV{\,\mathrm{mV}}
\def\SIeV{\,\mathrm{eV}}
\def\SIkeV{\,\mathrm{keV}}
\def\SIMeV{\,\mathrm{MeV}}
\def\SIGeV{\,\mathrm{GeV}}
\def\SIcal{\,\mathrm{cal}}
\def\SIkcal{\,\mathrm{kcal}}
\def\SImol{\,\mathrm{mol}}
\def\SIN{\,\mathrm{N}}
\def\SIHz{\,\mathrm{Hz}}
\def\SIm{\,\mathrm{m}}
\def\SIcm{\,\mathrm{cm}}
\def\SIfm{\,\mathrm{fm}}
\def\SImm{\,\mathrm{mm}}
\def\SInm{\,\mathrm{nm}}
\def\SImum{\,\mathrm{\mu m}}
\def\SIJ{\,\mathrm{J}}
\def\SIW{\,\mathrm{W}}
\def\SIkJ{\,\mathrm{kJ}}
\def\SIs{\,\mathrm{s}}
\def\SIkg{\,\mathrm{kg}}
\def\SIg{\,\mathrm{g}}
\def\SIK{\,\mathrm{K}}
\def\SImmHg{\,\mathrm{mmHg}}
\def\SIPa{\,\mathrm{Pa}}

\def\courseurl{https://github.com/zqhuang/SYSU\_TD}

\def\tpage#1#2{
\begin{frame}
\begin{center}
\begin{Large}
\bch
热学 \\
第#1讲 #2

{\vskip 0.3in}

黄志琦

\ech
\end{Large}
\end{center}

\vskip 0.2in

\bch
教材:《热学》第二版,赵凯华,罗蔚茵,高等教育出版社
\ech

\bch
课件下载
\ech
\courseurl
\end{frame}
}

\def\bfr#1{
\begin{frame}
\chtitle{#1} 
\bch
}

\def\efr{
\ech 
\end{frame}
}

  \date{}
  \begin{document}
  \bch
\tpage{11}{数理方程综述}

\begin{frame}
  \frametitle{本讲内容}  
  \bitem
\item{数理方程综述}
\item{简单的数理方程求解范例}
  \eitem
\end{frame}

\section{Introduction}

\secpage{宇宙中的一切都是数理方程}{但是我们只讨论真空中的可线性迭加的球形奶牛}

\begin{frame}
  \frametitle{偏微分方程}
  我们经常会遇到对空间、时间坐标的偏微分算符和未知函数组合而成的偏微分方程。例如,静电学的泊松方程:
  $$ \nabla^2\phi = -\frac{\rho}{\epsilon_0};$$
  法拉第电磁感应定律:
  $$ \nabla\times \mathbf{E} = -\frac{\partial \mathbf{B}}{\partial t};$$
  热传导方程
  $$\frac{\partial u}{\partial t} - a\nabla^2 u = 0;$$
\end{frame}


\begin{frame}
  \frametitle{边界条件和初始条件}
  一个完整的物理问题除了运动方程之外,还有边界条件和初始条件。

  例如一根长为$L$的两端固定的弦的横向小振动$u(x, t)$ ($0\le x\le L, t\ge 0$) 满足波动方程
  $$\frac{\partial^2u}{\partial t^2} - a^2\frac{\partial^2u}{\partial x^2} = 0 .$$
  “两端固定”分别对应的边界条件:
  $$ \left.u\right\vert_{x=0} = 0;\ \left.u\right\vert_{x=L} = 0.$$
  初始时刻弦上各点的位置和速度就是初始条件:
  $$ \left.u\right\vert_{t=0} = \phi(x);\ \left.\frac{\partial u}{\partial t}\right\vert_{t=0} = \psi(x),$$
  其中$\phi(x)$和$\psi(x)$都是已知函数。
\end{frame}

\begin{frame}
  \frametitle{数理方程三要素,解存在且唯一性}
  
  {\blue 方程,边界条件,初始条件构成了数理方程的三个要素。}

  \skipline

  我们默认“给够了条件的”物理问题总是存在唯一解。
\end{frame}


\begin{frame}
  \frametitle{“线性”(可迭加性)}
  (为了减少本课重修人数,){\blue 我们只要求掌握线性问题或能简单转化为线性问题的情况。}

  “线性”是指这样的特性:如果$f$,$g$都是满足条件的解,则对任意常数$c_1$, $c_2$,$c_1f+c_2g$也是满足条件的解。

  \addfig{0.5}{think1.jpg}

  诶?刚刚不是假设了问题存在唯一解吗?“$f$, $g$都是满足条件的解”是神马操作?
  
\end{frame}


\begin{frame}
  \frametitle{退而求其次}
显然“方程+边界条件+初始条件”不是线性的。

\skipline

我们所说的“线性”,一般是指退而求其次,只要求 {\blue “方程+边界条件”线性}。

\end{frame}


\begin{frame}
  \frametitle{思考题}

  刚才我们写出的两端固定弦横向小振动的方程和边界条件是线性的吗?

\end{frame}


\begin{frame}
  \frametitle{求解线性问题的套路}

  满足方程+边界条件的解有很多个:$f_1$, $f_2$, \ldots (它们的任意线性组合仍然是满足方程+边界条件的解)。

  所以解决问题的思路很简单:

  \bitem
\item{先找出这么一堆$f_1$, $f_2$,\ldots;设它们分别对应的初始条件为$I_1$, $I_2$, \ldots}
\item{把初始条件$I$分解为$I_1$, $I_2$, \ldots 的线性组合;把$f_1$, $f_2$, \ldots 作同样的线性组合即得到初始条件$I$对应的解。}
  \eitem
  
\end{frame}

\section{Examples}

\begin{frame}
  \frametitle{练习1}
  求解两端固定弦的横振动$u(x, t)$ ($0\le x\le L, t\ge 0$):
  $$\frac{\partial^2u}{\partial t^2} - a^2\frac{\partial^2u}{\partial x^2} = 0 .$$
  $$ \left.u\right\vert_{x=0} = 0;\ \left.u\right\vert_{x=L} = 0.$$
  $$ \left.u\right\vert_{t=0} = A\sin\frac{\pi x}{L};\ \left.\frac{\partial u}{\partial t}\right\vert_{t=0} = 0.$$
  这里$a>0$, $A>0$都是常量。$L$是弦的长度。
\end{frame}



\begin{frame}
  \frametitle{练习2}
  求解两端固定弦的横振动$u(x, t)$ ($0\le x\le L, t\ge 0$):
  $$\frac{\partial^2u}{\partial t^2} - a^2\frac{\partial^2u}{\partial x^2} = 0 .$$
  $$ \left.u\right\vert_{x=0} = 0;\ \left.u\right\vert_{x=L} = 0.$$
  $$ \left.u\right\vert_{t=0} = A\sin\frac{2\pi x}{L}\cos\frac{\pi x}{L};\ \left.\frac{\partial u}{\partial t}\right\vert_{t=0} = 0.$$
  这里$a>0$, $A>0$都是常量。$L$是弦的长度。
\end{frame}


\begin{frame}
  \frametitle{练习3}
  求解两端固定弦的横振动$u(x, t)$ ($0\le x\le L, t\ge 0$):
  $$\frac{\partial^2u}{\partial t^2} - a^2\frac{\partial^2u}{\partial x^2} = 0 .$$
  $$ \left.u\right\vert_{x=0} = 0;\ \left.u\right\vert_{x=L} = 0.$$
  $$ \left.u\right\vert_{t=0} = Ax(1-x);\ \left.\frac{\partial u}{\partial t}\right\vert_{t=0} = 0.$$
  这里$a>0$, $A>0$都是常量。$L$是弦的长度。
\end{frame}

\begin{frame}
  \frametitle{练习4}
  求解两端固定弦的横振动$u(x, t)$ ($0\le x\le L, t\ge 0$):
  $$\frac{\partial^2u}{\partial t^2} - a^2\frac{\partial^2u}{\partial x^2} = 0 .$$
  $$ \left.u\right\vert_{x=0} = 0;\ \left.u\right\vert_{x=L} = 0.$$
  $$ \left.u\right\vert_{t=0} = \delta(x-x_0);\ \left.\frac{\partial u}{\partial t}\right\vert_{t=0} = 0.$$
  这里$L$是弦的长度;$a>0$, $0<x_0<L$都是常量。
\end{frame}


\begin{frame}
  \frametitle{练习5}
  求解两端固定弦的横振动$u(x, t)$ ($0\le x\le L, t\ge 0$):
  $$\frac{\partial^2u}{\partial t^2} - a^2\frac{\partial^2u}{\partial x^2} = 0 .$$
  $$ \left.u\right\vert_{x=0} = 0;\ \left.u\right\vert_{x=L} = 0.$$
  $$ \left.u\right\vert_{t=0} = \phi(x);\ \left.\frac{\partial u}{\partial t}\right\vert_{t=0} = 0.$$
  这里$a$是常量;$L$是弦的长度;$\phi(x)$是已知函数。

  \skipline
  你能用两种不同的方法求解该问题吗?
\end{frame}


\begin{frame}
  \frametitle{Homework for Quizphobias}
  {\small
    \bitem
  \item[31]{初始条件和边界条件的区别是什么?}
  \item[32]{简要推导弦的横向小振动方程(可参考教材);解释$a$的物理意义。}
  \item[33]{求解两端固定弦的横振动$u(x, t)$ ($0\le x\le L, t\ge 0$):
  $$\frac{\partial^2u}{\partial t^2} - a^2\frac{\partial^2u}{\partial x^2} = 0 .$$
  $$ \left.u\right\vert_{x=0} = 0;\ \left.u\right\vert_{x=L} = 0.$$
  $$ \left.u\right\vert_{t=0} = 0;\ \left.\frac{\partial u}{\partial t}\right\vert_{t=0} = Ax(1-x).$$
    这里$a>0$, $A>0$都是常量。$L$是弦的长度。}
    \eitem}
\end{frame}

\ech
\end{document}
