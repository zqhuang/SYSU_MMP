\documentclass[CJK]{beamer}
\usepackage{CJKutf8}
\usepackage{beamerthemesplit}
\usetheme{Malmoe}
\useoutertheme[footline=authortitle]{miniframes}
\usepackage{amsmath}
\usepackage{amssymb}
\usepackage{graphicx}
\usepackage{eufrak}
\usepackage{color}
\usepackage{slashed}
\usepackage{simplewick}
\usepackage{tikz}
\usepackage{tcolorbox}
\graphicspath{{../figures/}}
%%figures
\def\lfig#1#2{\includegraphics[width=#1 in]{#2}}
\def\addfig#1#2{\begin{center}\includegraphics[width=#1 in]{#2}\end{center}}
\def\wulian{\includegraphics[width=0.18in]{emoji_wulian.jpg}}
\def\bigwulian{\includegraphics[width=0.35in]{emoji_wulian.jpg}}
\def\bye{\includegraphics[width=0.18in]{emoji_bye.jpg}}
\def\bigbye{\includegraphics[width=0.35in]{emoji_bye.jpg}}
\def\huaixiao{\includegraphics[width=0.18in]{emoji_huaixiao.jpg}}
\def\bighuaixiao{\includegraphics[width=0.35in]{emoji_huaixiao.jpg}}
\def\jianxiao{\includegraphics[width=0.18in]{emoji_jianxiao.jpg}}
\def\bigjianxiao{\includegraphics[width=0.35in]{emoji_jianxiao.jpg}}
%% colors
\def\blacktext#1{{\color{black}#1}}
\def\bluetext#1{{\color{blue}#1}}
\def\redtext#1{{\color{red}#1}}
\def\darkbluetext#1{{\color[rgb]{0,0.2,0.6}#1}}
\def\skybluetext#1{{\color[rgb]{0.2,0.7,1.}#1}}
\def\cyantext#1{{\color[rgb]{0.,0.5,0.5}#1}}
\def\greentext#1{{\color[rgb]{0,0.7,0.1}#1}}
\def\darkgray{\color[rgb]{0.2,0.2,0.2}}
\def\lightgray{\color[rgb]{0.6,0.6,0.6}}
\def\gray{\color[rgb]{0.4,0.4,0.4}}
\def\blue{\color{blue}}
\def\red{\color{red}}
\def\green{\color{green}}
\def\darkgreen{\color[rgb]{0,0.4,0.1}}
\def\darkblue{\color[rgb]{0,0.2,0.6}}
\def\skyblue{\color[rgb]{0.2,0.7,1.}}
%%control
\def\be{\begin{equation}}
\def\ee{\nonumber\end{equation}}
\def\bea{\begin{eqnarray}}
\def\eea{\nonumber\end{eqnarray}}
\def\bch{\begin{CJK}{UTF8}{gbsn}}
\def\ech{\end{CJK}}
\def\bitem{\begin{itemize}}
\def\eitem{\end{itemize}}
\def\bcenter{\begin{center}}
\def\ecenter{\end{center}}
\def\bex{\begin{minipage}{0.2\textwidth}\includegraphics[width=0.6in]{jugelizi.png}\end{minipage}\begin{minipage}{0.76\textwidth}}
\def\eex{\end{minipage}}
\def\chtitle#1{\frametitle{\bch#1\ech}}
\def\bmat#1{\left(\begin{array}{#1}}
\def\emat{\end{array}\right)}
\def\bcase#1{\left\{\begin{array}{#1}}
\def\ecase{\end{array}\right.}
\def\bmini#1{\begin{minipage}{#1\textwidth}}
\def\emini{\end{minipage}}
\def\tbox#1{\begin{tcolorbox}#1\end{tcolorbox}}
\def\pfrac#1#2#3{\left(\frac{\partial #1}{\partial #2}\right)_{#3}}
%%symbols
\def\bropt{\,(\ \ \ )}
\def\sone{$\star$}
\def\stwo{$\star\star$}
\def\sthree{$\star\star\star$}
\def\sfour{$\star\star\star\star$}
\def\sfive{$\star\star\star\star\star$}
\def\rint{{\int_\leftrightarrow}}
\def\roint{{\oint_\leftrightarrow}}
\def\stdHf{{\textit{\r H}_f}}
\def\deltaH{{\Delta \textit{\r H}}}
\def\ii{{\dot{\imath}}}
\def\skipline{{\vskip0.1in}}
\def\skiplines{{\vskip0.2in}}
\def\lagr{{\mathcal{L}}}
\def\hamil{{\mathcal{H}}}
\def\vecv{{\mathbf{v}}}
\def\vecx{{\mathbf{x}}}
\def\vecy{{\mathbf{y}}}
\def\veck{{\mathbf{k}}}
\def\vecp{{\mathbf{p}}}
\def\vecn{{\mathbf{n}}}
\def\vecA{{\mathbf{A}}}
\def\vecP{{\mathbf{P}}}
\def\vecsigma{{\mathbf{\sigma}}}
\def\hatJn{{\hat{J_\vecn}}}
\def\hatJx{{\hat{J_x}}}
\def\hatJy{{\hat{J_y}}}
\def\hatJz{{\hat{J_z}}}
\def\hatj#1{\hat{J_{#1}}}
\def\hatphi{{\hat{\phi}}}
\def\hatq{{\hat{q}}}
\def\hatpi{{\hat{\pi}}}
\def\vel{\upsilon}
\def\Dint{{\mathcal{D}}}
\def\adag{{\hat{a}^\dagger}}
\def\bdag{{\hat{b}^\dagger}}
\def\cdag{{\hat{c}^\dagger}}
\def\ddag{{\hat{d}^\dagger}}
\def\hata{{\hat{a}}}
\def\hatb{{\hat{b}}}
\def\hatc{{\hat{c}}}
\def\hatd{{\hat{d}}}
\def\hatN{{\hat{N}}}
\def\hatH{{\hat{H}}}
\def\hatp{{\hat{p}}}
\def\Fup{{F^{\mu\nu}}}
\def\Fdown{{F_{\mu\nu}}}
\def\newl{\nonumber \\}
\def\vece{\mathrm{e}}
\def\calM{{\mathcal{M}}}
\def\calT{{\mathcal{T}}}
\def\calR{{\mathcal{R}}}
\def\barpsi{\bar{\psi}}
\def\baru{\bar{u}}
\def\barv{\bar{\upsilon}}
\def\qeq{\stackrel{?}{=}}
\def\torder#1{\mathcal{T}\left(#1\right)}
\def\rorder#1{\mathcal{R}\left(#1\right)}
\def\contr#1#2{\contraction{}{#1}{}{#2}#1#2}
\def\trof#1{\mathrm{Tr}\left(#1\right)}
\def\trace{\mathrm{Tr}}
\def\comm#1{\ \ \ \left(\mathrm{used}\ #1\right)}
\def\tcomm#1{\ \ \ (\text{#1})}
\def\slp{\slashed{p}}
\def\slk{\slashed{k}}
\def\calp{{\mathfrak{p}}}
\def\veccalp{\mathbf{\mathfrak{p}}}
\def\Tthree{T_{\tiny \textcircled{3}}}
\def\pthree{p_{\tiny \textcircled{3}}}
\def\dbar{{\,\mathchar'26\mkern-12mu d}}
\def\erf{\mathrm{erf}}
\def\const{\mathrm{constant}}
\def\pheat{\pfrac p{\ln T}V}
\def\vheat{\pfrac V{\ln T}p}
%%units
\def\fdeg{{^\circ \mathrm{F}}}
\def\cdeg{^\circ \mathrm{C}}
\def\atm{\,\mathrm{atm}}
\def\angstrom{\,\text{\AA}}
\def\SIL{\,\mathrm{L}}
\def\SIkm{\,\mathrm{km}}
\def\SIyr{\,\mathrm{yr}}
\def\SIGyr{\,\mathrm{Gyr}}
\def\SIV{\,\mathrm{V}}
\def\SImV{\,\mathrm{mV}}
\def\SIeV{\,\mathrm{eV}}
\def\SIkeV{\,\mathrm{keV}}
\def\SIMeV{\,\mathrm{MeV}}
\def\SIGeV{\,\mathrm{GeV}}
\def\SIcal{\,\mathrm{cal}}
\def\SIkcal{\,\mathrm{kcal}}
\def\SImol{\,\mathrm{mol}}
\def\SIN{\,\mathrm{N}}
\def\SIHz{\,\mathrm{Hz}}
\def\SIm{\,\mathrm{m}}
\def\SIcm{\,\mathrm{cm}}
\def\SIfm{\,\mathrm{fm}}
\def\SImm{\,\mathrm{mm}}
\def\SInm{\,\mathrm{nm}}
\def\SImum{\,\mathrm{\mu m}}
\def\SIJ{\,\mathrm{J}}
\def\SIW{\,\mathrm{W}}
\def\SIkJ{\,\mathrm{kJ}}
\def\SIs{\,\mathrm{s}}
\def\SIkg{\,\mathrm{kg}}
\def\SIg{\,\mathrm{g}}
\def\SIK{\,\mathrm{K}}
\def\SImmHg{\,\mathrm{mmHg}}
\def\SIPa{\,\mathrm{Pa}}

\def\courseurl{https://github.com/zqhuang/SYSU\_TD}

\def\tpage#1#2{
\begin{frame}
\begin{center}
\begin{Large}
\bch
热学 \\
第#1讲 #2

{\vskip 0.3in}

黄志琦

\ech
\end{Large}
\end{center}

\vskip 0.2in

\bch
教材:《热学》第二版,赵凯华,罗蔚茵,高等教育出版社
\ech

\bch
课件下载
\ech
\courseurl
\end{frame}
}

\def\bfr#1{
\begin{frame}
\chtitle{#1} 
\bch
}

\def\efr{
\ech 
\end{frame}
}

  \date{}
\begin{document}
\tpage{2}{Risidue Theorem}


\begin{frame}
\chtitle{本讲内容}
\bch
\bitem
\item{上讲内容摘要}
\item{洛朗展开}
\item{留数定理}
\eitem
\ech
\end{frame}

%\begin{frame}
%  \chtitle{据说你们学过积分, 下面的积分你们会几个?}
%  \bch
%  \bitem
%\item{$$\int_{-\infty}^{\infty}e^{-x^2}dx$$}
%\item{$$\int_{0}^{\pi/2}\ln{\left(\sin x\right)}dx$$}
%\item{$$\int_{0}^{\infty}\frac{\sin x}{x} dx$$}
%\item{$$\int_{0}^{\infty}\frac{\sin x}{e^x-1} dx$$}
%\item{$$\int_{0}^{\infty}\frac{x}{e^x+1} dx$$}    
%  \eitem
%  \ech
%\end{frame}

\section{Laurant Series}

\begin{frame}
  \chtitle{上讲内容摘要}
  \bch
  柯西定理:{\bf \blue 解析函数沿解析区域边界上的积分为零。}
  (当然,要求函数在边界上连续)。
  
  \addfig{1.5}{simplecontour.png}

  $$\oint_C f(z) dz  = 0 $$
  \ech
\end{frame}


\begin{frame}
  \chtitle{上讲内容摘要}
  \bch
  柯西积分公式:{\bf \blue 解析函数在解析区域内任意一点的任意阶导数都可以用边界上的积分表示出来。}
  \addfig{1.5}{simplecontourwithz0.png}
  
  $$\frac{f^{(n)}(z_0)}{n!} =  \frac{1}{2\pi \ii}\oint_C \frac{f(z)}{(z-z_0)^{n+1}}dz$$
  $C$可以是包含$z_0$的任意解析区域的边界(当然,要求函数在边界上连续)。
  \ech
\end{frame}

\begin{frame}
  \chtitle{上讲内容摘要}
  \bch
  {\bf \blue 在一个圆内解析的函数可以看成在圆心展开的泰勒级数。}
  设$f(z)$在圆$|z-z_0|<R$ (这里允许$R=\infty$)内解析,则
  $$f(z) = \sum_{n=0}^{\infty} a_n (z-z_0)^n, |z-z_0|<R.$$
  其中
  $$a_n = \frac{f^{(n)}(z_0)}{n!} =  \frac{1}{2\pi \ii}\oint_C \frac{f(z)}{(z-z_0)^{n+1}}dz$$
  积分路径$C$可以取解析区域内逆时针方向围绕$z_0$一周的简单围道。
  \ech
\end{frame}


\begin{frame}
  \chtitle{思考题}
  \bch
  \addfig{1}{think2.jpg}
  
  柯西积分公式
    $$\frac{f^{(n)}(z_0)}{n!} =  \frac{1}{2\pi \ii}\oint_{\partial T} \frac{f(z)}{(z-z_0)^{n+1}}dz$$
  等式右边是沿边界的积分,按柯西定理结果恒为零?感觉哪里不对了…
  \ech
\end{frame}



\begin{frame}
  \chtitle{思考题}
  \bch

  一个比较直观的说法就是,{\blue 解析函数就是能用加减乘除表示出来的函数}。而诸如$|z|$, $\Arg z$之类的函数因为无法用$z$的加减乘除表示出来,所以不是解析函数。(这是快速寻求第一感的方法,不能作为证明。)
  \addfig{1}{think2.jpg}


  那么问题来了,为什么在圆内解析的函数用加减乘除表示出来(即泰勒展开)之后,只包含$z-z_0$的非负次幂项,而不包含$(z-z_0)^{-1}, (z-z_0)^{-2},\ldots$等负次幂项?
  \ech
\end{frame}


\begin{frame}
  \chtitle{在一个环里解析的函数: 洛朗展开定理}
  \bch

  {\blue
  设$f(z)$在环区域$r<|z-z_0|<R$ (这里允许$r=0$和$R=\infty$)内解析,则
  $$f(z) = \sum_{n=-\infty}^\infty a_n (z-z_0)^n,$$
  其中
  $$a_n = \frac{1}{2\pi\ii}\oint_C\frac{f(\zeta)}{(\zeta-z_0)^{n+1}}d\zeta,$$
  积分路径$C$可以是环内任意的一条逆时针绕$z_0$一周的分段光滑曲线。}

  这个展开式称为洛朗展开。注意和泰勒展开不同,指标的求和范围从$-\infty$到$\infty$。而且$a_n$不再和$f$在$z_0$处的$n$阶导数相关(\bye $z_0$根本就不在函数的定义域内)。

  \ech
\end{frame}

\begin{frame}
  \chtitle{第一讲``五步曲''最后一步的另一种证法}
  \bch
  {\small   
    对$|z-z_0|=s<r$,取$s<q<r$并以$q$为半径作圆区域$C_q$:$|z-z_0| < q$。根据柯西积分公式:
  \bea
  f(z) = \frac{1}{2\pi\ii}\oint_{\partial C_q}\frac{f(\zeta)}{\zeta - z} d\zeta   &=& \frac{1}{2\pi\ii}\oint_{\partial  C_q}\frac{f(\zeta)}{(\zeta - z_0) - (z-z_0)} d\zeta \newl
  &=& \frac{1}{2\pi\ii}\oint_{\partial C_q}\frac{1}{\zeta-z_0}\frac{f(\zeta)}{1 - \frac{z-z_0}{\zeta-z_0}} d\zeta \newl
  &=& \frac{1}{2\pi\ii}\oint_{\partial C_q}\frac{1}{\zeta-z_0}f(\zeta)\sum_{n=0}^\infty\left(\frac{z-z_0}{\zeta-z_0}\right)^n d\zeta \newl
  &=& \sum_{n=0}^\infty a_n(z-z_0)^n  
  \eea
  其中$a_n = \frac{1}{2\pi\ii}\oint_{\partial  C_q}\frac{f(\zeta)}{(\zeta-z_0)^{n+1}}d\zeta.$

  请用类似的方法证明洛朗展开定理。
  }
  \ech
\end{frame}

\begin{frame}
  \chtitle{求洛朗展开的例子}
  \bch
  $f(z) = \frac{1}{z(1+z)}$在环区域$0<|z|<1$内解析,洛朗展开为:
  
  \bea
  \frac{1}{z(1+z)} &=& \frac{1}{z}\,\cdot\, \frac{1}{1+z} \newl
  &=&\frac{1}{z}\left(1-z+z^2-z^3+\ldots\right) \newl
  &=& \frac{1}{z} - 1 + z - z^2 + \ldots
  \eea

  级数展开的很多技巧我们之后会进行专门训练,先不展开讨论。
  \ech
\end{frame}


\section{Residue Theorem}

\begin{frame}
  \chtitle{留数定理}
  \bch
  
  \ech
\end{frame}


\section{Homework}



\end{document}
