\documentclass[CJK]{beamer}
\usepackage{CJKutf8}
\usepackage{beamerthemesplit}
\usetheme{Malmoe}
\useoutertheme[footline=authortitle]{miniframes}
\usepackage{amsmath}
\usepackage{amssymb}
\usepackage{graphicx}
\usepackage{eufrak}
\usepackage{color}
\usepackage{slashed}
\usepackage{simplewick}
\usepackage{tikz}
\usepackage{tcolorbox}
\graphicspath{{../figures/}}
%%figures
\def\lfig#1#2{\includegraphics[width=#1 in]{#2}}
\def\addfig#1#2{\begin{center}\includegraphics[width=#1 in]{#2}\end{center}}
\def\wulian{\includegraphics[width=0.18in]{emoji_wulian.jpg}}
\def\bigwulian{\includegraphics[width=0.35in]{emoji_wulian.jpg}}
\def\bye{\includegraphics[width=0.18in]{emoji_bye.jpg}}
\def\bigbye{\includegraphics[width=0.35in]{emoji_bye.jpg}}
\def\huaixiao{\includegraphics[width=0.18in]{emoji_huaixiao.jpg}}
\def\bighuaixiao{\includegraphics[width=0.35in]{emoji_huaixiao.jpg}}
\def\jianxiao{\includegraphics[width=0.18in]{emoji_jianxiao.jpg}}
\def\bigjianxiao{\includegraphics[width=0.35in]{emoji_jianxiao.jpg}}
%% colors
\def\blacktext#1{{\color{black}#1}}
\def\bluetext#1{{\color{blue}#1}}
\def\redtext#1{{\color{red}#1}}
\def\darkbluetext#1{{\color[rgb]{0,0.2,0.6}#1}}
\def\skybluetext#1{{\color[rgb]{0.2,0.7,1.}#1}}
\def\cyantext#1{{\color[rgb]{0.,0.5,0.5}#1}}
\def\greentext#1{{\color[rgb]{0,0.7,0.1}#1}}
\def\darkgray{\color[rgb]{0.2,0.2,0.2}}
\def\lightgray{\color[rgb]{0.6,0.6,0.6}}
\def\gray{\color[rgb]{0.4,0.4,0.4}}
\def\blue{\color{blue}}
\def\red{\color{red}}
\def\green{\color{green}}
\def\darkgreen{\color[rgb]{0,0.4,0.1}}
\def\darkblue{\color[rgb]{0,0.2,0.6}}
\def\skyblue{\color[rgb]{0.2,0.7,1.}}
%%control
\def\be{\begin{equation}}
\def\ee{\nonumber\end{equation}}
\def\bea{\begin{eqnarray}}
\def\eea{\nonumber\end{eqnarray}}
\def\bch{\begin{CJK}{UTF8}{gbsn}}
\def\ech{\end{CJK}}
\def\bitem{\begin{itemize}}
\def\eitem{\end{itemize}}
\def\bcenter{\begin{center}}
\def\ecenter{\end{center}}
\def\bex{\begin{minipage}{0.2\textwidth}\includegraphics[width=0.6in]{jugelizi.png}\end{minipage}\begin{minipage}{0.76\textwidth}}
\def\eex{\end{minipage}}
\def\chtitle#1{\frametitle{\bch#1\ech}}
\def\bmat#1{\left(\begin{array}{#1}}
\def\emat{\end{array}\right)}
\def\bcase#1{\left\{\begin{array}{#1}}
\def\ecase{\end{array}\right.}
\def\bmini#1{\begin{minipage}{#1\textwidth}}
\def\emini{\end{minipage}}
\def\tbox#1{\begin{tcolorbox}#1\end{tcolorbox}}
\def\pfrac#1#2#3{\left(\frac{\partial #1}{\partial #2}\right)_{#3}}
%%symbols
\def\bropt{\,(\ \ \ )}
\def\sone{$\star$}
\def\stwo{$\star\star$}
\def\sthree{$\star\star\star$}
\def\sfour{$\star\star\star\star$}
\def\sfive{$\star\star\star\star\star$}
\def\rint{{\int_\leftrightarrow}}
\def\roint{{\oint_\leftrightarrow}}
\def\stdHf{{\textit{\r H}_f}}
\def\deltaH{{\Delta \textit{\r H}}}
\def\ii{{\dot{\imath}}}
\def\skipline{{\vskip0.1in}}
\def\skiplines{{\vskip0.2in}}
\def\lagr{{\mathcal{L}}}
\def\hamil{{\mathcal{H}}}
\def\vecv{{\mathbf{v}}}
\def\vecx{{\mathbf{x}}}
\def\vecy{{\mathbf{y}}}
\def\veck{{\mathbf{k}}}
\def\vecp{{\mathbf{p}}}
\def\vecn{{\mathbf{n}}}
\def\vecA{{\mathbf{A}}}
\def\vecP{{\mathbf{P}}}
\def\vecsigma{{\mathbf{\sigma}}}
\def\hatJn{{\hat{J_\vecn}}}
\def\hatJx{{\hat{J_x}}}
\def\hatJy{{\hat{J_y}}}
\def\hatJz{{\hat{J_z}}}
\def\hatj#1{\hat{J_{#1}}}
\def\hatphi{{\hat{\phi}}}
\def\hatq{{\hat{q}}}
\def\hatpi{{\hat{\pi}}}
\def\vel{\upsilon}
\def\Dint{{\mathcal{D}}}
\def\adag{{\hat{a}^\dagger}}
\def\bdag{{\hat{b}^\dagger}}
\def\cdag{{\hat{c}^\dagger}}
\def\ddag{{\hat{d}^\dagger}}
\def\hata{{\hat{a}}}
\def\hatb{{\hat{b}}}
\def\hatc{{\hat{c}}}
\def\hatd{{\hat{d}}}
\def\hatN{{\hat{N}}}
\def\hatH{{\hat{H}}}
\def\hatp{{\hat{p}}}
\def\Fup{{F^{\mu\nu}}}
\def\Fdown{{F_{\mu\nu}}}
\def\newl{\nonumber \\}
\def\vece{\mathrm{e}}
\def\calM{{\mathcal{M}}}
\def\calT{{\mathcal{T}}}
\def\calR{{\mathcal{R}}}
\def\barpsi{\bar{\psi}}
\def\baru{\bar{u}}
\def\barv{\bar{\upsilon}}
\def\qeq{\stackrel{?}{=}}
\def\torder#1{\mathcal{T}\left(#1\right)}
\def\rorder#1{\mathcal{R}\left(#1\right)}
\def\contr#1#2{\contraction{}{#1}{}{#2}#1#2}
\def\trof#1{\mathrm{Tr}\left(#1\right)}
\def\trace{\mathrm{Tr}}
\def\comm#1{\ \ \ \left(\mathrm{used}\ #1\right)}
\def\tcomm#1{\ \ \ (\text{#1})}
\def\slp{\slashed{p}}
\def\slk{\slashed{k}}
\def\calp{{\mathfrak{p}}}
\def\veccalp{\mathbf{\mathfrak{p}}}
\def\Tthree{T_{\tiny \textcircled{3}}}
\def\pthree{p_{\tiny \textcircled{3}}}
\def\dbar{{\,\mathchar'26\mkern-12mu d}}
\def\erf{\mathrm{erf}}
\def\const{\mathrm{constant}}
\def\pheat{\pfrac p{\ln T}V}
\def\vheat{\pfrac V{\ln T}p}
%%units
\def\fdeg{{^\circ \mathrm{F}}}
\def\cdeg{^\circ \mathrm{C}}
\def\atm{\,\mathrm{atm}}
\def\angstrom{\,\text{\AA}}
\def\SIL{\,\mathrm{L}}
\def\SIkm{\,\mathrm{km}}
\def\SIyr{\,\mathrm{yr}}
\def\SIGyr{\,\mathrm{Gyr}}
\def\SIV{\,\mathrm{V}}
\def\SImV{\,\mathrm{mV}}
\def\SIeV{\,\mathrm{eV}}
\def\SIkeV{\,\mathrm{keV}}
\def\SIMeV{\,\mathrm{MeV}}
\def\SIGeV{\,\mathrm{GeV}}
\def\SIcal{\,\mathrm{cal}}
\def\SIkcal{\,\mathrm{kcal}}
\def\SImol{\,\mathrm{mol}}
\def\SIN{\,\mathrm{N}}
\def\SIHz{\,\mathrm{Hz}}
\def\SIm{\,\mathrm{m}}
\def\SIcm{\,\mathrm{cm}}
\def\SIfm{\,\mathrm{fm}}
\def\SImm{\,\mathrm{mm}}
\def\SInm{\,\mathrm{nm}}
\def\SImum{\,\mathrm{\mu m}}
\def\SIJ{\,\mathrm{J}}
\def\SIW{\,\mathrm{W}}
\def\SIkJ{\,\mathrm{kJ}}
\def\SIs{\,\mathrm{s}}
\def\SIkg{\,\mathrm{kg}}
\def\SIg{\,\mathrm{g}}
\def\SIK{\,\mathrm{K}}
\def\SImmHg{\,\mathrm{mmHg}}
\def\SIPa{\,\mathrm{Pa}}

\def\courseurl{https://github.com/zqhuang/SYSU\_TD}

\def\tpage#1#2{
\begin{frame}
\begin{center}
\begin{Large}
\bch
热学 \\
第#1讲 #2

{\vskip 0.3in}

黄志琦

\ech
\end{Large}
\end{center}

\vskip 0.2in

\bch
教材:《热学》第二版,赵凯华,罗蔚茵,高等教育出版社
\ech

\bch
课件下载
\ech
\courseurl
\end{frame}
}

\def\bfr#1{
\begin{frame}
\chtitle{#1} 
\bch
}

\def\efr{
\ech 
\end{frame}
}

  \date{}
\begin{document}
\tpage{2}{Risidue Theorem}


\begin{frame}
\chtitle{本讲内容}
\bch
\bitem
\item{Laurent展开}
\item{留数定理}
\eitem
\ech
\end{frame}

\section{Laurent Series}
\secpage{Laurent展开}{把负次幂补上,但其实最重要的只是$-1$次幂}

\begin{frame}
  \chtitle{上讲内容摘要}
  \bch
  柯西定理:{\bf \blue 解析函数沿解析区域边界上的积分为零。}
  (当然,要求函数在边界上连续)。
  
  \addfig{1.5}{simplecontour.png}

  $$\oint_C f(z) dz  = 0 $$
  \ech
\end{frame}


\begin{frame}
  \chtitle{上讲内容摘要}
  \bch
  柯西积分公式:{\bf \blue 解析函数在解析区域内任意一点的任意阶导数都可以用边界上的积分表示出来。}
  \addfig{1.5}{simplecontourwithz0.png}
  
  $$\frac{f^{(n)}(z_0)}{n!} =  \frac{1}{2\pi \ii}\oint_C \frac{f(z)}{(z-z_0)^{n+1}}dz$$
  $C$可以是包含$z_0$的任意解析区域的边界(当然,要求函数在边界上连续)。
  \ech
\end{frame}

\begin{frame}
  \chtitle{上讲内容摘要}
  \bch
      {\bf \blue 在一个圆内解析的函数可以看成在圆心展开的Taylor级数。}

      \addfig{1}{TaylorExpansion.png}
      
  设$f(z)$在圆$|z-z_0|<R$ (这里允许$R=\infty$)内解析,则
  $$f(z) = \sum_{n=0}^{\infty} a_n (z-z_0)^n, |z-z_0|<R.$$
  其中
  $$a_n = \frac{f^{(n)}(z_0)}{n!} =  \frac{1}{2\pi \ii}\oint_C \frac{f(z)}{(z-z_0)^{n+1}}dz$$
  围道$C$为任意逆时针方向围绕$z_0$一周的分段光滑曲线。
  \ech
\end{frame}


\begin{frame}
  \chtitle{思考题}
  \bch
  \addfig{1}{think2.jpg}
  
  柯西积分公式
    $$\frac{f^{(n)}(z_0)}{n!} =  \frac{1}{2\pi \ii}\oint_{C} \frac{f(z)}{(z-z_0)^{n+1}}dz$$
  等式右边是沿边界的积分,按柯西定理结果恒为零……吗?感觉哪里不对了\wulian
  \ech
\end{frame}



\begin{frame}
  \chtitle{思考题}
  \bch

  一个比较直观的说法就是,{\blue 解析函数就是能用加减乘除表示出来的函数}。例如$z$, $1/(1+z)$等都是(定义域内的)解析函数,而诸如$|z|$, $\Arg z$之类的函数因为无法用$z$的加减乘除表示出来,所以不是解析函数。
  \addfig{1}{think2.jpg}


  为什么在圆内解析的函数用加减乘除表示出来(即Taylor展开)之后,只包含$z-z_0$的非负次幂项,而不包含如$(z-z_0)^{-1}, (z-z_0)^{-2},\ldots$等的负次幂项?
  \ech
\end{frame}


\begin{frame}
  \chtitle{在一个环里解析的函数: Laurent展开定理}
  \bch

  {\blue
  设$f(z)$在环区域$r<|z-z_0|<R$ (这里允许$r=0$和$R=\infty$)内解析,则
  $$f(z) = \sum_{n=-\infty}^\infty a_n (z-z_0)^n,$$
  其中
  $$a_n = \frac{1}{2\pi\ii}\oint_C\frac{f(\zeta)}{(\zeta-z_0)^{n+1}}d\zeta,$$
  积分路径$C$可以是环内任意的一条逆时针绕$z_0$一周的分段光滑曲线。}

  这个展开式称为Laurent展开。注意和Taylor展开不同,指标的求和范围从$-\infty$到$\infty$。而且$a_n$不再和$f$在$z_0$处的$n$阶导数相关(\bye $z_0$根本就不在函数的定义域内)。

  \ech
\end{frame}

\begin{frame}
  \chtitle{Laurent}
  \bch
  \bcenter
  \lfig{1.5}{Laurent.jpg}  
  
  Pierre Alphonse Laurent (1813-1854)
  \ecenter
  \ech
\end{frame}


\begin{frame}
  \chtitle{Laurent和他的定理}
  \bch
  \bcenter
  \lfig{1.5}{LaurentwithRing.jpg}
  
  $$f(z) = \sum_{n=-\infty}^\infty a_n (z-z_0)^n;\ \ a_n = \frac{1}{2\pi\ii}\oint_C\frac{f(\zeta)}{(\zeta-z_0)^{n+1}}d\zeta$$
  \ecenter
  \ech
\end{frame}

\begin{frame}
  \chtitle{第一讲``五步曲''最后一步的另一种证法}
  \bch
  {\small   
    对$|z-z_0|=s<r$,取$s<q<r$并以$q$为半径作圆区域$C_q$:$|z-z_0| < q$。根据柯西积分公式:
  \bea
  f(z) = \frac{1}{2\pi\ii}\oint_{\partial C_q}\frac{f(\zeta)}{\zeta - z} d\zeta   &=& \frac{1}{2\pi\ii}\oint_{\partial  C_q}\frac{f(\zeta)}{(\zeta - z_0) - (z-z_0)} d\zeta \newl
  &=& \frac{1}{2\pi\ii}\oint_{\partial C_q}\frac{1}{\zeta-z_0}\frac{f(\zeta)}{1 - \frac{z-z_0}{\zeta-z_0}} d\zeta \newl
  &=& \frac{1}{2\pi\ii}\oint_{\partial C_q}\frac{1}{\zeta-z_0}f(\zeta)\sum_{n=0}^\infty\left(\frac{z-z_0}{\zeta-z_0}\right)^n d\zeta \newl
  &=& \sum_{n=0}^\infty a_n(z-z_0)^n  
  \eea
  其中$a_n = \frac{1}{2\pi\ii}\oint_{\partial  C_q}\frac{f(\zeta)}{(\zeta-z_0)^{n+1}}d\zeta.$

  请用类似的方法证明Laurent展开定理。
  }
  \ech
\end{frame}

\begin{frame}
  \chtitle{求Laurent展开的例子}
  \bch
  $f(z) = \frac{1}{z(1+z)}$在环区域$0<|z|<1$内解析,Laurent展开为:
  
  \bea
  \frac{1}{z(1+z)} &=& \frac{1}{z}\,\cdot\, \frac{1}{1+z} \newl
  &=&\frac{1}{z}\left(1-z+z^2-z^3+\ldots\right) \newl
  &=& \frac{1}{z} - 1 + z - z^2 + \ldots
  \eea

  级数展开的很多技巧我们之后会结合习题进行专门讲解,先不展开讨论。
  \ech
\end{frame}


\begin{frame}
  \chtitle{总结: Taylor展开和Laurent展开}
  \bch
  \bmini{0.48}
  \addfig{1}{TaylorExpansion.png}

  {\small
  $$f(z) = \sum_{n=0}^\infty a_n(z-z_0)^n$$
  }{
    \scriptsize
    $$a_n = \frac{1}{2\pi\ii}\oint_{C}\frac{f(\zeta)}{(\zeta-z_0)^{n+1}}d\zeta {\blue = \frac{f^{(n)}(z_0)}{n!}}$$
  
  }
  \emini
  \bmini{0.48}
  \addfig{1}{LaurentExpansion.png}

  {\small
    $$f(z) = \sum^\infty_{{\blue n=-\infty}} a_n (z-z_0)^n$$
}{    \scriptsize
    $$a_n = \frac{1}{2\pi\ii}\oint_{C}\frac{f(\zeta)}{(\zeta-z_0)^{n+1}}d\zeta$$
  }
  \emini
  \ech
\end{frame}

\begin{frame}
  \chtitle{快速理解展开系数}
  \bch
  \addfig{1.5}{aroundz0.png}
  
  设$C$是逆时针围绕$z_0$一圈的简单围道,计算积分
  $$\oint_C (z-z_0)^n dz.$$
  由此你能说出为什么Taylor和Laurent级数展开中$n$次幂项的系数总是
  $$a_n = \frac{1}{2\pi\ii}\oint_{C}\frac{f(z)}{(z-z_0)^{n+1}}dz$$
  吗?
  
  \ech
\end{frame}

\section{Residue Theorem}

\secpage{留数定理}{其实就是$\frac{1}{z-z_0}$的原函数$\ln (z-z_0)$绕$z_0$一圈变化$2\pi \ii$}

\begin{frame}
  \chtitle{孤立奇点}
  \bch
  如果$f(z)$在$z_0$的邻域$0<|z-z_0|<\delta$内解析($\delta$可以是任意一个小的正数),但在$z=z_0$不解析(没有定义或者有定义却不可导),则称$z_0$为$f(z)$的{\bf 孤立奇点}。

    \skiplines
    
    \bex
    $0, 1, 2$都是函数$\frac{1}{z^2(z^2-3z+2)}$的孤立奇点。
    
    $\pm \ii$都是函数$\frac{1}{z^2+1}$的孤立奇点。
    \eex
  \ech
\end{frame}

\begin{frame}
  \chtitle{孤立奇点的留数}
  \bch
  如果$z_0$是$f(z)$的孤立奇点,则$f$可在邻域$0<|z-z_0|<\delta$ 内Laurent展开。因为我们知道$(z-z_0)^n$绕$z_0$积分一圈当且仅当$n= -1$时结果不为零(为$2 \pi\ii$),所以我们特别关注$(z-z_0)^{-1}$前的系数$a_{-1}$,并把它称为$f$在$z_0$处的{\bf 留数,记作$\res{f}{z_0}$}.
  \tbox{$$\res{f}{z_0} \equiv a_{-1}$$}
  \ech
\end{frame}


\begin{frame}
  \chtitle{留数定理}
  \bch
      {\blue \bf 留数定理:设$f$在区域$T$内除有限个孤立奇点$b_1,b_2,\ldots,b_n$之外解析,在$T$的边界上连续,则$f$沿$T$的边界的积分等于$2\pi\ii$乘以$f$在所有孤立奇点处的留数之和}
      \tbox{$$\oint_{\partial T} f(z)dz  = 2\pi\ii\sum_{k=1}^n\res{f}{b_k}$$}


  \ech
\end{frame}

\begin{frame}
  \chtitle{图解留数定理}
  \bch
  \bmini{0.53}
  两个孤立奇点的例子。根据柯西定理
  $$\oint_C f(z) dz - \oint_{C_1}f(z)dz - \oint_{C_2}f(z) dz = 0$$
  {\scriptsize 在$C_1$和$C_2$上的积分前多了个负号,是因为$C_1$和$C_2$方向和左内法则规定的正方向相反。}  
  \emini
  \bmini{0.43}
  \addfig{1.2}{ResidueTheorem.png}
  \emini
  


  再利用$(z-b_1)^n$沿$C_1$积分当且仅当$n=-1$时不为零(为$2\pi\ii$),即可得到
  $$\oint_{C_1}f(z)dz = 2\pi\ii \res{f}{b_1}$$
  同理有
  $$\oint_{C_2}f(z)dz = 2\pi\ii \res{f}{b_2}$$
  代回最上面的式子就得到留数定理。当孤立奇点个数更多时同理可证。 
  \ech
\end{frame}

\begin{frame}
  \chtitle{留数定理应用举例}
  \bch
  计算定积分
  $$I = \int_0^\infty \frac{\sin x}{x} dx $$
  \ech
\end{frame}

\begin{frame}
  \chtitle{留数定理应用举例}
  \bch
  思路:首先$\frac{\sin x}{x}$是偶函数,所以
  $$I = \frac{1}{2} \int_{-\infty}^\infty \frac{\sin x}{x} dx. $$

  $\sin x$是$e^{\ii x}$的虚部,即
  $$I = \frac{1}{2}\mathrm{Im} \left(\int_{-\infty}^\infty \frac{e^{ix}}{x} dx \right)$$
  \ech
\end{frame}

\begin{frame}
  \chtitle{留数定理应用举例}
  \bch
  想使用留数定理就必然要把积分路径补成一个闭合围道。考虑到$e^{iz}$当$z$的虚部很大时趋向于零,我们就尽量在上半平面补。

  最简单当然是补一个上半圆$C_R$(半径$R\rightarrow \infty$),如下图所示:
  
  \addfig{2.5}{sinxbyxpath0.png}

  
  
  \ech
\end{frame}

\begin{frame}
  \chtitle{留数定理应用举例}
  \bch
  问题是,在这个围道上$\frac{e^{iz}}{z}$在$z=0$处没有定义,我们就要想办法绕开这个奇点。

  如下图所示,在原点附近拐个弯,再取一个小半圆$C_\delta$ (半径$\delta \rightarrow 0^+$)。
  
  \addfig{3}{sinxbyxpath.png}

  \ech
\end{frame}


\begin{frame}
  \chtitle{留数定理应用举例}
  \bch  
  \addfig{1.45}{sinxbyxpath.png}
{\small  在这个围道内部,$\frac{e^{iz}}{z}$处处解析,根据柯西定理或留数定理,
  $$\left(\int_{-R}^{-\delta} + \int_{\delta}^{R} + \int_{C_R} + \int_{C_\delta}\right)\frac{e^{iz}}{z} dz = 0  $$
 当$\delta \rightarrow 0^+$时,
 $$\int_{C_\delta} \frac{e^{iz}}{z}dz  \rightarrow \int_{C_\delta} \frac{1}{z}dz = \Delta \ln z = - \ii \pi $$
 当$R\rightarrow \infty$时,可以(这里还是要稍作思考)估算出
 $$\int_{C_R} \frac{e^{iz}}{z}dz \rightarrow 0$$}
  \ech
\end{frame}

\begin{frame}
  \chtitle{留数定理应用举例}
  \bch  
  \addfig{1.45}{sinxbyxpath.png}
    所以我们最后得到$\delta\rightarrow 0$, $R\rightarrow \infty$时,
    $$\left(\int_{-R}^{-\delta} + \int_{\delta}^{R}\right)  \frac{e^{iz}}{z} dz \rightarrow \ii\pi $$
    即
    $$ I = \frac{1}{2}\mathrm{Im}(\ii\pi) = \frac{\pi}{2}.$$
  \ech
\end{frame}

\section{Homework}
\begin{frame}
  \chtitle{课后作业}
  \bch
  \bitem
\item[4.]{函数$\frac{e^{iz}}{z}$在环形区域$0<|z|<\infty$内解析,试求它的Laurent展开。}
\item[5.]{计算函数$\frac{e^{iz}}{z}$在$z=0$处的留数。}
\item[6.]{仿照课上用$\frac{e^{iz}}{z}$的围道积分计算
  $$I=\int_0^{\infty}\frac{\sin x}{x}dx$$
  的方法,但是取围道时把小半圆$C_\delta$取成下半圆,如图所示。用这个围道重新计算$I$。
  
  \addfig{1.8}{sinxbyxpath1.png}
}
  \eitem
  \ech  
\end{frame}


\end{document}
