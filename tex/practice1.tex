\documentclass[12pt,CJK]{article}
\usepackage{geometry}
\input{reduced_macros.tex}
\geometry{tmargin=0.3in, bmargin=0.5in, lmargin=0.5in, rmargin=0.9in, nohead, nofoot}
\def\mark#1{{\color{blue} (#1分)}}
\renewcommand{\thepage}{}
\begin{document}
\bch
{\large 数理方法 课堂小测I}

{\vskip 0.2in}

姓名 ....................... {\hskip 0.5in}    学号 .......................{\hskip 0.5in}  分数 ...................

\bitem
\item[(一)]{选择题,每题3分,共45分。

  \bitem
\item[(1)]{ $e^{\ii \pi}$的值为  \bropt
  
  \optlist{$-1$}{$0$}{$1$}{$\ii$} }
\item[(2)]{方程$e^z = 1 +\ii$ 的全部复数解为 $z=$ \bropt


\foptlist{$\ln 2 + \frac{\pi}{4}\ii$}{$\ln 2 + \left(2n+\frac{1}{4}\right)\pi \ii,\,n\in Z$}{$\frac{1}{2}\ln 2 + \left(2n+\frac{1}{4}\right)\pi \ii,\,n\in Z$}{$\frac{1}{2}\ln 2 + \left(n+\frac{1}{4}\right)\pi \ii,\,n\in Z$}}


\item[(3)]{复变函数 $z\cos z$ 在 $z=0$ 处的导数为 \bropt
  
  \optlist{$-1$}{$0$}{$1$}{$2\pi\ii$}}

\item[(4)]{复变函数$f(z) = (1+z)e^z$的原函数为(忽略不写积分常数): \bropt

  \optlist{$e^z$}{$ze^z$}{$\left(z+\frac{z^2}{2}\right)e^z$}{$\frac{e^z}{1+z}$}}
\item[(5)]{方程 $z^5+ z^4+z^3+2=0$ 的所有复数解的平方和为 \bropt

  \optlist{$-1$}{$0$}{$1$}{$2$}}  

\item[(6)]{已知$f$和$g$在全复平面上解析,且在某个复数集$S$上$f$和$g$恒等。当$S$为下述哪个集合时,我们{\bf 不能}断言$f$和$g$在全复平面上恒等? \bropt

  \optlist{单位圆$|z|=1$内部}{区间$(0,1)$}{$(0,1)$上的所有有理数}{整数集}}
  
\item[(7)]{$\frac{1}{(1+e^z)\sin z}$ 在区域$|z|< 5 $内有多少个孤立奇点? \bropt
  
  \optlist{$3$}{$4$}{$5$}{$6$}}
\item[(8)]{$\frac{1}{z^2-3z+2}$ 在 $z=1$ 处的留数等于 \bropt

    \optlist{$2$}{$1$}{$0$}{$-1$}}  

\item[(9)]{下列哪个多值函数在区域 $1<|z|<2$ 内可以规定适当的幅角范围成为解析函数? \bropt

  \optlist{$\ln(z-1)$}{$\ln (z+1)$}{$\ln [(z-1)(z+1)]$}{$\ln \frac{z-1}{z+1}$}}


\item[(10)]{$\sqrt{2\pi}\,\delta (x^2-1)$的傅立叶变换为 \bropt

  \optlist{$1$}{$\cos k$}{$\sin k$}{$\frac{\sin k}{k}$}}

\item[(11)]{$e^{-t}\sin{2t}$的拉普拉斯变换为 \bropt

  \optlist{$\frac{2}{(p+1)^2+4}$}{$\frac{2}{(p-1)^2+4}$}{$\frac{2p}{p^2+4}$}{$\frac{2e^{-p}}{p^2+4}$}}

\item[(12)]{$ \left\vert\oint_{|z|=1}\, \frac{\cos z}{z^{2719}}\,dz\right\vert$ 和下列哪个数量级最接近? \bropt

  \optlist{$10^{-900}$}{$10^{-2700}$}{$10^{-5100}$}{$10^{-15300}$}}

\item[(13)]{ 在单位球内的积分 $\iiint_{x^2+y^2+z^2\le 1} (|x|+|y|+|z|)dxdydz$ 等于\bropt

  \optlist{$\frac{3}{2}\pi$}{$\sqrt{3}\pi$}{$2\pi$}{$2\sqrt{3}\pi$}}
  
\item[(14)]{用$z^*$表示$z$的共轭复数,按逆时针沿着曲线$|z-3|+|z+3|=10$的围道积分$\oint_{|z-3|+|z+3|=10}\, (z^*dz - zdz^*)$  等于 \bropt
  
  \optlist{$48\pi\ii$}{$60\pi\ii$}{$80\pi\ii$}{$96\pi\ii$}}

\item[(15)]{下列哪一个复变函数在复平面上处处不可导? \bropt

  \optlist{$\frac{1}{z}$}{$|z|^2$}{$\sin{|z|}$}{$e^{|z|}$} }  
    
  \eitem
  
\vspace{1in}
  }
\item[(二)]{ 填空题,每题5分,共35分。

  \bitem  
\item[(1)]{复变函数 $e^{\frac{\cos z}{1+z^2+z^4}}\cos{(z^3)}$ 在 $z=0$ 处的导数为\_\_\_\_\_\_\_\_。 }  
\item[(2)]{函数 $\frac{1}{z^{5}+z+1}$ 的所有孤立奇点处的留数之和为\_\_\_\_\_\_\_\_。}  
\item[(3)]{积分 $\int_0^{\frac{\pi}{2}}\, \cos^5\theta \, \sin^4\theta\, d\theta$ 等于\_\_\_\_\_\_\_\_。   }
\item[(4)]{实积分 $\int_0^\infty e^{-x^2}\cos{2x}dx$ 等于\_\_\_\_\_\_\_\_。}
\item[(5)]{逆时针方向沿着上半个单位圆的积分$\int_{|z|=1, \mathrm{Im}(z)\ge 0}\, z^{\frac{1}{3}}\,dz$ 等于\_\_\_\_\_\_\_\_。}
\item[(6)]{复变函数$f(z) = |1-z|^4 + (1+|z|)^4$的最小可能值等于\_\_\_\_\_\_\_\_。}   
\item[(7)]{定义函数  $f(x) = \frac{1}{\pi}\int_0^\pi \cos{\left( x \cos t\right)} \, dt. $,则积分 $\int_0^{\frac{\pi}{2}} f(x)f(\frac{\pi}{2}-x)\,dx$ 等于 \_\_\_\_\_\_\_\_。}
\eitem

  }
  

\item[(三)]{长度为$L$的不良导体棒一端和温度为$T_0$的热库接触,并在$t=0$时刻和热库处于热平衡。从$t=0$时刻开始,在不良导体棒的另一端注入恒定大小为$j$的热流。设不良导体棒的导热系数$\lambda$,单位质量的比热$c$和质量密度$\rho$均已知。写出不良导体棒上温度$T(x, t)$ ($0\le x\le L, t\ge 0$)满足的方程和边界条件。(10分) 并简要分析当$t$很大时的解的渐近行为。(10分)

}
\eitem    


\ech
\end{document}
