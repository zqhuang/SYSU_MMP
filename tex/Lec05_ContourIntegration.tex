\documentclass[CJK]{beamer}
\usepackage{CJKutf8}
\usepackage{beamerthemesplit}
\usetheme{Malmoe}
\useoutertheme[footline=authortitle]{miniframes}
\usepackage{amsmath}
\usepackage{amssymb}
\usepackage{graphicx}
\usepackage{eufrak}
\usepackage{color}
\usepackage{slashed}
\usepackage{simplewick}
\usepackage{tikz}
\usepackage{tcolorbox}
\graphicspath{{../figures/}}
%%figures
\def\lfig#1#2{\includegraphics[width=#1 in]{#2}}
\def\addfig#1#2{\begin{center}\includegraphics[width=#1 in]{#2}\end{center}}
\def\wulian{\includegraphics[width=0.18in]{emoji_wulian.jpg}}
\def\bigwulian{\includegraphics[width=0.35in]{emoji_wulian.jpg}}
\def\bye{\includegraphics[width=0.18in]{emoji_bye.jpg}}
\def\bigbye{\includegraphics[width=0.35in]{emoji_bye.jpg}}
\def\huaixiao{\includegraphics[width=0.18in]{emoji_huaixiao.jpg}}
\def\bighuaixiao{\includegraphics[width=0.35in]{emoji_huaixiao.jpg}}
\def\jianxiao{\includegraphics[width=0.18in]{emoji_jianxiao.jpg}}
\def\bigjianxiao{\includegraphics[width=0.35in]{emoji_jianxiao.jpg}}
%% colors
\def\blacktext#1{{\color{black}#1}}
\def\bluetext#1{{\color{blue}#1}}
\def\redtext#1{{\color{red}#1}}
\def\darkbluetext#1{{\color[rgb]{0,0.2,0.6}#1}}
\def\skybluetext#1{{\color[rgb]{0.2,0.7,1.}#1}}
\def\cyantext#1{{\color[rgb]{0.,0.5,0.5}#1}}
\def\greentext#1{{\color[rgb]{0,0.7,0.1}#1}}
\def\darkgray{\color[rgb]{0.2,0.2,0.2}}
\def\lightgray{\color[rgb]{0.6,0.6,0.6}}
\def\gray{\color[rgb]{0.4,0.4,0.4}}
\def\blue{\color{blue}}
\def\red{\color{red}}
\def\green{\color{green}}
\def\darkgreen{\color[rgb]{0,0.4,0.1}}
\def\darkblue{\color[rgb]{0,0.2,0.6}}
\def\skyblue{\color[rgb]{0.2,0.7,1.}}
%%control
\def\be{\begin{equation}}
\def\ee{\nonumber\end{equation}}
\def\bea{\begin{eqnarray}}
\def\eea{\nonumber\end{eqnarray}}
\def\bch{\begin{CJK}{UTF8}{gbsn}}
\def\ech{\end{CJK}}
\def\bitem{\begin{itemize}}
\def\eitem{\end{itemize}}
\def\bcenter{\begin{center}}
\def\ecenter{\end{center}}
\def\bex{\begin{minipage}{0.2\textwidth}\includegraphics[width=0.6in]{jugelizi.png}\end{minipage}\begin{minipage}{0.76\textwidth}}
\def\eex{\end{minipage}}
\def\chtitle#1{\frametitle{\bch#1\ech}}
\def\bmat#1{\left(\begin{array}{#1}}
\def\emat{\end{array}\right)}
\def\bcase#1{\left\{\begin{array}{#1}}
\def\ecase{\end{array}\right.}
\def\bmini#1{\begin{minipage}{#1\textwidth}}
\def\emini{\end{minipage}}
\def\tbox#1{\begin{tcolorbox}#1\end{tcolorbox}}
\def\pfrac#1#2#3{\left(\frac{\partial #1}{\partial #2}\right)_{#3}}
%%symbols
\def\bropt{\,(\ \ \ )}
\def\sone{$\star$}
\def\stwo{$\star\star$}
\def\sthree{$\star\star\star$}
\def\sfour{$\star\star\star\star$}
\def\sfive{$\star\star\star\star\star$}
\def\rint{{\int_\leftrightarrow}}
\def\roint{{\oint_\leftrightarrow}}
\def\stdHf{{\textit{\r H}_f}}
\def\deltaH{{\Delta \textit{\r H}}}
\def\ii{{\dot{\imath}}}
\def\skipline{{\vskip0.1in}}
\def\skiplines{{\vskip0.2in}}
\def\lagr{{\mathcal{L}}}
\def\hamil{{\mathcal{H}}}
\def\vecv{{\mathbf{v}}}
\def\vecx{{\mathbf{x}}}
\def\vecy{{\mathbf{y}}}
\def\veck{{\mathbf{k}}}
\def\vecp{{\mathbf{p}}}
\def\vecn{{\mathbf{n}}}
\def\vecA{{\mathbf{A}}}
\def\vecP{{\mathbf{P}}}
\def\vecsigma{{\mathbf{\sigma}}}
\def\hatJn{{\hat{J_\vecn}}}
\def\hatJx{{\hat{J_x}}}
\def\hatJy{{\hat{J_y}}}
\def\hatJz{{\hat{J_z}}}
\def\hatj#1{\hat{J_{#1}}}
\def\hatphi{{\hat{\phi}}}
\def\hatq{{\hat{q}}}
\def\hatpi{{\hat{\pi}}}
\def\vel{\upsilon}
\def\Dint{{\mathcal{D}}}
\def\adag{{\hat{a}^\dagger}}
\def\bdag{{\hat{b}^\dagger}}
\def\cdag{{\hat{c}^\dagger}}
\def\ddag{{\hat{d}^\dagger}}
\def\hata{{\hat{a}}}
\def\hatb{{\hat{b}}}
\def\hatc{{\hat{c}}}
\def\hatd{{\hat{d}}}
\def\hatN{{\hat{N}}}
\def\hatH{{\hat{H}}}
\def\hatp{{\hat{p}}}
\def\Fup{{F^{\mu\nu}}}
\def\Fdown{{F_{\mu\nu}}}
\def\newl{\nonumber \\}
\def\vece{\mathrm{e}}
\def\calM{{\mathcal{M}}}
\def\calT{{\mathcal{T}}}
\def\calR{{\mathcal{R}}}
\def\barpsi{\bar{\psi}}
\def\baru{\bar{u}}
\def\barv{\bar{\upsilon}}
\def\qeq{\stackrel{?}{=}}
\def\torder#1{\mathcal{T}\left(#1\right)}
\def\rorder#1{\mathcal{R}\left(#1\right)}
\def\contr#1#2{\contraction{}{#1}{}{#2}#1#2}
\def\trof#1{\mathrm{Tr}\left(#1\right)}
\def\trace{\mathrm{Tr}}
\def\comm#1{\ \ \ \left(\mathrm{used}\ #1\right)}
\def\tcomm#1{\ \ \ (\text{#1})}
\def\slp{\slashed{p}}
\def\slk{\slashed{k}}
\def\calp{{\mathfrak{p}}}
\def\veccalp{\mathbf{\mathfrak{p}}}
\def\Tthree{T_{\tiny \textcircled{3}}}
\def\pthree{p_{\tiny \textcircled{3}}}
\def\dbar{{\,\mathchar'26\mkern-12mu d}}
\def\erf{\mathrm{erf}}
\def\const{\mathrm{constant}}
\def\pheat{\pfrac p{\ln T}V}
\def\vheat{\pfrac V{\ln T}p}
%%units
\def\fdeg{{^\circ \mathrm{F}}}
\def\cdeg{^\circ \mathrm{C}}
\def\atm{\,\mathrm{atm}}
\def\angstrom{\,\text{\AA}}
\def\SIL{\,\mathrm{L}}
\def\SIkm{\,\mathrm{km}}
\def\SIyr{\,\mathrm{yr}}
\def\SIGyr{\,\mathrm{Gyr}}
\def\SIV{\,\mathrm{V}}
\def\SImV{\,\mathrm{mV}}
\def\SIeV{\,\mathrm{eV}}
\def\SIkeV{\,\mathrm{keV}}
\def\SIMeV{\,\mathrm{MeV}}
\def\SIGeV{\,\mathrm{GeV}}
\def\SIcal{\,\mathrm{cal}}
\def\SIkcal{\,\mathrm{kcal}}
\def\SImol{\,\mathrm{mol}}
\def\SIN{\,\mathrm{N}}
\def\SIHz{\,\mathrm{Hz}}
\def\SIm{\,\mathrm{m}}
\def\SIcm{\,\mathrm{cm}}
\def\SIfm{\,\mathrm{fm}}
\def\SImm{\,\mathrm{mm}}
\def\SInm{\,\mathrm{nm}}
\def\SImum{\,\mathrm{\mu m}}
\def\SIJ{\,\mathrm{J}}
\def\SIW{\,\mathrm{W}}
\def\SIkJ{\,\mathrm{kJ}}
\def\SIs{\,\mathrm{s}}
\def\SIkg{\,\mathrm{kg}}
\def\SIg{\,\mathrm{g}}
\def\SIK{\,\mathrm{K}}
\def\SImmHg{\,\mathrm{mmHg}}
\def\SIPa{\,\mathrm{Pa}}

\def\courseurl{https://github.com/zqhuang/SYSU\_TD}

\def\tpage#1#2{
\begin{frame}
\begin{center}
\begin{Large}
\bch
热学 \\
第#1讲 #2

{\vskip 0.3in}

黄志琦

\ech
\end{Large}
\end{center}

\vskip 0.2in

\bch
教材:《热学》第二版,赵凯华,罗蔚茵,高等教育出版社
\ech

\bch
课件下载
\ech
\courseurl
\end{frame}
}

\def\bfr#1{
\begin{frame}
\chtitle{#1} 
\bch
}

\def\efr{
\ech 
\end{frame}
}

  \date{}
\begin{document}
\tpage{5}{Contour Integration}


\begin{frame}
  \chtitle{响应教务部号召,开动点名神器}
  \addfig{1.5}{drive.jpg}
  \bch
  请一位同学计算一下$$f(z) = \frac{1}{z^2+4z+1}$$在每个孤立奇点处的留数。
  \ech
\end{frame}


\begin{frame}
\chtitle{本讲内容}
\bch
\bitem
\item{用留数定理计算定积分}
 \eitem
\ech
\end{frame}


\section{Did we ever learn integration?}
\secpage{据说我们学过积分?}{不信谣,不传谣\bye}


\begin{frame}
  \chtitle{例1 (研究生面试题)}
  \bch
  \addfig{0.5}{think1.jpg}

  计算积分$$ \int_0^1 \ln x \, dx. $$
  \ech
\end{frame}

\begin{frame}
  \chtitle{例 1 解答}
  \bch
  $$\int_0^1 \ln x\,dx =\left. \left( x\ln x-x \right)\right\vert^{1}_0 = -1 $$

  \ech
\end{frame}


\begin{frame}
  \chtitle{例2 }
  \bch
  \addfig{0.5}{think1.jpg}

  计算积分$$ \int_0^{2\pi}\frac{1}{2+\cos x} \, dx. $$
  \ech
\end{frame}

\begin{frame}
  \chtitle{例2 解法一(三角函数有理分式标准推土法)}
  \bch
    回忆下高数课学过的(?)的知识:{\blue 凡是三角函数有理分式,{\bf 原则上}都能用变量替换$t = \tan\frac{x}{2}$求出原函数。}

    请先验算:{\blue
    $$ dx = \frac{2dt}{1+t^2}, $$
    $$ \cos x = \frac{1-t^2}{1+t^2} $$
    $$ \sin x = \frac{2t}{1+t^2} $$    
    $$ \tan x = \frac{2t}{1-t^2} $$
    }
  \ech
\end{frame}

\begin{frame}
  \chtitle{例2 解法一(三角函数有理分式标准推土法)}
  \bch
  {\small  
  剩下的就是毫无技巧地推土:
    \bea
    \int \frac{1}{2+\cos x} dx &=& \int \frac{1}{2 + \frac{1-t^2}{1+t^2}} \frac{2dt}{1+t^2} \newl
    &=& \int \frac{2}{3 + t^2 } dt \newl
    &=& \frac{2}{\sqrt{3}} \arctan\frac{t}{\sqrt{3}} + c \newl
    &=& \frac{2}{\sqrt{3}} \arctan\frac{\tan\frac{x}{2}}{\sqrt{3}} + c    
    \eea
  }
  不必为原函数在$x=\pi$不连续而惊慌,利用一下三角函数周期性把积分换到$(-\pi, \pi)$:
  $$ \int_0^{2\pi}\frac{dx}{2+\cos x}  = \int_{-\pi}^{\pi}\frac{dx}{2+\cos x} =\left. \frac{2}{\sqrt{3}} \arctan\frac{\tan\frac{x}{2}}{\sqrt{3}}\right\vert_{-\pi}^{\pi}  = \frac{2\pi}{\sqrt{3}} .$$
  \ech
\end{frame}


\begin{frame}
  \chtitle{例2 解法二 (围道积分法)}
  \bch
  考虑逆时针方向的单位圆$|z|=1$,
  
  \addfig{1}{contour01.png}

  记幅角为$x$,则$z = e^{\ii x}$,请先验算:
  {\blue
    $$dx =\frac{dz}{\ii z}$$
    $$\cos x = \frac{z+\frac{1}{z}}{2}$$
    $$\sin x = \frac{z-\frac{1}{z}}{2\ii}$$    
  }
  \ech
\end{frame}

\begin{frame}
  \chtitle{例2 解法二 (围道积分法)}
  \bch
    \addfig{1}{contour01.png}
  \bea
  \int_0^{2\pi}\frac{1}{2+\cos x} \, dx &=& \oint_C \frac{1}{2+\frac{z+\frac{1}{z}}{2}}\frac{dz}{\ii z} \newl
  &=& -2\ii\oint_C\frac{1}{4z+z^2+1} dz \newl
  &=& -2\ii \left(2\pi\ii \frac{1}{2\sqrt{3}}\right)  \newl
  &=& \frac{2\pi}{\sqrt{3}}
  \eea
  \ech
\end{frame}

\begin{frame}
  \bch

  \bcenter
  \lfig{1.5}{kaisen.jpg}

  其实我更喜欢毫无技巧地推土
  \ecenter
  
  \ech
\end{frame}


\begin{frame}
  \chtitle{例3}
  \bch
  \addfig{1}{think2.jpg}

  计算积分$$ \int_0^{2\pi}\cos^{2n}\theta \, d\theta$$

  \ech
\end{frame}


\begin{frame}
  \chtitle{例3 解法一}
  \bch
  利用一个非常有用的{\blue $B$函数和$\Gamma$函数的关系}来解决问题。

  \skipline
  
  对$\alpha, \beta>0$,$B$函数定义为:
  $$B(\alpha,\beta) = \int_0^1 x^{\alpha-1}(1-x)^{\beta-1}\, dx$$
  它和$\Gamma$函数有如下关系:
  {\blue $$B(\alpha,\beta) = \frac{\Gamma(\alpha)\Gamma(\beta)}{\Gamma(\alpha+\beta)} $$}

  令$x=\cos^2\theta$ 即得到所求积分:
 {\small $$\int_0^{2\pi}\cos^{2n}\theta \,d\theta = 2 \int_0^1x^{n-1/2}(1-x)^{-1/2} dx =2 B(n+\frac{1}{2},\frac{1}{2}) = \frac{\pi}{2^{2n-1}}\frac{(2n)!}{(n!)^2} $$}
  
  \ech
\end{frame}


\begin{frame}
  \chtitle{附:补充证明$B$函数和$\Gamma$函数的关系(技巧似曾相识)}
  \bch
  根据$\Gamma$函数定义:
  $$  \Gamma(\alpha)\Gamma(\beta) = \int_0^\infty u^{\alpha-1}e^{-u} du \int_0^\infty \upsilon^{\beta-1}e^{-\upsilon} d\upsilon    $$
  做替换$u=s^2, \upsilon = t^2$;
  $$  \Gamma(\alpha)\Gamma(\beta) = 4 \int_0^\infty s^{2\alpha-1}e^{-s^2} ds \int_0^\infty t^{2\beta-1}e^{-t^2} dt $$
  然后转换到极坐标$s = r\cos\theta, t = r\sin\theta$;
  $$ \Gamma(\alpha)\Gamma(\beta) = 4 \int_0^\infty  e^{-r^2}r^{2\alpha+2\beta-2}r\,dr \int_0^{\pi/2}\cos^{2\alpha-1}\theta \sin^{2\beta-1}\theta d\theta $$
  最后做变量替换$w = r^2$, $x = \cos^2\theta$,即得证。

  \ech
\end{frame}

\begin{frame}
  \chtitle{例3 解法二 围道积分}
  \bch
  还是取单位圆进行围道积分:

  \addfig{1}{contour01.png}

  
  $$ \int_0^{2\pi}\cos^{2n}x \, dx = \oint_C \left(\frac{1+z^2}{2z}\right)^{2n}\frac{dz}{\ii z}.$$
  等式右边是标准的围道积分,围道内只有$z=0$一个孤立奇点。易看出留数为
  $$\frac{(2n)!}{2^{2n}(n!)^2\ii}$$
  乘以$2\pi\ii$即得积分结果。

  

  
  \ech
\end{frame}



\begin{frame}
  \chtitle{例4}
  \bch
  \addfig{1}{think2.jpg}

  计算积分$$ \int_{-\infty}^\infty e^{-x^2}\cos x \, dx .$$

  
  \ech
\end{frame}


\begin{frame}
  \chtitle{例4 解法一 (级数展开大法)}
  \bch
  回忆热学课上的高斯积分:
  $$\int_{-\infty}^{\infty} e^{-ax^2}\,dx = \sqrt{\pi} a^{-1/2}\, .$$
  两边对$a$求导$n$次并除以$(2n)!$:
  $$\int_{-\infty}^{\infty} \frac{(-1)^nx^{2n}}{(2n)!} e^{-ax^2} \,dx= \sqrt{\pi}\frac{(-\frac{1}{4})^n}{n!} a^{-\frac{1}{2}-n}$$
  {\scriptsize (上面我们用到了$(2n)! = (2^n n!) \times 1\times 3\times 5\times\ldots \times(2n-1)$.)}
    
    令$a=1$并两边对$n$从$0$到$\infty$求和
  $$\int_{-\infty}^{\infty}  e^{-x^2}\cos x\,dx = \sqrt{\pi}e^{-1/4}$$
  \ech
\end{frame}

\begin{frame}
  \chtitle{例4 解法二 围道积分}
  \bch

  \addfig{3.5}{contour02.png}

  在如图的围道上对函数$e^{-x^2}$进行积分。

  容易验证当$R\rightarrow \infty$时,两条短边上的积分趋向于零。因此

  $$\int_{-\infty}^{\infty} e^{-(x+\frac{\ii}{2})^2}dx = \int_{-\infty}^{\infty} e^{-x^2}dx  =\sqrt{\pi}$$
  化简并取实部即得到
  $$\int_{-\infty}^{\infty}  e^{-x^2}\cos x\,dx = \sqrt{\pi}e^{-1/4},$$
  {\scriptsize (思考:取虚部可以得到什么结论)}

  \ech
\end{frame}



\begin{frame}
  \chtitle{例5}
  \bch
  \addfig{1}{think2.jpg}
  计算积分$$ \int_0^{\infty} \frac{\cos{x} - e^{-x}}{x}\, dx.$$
  \ech
\end{frame}


\begin{frame}
  \chtitle{例5 解答概要}
  \bch
  \bmini{0.41}
  \addfig{1.5}{contour03.png}
  \emini
  \bmini{0.54}
  对$\frac{e^{\ii z}}{z}$在如图围道上积分,易估算出当$R\rightarrow \infty$时在$C_R$上积分趋向于零。$\delta\rightarrow 0^+$时在$C_\delta$上积分为$-\frac{\pi}{2}\ii$。因此得到
  \emini
  
  $$\int_0^\infty \frac{e^{\ii x}}{x} dx - \int_0^\infty \frac{e^{- x}}{\ii x} d(\ii x)  = \frac{\pi}{2}\ii$$
  对比两边实部即得
  $$ \int_0^{\infty} \frac{\cos{x} - e^{-x}}{x}\, dx = 0.$$
  {\scriptsize (思考:对比两边虚部得到什么结论?)}
  \ech
\end{frame}



\begin{frame}
  \chtitle{例 6}
  \bch
  \addfig{1}{think3.jpg}
  
  计算积分 $$ \int_0^{\infty} \sin(x^2)\, dx $$
  \ech
\end{frame}




\begin{frame}
  \chtitle{例6 解答概要}
  \bch
  \bmini{0.41}
  \addfig{1.5}{contour04.png}
  \emini
  \bmini{0.54}
  对$e^{\ii z^2}$在如图围道上积分。不难估算出当$R\rightarrow \infty$时在$C_R$上积分趋向于零(请自行补充完整这部分证明)。
  \emini
  
  $$\int_0^\infty e^{\ii x^2} dx - e^{\frac{\pi}{4}\ii}\, \int_0^\infty e^{- x^2}dx  = 0 $$
  对比两边虚部即得
  $$ \int_0^{\infty} \sin(x^2)\,dx = \sqrt{\frac{\pi}{8}} $$ 
  {\scriptsize (思考:对比两边实部得到什么结论?)}
  \ech
\end{frame}

\begin{frame}
  \chtitle{例 7}
  \bch
  \addfig{1}{think4.jpg}
  
  计算积分$$\int_{-\infty}^{\infty}\frac{e^{\alpha x}}{1+e^x} \, dx,$$
  其中实常数$\alpha$满足$0<\alpha<1$。

  \ech
\end{frame}




\begin{frame}
  \chtitle{例7 解答概要}
  \bch
  \addfig{3.5}{contour05.png}
  按如图的围道对$f(z) = \frac{e^{\alpha z}}{1+e^z}$进行积分,并注意到围道内有一个孤立奇点$z=\pi\ii$。 记所求积分为$I$,则有
  $$ (1 - e^{2\pi\alpha\ii})I =2\pi\ii\res{f}{\pi\ii} =  -2\pi \ii e^{\alpha\pi\ii}$$
  从而得出
  $$I = \frac{\pi}{\sin(\alpha\pi)}.$$
  \ech
\end{frame}



\begin{frame}
  \chtitle{$\Gamma$函数的互宇宗量关系}
  \bch
  作变量替换$t =  \frac{e^x}{1+e^x}$,则

  $$ \int_{-\infty}^{\infty}\frac{e^{\alpha x}}{1+e^x} \, dx = \int_0^1 t^{\alpha-1}(1-t)^{-\alpha} dt = B(\alpha, 1-\alpha) $$
  再结合之前的$B$函数和$\Gamma$函数的关系,就得到著名的{\blue
    $\Gamma$函数的互宇宗量关系
    $$\Gamma(\alpha)\Gamma(1-\alpha) = \frac{\pi}{\sin(\pi\alpha)}$$
  }
  用解析延拓的办法可以把上式推广到$\alpha$为任何非整数的情形。
  
 {\scriptsize (请思考:我们是怎样做到从一条线($0<\alpha<1$)出发进行解析延拓的.) }
  \ech
\end{frame}

\begin{frame}
  \chtitle{总结}
  \bch
  \bcenter
  \lfig{2}{blackq.jpg}
  
  完全不知道这些围道哪来的.jpg  
  \ecenter
  \ech
\end{frame}


\begin{frame}
  \chtitle{技艺无止境:留给学神们的思考题\bye}
  \bch
\bea
&& \int_0^{\pi/2}\frac{\sin x}{x^\alpha}\,dx,\ \ \  0<\alpha<1; \newl
&& \int_0^1 \frac{x^{1/4}(1-x)^{3/4}}{(1+x)^3}\, dx; \newl
&& \int_0^{\pi/2} \ln(\sin x) \,dx\,; \newl
&& \int_0^{\infty}\frac{x^{p-\frac{1}{2}}}{(x+a)^p(x+b)^p}\, dx,\ \ \ p>\frac{1}{2}, a>0, b>0. 
\eea
\ech
\end{frame}

\section{Homework}

\begin{frame}
  \chtitle{课后作业}
  \bch
  \bitem
\item[13]{计算积分$$\int_0^{2\pi}\frac{dx}{1-2\alpha\cos x + \alpha^2},$$
其中常数$\alpha$满足$0<\alpha<1$。}
\item[14]{计算积分$$\int_0^\infty \frac{x\sin x}{1+x^2} \,dx.$$}
\item[15]{计算积分$$\int_0^{\infty}\frac{\ln x}{x^2+\alpha^2}dx, $$
其中常数$\alpha>0$。}
  \eitem
  \ech
\end{frame}

\end{document}
