\documentclass[CJK]{beamer}
\usepackage{CJKutf8}
\usepackage{beamerthemesplit}
\usetheme{Malmoe}
\useoutertheme[footline=authortitle]{miniframes}
\usepackage{amsmath}
\usepackage{amssymb}
\usepackage{graphicx}
\usepackage{eufrak}
\usepackage{color}
\usepackage{slashed}
\usepackage{simplewick}
\usepackage{tikz}
\usepackage{tcolorbox}
\graphicspath{{../figures/}}
%%figures
\def\lfig#1#2{\includegraphics[width=#1 in]{#2}}
\def\addfig#1#2{\begin{center}\includegraphics[width=#1 in]{#2}\end{center}}
\def\wulian{\includegraphics[width=0.18in]{emoji_wulian.jpg}}
\def\bigwulian{\includegraphics[width=0.35in]{emoji_wulian.jpg}}
\def\bye{\includegraphics[width=0.18in]{emoji_bye.jpg}}
\def\bigbye{\includegraphics[width=0.35in]{emoji_bye.jpg}}
\def\huaixiao{\includegraphics[width=0.18in]{emoji_huaixiao.jpg}}
\def\bighuaixiao{\includegraphics[width=0.35in]{emoji_huaixiao.jpg}}
\def\jianxiao{\includegraphics[width=0.18in]{emoji_jianxiao.jpg}}
\def\bigjianxiao{\includegraphics[width=0.35in]{emoji_jianxiao.jpg}}
%% colors
\def\blacktext#1{{\color{black}#1}}
\def\bluetext#1{{\color{blue}#1}}
\def\redtext#1{{\color{red}#1}}
\def\darkbluetext#1{{\color[rgb]{0,0.2,0.6}#1}}
\def\skybluetext#1{{\color[rgb]{0.2,0.7,1.}#1}}
\def\cyantext#1{{\color[rgb]{0.,0.5,0.5}#1}}
\def\greentext#1{{\color[rgb]{0,0.7,0.1}#1}}
\def\darkgray{\color[rgb]{0.2,0.2,0.2}}
\def\lightgray{\color[rgb]{0.6,0.6,0.6}}
\def\gray{\color[rgb]{0.4,0.4,0.4}}
\def\blue{\color{blue}}
\def\red{\color{red}}
\def\green{\color{green}}
\def\darkgreen{\color[rgb]{0,0.4,0.1}}
\def\darkblue{\color[rgb]{0,0.2,0.6}}
\def\skyblue{\color[rgb]{0.2,0.7,1.}}
%%control
\def\be{\begin{equation}}
\def\ee{\nonumber\end{equation}}
\def\bea{\begin{eqnarray}}
\def\eea{\nonumber\end{eqnarray}}
\def\bch{\begin{CJK}{UTF8}{gbsn}}
\def\ech{\end{CJK}}
\def\bitem{\begin{itemize}}
\def\eitem{\end{itemize}}
\def\bcenter{\begin{center}}
\def\ecenter{\end{center}}
\def\bex{\begin{minipage}{0.2\textwidth}\includegraphics[width=0.6in]{jugelizi.png}\end{minipage}\begin{minipage}{0.76\textwidth}}
\def\eex{\end{minipage}}
\def\chtitle#1{\frametitle{\bch#1\ech}}
\def\bmat#1{\left(\begin{array}{#1}}
\def\emat{\end{array}\right)}
\def\bcase#1{\left\{\begin{array}{#1}}
\def\ecase{\end{array}\right.}
\def\bmini#1{\begin{minipage}{#1\textwidth}}
\def\emini{\end{minipage}}
\def\tbox#1{\begin{tcolorbox}#1\end{tcolorbox}}
\def\pfrac#1#2#3{\left(\frac{\partial #1}{\partial #2}\right)_{#3}}
%%symbols
\def\bropt{\,(\ \ \ )}
\def\sone{$\star$}
\def\stwo{$\star\star$}
\def\sthree{$\star\star\star$}
\def\sfour{$\star\star\star\star$}
\def\sfive{$\star\star\star\star\star$}
\def\rint{{\int_\leftrightarrow}}
\def\roint{{\oint_\leftrightarrow}}
\def\stdHf{{\textit{\r H}_f}}
\def\deltaH{{\Delta \textit{\r H}}}
\def\ii{{\dot{\imath}}}
\def\skipline{{\vskip0.1in}}
\def\skiplines{{\vskip0.2in}}
\def\lagr{{\mathcal{L}}}
\def\hamil{{\mathcal{H}}}
\def\vecv{{\mathbf{v}}}
\def\vecx{{\mathbf{x}}}
\def\vecy{{\mathbf{y}}}
\def\veck{{\mathbf{k}}}
\def\vecp{{\mathbf{p}}}
\def\vecn{{\mathbf{n}}}
\def\vecA{{\mathbf{A}}}
\def\vecP{{\mathbf{P}}}
\def\vecsigma{{\mathbf{\sigma}}}
\def\hatJn{{\hat{J_\vecn}}}
\def\hatJx{{\hat{J_x}}}
\def\hatJy{{\hat{J_y}}}
\def\hatJz{{\hat{J_z}}}
\def\hatj#1{\hat{J_{#1}}}
\def\hatphi{{\hat{\phi}}}
\def\hatq{{\hat{q}}}
\def\hatpi{{\hat{\pi}}}
\def\vel{\upsilon}
\def\Dint{{\mathcal{D}}}
\def\adag{{\hat{a}^\dagger}}
\def\bdag{{\hat{b}^\dagger}}
\def\cdag{{\hat{c}^\dagger}}
\def\ddag{{\hat{d}^\dagger}}
\def\hata{{\hat{a}}}
\def\hatb{{\hat{b}}}
\def\hatc{{\hat{c}}}
\def\hatd{{\hat{d}}}
\def\hatN{{\hat{N}}}
\def\hatH{{\hat{H}}}
\def\hatp{{\hat{p}}}
\def\Fup{{F^{\mu\nu}}}
\def\Fdown{{F_{\mu\nu}}}
\def\newl{\nonumber \\}
\def\vece{\mathrm{e}}
\def\calM{{\mathcal{M}}}
\def\calT{{\mathcal{T}}}
\def\calR{{\mathcal{R}}}
\def\barpsi{\bar{\psi}}
\def\baru{\bar{u}}
\def\barv{\bar{\upsilon}}
\def\qeq{\stackrel{?}{=}}
\def\torder#1{\mathcal{T}\left(#1\right)}
\def\rorder#1{\mathcal{R}\left(#1\right)}
\def\contr#1#2{\contraction{}{#1}{}{#2}#1#2}
\def\trof#1{\mathrm{Tr}\left(#1\right)}
\def\trace{\mathrm{Tr}}
\def\comm#1{\ \ \ \left(\mathrm{used}\ #1\right)}
\def\tcomm#1{\ \ \ (\text{#1})}
\def\slp{\slashed{p}}
\def\slk{\slashed{k}}
\def\calp{{\mathfrak{p}}}
\def\veccalp{\mathbf{\mathfrak{p}}}
\def\Tthree{T_{\tiny \textcircled{3}}}
\def\pthree{p_{\tiny \textcircled{3}}}
\def\dbar{{\,\mathchar'26\mkern-12mu d}}
\def\erf{\mathrm{erf}}
\def\const{\mathrm{constant}}
\def\pheat{\pfrac p{\ln T}V}
\def\vheat{\pfrac V{\ln T}p}
%%units
\def\fdeg{{^\circ \mathrm{F}}}
\def\cdeg{^\circ \mathrm{C}}
\def\atm{\,\mathrm{atm}}
\def\angstrom{\,\text{\AA}}
\def\SIL{\,\mathrm{L}}
\def\SIkm{\,\mathrm{km}}
\def\SIyr{\,\mathrm{yr}}
\def\SIGyr{\,\mathrm{Gyr}}
\def\SIV{\,\mathrm{V}}
\def\SImV{\,\mathrm{mV}}
\def\SIeV{\,\mathrm{eV}}
\def\SIkeV{\,\mathrm{keV}}
\def\SIMeV{\,\mathrm{MeV}}
\def\SIGeV{\,\mathrm{GeV}}
\def\SIcal{\,\mathrm{cal}}
\def\SIkcal{\,\mathrm{kcal}}
\def\SImol{\,\mathrm{mol}}
\def\SIN{\,\mathrm{N}}
\def\SIHz{\,\mathrm{Hz}}
\def\SIm{\,\mathrm{m}}
\def\SIcm{\,\mathrm{cm}}
\def\SIfm{\,\mathrm{fm}}
\def\SImm{\,\mathrm{mm}}
\def\SInm{\,\mathrm{nm}}
\def\SImum{\,\mathrm{\mu m}}
\def\SIJ{\,\mathrm{J}}
\def\SIW{\,\mathrm{W}}
\def\SIkJ{\,\mathrm{kJ}}
\def\SIs{\,\mathrm{s}}
\def\SIkg{\,\mathrm{kg}}
\def\SIg{\,\mathrm{g}}
\def\SIK{\,\mathrm{K}}
\def\SImmHg{\,\mathrm{mmHg}}
\def\SIPa{\,\mathrm{Pa}}

\def\courseurl{https://github.com/zqhuang/SYSU\_TD}

\def\tpage#1#2{
\begin{frame}
\begin{center}
\begin{Large}
\bch
热学 \\
第#1讲 #2

{\vskip 0.3in}

黄志琦

\ech
\end{Large}
\end{center}

\vskip 0.2in

\bch
教材:《热学》第二版,赵凯华,罗蔚茵,高等教育出版社
\ech

\bch
课件下载
\ech
\courseurl
\end{frame}
}

\def\bfr#1{
\begin{frame}
\chtitle{#1} 
\bch
}

\def\efr{
\ech 
\end{frame}
}

\title{Lesson }
  \author{}
  \date{}
\begin{document}
\tpage{1}{复数的基本性质}


\begin{frame}
\chtitle{本讲内容}
\bch
\bitem
\item{微积分知识回顾}
\item{复数的基本性质}
\eitem
\ech
\end{frame}

\section{Calculus}

\begin{frame}
\chtitle{预备知识:导数的概念和图像}
\bch
\tbox{
  $$f'(x) \equiv \lim_{\Delta x \rightarrow 0} \frac{f(x+\Delta x) - f(x)}{\Delta x} $$}
\addfig{2.1}{derv.png}

等式右边的极限可以形象地写成$\frac{df}{dx}$,在脑海中的图像就是$f$的微小变化量除以$x$的微小变化量。

\ech
\end{frame}

\begin{frame}
\chtitle{预备知识:积分的概念和图像}
\bch
\tbox{
  $$ \int_a^b f(x) dx = F(b) - F(a) $$
}
(原函数$F(x)$满足$F'(x) = f(x)$)

\addfig{2.5}{integral.jpg}
\ech
\end{frame}

\begin{frame}
\chtitle{预备知识:不定积分}
\bch
原函数并不唯一,可以相差一个常数,写成不定积分:

\tbox{
  $$ \int f(x) dx = F(x) + c $$
}
这是假数学书的写法,左边的$x$和右边的$x$完全是不同的意思。

我推荐的写法是:
 \tbox{ $$ F(x) = \int_a^x f(t) dt$$}
即原函数可以看成为$f$从某个固定端点$a$开始到$x$的定积分。等式两边的$x$是一个意思。

下面我们用直观的几何图像来解释为什么可以这样写。

\ech
\end{frame}


\begin{frame}
\chtitle{几何图像}
\bch
考虑定义式$$ F(x) = \int_a^x f(t) dt$$
即$F(x)$为图中所示的青色区域面积。

\bmini{0.45}
考虑$x$变化小量$dx$,则$F$的变化量为图中红色区域面积,近似地有:
$$d F = f(x) dx$$
所以$F$确实满足原函数的定义:
$$F'(x) = \frac{dF}{dx} = f(x)$$

\emini
\bmini{0.5}
\addfig{2.2}{integral_derv.jpg}
\emini

\ech
\end{frame}

\begin{frame}
  \chtitle{检验一下学习成果}
  \bch
  \addfig{1}{think.jpg}
  
  设$f(x,t)$为二元函数,定义:
  $$ F(x) = \int_{a(x)}^{b(x)} f(x, t) dt $$
  用画图的方法直接说明下式成立:
  $$ F'(x) =b'(x)\,f\left(b(x)\right)- a'(x)\,f\left(a(x)\right) + \int_{a(x)}^{b(x)} \partial_x f(x, t)\, dt $$

  \ech
\end{frame}

\begin{frame}
  \chtitle{带参量的积分}
  \bch
  前面两个例子的共同点是,在对$t$的定积分里包含了参量$x$,从而结果依赖于$x$,可以把结果看成$x$的一个函数。
  $$ F({\color{red} x}) = \int_a^{{\color{red} x}} f(t) dt$$
  $$ F({\red x}) = \int_{a({\red x})}^{b({\red x})} f({\red x}, t) dt $$  
  \skiplines
  
  这种用带参量的积分定义新函数的方法在本课非常重要,很快我们会学习如何把它推广到复数的积分的情况。

  
  \ech
\end{frame}

\begin{frame}
  \chtitle{$\Gamma$函数}
  \bch
  不难证明$n$为非负整数时
  $$\int_0^\infty t^n e^{-t}dt = n!$$
  利用带参量的积分定义新函数的方法,我们很容易地可以把“阶乘”这个本来只适用于非负整数的函数推广到非负实数的情况。定义:
  \tbox{
  $$\Gamma\left(x\right) \equiv \int_0^\infty t^{x-1}e^{-t} dt $$}
  这就是著名的$\Gamma$函数。我们学习了复变函数之后,还会把它的定义域扩张到几乎整个复数域。
  \ech
\end{frame}


\begin{frame}
  \chtitle{$\Gamma$函数的基本性质I}
  \bch
  根据
  \tbox{
  $$\Gamma\left(x\right) \equiv \int_0^\infty t^{x-1}e^{-t} dt $$}
  证明:
  \tbox{$$\Gamma\left(x+1\right) = x\,\Gamma\left(x\right)$$}
由此易得 $\Gamma(n+1) = n!\ (n = 0,1,2,\ldots).$
\ech
\end{frame}

\begin{frame}
  \chtitle{$\Gamma$函数的基本性质II}
  \bch
  \addfig{1}{think2.jpg}
  对$0<x<1$,证明:
  \tbox{$$\Gamma\left(x\right)\Gamma\left(1-x\right) =  \frac{\pi}{\sin{(\pi x)}}$$}

  由此可以推出高斯积分公式(瞬间勾起对麦克斯韦分布的美好回忆\wulian)
  $$\Gamma(\frac{1}{2}) = \sqrt{\pi}\, ,$$
  
  \ech
\end{frame}


\begin{frame}
\chtitle{预备知识:二阶导数}
\bch
二阶导数定义为导数的导数,即“斜率的变化率”。

若$f''(x)>0$,则斜率持续增大,$f(x)$为凹函数。

若$f''(x)<0$,则斜率持续减少,$f(x)$为凸函数。

\addfig{3}{convex_concave.png}

\ech
\end{frame}

\begin{frame}
\chtitle{预备知识:高阶导数}
\bch
一般地,可定义$n$阶导数为$n-1$阶导数的导数。即

\tbox{
$$f^{(n)}(x) = \lim_{dx\rightarrow 0}\frac{f^{(n-1)}(x+dx) - f^{(n-1)}(x)}{dx}$$
}

\ech
\end{frame}

\begin{frame}
\chtitle{预备知识:用导数近似求值}
\bch

假设$f$在$x_0$处的各阶导数均已知,在$x_0$附近的$x$处的函数值可以近似写成
$$f(x) \approx f(x_0) + f'(x_0) (x-x_0)$$
为了改善这个估算的精度,我们可以先估算一阶导数:
$$f'(x) \approx f'(x_0) + f''(x_0)(x-x_0)$$
再对上式从$x_0$到$x$积分得到二阶近似:
$$f(x) \approx f(x_0) + f'(x_0)(x-x_0) + \frac{1}{2}f''(x_0)(x-x_0)^2$$
\ech
\end{frame}

\begin{frame}
\chtitle{预备知识:用导数近似求值(续)}
\bch
还不够精确?那好,把$f'$看成新的函数,并对$f'$使用上面得到的二阶近似公式:
$$ f'(x) \approx f'(x_0) + f''(x_0)(x-x_0) + \frac{1}{2}f'''(x_0)(x-x_0)^2$$
然后从$x_0$到$x$积分得到
$$f(x) \approx f(x_0) + f'(x_0)(x-x_0) + \frac{1}{2}f''(x_0)(x-x_0)^2 + \frac{1}{6}f'''(x_0) (x-x_0)^3$$
\ech
\end{frame}


\begin{frame}
\chtitle{预备知识:泰勒级数展开}
\bch
把上述过程进行无限多次,得到:
\tbox{
  $$ f(x)  = \sum_{n=0}^\infty \frac{f^{(n)}(x_0)}{n!} (x-x_0)^n $$
}
特别地如取$x_0=0$,
\tbox{
  $$ f(x)  = \sum_{n=0}^\infty \frac{f^{(n)}(0)}{n!} x^n $$
}

\skipline

什么?收敛性怎么判断?再见\bye
\ech
\end{frame}


\begin{frame}
\chtitle{用无穷级数定义函数}
\bch
反过来,对给定的无穷数列$c_0, c_1, c_2, \ldots$,可以定义
$$ f(x) = \sum_{n=0}^\infty c_n x^n $$
这个定义新的函数的方法也非常重要,很容易把上式中的所有数推广到复数的情形。
\ech
\end{frame}


\begin{frame}
\chtitle{贝塞尔函数 (Bessel Functions)}
\bch
$n$阶贝塞尔函数定义为
\tbox
{$$J_n(x) \equiv \sum_{k=0}^\infty \frac{(-1)^k}{k!(k+n)!} \left(\frac{x}{2}\right)^{2k+n},\ (n\ge 0)$$ }
\addfig{2.5}{bessel_functions.png}
\ech
\end{frame}


\begin{frame}
\chtitle{贝塞尔函数的积分表达式}
\bch
由定义式
\tbox
{$$J_n(x) \equiv \sum_{k=0}^\infty \frac{(-1)^k}{k!(k+n)!} \left(\frac{x}{2}\right)^{2k+n},\ (n\ge 0)$$ }
证明
\tbox
{$$J_n(x) = \frac{1}{\pi} \int_0^\pi\cos\left(nt - x \sin t\right)dt$$}
\ech
\end{frame}



\begin{frame}
\chtitle{附录:基本的微积分公式}
\bch
\tbox{$$d \left(x^n\right) = n x^{n-1} dx;\ d\left(\ln x\right) = \frac{dx}{x}$$}
\tbox{$$d\left(e^x\right) = e^x dx;\ d(\sin x) = \cos x\, dx;\ d(\cos x) = -\sin x\, dx$$}
\tbox{$$d(xy) = xdy + ydx$$}
\tbox{$$d\left[f\left(g(x)\right)\right] = f'\left(g(x)\right) g'(x) dx$$}
\ech
\end{frame}



\section{Complex Functions}


\begin{frame}
  \bch
  {\Large 下面进入对复变函数的讨论}
\ech
\end{frame}

\begin{frame}
\chtitle{虚数单位$\ii$ (imaginary unit)}
\bch
规定:
\tbox{$$\ii^2 = -1 $$}

\bmini{0.54}
一般的复数$z$可以写成$z=a +  \ii b $ ($a, b\in \Re$) 的形式,对应复平面上的点$(a, b)$。

$a$称为实部(real part),记作$\mathrm{Re}(z)$; $b$称为虚部(imaginary part),记作$\mathrm{Im}(z)$。

\emini
\bmini{0.42}
\lfig{2.}{complex_number.png}
\emini
\ech
\end{frame}



\begin{frame}
\chtitle{复数的模(modulus)和幅角(phase)}
\bch
\tbox{
$$ a+\ii b = r (\cos\theta + \ii \sin\theta)  $$}

模$r$和幅角$\theta$满足:
$$ r = \sqrt{a^2+b^2},\, \cos\theta = \frac{a}{r},\, \sin\theta = \frac{b}{r}$$
\bmini{0.54}
这相当于用极坐标来表示复平面上的点

\skipline

复数$z$的模记作$|z|$,幅角记作$\arg z$
\emini
\bmini{0.42}
\lfig{1.5}{complex_number_modulus.png}
\emini

\ech
\end{frame}


\begin{frame}
\chtitle{共轭复数}
\bch


\bmini{0.5}
$z = a+\ii b$ ($a, b\in \Re$)的共轭复数定义为$z$关于实轴的镜像:
$$\bar{z} \equiv a - \ii b$$

容易验证:

$$z\bar{z} = |z|^2$$
\emini
\bmini{0.46}
\addfig{2}{conjugate.png}
\emini
\ech
\end{frame}

\begin{frame}
\chtitle{复数的加法的几何意义}
\bch
把复数$x+\ii y$看成从原点出发到$(x, y)$的矢量,则两个复数的加法等效于矢量加法:

\addfig{2}{complex_addition.png}
\ech
\end{frame}

\begin{frame}
\chtitle{复数的乘法的几何意义}
\bch

复数相乘,可以把模相乘,幅角相加。若
$$z_1=r_1(\cos\theta_1+\ii \sin\theta_1),\ z_2= r_2(\cos\theta_2+\ii \sin\theta_2)$$
则
$$z_1z_2 = (r_1r_2)\left[\cos(\theta_1+\theta_2) + \ii \sin(\theta_1+\theta_2)\right]$$

这很容易用三角函数公式进行验证,但背后是否有什么更深刻的原因呢?

\ech
\end{frame}


\begin{frame}
  \chtitle{第一个复变函数:指数函数}
  \bch
  用级数定义新函数的方法,定义复数的指数函数为:
  $$ e^z\equiv \sum_{n=0}^\infty \frac{1}{n!}z^n$$
  \ech
\end{frame}

\begin{frame}
\chtitle{欧拉(Euler)公式}
\bch
若$\theta$为实数,有
\tbox{$$ e^{\ii \theta} = \cos\theta + \ii \sin\theta $$}

\addfig{1}{think3.jpg}

哪位dalao自告奋勇来证明一下!

\ech
\end{frame}


\begin{frame}
\chtitle{复数的指数表示}
\bch
利用欧拉公式,复数的极坐标表达式就可以很简洁地写成$z = r e^{\ii \theta}$。这叫复数的指数表示。

容易证明,指数函数在实数域内的性质在复数域内仍成立
$$ e^{z_1}e^{z_2} = e^{z_1+z_2} $$
(证明留为作业)。于是
$$r_1e^{\ii\theta_1} r_2e^{\ii \theta_2} = (r_1r_2) e^{\ii(\theta_1+\theta_2)}$$
这样书写,复数乘法的几何意义就很清晰了。

\ech
\end{frame}


\begin{frame}
\chtitle{$n$次方程的复数根}
\bch
$n$次方程
$$z^n + c_{n-1}z^{n-1} + c_{n-2}z^{n-2}+\ldots + c_1 z+c_0=0$$
有$n$个复数根$z_0, z_1, \ldots, z_{n-1}$。于是有因式分解
$$z^n + c_{n-1}z^{n-1} + c_{n-2}z^{n-2}+\ldots + c_1 z+c_0=\prod_{k=0}^{n-1}\left(z-z_k\right)$$
通过比较两边的各项系数,有
$$ \sum_{k=0}^{n-1}z_k = -c_{n-1},\  \sum_{i\ne j} z_iz_j = c_{n-2},\ \ldots,\ \prod_{k=0}^{n-1}z_k = (-1)^nc_0$$

\ech
\end{frame}


\begin{frame}
\chtitle{$z^n=1$的复数根}
\bch
特别地,方程$z^n=1$的$n$个复数根为
$$e^{\frac{2\pi k \ii}{n}}, \ (k=0,1,2,\ldots,n-1)$$
这些根均匀地分布在单位圆$|z|=1$上。

比较
$$z^n-1 = \prod_{k=0}^{n-1}\left(z-e^{\frac{2\pi k \ii}{n}}\right)$$
两边的$z^{n-1}$项,可以得到
$$\sum_{k=0}^{n-1} e^{\frac{2\pi k \ii}{n}} = 0\, .$$
从复数的矢量意义来看,上式是显然的。
\ech
\end{frame}



\section{Homework}

\begin{frame}
\chtitle{课后作业}
\bch
\bitem
\item[1]{计算$\Gamma(5/2)$的精确值。}
\item[2]{计算$J_0(0.1)$和$J_1(0.2)$的近似值,保留三位有效数。}
\item[3]{从指数函数的级数定义出发,证明$$ e^{z_1+z_2}= e^{z_1}e^{z_2}\, .$$}
\item[*4]{证明对正整数$n>1$,有
  $$\sin{\frac{\pi}{n}} \,  \sin{\frac{2\pi}{n}} \ldots  \sin{\frac{(n-1)\pi}{n}} = \frac{n}{2^{n-1}}$$
}
  
\eitem
\ech
\end{frame}

\end{document}
