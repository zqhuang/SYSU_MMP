\documentclass[CJK]{beamer}
\usepackage{CJKutf8}
\usepackage{beamerthemesplit}
\usetheme{Malmoe}
\useoutertheme[footline=authortitle]{miniframes}
\usepackage{amsmath}
\usepackage{amssymb}
\usepackage{graphicx}
\usepackage{eufrak}
\usepackage{color}
\usepackage{slashed}
\usepackage{simplewick}
\usepackage{tikz}
\usepackage{tcolorbox}
\graphicspath{{../figures/}}
%%figures
\def\lfig#1#2{\includegraphics[width=#1 in]{#2}}
\def\addfig#1#2{\begin{center}\includegraphics[width=#1 in]{#2}\end{center}}
\def\wulian{\includegraphics[width=0.18in]{emoji_wulian.jpg}}
\def\bigwulian{\includegraphics[width=0.35in]{emoji_wulian.jpg}}
\def\bye{\includegraphics[width=0.18in]{emoji_bye.jpg}}
\def\bigbye{\includegraphics[width=0.35in]{emoji_bye.jpg}}
\def\huaixiao{\includegraphics[width=0.18in]{emoji_huaixiao.jpg}}
\def\bighuaixiao{\includegraphics[width=0.35in]{emoji_huaixiao.jpg}}
\def\jianxiao{\includegraphics[width=0.18in]{emoji_jianxiao.jpg}}
\def\bigjianxiao{\includegraphics[width=0.35in]{emoji_jianxiao.jpg}}
%% colors
\def\blacktext#1{{\color{black}#1}}
\def\bluetext#1{{\color{blue}#1}}
\def\redtext#1{{\color{red}#1}}
\def\darkbluetext#1{{\color[rgb]{0,0.2,0.6}#1}}
\def\skybluetext#1{{\color[rgb]{0.2,0.7,1.}#1}}
\def\cyantext#1{{\color[rgb]{0.,0.5,0.5}#1}}
\def\greentext#1{{\color[rgb]{0,0.7,0.1}#1}}
\def\darkgray{\color[rgb]{0.2,0.2,0.2}}
\def\lightgray{\color[rgb]{0.6,0.6,0.6}}
\def\gray{\color[rgb]{0.4,0.4,0.4}}
\def\blue{\color{blue}}
\def\red{\color{red}}
\def\green{\color{green}}
\def\darkgreen{\color[rgb]{0,0.4,0.1}}
\def\darkblue{\color[rgb]{0,0.2,0.6}}
\def\skyblue{\color[rgb]{0.2,0.7,1.}}
%%control
\def\be{\begin{equation}}
\def\ee{\nonumber\end{equation}}
\def\bea{\begin{eqnarray}}
\def\eea{\nonumber\end{eqnarray}}
\def\bch{\begin{CJK}{UTF8}{gbsn}}
\def\ech{\end{CJK}}
\def\bitem{\begin{itemize}}
\def\eitem{\end{itemize}}
\def\bcenter{\begin{center}}
\def\ecenter{\end{center}}
\def\bex{\begin{minipage}{0.2\textwidth}\includegraphics[width=0.6in]{jugelizi.png}\end{minipage}\begin{minipage}{0.76\textwidth}}
\def\eex{\end{minipage}}
\def\chtitle#1{\frametitle{\bch#1\ech}}
\def\bmat#1{\left(\begin{array}{#1}}
\def\emat{\end{array}\right)}
\def\bcase#1{\left\{\begin{array}{#1}}
\def\ecase{\end{array}\right.}
\def\bmini#1{\begin{minipage}{#1\textwidth}}
\def\emini{\end{minipage}}
\def\tbox#1{\begin{tcolorbox}#1\end{tcolorbox}}
\def\pfrac#1#2#3{\left(\frac{\partial #1}{\partial #2}\right)_{#3}}
%%symbols
\def\bropt{\,(\ \ \ )}
\def\sone{$\star$}
\def\stwo{$\star\star$}
\def\sthree{$\star\star\star$}
\def\sfour{$\star\star\star\star$}
\def\sfive{$\star\star\star\star\star$}
\def\rint{{\int_\leftrightarrow}}
\def\roint{{\oint_\leftrightarrow}}
\def\stdHf{{\textit{\r H}_f}}
\def\deltaH{{\Delta \textit{\r H}}}
\def\ii{{\dot{\imath}}}
\def\skipline{{\vskip0.1in}}
\def\skiplines{{\vskip0.2in}}
\def\lagr{{\mathcal{L}}}
\def\hamil{{\mathcal{H}}}
\def\vecv{{\mathbf{v}}}
\def\vecx{{\mathbf{x}}}
\def\vecy{{\mathbf{y}}}
\def\veck{{\mathbf{k}}}
\def\vecp{{\mathbf{p}}}
\def\vecn{{\mathbf{n}}}
\def\vecA{{\mathbf{A}}}
\def\vecP{{\mathbf{P}}}
\def\vecsigma{{\mathbf{\sigma}}}
\def\hatJn{{\hat{J_\vecn}}}
\def\hatJx{{\hat{J_x}}}
\def\hatJy{{\hat{J_y}}}
\def\hatJz{{\hat{J_z}}}
\def\hatj#1{\hat{J_{#1}}}
\def\hatphi{{\hat{\phi}}}
\def\hatq{{\hat{q}}}
\def\hatpi{{\hat{\pi}}}
\def\vel{\upsilon}
\def\Dint{{\mathcal{D}}}
\def\adag{{\hat{a}^\dagger}}
\def\bdag{{\hat{b}^\dagger}}
\def\cdag{{\hat{c}^\dagger}}
\def\ddag{{\hat{d}^\dagger}}
\def\hata{{\hat{a}}}
\def\hatb{{\hat{b}}}
\def\hatc{{\hat{c}}}
\def\hatd{{\hat{d}}}
\def\hatN{{\hat{N}}}
\def\hatH{{\hat{H}}}
\def\hatp{{\hat{p}}}
\def\Fup{{F^{\mu\nu}}}
\def\Fdown{{F_{\mu\nu}}}
\def\newl{\nonumber \\}
\def\vece{\mathrm{e}}
\def\calM{{\mathcal{M}}}
\def\calT{{\mathcal{T}}}
\def\calR{{\mathcal{R}}}
\def\barpsi{\bar{\psi}}
\def\baru{\bar{u}}
\def\barv{\bar{\upsilon}}
\def\qeq{\stackrel{?}{=}}
\def\torder#1{\mathcal{T}\left(#1\right)}
\def\rorder#1{\mathcal{R}\left(#1\right)}
\def\contr#1#2{\contraction{}{#1}{}{#2}#1#2}
\def\trof#1{\mathrm{Tr}\left(#1\right)}
\def\trace{\mathrm{Tr}}
\def\comm#1{\ \ \ \left(\mathrm{used}\ #1\right)}
\def\tcomm#1{\ \ \ (\text{#1})}
\def\slp{\slashed{p}}
\def\slk{\slashed{k}}
\def\calp{{\mathfrak{p}}}
\def\veccalp{\mathbf{\mathfrak{p}}}
\def\Tthree{T_{\tiny \textcircled{3}}}
\def\pthree{p_{\tiny \textcircled{3}}}
\def\dbar{{\,\mathchar'26\mkern-12mu d}}
\def\erf{\mathrm{erf}}
\def\const{\mathrm{constant}}
\def\pheat{\pfrac p{\ln T}V}
\def\vheat{\pfrac V{\ln T}p}
%%units
\def\fdeg{{^\circ \mathrm{F}}}
\def\cdeg{^\circ \mathrm{C}}
\def\atm{\,\mathrm{atm}}
\def\angstrom{\,\text{\AA}}
\def\SIL{\,\mathrm{L}}
\def\SIkm{\,\mathrm{km}}
\def\SIyr{\,\mathrm{yr}}
\def\SIGyr{\,\mathrm{Gyr}}
\def\SIV{\,\mathrm{V}}
\def\SImV{\,\mathrm{mV}}
\def\SIeV{\,\mathrm{eV}}
\def\SIkeV{\,\mathrm{keV}}
\def\SIMeV{\,\mathrm{MeV}}
\def\SIGeV{\,\mathrm{GeV}}
\def\SIcal{\,\mathrm{cal}}
\def\SIkcal{\,\mathrm{kcal}}
\def\SImol{\,\mathrm{mol}}
\def\SIN{\,\mathrm{N}}
\def\SIHz{\,\mathrm{Hz}}
\def\SIm{\,\mathrm{m}}
\def\SIcm{\,\mathrm{cm}}
\def\SIfm{\,\mathrm{fm}}
\def\SImm{\,\mathrm{mm}}
\def\SInm{\,\mathrm{nm}}
\def\SImum{\,\mathrm{\mu m}}
\def\SIJ{\,\mathrm{J}}
\def\SIW{\,\mathrm{W}}
\def\SIkJ{\,\mathrm{kJ}}
\def\SIs{\,\mathrm{s}}
\def\SIkg{\,\mathrm{kg}}
\def\SIg{\,\mathrm{g}}
\def\SIK{\,\mathrm{K}}
\def\SImmHg{\,\mathrm{mmHg}}
\def\SIPa{\,\mathrm{Pa}}

\def\courseurl{https://github.com/zqhuang/SYSU\_TD}

\def\tpage#1#2{
\begin{frame}
\begin{center}
\begin{Large}
\bch
热学 \\
第#1讲 #2

{\vskip 0.3in}

黄志琦

\ech
\end{Large}
\end{center}

\vskip 0.2in

\bch
教材:《热学》第二版,赵凯华,罗蔚茵,高等教育出版社
\ech

\bch
课件下载
\ech
\courseurl
\end{frame}
}

\def\bfr#1{
\begin{frame}
\chtitle{#1} 
\bch
}

\def\efr{
\ech 
\end{frame}
}

  \date{}
  \begin{document}
  \bch
\tpage{18}{Equations with Source Terms}


\begin{frame}
  \frametitle{Outline}
  \tableofcontents
\end{frame}


\section{Source in the Equation}
\secpage{有源的方程}{把源进行分解}

\begin{frame}
\frametitle{例题1}
求解$0\le x\le L$上的烤串问题:
\bea
\frac{\partial u}{\partial t} - a\frac{\partial^2u}{\partial x^2} &=& \phi(x,t), \newl
\left. u \right\vert_{x=0} = 0, \newl
\left. u \right\vert_{x=L} = 0, \newl
\left. u \right\vert_{t=0} = 0. \newl
\eea
这里热源$\phi(x,t)$是已知函数。
\end{frame}

\begin{frame}
\frametitle{改变时间演化因子}
如果没有$\phi(x,t)$的存在,我们将进行分解:
$$u = \sum_{n = 0}^\infty c_ne^{-\frac{an^2\pi^2t}{L^2} } \sin\frac{n\pi x}{L}. $$
有$\phi(x,t)$的情况,我们猜想 $e^{-\frac{an^2\pi^2t}{L^2} }$有可能需要替换成一个其他什么东西:
$$u = \sum_{n = 0}^\infty T_n(t) \sin\frac{n\pi x}{L}. $$
其中每个$T_n(t)$都是待定函数。利用初始条件显然有
$$T_n(0) = 0. $$
\end{frame}


\begin{frame}
\frametitle{对源的分解}

为了求出$T_n$,我们把等式右边的$\phi(x,t)$也进行级数展开:
$$\phi(x,t) = \sum_{n=0}^\infty G_n(t) \sin\frac{n\pi x}{L}. $$
代入到原方程,得到
$$ \sum_{n = 0}^\infty\left[ T_n'(t) +\frac{n^2\pi^2a}{L^2}T(t)\right] \sin\frac{n\pi x}{L} = \sum_{n=0}^\infty G_n(t) \sin\frac{n\pi x}{L}.$$
两边比较系数得到
$$ T_n'(t) + \frac{n^2\pi^2a}{L^2}T(t) = G_n(t). $$
再利用初始条件$T_n(0)  = 0$,原则上(\wulian)可以求出$T_n(t)$。
\end{frame}

\begin{frame}
  \frametitle{思考题}
  例题中如果给定$\phi(x,t) = \frac{1}{\tau}\sin\frac{\pi x}{L}e^{-\frac{a\pi^2t}{L^2}}$(这里$\tau>0$为常量),试求出具体的$u(x,t)$。
\end{frame}


\section{Nonlinear Boundary Conditions}
\secpage{非齐次边界条件}{特解=渐近解}




\begin{frame}
  \frametitle{例题2}
  
  \addfig{3}{HeatEq2.png}
  
  长度为$L$的导热棒一端和温度为零(这里是随意规定了一个温度零点,不是绝对零度)的热库接触,并在$t=0$时刻和热库处于热平衡。从$t=0$时刻开始,在导热棒的另一端注入恒定大小为$j$的热流。设已知导热棒的导热系数$\lambda$和热传导方程参数$a$,求解导热棒上温度$u(x, t)$ ($0\le x\le L, t\ge 0$)。
  
\end{frame}

\begin{frame}
  \frametitle{数理方程}
      \bea
    \frac{\partial u}{\partial t} -a\frac{\partial^2u}{\partial x^2} &=& 0, \newl
    \left.u\right\vert_{x=0} &=& 0, \newl
    \left.\frac{\partial u}{\partial x}\right\vert_{x=L} &=& \frac{j}{\lambda}, \newl
    \left.u\right\vert_{t=0} &=& 0.
    \eea

    下一步目标是通过寻找特解,把非线性的边界条件转化为非线性的初始条件,回到标准套路。{\blue 对热传导方程,寻找特解有一个非常简便的办法:分析渐近行为。}
\end{frame}

\begin{frame}
  \frametitle{渐近行为分析}
  
  猜想当$t$远大于典型热扩散时间$L^2/a$时,系统处于稳恒状态(温度梯度不再变化)。因为一端温度是固定的,要得到稳恒状态的必要条件是热量不在导热棒上积累,也就是说进来的热流$j$必须保持不变地通过整个导热棒,最后从另一端进入热库。这说明稳恒状态下$\frac{\partial u}{\partial x}$处处等于$\frac{j}{\lambda}$。由此得出:
    $$u(x, t) \rightarrow \frac{j}{\lambda} x$$
  这就是我们需要的特解。
\end{frame}

\begin{frame}
  \frametitle{分离变量法}
  令
  $$u(x,t) = \frac{j}{\lambda}x + \upsilon(x,t)$$
  易见$\upsilon$也满足热传导方程,且
  \bea
  \left.\upsilon\right\vert_{x=0} &=& 0, \newl
  \left.\frac{\partial \upsilon}{\partial x}\right\vert_{x=L} &=& 0, \newl
  \left.\upsilon\right\vert_{t=0} &=& -\frac{j}{\lambda}x.
  \eea
  显然$\upsilon$可以用我们熟悉的“标准套路”解出来。
\end{frame}


\begin{frame}
  \frametitle{最终结果}
  解出$\upsilon$之后得到
  $$u(x,t) = \frac{j}{\lambda}x -\frac{2jL}{\lambda \pi^2}\sum_{n=0}^\infty \frac{(-1)^n}{\left(n+\frac{1}{2}\right)^2}e^{-a\frac{(n+\frac{1}{2})^2\pi^2}{L^2}t}\sin{\left(\frac{(n+\frac{1}{2})\pi}{L}x\right)}.$$
\end{frame}




\begin{frame}
  \frametitle{例题3}
  
  \addfig{3}{problem12-1.png}
  
  有长为$2L$,温度为$T_0$的均匀导热棒,其材质的热传导方程参数为 $a$。在$t=0$时刻,在它的两端$x=\pm L$处分别接上温度为$T_1$的相同材质相同截面形状的非常长的均匀导热棒。求之后导热棒上的温度变化。
  
\end{frame}




\begin{frame}
  \frametitle{写出方程}
  
  \addfig{3}{problem12-1.png}

  显然渐近解是$T=T_1$,所以令$T(x,t) = T_1 + u(x,t)$,$u$满足
  \bea
  \frac{\partial u}{\partial t} - a\frac{\partial^2 u}{\partial x^2} &=& 0, \newl
  u_{t=0} &=&  (T_0-T_1) \theta_L(x) .
  \eea
  其中$\theta_L(x)$当且仅当$|x|<L$时为$1$,否则为零。
  
  
\end{frame}


\begin{frame}
  \frametitle{格林函数方法求解}
  \bea
  u(x,t) &=& \int_{-L}^L \frac{1}{\sqrt{4\pi at}}e^{-\frac{(x-x_0)^2}{4at}} (T_0-T_1)dx_0 \newl
  &=& \frac{T_0-T_1}{\sqrt{4\pi at}} \int_{-L}^L e^{-\frac{(x-x_0)^2}{4at}} dx_0
  \eea
  当然,如果你喜欢,可以把上述积分写成误差函数。

  最后结果为
  $$T(x,t) = T_1 +  \frac{T_0-T_1}{\sqrt{4\pi at}} \int_{-L}^L e^{-\frac{(x-x_0)^2}{4at}} dx_0 $$
  
\end{frame}



  
\section{Homework}

\begin{frame}
  \frametitle{Homework}
  \bitem
  \item{请补充完整例题2中的$\upsilon$的求解过程。}
  \item{有长为$L$,温度为$T_0$的均匀导热棒,材质的热传导方程参数为 $a$。在$t=0$时刻,在它的一端$x= L$处接上温度为$T_1$的相同材质相同截面形状的非常长的均匀导热棒。求之后导热棒上的温度变化。

  \addfig{3}{problem12-2.png}
}
  \eitem
\end{frame}

\section{Appendix}

\begin{frame}
  \frametitle{附录1:平均温度一直变化的例子}
  
  \addfig{2.5}{heatflux2.png}
  
  在一根长为$2L$的导热棒在$t=0$时刻温度为$T_0$。在$t>0$时刻,导热棒两端均有强度为$j$的热流进入。设材料的导热系数$\lambda$,质量密度$\rho$,单位质量的比热$c$均已知,试计算$t\ge 0$时刻导热棒各处的温度$T(x,t)$。
  
\end{frame}

\begin{frame}
  \frametitle{附录1:平均温度一直变化的例子}
  
  根据对称性,在棒中间处热流和温度梯度均为零。写出如下的方程和边界条件:
  \bea
  \frac{\partial T}{\partial t} - a\frac{\partial^2 T}{\partial x^2} &=& 0 \newl
  \left.\frac{\partial T}{\partial x}\right\vert_{x=0} &=& 0 \newl
  \left.\frac{\partial T}{\partial x}\right\vert_{x=L} &=& \frac{j}{\lambda}  \newl
  \left.T\right\vert_{t=0} &=&  T_0 
  \eea
  其中$a = \frac{\lambda}{\rho c} $。
  
  
\end{frame}


\begin{frame}
  \frametitle{附录1:平均温度一直变化的例子}
  
  先分析主要图像。

  \skiplines

  在$t$时刻,累计流入的热量为$Q =  2 j St$ (其中$S$为横截面积)。棒的热容为$C =  c \rho (2SL)$。所以$t$时刻棒的的平均温度为
  $$ \bar{T} =   T_0  + \frac{Q}{C} = T_0 + \frac{j}{\rho cL}t $$
  
  
\end{frame}


\begin{frame}
  \frametitle{附录1:平均温度一直变化的例子}
  
  把平均温度去掉,研究各处温度起伏:$\Delta T(x, t) = T(x, t) - \left(T_0+\frac{j}{\rho cL} t\right)$。显然$\Delta T$满足方程:
  \bea
  \frac{\partial \Delta T }{\partial t} - a \frac{\partial^2 \Delta T}{\partial x^2} &=&  -\frac{j}{\rho cL} \newl
  \left.\frac{\partial \Delta T}{\partial x}\right\vert_{x=0} &=& 0 \newl
  \left.\frac{\partial \Delta  T}{\partial x}\right\vert_{x=L} &=& \frac{j}{\lambda}  \newl
  \left.\Delta T\right\vert_{t=0} = 0
  \eea
  因为$\Delta T$描述的是温度起伏,还有一个额外条件:
  $$\int_0^L \Delta T(x, t) dx = 0 $$
  
\end{frame}

\begin{frame}
  \frametitle{附录1:平均温度一直变化的例子}
  
  当$t$很大时,棒上的温度梯度趋于稳定,即$\Delta T$仅仅依赖于$x$,满足
  \bea
  - a \frac{\partial^2 \Delta T}{\partial x^2} &=&  -\frac{j}{\rho cL} \newl
  \left.\frac{\partial \Delta T}{\partial x}\right\vert_{x=0} &=& 0 \newl
  \left.\frac{\partial \Delta  T}{\partial x}\right\vert_{x=L} &=& \frac{j}{\lambda}  \newl
  \int_0^L \Delta T(x, t) dx &=& 0  
  \eea
  由此不难解出
  $$\Delta T = \frac{j}{2\lambda} \left(\frac{x^2}{L} - \frac{L}{3}\right) $$
  
\end{frame}


\begin{frame}
  \frametitle{附录1:平均温度一直变化的例子}  
  把“稳恒解”(虽然平均温度不断变化,但各处的温度梯度稳定不变)当作特解,令
  $$ T = \left(T_0+\frac{j}{\rho cL} t\right) + \frac{j}{2\lambda} \left(\frac{x^2}{L} - \frac{L}{3}\right) +\delta T(x, t), $$
  这里的$\delta T(x,t)$描述了解对稳恒态的偏差如何衰减。把$T(x,t)$直接代入初始的方程和边界条件,易见$\delta T$也满足热传导方程,并满足:
  \bea
  \left.\frac{\partial \delta T}{\partial x}\right\vert_{x=0} &=& 0 \newl
  \left.\frac{\partial \delta T}{\partial x}\right\vert_{x=L} &=& 0 \newl
  \left.\delta T\right\vert_{t=0} &=&  -\frac{j}{2\lambda} \left(\frac{x^2}{L} - \frac{L}{3}\right) 
  \eea

\end{frame}

\begin{frame}
  \frametitle{附录1:平均温度一直变化的例子}  
  显然$\delta T$可以用标准套路解出来。 请自行完成这部分计算。  最后的完整解是:
{\small  $$ T = \left(T_0+\frac{j}{\rho cL} t\right) + \frac{j}{2\lambda} \left(\frac{x^2}{L} - \frac{L}{3}\right) - \frac{2jL}{\lambda \pi^2}\sum_{n=1}^\infty \frac{(-1)^n}{n^2}e^{-a\frac{n^2\pi^2}{L^2}t}\cos{\left(\frac{n\pi }{L}x\right)}$$}

  这个解的第一个括号内是平均温度的变化,第二部分描述稳恒态的形状,第三部分描述初始时对稳恒态的偏离是如何衰减掉的。
\end{frame}


\begin{frame}
  \frametitle{附录2:波动方程,不好借助渐近解的例子}
求解一端固定,另一端正弦驱动的弦振动:
\bea
\frac{\partial ^2u}{\partial t^2}  -  a^2 \frac{\partial^2u}{\partial x^2} &=& 0, \newl
\left.u\right\vert_{x=0} &=& 0,\newl
\left.u\right\vert_{x=L} &=& A\sin (\omega t),\newl
\left.u\right\vert_{t=0} &=& 0 , \newl
\left.\frac{\partial u}{\partial t}\right\vert_{t=0} &=&  0.
\eea
\end{frame}

\begin{frame}
\frametitle{附录2:波动方程,不好借助渐近解的例子(解法一)}
考虑如下的{\blue 满足边界条件但不满足初始条件的特解}:
$$ u_0(x,t) = A \sin(\omega t) \frac{\sin\frac{\omega x}{a}}{\sin\frac{\omega L}{a}}  $$
令$u(x,t) = u_0(x,t) + \upsilon(x,t)$,则$\upsilon(x,t)$满足:
\bea
\frac{\partial ^2\upsilon}{\partial t^2}  -  a^2 \frac{\partial^2\upsilon}{\partial x^2} &=& 0, \newl
\left.\upsilon\right\vert_{x=0} &=& 0, \newl
\left.\upsilon \right\vert_{x=L} &=& 0,\newl
\left.\upsilon \right\vert_{t=0} &=& 0 , \newl
\left.\frac{\partial \upsilon}{\partial t}\right\vert_{t=0} &=&  -A\omega \frac{\sin\frac{\omega x}{a}}{\sin\frac{\omega L}{a}}
\eea
$\upsilon(x,t)$显然可以用标准套路求解,请自行完成。
\end{frame}


% final solution is
% $$ u(x,t) = \frac{A}{\sin\frac{\omega L}{a}}\left\{\sin(\omega t)\sin\frac{\omega x}{a} + \frac{\omega L}{\pi a}\sum_{n=1}^\infty \frac{1}{n}\left[\mathrm{sinc}\left(\frac{\omega L}{a}+n\pi\right) - \mathrm{sinc}\left(\frac{\omega L}{a}-n\pi\right) \right]\right\} $$

\begin{frame}
\frametitle{附录2:波动方程,不好借助渐近解的例子(解法二)}

解法一的办法是把非线性的边界条件转化为非线性的初始条件,手段是找一个既满足方程又满足边界条件的特解,这往往需要高超的技巧。

事实上,我们还可以把非线性的边界条件转化为非线性的方程(准确地说是带源的方程),而且这样做对技巧的要求降低了很多: 你只要随便找一个满足边界条件但并不满足方程的“瞎解”。

\skipline

写一个满足边界条件但并不满足方程的解显然要容易得多,例如:
$$\frac{Ax}{L} \sin(\omega t)$$

\end{frame}

\begin{frame}
\frametitle{附录2:波动方程,不好借助渐近解的例子(解法二)}

令
$$ u(x,t)  = \frac{Ax}{L} \sin(\omega t) + \upsilon(x,t)$$
代回原方程,得到:

\bea
\frac{\partial ^2\upsilon}{\partial t^2}  -  a^2 \frac{\partial^2\upsilon}{\partial x^2} &=& \frac{A\omega^2x}{L} \sin(\omega t) , \newl
\left.\upsilon \right\vert_{x=0} &=& 0,\newl
\left.\upsilon \right\vert_{x=L} &=& 0,\newl
\left.\upsilon \right\vert_{t=0} &=& 0 , \newl
\left.\frac{\partial \upsilon}{\partial t}\right\vert_{t=0} &=&  -\frac{A\omega x}{L}
\eea
这样就把问题转化为求有源的方程。可以参考例题1的标准套路解决这个问题,请自行完成。

\end{frame}

\thinka{如果驱动频率为共振频率:$\omega = \frac{n\pi a}{L}$ ($n$为某正整数) ,上述解法还可行吗?}

%% when n=1 the solution is
%% $$ A\left[ -\frac{at}{L}\cos\frac{\pi at}{L}\sin\frac{\pi x}{L} - \frac{x}{L}\sin\frac{\pi at}{L}\cos\frac{\pi x}{L} + \frac{1}{2\pi}\sin\frac{\pi x}{L}\sin\frac{\pi a t}{L} + \frac{2}{\pi}\sum_{n=2}^\infty\frac{(-1)^n}{n^2-1}\sin\frac{n\pi x}{L}\sin\frac{n\pi a t}{L}\right]$$ 
\ech
\end{document}
