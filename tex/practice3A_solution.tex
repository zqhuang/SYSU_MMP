\documentclass[12pt,CJK]{article}
\usepackage{geometry}
\input{reduced_macros.tex}
\geometry{tmargin=0.3in, bmargin=0.5in, lmargin=0.5in, rmargin=0.9in, nohead, nofoot}
\def\mark#1{{\color{blue} (#1分)}}
\renewcommand{\thepage}{}
\begin{document}
\bch
{\large 数理方法 课堂小测III 初入江湖版}

{\vskip 0.2in}

\bitem
\item[(一)]{选择题,每题3分,共30分。

  \bitem

\item[(1)]{下列哪个物理问题对应的方程{\bf 不是}波动方程 \brans{C}
  
  \foptlist{均匀弦的横向小振动}{均匀弹性杆的纵向小振动}{{\blue 孤立均匀球的温度变化}}{引力波的传播}
}


\item[(2)]{下列哪个函数具有分离变量的形式 \brans{B}
  
  \optlist{$e^{-xt}$}{{\blue $\frac{x^2}{1+e^t}$}}{$\sqrt{t^2+x^2}$}{$e^{\frac{x}{t}}$}

{\red 所谓分离变量形式就是能写成$f(x)g(t)$形式的函数。}
}
  

\item[(3)]{下列关于格林函数的说法{\bf 错误}的是 \brans{D}
  
  \optlist{是线性系统对脉冲输入的响应}{依赖于脉冲输入的位置}{依赖于边界条件}{{\blue 只适用于无限大空间的问题}}

  {\red 格林函数是线性系统(不需要无限大)对单位脉冲输入的响应。如果改变边界条件(例如把边界上为零改为边界上导数为零),显然响应会发生变化。}
}


\item[(4)]{球坐标系$(r,\theta,\phi)$的拉普拉斯算符为: \brans{A}
  
  \goptlist{{\blue 
  $\frac{1}{r^2}\frac{\partial}{\partial r}\left(r^2\frac{\partial}{\partial r}\right) + \frac{1}{r^2\sin\theta}\frac{\partial}{\partial \theta}\left(\sin\theta\frac{\partial}{\partial\theta}\right) + \frac{1}{r^2\sin^2\theta}\frac{\partial^2}{\partial \phi^2} $}}
           {$\frac{1}{r}\frac{\partial}{\partial r}\left(r\frac{\partial}{\partial r}\right) + \frac{1}{r^2}\frac{\partial^2}{\partial \theta^2} + \frac{1}{r^2\sin^2\theta}\frac{\partial^2}{\partial \phi^2} $}
           {$\frac{\partial^2}{\partial r^2} + \frac{\partial^2}{\partial \theta^2} +\frac{\partial^2}{\partial \phi^2}$}
           {$\frac{\partial^2}{\partial r^2} + \frac{1}{r^2}\frac{\partial^2}{\partial \theta^2} + \frac{1}{r^2\sin^2\theta}\frac{\partial^2}{\partial \phi^2}$}
}  

\item[(5)]{贝塞尔函数的{\bf 导函数} $J'_{100}(x)$ 的最小正实数零点大约为 \brans{C}

  \optlist{$1$}{$10$}{{\blue $100$}}{$10^4$}

{\red 知识点: $J_m$的第一个峰值在$m$附近。}
}
  
\item[(6)]{ 下列哪个贝塞尔函数在 $x=0$ 处发散? \brans{D}
  
  \optlist{$J_1(x)$}{$J_{-1}(x)$}{$J_{1/2}(x)$}{\blue $J_{-1/2}(x)$}

{\red 知识点:对整数$m$,$J_{-m}(x) = (-1)^mJ_m(x)$;非整数阶$J_\nu$在$x\rightarrow 0^+$时的行为$\sim x^\nu$。注意$J_{-m}$的展开虽然粗看形式上有负次幂项,但那些项的系数为零,其实并不存在。}
}

\item[(7)]{ 积分$\int_0^a J_1(x) dx= $ \brans{B}
  
  \optlist{$J_0(a)$}{{\blue $1-J_0(a)$}}{$aJ_0(a)$}{$a J_{1}(a)$}

  {\red 见 Lec.16 贝塞尔函数的递推公式 $J_{m-1}-J_{m+1} = 2 J_{m}' $,令$m=0$并利用$J_{-1} = -J_1$。或者利用$(x^{-\nu} J_\nu)' = -x^{-\nu} J_{\nu +1}$并令$\nu=0$。 }
}

\item[(8)]{两类贝塞尔函数的平方和 $[J_{5}(100)]^2+[Y_{5}(100)]^2$ 大约为 \brans{A}

  \optlist{{\blue $0.006$}}{$0.01$}{$0.06$}{$1$}

{\red 见 Lec. 17 $J_m$在无穷远处的渐近展开以及 Lec.18 第二类贝塞尔函数的引入思想(即$x$很大时,$J_m\approx \sqrt{\frac{2}{\pi x}}\cos(x-\ldots)$, $Y_m\approx \sqrt{\frac{2}{\pi x}}\sin(x-\ldots)$)。}
}  

\item[(9)]{设$\mu, \nu$是球贝塞尔函数$j_\ell(x)$ ($\ell\ge 0$)的两个不同的零点,则下列哪个等式一定正确? \brans{C}
  
  \goptlist{$\int_0^1j_\ell(\mu x)j_\ell(\nu x)dx = 0 $}{ $\int_0^1j_\ell(\mu x)j_\ell(\nu x)xdx = 0 $}{{\blue $\int_0^1j_\ell(\mu x)j_\ell(\nu x)x^2dx = 0 $}}{$\int_0^{1}(1-x^2)j_\ell(\mu x)j_\ell(\nu x)dx = 0 $}

  {\red 解法一: 利用一般正交定理得出 $j_\ell(\mu x)Y_{\ell 0}$ 和 $j_\ell(\nu x)Y_{\ell 0}$ 在单位球内的正交性。

  解法二:把$j_\ell$转化为第一类贝塞尔函数并利用第一类贝塞尔函数的正交性(虽然严格来说我们在课上并未讨论分数阶的情形)。}
}
  
\item[(10)]{$\cos\theta\, Y_{4,2}(\theta,\phi)$ 可写成哪两个球面谐函数的线性组合? \brans{A}
  
  \foptlist{{\blue $ Y_{3,2}(\theta,\phi)$ 和 $Y_{5,2}(\theta,\phi)$}}{$Y_{4,2}(\theta,\phi)$ 和 $Y_{5,2}(\theta,\phi)$}{$Y_{3,2}(\theta,\phi)$ 和 $Y_{4,2}(\theta,\phi)$}{$Y_{4,1}(\theta,\phi)$ 和 $Y_{4,3}(\theta,\phi)$}

{\red  假设$\cos\theta Y_{4,2} = \sum_{\ell m} c_{\ell m}Y_{\ell m}$,利用正交归一条件写出$c_{\ell m}$的积分表达式,并考虑何时$c_{\ell m}$非零。注意$\cos\theta$正比于$Y_{10}$,见 Lec.23 三个球谐函数乘积的积分非零条件,以及$Y_{\ell m}^* = (-1)^m Y_{\ell,- m}$的对称性。}
}  

\eitem  
}
\item[(二)]{填空题(每题5分,共30分)
  \bitem
\item[(1)]{设二维正交曲面坐标系$(x,y)$的正交线元长度依次为 $dx$ 和 $e^{-x^2}dy$,写出该坐标系的拉普拉斯算符的显式表达: \underline{ \blue $e^{x^2}\frac{\partial}{\partial x}\left(e^{-x^2}\frac{\partial}{\partial x}\right) + e^{2x^2}\frac{\partial^2}{\partial y^2}$}

{\red 见 Lec.15 正交曲面坐标系的知识。由于这个知识在物理中十分重要,请确认你掌握了任意维空间的情况。}
}
\item[(2)]{积分$\int_0^1 x^4 P_6(x) dx = $ \underline{\blue $0$}。

  {\red 利用罗巨格公式容易看出偶数阶勒让德多项式都是偶函数,奇数阶勒让德函数都是奇函数。因此这个积分可以写成 $[-1,1]$上的积分的一半。而在$[-1,1]$内, $P_6$和低于6次的多项式都正交(因为低于6次的多项式总是能展开为$P_0,P_1,\ldots, P_5$的线性组合)。

    \skipline
    
    罗巨格公式是对付勒让德多项式的首选工具。对积分问题,除了用正交性直接判断之外,还能用分部积分$\ell$次的套路硬来。对求$P_\ell(t)$ ($t$为常数,特别是为某个特殊点$\pm 1$, $0$)的值问题,可以把$(x^2-1)^\ell$展开为$x-t$的多项式然后逐项求导$\ell$次。
  }
}
\item[(3)]{函数 $f_1(x,y)$, $f_2(x,y)$ 在椭圆 $2x^2+ y^2 = 1$ 上的取值处处为零,在该椭圆内部分别满足

  $ \nabla^2 f_1 =-k_1^2f_1$ 以及 $\nabla^2 f_2 = -k_2^2f_2$,其中$k_1>k_2>0$。

  则在该椭圆内的积分$\iint f_1(x,y)f_2(x,y) dx dy = $ \underline{\blue $0$}。


  {\red 见 Lec.18 一般正交定理。请确认你理解了一般零边界条件的意思。}
}
\item[(4)]{设 $\mu>0$,$J_2(\mu)=0$,则积分$\int_0^1\left[J_2(\mu x)\right]^2 x dx =$ \underline{\blue $\frac{\left[J_3(\mu)\right]^2}{2}$ (或$\frac{\left[J_2'(\mu)\right]^2}{2}$)}。

  {\red 见 Lec.16 贝塞尔函数的第一条正交定理。当然,贝塞尔函数的三条正交定理都非常重要。}
}  
\item[(5)]{一维无边界的空间内满足方程 $\frac{\partial^2u}{\partial t^2}-a^2\frac{\partial^2u}{\partial x^2} = 0$ 和初始条件  $\left.u\right\vert_{t=0} = e^{-x^2},\ \ \left.\frac{\partial u}{\partial t}\right\vert_{t=0} = 0$ 的解是: $u(x,t)=$ \underline{\blue $\frac{1}{2}\left[e^{-(x+at)^2}+e^{-(x-at)^2}\right]$ }。

{\red 见 Lec.13 一维无边界波动方程的通解;注意这种方法虽然解决无边界问题十分有效,计算有限区域内的波动问题就十分容易出问题。请参考本卷的计算题。}
}
\item[(6)]{一维无边界的空间内满足方程 $\frac{\partial u}{\partial t}-a\frac{\partial^2u}{\partial x^2} = 0$ 和初始条件  $\left.u\right\vert_{t=0} = e^{-x^2}$ 的解是: $u(x,t)=$ \underline{\blue $\frac{1}{\sqrt{4at+1}}e^{-\frac{x^2}{4at+1}}$}。

  {\red 解法一:用格林函数方法写出$u = \int_{-\infty}^\infty e^{-x_0^2} \frac{1}{\sqrt{4\pi at}} e^{-\frac{(x-x_0)^2}{4at}} dx_0$然后轻轻一化简…

    解法二:直接把这个解看成在原点的脉冲输入经过一段时间扩散后的结果,把$x_0=0$的格林函数做一下时间平移并乘以适当常数。}}

  \eitem
}
\item[(三)]{两端固定的,长度为$L$的均匀弦的横向振动$u(x,t)$ ($0\le x\le L$)满足初始条件:
  \bea
  \left.u\right\vert_{t=0} &=& A\sin \frac{\pi x}{L} \newl
  \left. \frac{\partial u}{\partial t}\right\vert_{t=0} &=& 0.
  \eea
  求解$u(x,t)$。设弦的线密度$\lambda$,张力$f$均已知。(20分)


    {\vskip 0.1in}

    {\blue

      方程为波动方程
      \be
      \frac{\partial ^2 u}{\partial t^2} - a^2\frac{\partial^2u}{\partial x^2} = 0,
      \ee
      其中$a^2 = \frac{f}{\lambda}$。
      

      解法1:
      
      按边界条件确定谐函数形式为$\sin\frac{n\pi x}{L}$,配上波动方程的时间演化因子:
      $$ u(x,t) = \sum_{n=1}^\infty \sin\frac{n\pi x}{L} \left(c_n \cos\frac{n\pi at}{L} +  s_n\sin\frac{n\pi at}{L}\right). $$
      由初始条件速度为零,确定所有$s_n = 0$。

      由初始位移条件:得到
      \bea
      \sum_{n=1}^\infty c_n\sin\frac{n\pi x}{L} = A\sin{\frac{\pi x}{L}} 
      \eea
      观察该式子左右两边,显然$c_1 = A$,其余$c_n$均为零。于是最后结果为
      \be
      u= A\sin\frac{\pi x}{L}\cos{\frac{\pi at}{L}}
      \ee


      解法2: 设通解
      $u = f(x-at)+g(x+at)$
      由初始条件得到
      \bea
      f(x)+g(x) &=& A\sin\frac{\pi x}{L}, \newl
      f'(x) -g'(x) &=& 0
      \eea
      由此得到$f(x) = g(x) = \frac{A}{2} \sin\frac{\pi x}{L}$

      但是注意这个等式是在$x\in [0, L]$内求出来的。在这个范围之外需要拓展定义。

      在这个问题里,边界条件的要求为
      $$ f(at) + g(-at) = 0; $$
      $$ f(L+at)+g(L-at) = 0. $$
      这两个条件交替使用可以把$[0,L]$的函数拓展到任意范围。在本题中(由于运气好),拓展的结果恰好是$\frac{A}{2}\sin{\pi x}{L}$的自然延拓。所以最后结果为
      $$ u(x,t) = \frac{A}{2}\left( \sin\frac{\pi (x+at)}{L}+ \sin\frac{\pi (x-at)}{L}\right) = A\sin\frac{\pi x}{L}\cos\frac{\pi at}{L}.$$

      \skiplines
      
      后记: 一般来说,有限区域内的问题用第二种方法比较容易出错。要理解本题第二种解法里的“运气好”的意思,你可以把初始位移替换成$ \left.u\right\vert_{t=0} = A\left(1-\cos\frac{2\pi x}{L}\right)$然后分别尝试使用第一种和第二种解法,看看结果是否相同。

    }
}

\item[(四)]{有半径为 $R$,导热系数为 $\lambda$,单位质量比热为 $c$,质量密度为 $\rho$ 的孤立均匀薄圆盘。以圆盘中心为原点建立极坐标 $(r,\theta)$。初始时刻各点的温度为 $\left.T\right\vert_{t=0}= T_0\left(2 + \frac{r^2}{R^2}\cos\theta\right)$。计算之后圆盘上各点的温度变化。(20分)

  {\blue
    令$u = T - 2T_0$ (这一步并非必须,但能免去之后讨论$k=0$的谐函数的各种麻烦,是很好的处理习惯)。

    $u$满足热传导方程:
    $$\frac{\partial u}{\partial t} - a \nabla^2 u = 0. $$
    和绝热边界条件:
    $$\left.\frac{\partial u}{\partial r}\right\vert_{r=R} = 0 .$$
    初始条件为:
    $$\left. u \right\vert_{t=0} = T_0\frac{r^2}{R^2}\cos\theta .$$

    一般的谐函数为$J_m(kr)\cos m\theta$和$J_m(kr)\sin m\theta$。但既然初始条件正比于$\cos\theta$,就可以只考虑$J_1(kr)\cos\theta$。

    根据边界绝热条件,并添上热传导方程的时间依赖因子:
    $$u = \sum_{i} c_i J_1\left(\frac{\mu_i}{R}r\right)\cos\theta \,e^{-a\frac{\mu_i^2}{R^2}t}, $$
    其中$\mu_i$是$J_1'$的第$i$个正实数零点,$c_i$为待定系数。

    初始条件给出:
    $$  \sum_{i} c_i J_1\left(\frac{\mu_i}{R}r\right) = T_0 \frac{r^2}{R^2}. $$
    记$x = \frac{r}{R}$,则上式成为:
    $$  \sum_{i} c_i J_1\left(\mu_i x\right) = T_0 x^2. $$
    两边同乘以$xJ_1(\mu_j x)$并在$[0,1]$内积分,得到
    $$ \frac{\left(1-\frac{1}{\mu_j^2}\right)\left[J_1(\mu_j)\right]^2}{2} c_j  = T_0 \int_0^1 x^3 J_1(\mu_j x) dx.  $$
    即
    $$c_j = \frac{2T_0 \int_0^1 x^3 J_1(\mu_j x) dx}{\left(1-\frac{1}{\mu_j^2}\right)\left[J_1(\mu_j)\right]^2}. $$
    即最后结果为:
    $$T = 2T_0 \left[ 1 +  \sum_{i}  \frac{\int_0^1 x^3 J_1(\mu_i x) dx}{\left(1-\frac{1}{\mu_i^2}\right)\left[J_1(\mu_i)\right]^2} J_1\left(\frac{\mu_i}{R}r\right)\cos\theta\, e^{-a\frac{\mu_i^2}{R^2}t}\right], $$
    
    \skiplines
    
    后记:
    其中的积分可以化简为(1,2)类超几何函数:
    \bea
    \int_0^1 x^3J_1(\mu_i x) dx  &=& \int_0^1 \sum_{k=0}^\infty \frac{(-1)^k\mu_i^{2k+1}}{k!(k+1)!2^{2k+1}} x^{2k+4} dx \newl
    &=&  \sum_{k=0}^\infty \frac{(-1)^k\mu_i^{2k+1}}{k!(k+1)!(2k+5)2^{2k+1}}  \newl
    &=& \frac{\mu_i}{10} \ _1F_2\left(\frac{5}{2};2,\frac{7}{2}; -\frac{\mu_i^2}{4}\right) 
    \eea
    一般地,积分 $\int x^nJ_m(x)dx$ 当 $n+m$ 为奇数时总能通过分部积分把结果写成贝塞尔函数;当$n+m$为偶数则结果无法写成贝塞尔函数,只能写成超几何函数。因为养生MMP课没有学习超几何函数,当 $n+m$ 为偶数时,你完全有理由把积分扔在那里不管。

    
  }
}

  
\eitem  





\ech
\end{document}
