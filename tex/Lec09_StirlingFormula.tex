\documentclass[CJK,13pt]{beamer}

\usepackage{CJKutf8}
\usepackage{beamerthemesplit}
\usetheme{Malmoe}
\useoutertheme[footline=authortitle]{miniframes}
\usepackage{amsmath}
\usepackage{amssymb}
\usepackage{graphicx}
\usepackage{eufrak}
\usepackage{color}
\usepackage{slashed}
\usepackage{simplewick}
\usepackage{tikz}
\usepackage{tcolorbox}
\graphicspath{{../figures/}}
%%figures
\def\lfig#1#2{\includegraphics[width=#1 in]{#2}}
\def\addfig#1#2{\begin{center}\includegraphics[width=#1 in]{#2}\end{center}}
\def\wulian{\includegraphics[width=0.18in]{emoji_wulian.jpg}}
\def\bigwulian{\includegraphics[width=0.35in]{emoji_wulian.jpg}}
\def\bye{\includegraphics[width=0.18in]{emoji_bye.jpg}}
\def\bigbye{\includegraphics[width=0.35in]{emoji_bye.jpg}}
\def\huaixiao{\includegraphics[width=0.18in]{emoji_huaixiao.jpg}}
\def\bighuaixiao{\includegraphics[width=0.35in]{emoji_huaixiao.jpg}}
\def\jianxiao{\includegraphics[width=0.18in]{emoji_jianxiao.jpg}}
\def\bigjianxiao{\includegraphics[width=0.35in]{emoji_jianxiao.jpg}}
%% colors
\def\blacktext#1{{\color{black}#1}}
\def\bluetext#1{{\color{blue}#1}}
\def\redtext#1{{\color{red}#1}}
\def\darkbluetext#1{{\color[rgb]{0,0.2,0.6}#1}}
\def\skybluetext#1{{\color[rgb]{0.2,0.7,1.}#1}}
\def\cyantext#1{{\color[rgb]{0.,0.5,0.5}#1}}
\def\greentext#1{{\color[rgb]{0,0.7,0.1}#1}}
\def\darkgray{\color[rgb]{0.2,0.2,0.2}}
\def\lightgray{\color[rgb]{0.6,0.6,0.6}}
\def\gray{\color[rgb]{0.4,0.4,0.4}}
\def\blue{\color{blue}}
\def\red{\color{red}}
\def\green{\color{green}}
\def\darkgreen{\color[rgb]{0,0.4,0.1}}
\def\darkblue{\color[rgb]{0,0.2,0.6}}
\def\skyblue{\color[rgb]{0.2,0.7,1.}}
%%control
\def\be{\begin{equation}}
\def\ee{\nonumber\end{equation}}
\def\bea{\begin{eqnarray}}
\def\eea{\nonumber\end{eqnarray}}
\def\bch{\begin{CJK}{UTF8}{gbsn}}
\def\ech{\end{CJK}}
\def\bitem{\begin{itemize}}
\def\eitem{\end{itemize}}
\def\bcenter{\begin{center}}
\def\ecenter{\end{center}}
\def\bex{\begin{minipage}{0.2\textwidth}\includegraphics[width=0.6in]{jugelizi.png}\end{minipage}\begin{minipage}{0.76\textwidth}}
\def\eex{\end{minipage}}
\def\chtitle#1{\frametitle{\bch#1\ech}}
\def\bmat#1{\left(\begin{array}{#1}}
\def\emat{\end{array}\right)}
\def\bcase#1{\left\{\begin{array}{#1}}
\def\ecase{\end{array}\right.}
\def\bmini#1{\begin{minipage}{#1\textwidth}}
\def\emini{\end{minipage}}
\def\tbox#1{\begin{tcolorbox}#1\end{tcolorbox}}
\def\pfrac#1#2#3{\left(\frac{\partial #1}{\partial #2}\right)_{#3}}
%%symbols
\def\bropt{\,(\ \ \ )}
\def\sone{$\star$}
\def\stwo{$\star\star$}
\def\sthree{$\star\star\star$}
\def\sfour{$\star\star\star\star$}
\def\sfive{$\star\star\star\star\star$}
\def\rint{{\int_\leftrightarrow}}
\def\roint{{\oint_\leftrightarrow}}
\def\stdHf{{\textit{\r H}_f}}
\def\deltaH{{\Delta \textit{\r H}}}
\def\ii{{\dot{\imath}}}
\def\skipline{{\vskip0.1in}}
\def\skiplines{{\vskip0.2in}}
\def\lagr{{\mathcal{L}}}
\def\hamil{{\mathcal{H}}}
\def\vecv{{\mathbf{v}}}
\def\vecx{{\mathbf{x}}}
\def\vecy{{\mathbf{y}}}
\def\veck{{\mathbf{k}}}
\def\vecp{{\mathbf{p}}}
\def\vecn{{\mathbf{n}}}
\def\vecA{{\mathbf{A}}}
\def\vecP{{\mathbf{P}}}
\def\vecsigma{{\mathbf{\sigma}}}
\def\hatJn{{\hat{J_\vecn}}}
\def\hatJx{{\hat{J_x}}}
\def\hatJy{{\hat{J_y}}}
\def\hatJz{{\hat{J_z}}}
\def\hatj#1{\hat{J_{#1}}}
\def\hatphi{{\hat{\phi}}}
\def\hatq{{\hat{q}}}
\def\hatpi{{\hat{\pi}}}
\def\vel{\upsilon}
\def\Dint{{\mathcal{D}}}
\def\adag{{\hat{a}^\dagger}}
\def\bdag{{\hat{b}^\dagger}}
\def\cdag{{\hat{c}^\dagger}}
\def\ddag{{\hat{d}^\dagger}}
\def\hata{{\hat{a}}}
\def\hatb{{\hat{b}}}
\def\hatc{{\hat{c}}}
\def\hatd{{\hat{d}}}
\def\hatN{{\hat{N}}}
\def\hatH{{\hat{H}}}
\def\hatp{{\hat{p}}}
\def\Fup{{F^{\mu\nu}}}
\def\Fdown{{F_{\mu\nu}}}
\def\newl{\nonumber \\}
\def\vece{\mathrm{e}}
\def\calM{{\mathcal{M}}}
\def\calT{{\mathcal{T}}}
\def\calR{{\mathcal{R}}}
\def\barpsi{\bar{\psi}}
\def\baru{\bar{u}}
\def\barv{\bar{\upsilon}}
\def\qeq{\stackrel{?}{=}}
\def\torder#1{\mathcal{T}\left(#1\right)}
\def\rorder#1{\mathcal{R}\left(#1\right)}
\def\contr#1#2{\contraction{}{#1}{}{#2}#1#2}
\def\trof#1{\mathrm{Tr}\left(#1\right)}
\def\trace{\mathrm{Tr}}
\def\comm#1{\ \ \ \left(\mathrm{used}\ #1\right)}
\def\tcomm#1{\ \ \ (\text{#1})}
\def\slp{\slashed{p}}
\def\slk{\slashed{k}}
\def\calp{{\mathfrak{p}}}
\def\veccalp{\mathbf{\mathfrak{p}}}
\def\Tthree{T_{\tiny \textcircled{3}}}
\def\pthree{p_{\tiny \textcircled{3}}}
\def\dbar{{\,\mathchar'26\mkern-12mu d}}
\def\erf{\mathrm{erf}}
\def\const{\mathrm{constant}}
\def\pheat{\pfrac p{\ln T}V}
\def\vheat{\pfrac V{\ln T}p}
%%units
\def\fdeg{{^\circ \mathrm{F}}}
\def\cdeg{^\circ \mathrm{C}}
\def\atm{\,\mathrm{atm}}
\def\angstrom{\,\text{\AA}}
\def\SIL{\,\mathrm{L}}
\def\SIkm{\,\mathrm{km}}
\def\SIyr{\,\mathrm{yr}}
\def\SIGyr{\,\mathrm{Gyr}}
\def\SIV{\,\mathrm{V}}
\def\SImV{\,\mathrm{mV}}
\def\SIeV{\,\mathrm{eV}}
\def\SIkeV{\,\mathrm{keV}}
\def\SIMeV{\,\mathrm{MeV}}
\def\SIGeV{\,\mathrm{GeV}}
\def\SIcal{\,\mathrm{cal}}
\def\SIkcal{\,\mathrm{kcal}}
\def\SImol{\,\mathrm{mol}}
\def\SIN{\,\mathrm{N}}
\def\SIHz{\,\mathrm{Hz}}
\def\SIm{\,\mathrm{m}}
\def\SIcm{\,\mathrm{cm}}
\def\SIfm{\,\mathrm{fm}}
\def\SImm{\,\mathrm{mm}}
\def\SInm{\,\mathrm{nm}}
\def\SImum{\,\mathrm{\mu m}}
\def\SIJ{\,\mathrm{J}}
\def\SIW{\,\mathrm{W}}
\def\SIkJ{\,\mathrm{kJ}}
\def\SIs{\,\mathrm{s}}
\def\SIkg{\,\mathrm{kg}}
\def\SIg{\,\mathrm{g}}
\def\SIK{\,\mathrm{K}}
\def\SImmHg{\,\mathrm{mmHg}}
\def\SIPa{\,\mathrm{Pa}}


\def\courseurl{https://github.com/zqhuang/SYSU\_TD}

\def\tpage#1#2{
\begin{frame}
\begin{center}
\begin{Large}
\bch
热学 \\
第#1讲 #2

{\vskip 0.3in}

黄志琦

\ech
\end{Large}
\end{center}

\vskip 0.2in

\bch
教材:《热学》第二版,赵凯华,罗蔚茵,高等教育出版社
\ech

\bch
课件下载
\ech
\courseurl
\end{frame}
}

\def\bfr#1{
\begin{frame}
\chtitle{#1} 
\bch
}

\def\efr{
\ech 
\end{frame}
}


\date{}

\begin{document}
\bch

\tpage{9}{Stirling Formula, $n$-sphere Integral}

\begin{frame}
  \frametitle{本讲内容}
  
  \bitem
\item{Stirling 公式}
\item{$n$维限和积分公式}
  \eitem

\end{frame}

\section{Stirling's Formula}

\begin{frame}
  \frametitle{Stirling公式}
  
 当$x\gg 1$时,有如下的近似表达式 (Stirling公式):
      {\blue
\tbox{        
  \be
  x! \approx \sqrt{2\pi x}\left(\frac{x}{e}\right)^x
  \ee}
  }
  
\end{frame}

\begin{frame}
  \frametitle{Stirling公式的证明}
  
  利用$\Gamma$函数的定义:
  $$ x! = \int_0^\infty t^xe^{-t}dt =\int_0^\infty e^{-t+x\ln t}dt $$
  被积函数在极大值点$t=x$附近贡献较大,所以把$-t+x\ln t$在$t=x$附近泰勒展开:
  $$ -t+x\ln t \approx -x + x\ln x - \frac{(t-x)^2}{2x} $$
  \bea
  x! &\approx & \int_0^\infty e^{-x+x\ln x}e^{-\frac{(t-x)^2}{2x}}dt \newl
    &\approx & \left(\frac{x}{e}\right)^x\int_{-\infty}^\infty e^{-\frac{(t-x)^2}{2x}}dt \newl
    &= & \sqrt{2\pi x}\left(\frac{x}{e}\right)^x
  \eea
    
  
  
\end{frame}

\begin{frame}
  \frametitle{Stirling公式的加强版}
  
  Stirling公式的终级版本为:
  \be
  x! = \sqrt{2\pi x}\left(\frac{x}{e}\right)^x e^{\frac{1}{12x}-\frac{1}{360 x^3} + \frac{1}{1260 x^5} - \frac{1}{1680 x^7}\ldots}
  \ee
      (证明略)
      
      如果取截断
      {\blue
\tbox{        
  \be
  x!\approx \sqrt{2\pi x}\left(\frac{x}{e}\right)^x e^{\frac{1}{12x}-\frac{1}{360 x^3} }
  \ee}
      }
     对$x\ge 3$,结果的相对误差$\lesssim 3\times 10^{-6}$。因此,一个快速用初等函数计算 $\Gamma$函数的办法是用递推公式转化为$x\ge 3$时的$\Gamma$函数的计算问题。

  
\end{frame}


\begin{frame}
  \frametitle{例题}
  
  计算$\Gamma\left(\frac{1}{3}\right)$,要求误差小于$10^{-5}$。
  
\end{frame}


\begin{frame}
  \frametitle{ 解法一}
  
  大多数编程语言和脚本语言都有内置的$\Gamma$函数。例如,在python环境中输入命令:

  \skipline

  
  \tbox{\darkgreen
  from math import * \\
  gamma(1/3.)
  }
  
  得到输出结果

  \skipline
  
  \tbox
  {\darkgreen
  2.678938534707748
  }
  
\end{frame}


\begin{frame}
  \frametitle{ 解法二}
  
  如果手头没有可以现成计算$\Gamma$函数的工具,那么就利用Stirling公式的加强版:
    {\small
    $$
      \sfgamma{\frac{1}{3}} = \frac{ \left(\frac{10}{3}\right)!}{\frac{1}{3}\times \frac{4}{3} \times \frac{7}{3}\times\frac{10}{3}}  \approx \sqrt{\frac{20\pi}{3}}\left(\frac{10}{3e}\right)^{\frac{10}{3}} e^{\frac{1}{40} - \frac{3}{4000}}\times \frac{81}{280}\approx 2.678934   $$
      }
  
\end{frame}


\begin{frame}
  \frametitle{Euler常数 —— 重要性仅次于$\pi$和$e$的数学常数}
  
  可以用连续的积分$$ T_n = \int_1^{n} \frac{1}{x} dx = \ln n $$
  来近似级数和
  $$S_n = 1+\frac{1}{2}+\frac{1}{3} + \frac{1}{4} + \ldots + \frac{1}{n}$$
  误差当$n\rightarrow \infty$时趋向于一个常数,定义其为Euler常数:
  \tbox{$$\gamma = \lim_{n\rightarrow \infty} \left(1+\frac{1}{2}+\frac{1}{3}  + \ldots + \frac{1}{n} - \ln n\right) = 0.5772156649\ldots$$}
  
\end{frame}


\begin{frame}
  \frametitle{例题}
  
  证明$\Gamma$函数的无穷乘积表达式:
  \tbox{
    $$\frac{1}{\sfgamma{z}} = z e^{\gamma z}\prod_{n=1}^\infty \left(1+\frac{z}{n}\right) e^{-\frac{z}{n}}. $$
    }
  
\end{frame}


\begin{frame}
  \frametitle{证明:利用$\Gamma$函数递推关系}
  
  {\small  
  \bea
  \frac{1}{\sfgamma{z}} &=& \frac{z(z+1)(z+2)\ldots (z+n)}{(z+n)!} \newl
  &=& \frac{n!z(1+z)\left(1+\frac{z}{2}\right)\left(1+\frac{z}{3}\right)\ldots \left(1+\frac{z}{n}\right)}{(z+n)!}
  \eea
  令$n\rightarrow \infty$,并利用Stirling公式
  \bea
  \frac{1}{\sfgamma{z}} &=& \lim_{n\rightarrow \infty}\frac{(\frac{n}{e})^n}{(\frac{z+n}{e})^{n+z}} z(1+z)\left(1+\frac{z}{2}\right)\left(1+\frac{z}{3}\right)\ldots \left(1+\frac{z}{n}\right) \newl
  &=& \lim_{n\rightarrow \infty}\left(1+\frac{z}{n}\right)^{-n}e^zn^{-z} z(1+z)\left(1+\frac{z}{2}\right)\left(1+\frac{z}{3}\right)\ldots \left(1+\frac{z}{n}\right) \newl
  &=& \lim_{n\rightarrow \infty}n^{-z} z(1+z)\left(1+\frac{z}{2}\right)\left(1+\frac{z}{3}\right)\ldots \left(1+\frac{z}{n}\right)  \newl
  &=& \lim_{n\rightarrow \infty}e^{-z \ln n}  z(1+z)\left(1+\frac{z}{2}\right)\left(1+\frac{z}{3}\right)\ldots \left(1+\frac{z}{n}\right) \newl
  &=& \lim_{n\rightarrow \infty}e^{\gamma z - (1+\frac{1}{2}+\frac{1}{3}+\ldots+\frac{1}{n})z}  z(1+z)\left(1+\frac{z}{2}\right)\left(1+\frac{z}{3}\right)\ldots \left(1+\frac{z}{n}\right)  
  \eea
  }
  
\end{frame}

\begin{frame}
  \frametitle{例题}
  
  随机抛100次硬币,估算恰好有50次正面向上的概率。
  
\end{frame}


\begin{frame}
  \frametitle{解答}
  
  \bea
  p &=& \frac{1}{2^{100}} \times \frac{100!}{(50!)^2} \newl
  &\approx & \frac{1}{2^{100}} \times  \frac{\sqrt{200\pi}\left(\frac{100}{e}\right)^{100}}{100\pi\left(\frac{50}{e}\right)^{100}} \newl
  &=& \frac{1}{10}\sqrt{\frac{2}{\pi}} \newl
  &\approx & 0.080
  \eea


  \skiplines

  {请尝试用Stirling公式的加强版来计算更加精确的结果。}
  
  
\end{frame}

\section{Integral in a $n$D sphere}

\begin{frame}
  \frametitle{$n$维限和积分公式 ($B$函数的推广)}
{\blue
  \bea
 && \int_{\Omega_n} x_1^{\alpha_1-1}x_2^{\alpha_2-1}\ldots x_n^{\alpha_n-1} f(x_1+x_2+\ldots+x_n)dx_1dx_2\ldots dx_n \newl
  &=& \frac{\Gamma(\alpha_1)\Gamma(\alpha_2)\ldots \Gamma(\alpha_n)}{\Gamma(\alpha_1+\alpha_2+\ldots + \alpha_n)}\int_0^1f(u)u^{\alpha_1+\alpha_2+\ldots + \alpha_n-1} du \,. \nonumber
  \eea
  其中等式左边的积分区域$\Omega_n = \{(x_1,x_2,\ldots,x_n): x_1,x_2,\ldots, x_n\ge 0; x_1+x_2+\ldots+x_n\le 1 \}$.
}  
\end{frame}


\begin{frame}
  \frametitle{$n$维限和积分公式的证明}
  
  {\scriptsize
  证明:用归纳法,$n=1$时命题显然成立。假设命题对$n-1$成立,考虑积分
  \scriptsize
  $$ J  =  \int_{\Omega_n} x_1^{\alpha_1-1}x_2^{\alpha_2-1}\ldots x_{n}^{\alpha_{n}-1}\delta(x_1+x_2+\ldots+x_n-1)dx_1dx_2\ldots dx_n.$$
  先对$x_n$积分,得到
  $$ J = \int_{\Omega_{n-1}} x_1^{\alpha_1-1}x_2^{\alpha_2-1}\ldots x_{n-1}^{\alpha_{n-1}-1}(1-x_1-x_2-\ldots -x_{n-1})^{\alpha_n-1} dx_1dx_2\ldots dx_{n-1}. $$
  利用对$n-1$的归纳假设,以及$B$函数和$\Gamma$函数的关系,得到
  \bea
  J &=& \frac{\Gamma(\alpha_1)\Gamma(\alpha_2)\ldots\Gamma(\alpha_{n-1})}{\Gamma(\alpha_1+\alpha_2+\ldots+\alpha_{n-1})} \int_0^1(1-t)^{\alpha_n-1}t^{\alpha_1+\alpha_2+\ldots+\alpha_{n-1}-1} dt \newl
  &=& \frac{\Gamma(\alpha_1)\Gamma(\alpha_2)\ldots\Gamma(\alpha_{n-1})}{\Gamma(\alpha_1+\alpha_2+\ldots+\alpha_{n-1})} \frac{\Gamma(\alpha_n)\Gamma(\alpha_1+\alpha_2+\ldots+\alpha_{n-1})}{\Gamma(\alpha_1+\alpha_2+\ldots+\alpha_{n-1}+\alpha_n)} \newl
  &=& \frac{\Gamma(\alpha_1)\Gamma(\alpha_2)\ldots \Gamma(\alpha_n)}{\Gamma(\alpha_1+\alpha_2+\ldots + \alpha_n)}
  \eea
  }
  
\end{frame}

\begin{frame}
  \frametitle{$n$维限和积分公式的证明(续)}
  
      {\scriptsize
        然后对任意$0<u<1$,考虑积分
        \begin{equation}
        I(u)= \int_{\Omega_n} x_1^{\alpha_1-1}x_2^{\alpha_2-1}\ldots x_n^{\alpha_n-1} \delta(x_1+x_2+\ldots+x_n-u) dx_1dx_2\ldots dx_n.\label{eq:1}
        \end{equation}
        做变量替换$x_i = uy_i$,并利用前面得到的积分$J$,就得到
        \begin{eqnarray}
        I(u) &=& u^{\alpha_1+\alpha_2+\ldots+\alpha_n-1} \int_{\Omega_n} y_1^{\alpha_1-1}y_2^{\alpha_2-1}\ldots y_n^{\alpha_n-1} \delta(y_1+y_2+\ldots+y_n-1) dy_1dy_2\ldots dy_n \nonumber \\
        &=& u^{\alpha_1+\alpha_2+\ldots+\alpha_n-1} \frac{\Gamma(\alpha_1)\Gamma(\alpha_2)\ldots \Gamma(\alpha_n)}{\Gamma(\alpha_1+\alpha_2+\ldots + \alpha_n)} \label{eq:2}
        \end{eqnarray}
        最后,分别利用\eqref{eq:1}和\eqref{eq:2}可以看出所要求证的等式左右两边都等于
        $$ \int_0^1 I(u) f(u) du $$
        于是证毕。
        }
  
\end{frame}

\begin{frame}
  \frametitle{例题}
  
  计算$n$-维空间的球体的体积 ($n\in Z^+$)。
  
\end{frame}

\begin{frame}
  \frametitle{解答}
  
  {\small
  $$ V_n = \int_{x_1^2+x_2^2+\ldots +x_n^2\le 1} dx_1 dx_2 \ldots dx_n  $$
  利用对称性可以写成
  $$ V_n = 2^n\int_{x_1,x_2,\ldots,x_n\ge 0; x_1^2+x_2^2+\ldots +x_n^2\le 1} dx_1 dx_2 \ldots dx_n  $$
  作变量替换$x_i = \sqrt{y_i}$,
  $$  V_n = \int_{\Omega_n} y_1^{-\frac{1}{2}}y_2^{-\frac{1}{2}}\ldots y_n^{-\frac{1}{2}}dy_1 dy_2 \ldots dy_n  $$
  再利用$n$维限和积分公式:
$$V_n = \frac{\Gamma\left(\frac{1}{2}\right)^n}{\Gamma\left(\frac{n}{2}\right)}\int_0^1 u^{\frac{n}{2}-1} du = \frac{\pi^{n/2}}{\Gamma\left(\frac{n}{2}\right)}\frac{2}{n} = \frac{\pi^{n/2}}{\left(\frac{n}{2}\right)!} $$}
  
\end{frame}

\begin{frame}
  \frametitle{例题}
  
  满足标准正态分布$$P(x) = \frac{1}{\sqrt{2\pi}}e^{-\frac{x^2}{2}}$$的随机变量$x$的$100$次独立采样值为$x_1,x_2,\ldots, x_{100}$;请估算$x_1^2+x_2^2+\ldots +x_{100}^2 > 200 $ 的概率。
  
\end{frame}


\begin{frame}
  \frametitle{ 解答}
  
  我们先考虑$n$个独立地满足标准正态分布的变量$x_1,x_2,\ldots, x_n$的平方和不大于$s$ ($s\ge 0$)的概率:

  \be
  P_n(s) = \frac{1}{(2\pi)^{n/2}}\int_{x_1^2+x_2^2+\ldots x_n^2<s}  e^{-\frac{x_1^2+x_2^2+\ldots+x_n^2}{2}}dx_1dx_2\ldots dx_n.
  \ee
  由正态分布的对称性,可以把积分限定在$x_1,x_2,\ldots,x_n\ge 0$的范围内:
        {\scriptsize
  \be
  P_n(s) = \frac{2^n}{(2\pi)^{n/2}}\int_{x_1,x_2,\ldots,x_n\ge 0;x_1^2+x_2^2+\ldots x_n^2<s}  e^{-\frac{x_1^2+x_2^2+\ldots+x_n^2}{2}}dx_1dx_2\ldots dx_n .
  \ee}
\end{frame}

\begin{frame}
  \frametitle{ 解答(续)}
  
  做变量替换$x_i=\sqrt{sy_i}$ ($i=1,2,\ldots,n$),并利用$n$维限和积分公式:
  {\small
    \bea
  P_n(s) &=& \frac{s^{\frac{n}{2}}}{(2\pi)^{n/2}}\int_{\Omega_n} y_1^{-\frac{1}{2}}y_2^{-\frac{1}{2}}\ldots y_n^{-\frac{1}{2}} e^{-\frac{s(y_1+y_2+\ldots+y_n)}{2}}dy_1dy_2\ldots dy_n  \newl
  &=& \frac{s^{\frac{n}{2}}}{(2\pi)^{n/2}} \frac{\left[\Gamma\left(\frac{1}{2}\right)\right]^n}{\Gamma\left(\frac{n}{2}\right)}\int_0^1e^{-\frac{su}{2}}u^{\frac{n}{2}-1}du  \newl
  &=& \frac{1}{\Gamma\left(\frac{n}{2}\right)}\int_0^{\frac{s}{2}} e^{-t} t^{\frac{n}{2}-1} dt .
  \eea
  }
       我们需要计算的是:
        \be
        p = 1-P_{100}(200) = \frac{1}{\Gamma(50)}\int_{100}^\infty t^{49}e^{-t}dt  = \frac{1}{\Gamma(50)}\int_{100}^\infty e^{-t+49\ln t}dt 
        \ee
  
\end{frame}

\begin{frame}
  \frametitle{解答(续)}
  

        被积函数随着$t$增大而迅速减少。因此我们在$t=100$附近做泰勒展开
        $$ -t + 49\ln t \approx  -100 + 49 \ln 100 - 0.51(t-100) $$
        积分得到
        \be
        \int_{100}^\infty t^{49}e^{-t}dt  \approx  e^{-100+49\ln 100} \int_{100}^\infty e^{-0.51 (t-100)} dt \approx \frac{ e^{-100+49\ln 100} }{0.51}
        \ee
        再利用Stirling公式,
        \be
        p \approx \frac{ e^{-100 + 49\ln 100 - (-49+49\ln 49)} }{0.51\times \sqrt{98\pi}} \approx 1.2\times 10^{-8}
        \ee
        
  
\end{frame}

\section{Homework}

\begin{frame}
  \frametitle{Homework for quizphobia}
  \bitem
\item[25]{抛6000次骰子,每个面(1,2,3,4,5,6)恰好各出现1000次的概率大约为多少?}
\item[26]{计算质量为$m$半径为$r$的刚性球的转动惯量。}
\item[27]{理想气体中任取100个分子,这100个分子的方均根速率超过所有分子的方均根速率的2倍的概率大概是多大?}  

  \eitem
  
\end{frame}



\ech
\end{document}
