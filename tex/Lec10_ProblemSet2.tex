\documentclass[CJK]{beamer}
\usepackage{CJKutf8}
\usepackage{beamerthemesplit}
\usetheme{Malmoe}
\useoutertheme[footline=authortitle]{miniframes}
\usepackage{amsmath}
\usepackage{amssymb}
\usepackage{graphicx}
\usepackage{eufrak}
\usepackage{color}
\usepackage{slashed}
\usepackage{simplewick}
\usepackage{tikz}
\usepackage{tcolorbox}
\graphicspath{{../figures/}}
%%figures
\def\lfig#1#2{\includegraphics[width=#1 in]{#2}}
\def\addfig#1#2{\begin{center}\includegraphics[width=#1 in]{#2}\end{center}}
\def\wulian{\includegraphics[width=0.18in]{emoji_wulian.jpg}}
\def\bigwulian{\includegraphics[width=0.35in]{emoji_wulian.jpg}}
\def\bye{\includegraphics[width=0.18in]{emoji_bye.jpg}}
\def\bigbye{\includegraphics[width=0.35in]{emoji_bye.jpg}}
\def\huaixiao{\includegraphics[width=0.18in]{emoji_huaixiao.jpg}}
\def\bighuaixiao{\includegraphics[width=0.35in]{emoji_huaixiao.jpg}}
\def\jianxiao{\includegraphics[width=0.18in]{emoji_jianxiao.jpg}}
\def\bigjianxiao{\includegraphics[width=0.35in]{emoji_jianxiao.jpg}}
%% colors
\def\blacktext#1{{\color{black}#1}}
\def\bluetext#1{{\color{blue}#1}}
\def\redtext#1{{\color{red}#1}}
\def\darkbluetext#1{{\color[rgb]{0,0.2,0.6}#1}}
\def\skybluetext#1{{\color[rgb]{0.2,0.7,1.}#1}}
\def\cyantext#1{{\color[rgb]{0.,0.5,0.5}#1}}
\def\greentext#1{{\color[rgb]{0,0.7,0.1}#1}}
\def\darkgray{\color[rgb]{0.2,0.2,0.2}}
\def\lightgray{\color[rgb]{0.6,0.6,0.6}}
\def\gray{\color[rgb]{0.4,0.4,0.4}}
\def\blue{\color{blue}}
\def\red{\color{red}}
\def\green{\color{green}}
\def\darkgreen{\color[rgb]{0,0.4,0.1}}
\def\darkblue{\color[rgb]{0,0.2,0.6}}
\def\skyblue{\color[rgb]{0.2,0.7,1.}}
%%control
\def\be{\begin{equation}}
\def\ee{\nonumber\end{equation}}
\def\bea{\begin{eqnarray}}
\def\eea{\nonumber\end{eqnarray}}
\def\bch{\begin{CJK}{UTF8}{gbsn}}
\def\ech{\end{CJK}}
\def\bitem{\begin{itemize}}
\def\eitem{\end{itemize}}
\def\bcenter{\begin{center}}
\def\ecenter{\end{center}}
\def\bex{\begin{minipage}{0.2\textwidth}\includegraphics[width=0.6in]{jugelizi.png}\end{minipage}\begin{minipage}{0.76\textwidth}}
\def\eex{\end{minipage}}
\def\chtitle#1{\frametitle{\bch#1\ech}}
\def\bmat#1{\left(\begin{array}{#1}}
\def\emat{\end{array}\right)}
\def\bcase#1{\left\{\begin{array}{#1}}
\def\ecase{\end{array}\right.}
\def\bmini#1{\begin{minipage}{#1\textwidth}}
\def\emini{\end{minipage}}
\def\tbox#1{\begin{tcolorbox}#1\end{tcolorbox}}
\def\pfrac#1#2#3{\left(\frac{\partial #1}{\partial #2}\right)_{#3}}
%%symbols
\def\bropt{\,(\ \ \ )}
\def\sone{$\star$}
\def\stwo{$\star\star$}
\def\sthree{$\star\star\star$}
\def\sfour{$\star\star\star\star$}
\def\sfive{$\star\star\star\star\star$}
\def\rint{{\int_\leftrightarrow}}
\def\roint{{\oint_\leftrightarrow}}
\def\stdHf{{\textit{\r H}_f}}
\def\deltaH{{\Delta \textit{\r H}}}
\def\ii{{\dot{\imath}}}
\def\skipline{{\vskip0.1in}}
\def\skiplines{{\vskip0.2in}}
\def\lagr{{\mathcal{L}}}
\def\hamil{{\mathcal{H}}}
\def\vecv{{\mathbf{v}}}
\def\vecx{{\mathbf{x}}}
\def\vecy{{\mathbf{y}}}
\def\veck{{\mathbf{k}}}
\def\vecp{{\mathbf{p}}}
\def\vecn{{\mathbf{n}}}
\def\vecA{{\mathbf{A}}}
\def\vecP{{\mathbf{P}}}
\def\vecsigma{{\mathbf{\sigma}}}
\def\hatJn{{\hat{J_\vecn}}}
\def\hatJx{{\hat{J_x}}}
\def\hatJy{{\hat{J_y}}}
\def\hatJz{{\hat{J_z}}}
\def\hatj#1{\hat{J_{#1}}}
\def\hatphi{{\hat{\phi}}}
\def\hatq{{\hat{q}}}
\def\hatpi{{\hat{\pi}}}
\def\vel{\upsilon}
\def\Dint{{\mathcal{D}}}
\def\adag{{\hat{a}^\dagger}}
\def\bdag{{\hat{b}^\dagger}}
\def\cdag{{\hat{c}^\dagger}}
\def\ddag{{\hat{d}^\dagger}}
\def\hata{{\hat{a}}}
\def\hatb{{\hat{b}}}
\def\hatc{{\hat{c}}}
\def\hatd{{\hat{d}}}
\def\hatN{{\hat{N}}}
\def\hatH{{\hat{H}}}
\def\hatp{{\hat{p}}}
\def\Fup{{F^{\mu\nu}}}
\def\Fdown{{F_{\mu\nu}}}
\def\newl{\nonumber \\}
\def\vece{\mathrm{e}}
\def\calM{{\mathcal{M}}}
\def\calT{{\mathcal{T}}}
\def\calR{{\mathcal{R}}}
\def\barpsi{\bar{\psi}}
\def\baru{\bar{u}}
\def\barv{\bar{\upsilon}}
\def\qeq{\stackrel{?}{=}}
\def\torder#1{\mathcal{T}\left(#1\right)}
\def\rorder#1{\mathcal{R}\left(#1\right)}
\def\contr#1#2{\contraction{}{#1}{}{#2}#1#2}
\def\trof#1{\mathrm{Tr}\left(#1\right)}
\def\trace{\mathrm{Tr}}
\def\comm#1{\ \ \ \left(\mathrm{used}\ #1\right)}
\def\tcomm#1{\ \ \ (\text{#1})}
\def\slp{\slashed{p}}
\def\slk{\slashed{k}}
\def\calp{{\mathfrak{p}}}
\def\veccalp{\mathbf{\mathfrak{p}}}
\def\Tthree{T_{\tiny \textcircled{3}}}
\def\pthree{p_{\tiny \textcircled{3}}}
\def\dbar{{\,\mathchar'26\mkern-12mu d}}
\def\erf{\mathrm{erf}}
\def\const{\mathrm{constant}}
\def\pheat{\pfrac p{\ln T}V}
\def\vheat{\pfrac V{\ln T}p}
%%units
\def\fdeg{{^\circ \mathrm{F}}}
\def\cdeg{^\circ \mathrm{C}}
\def\atm{\,\mathrm{atm}}
\def\angstrom{\,\text{\AA}}
\def\SIL{\,\mathrm{L}}
\def\SIkm{\,\mathrm{km}}
\def\SIyr{\,\mathrm{yr}}
\def\SIGyr{\,\mathrm{Gyr}}
\def\SIV{\,\mathrm{V}}
\def\SImV{\,\mathrm{mV}}
\def\SIeV{\,\mathrm{eV}}
\def\SIkeV{\,\mathrm{keV}}
\def\SIMeV{\,\mathrm{MeV}}
\def\SIGeV{\,\mathrm{GeV}}
\def\SIcal{\,\mathrm{cal}}
\def\SIkcal{\,\mathrm{kcal}}
\def\SImol{\,\mathrm{mol}}
\def\SIN{\,\mathrm{N}}
\def\SIHz{\,\mathrm{Hz}}
\def\SIm{\,\mathrm{m}}
\def\SIcm{\,\mathrm{cm}}
\def\SIfm{\,\mathrm{fm}}
\def\SImm{\,\mathrm{mm}}
\def\SInm{\,\mathrm{nm}}
\def\SImum{\,\mathrm{\mu m}}
\def\SIJ{\,\mathrm{J}}
\def\SIW{\,\mathrm{W}}
\def\SIkJ{\,\mathrm{kJ}}
\def\SIs{\,\mathrm{s}}
\def\SIkg{\,\mathrm{kg}}
\def\SIg{\,\mathrm{g}}
\def\SIK{\,\mathrm{K}}
\def\SImmHg{\,\mathrm{mmHg}}
\def\SIPa{\,\mathrm{Pa}}

\def\courseurl{https://github.com/zqhuang/SYSU\_TD}

\def\tpage#1#2{
\begin{frame}
\begin{center}
\begin{Large}
\bch
热学 \\
第#1讲 #2

{\vskip 0.3in}

黄志琦

\ech
\end{Large}
\end{center}

\vskip 0.2in

\bch
教材:《热学》第二版,赵凯华,罗蔚茵,高等教育出版社
\ech

\bch
课件下载
\ech
\courseurl
\end{frame}
}

\def\bfr#1{
\begin{frame}
\chtitle{#1} 
\bch
}

\def\efr{
\ech 
\end{frame}
}

  \date{}
\begin{document}
\tpage{10}{Problem Set II}

\newcounter{chap}
\newcounter{problem}[chap]
\def\proid{{Problem \thechap.\theproblem}\ }

\begin{frame}
  \chtitle{本讲内容: 复变函数知识回顾}
  \bch
  \bitem
\item{复数的几何意义}
\item{柯西-黎曼条件}
\item{一元$n$次多项式的根和系数的关系}
\item{柯西积分公式用于估算}
\item{无穷远点是个伪概念}
  \eitem
  \ech
\end{frame}

\setcounter{chap}{2}
\setcounter{problem}{0}

\stepcounter{problem}
\begin{frame}
  \chtitle{\proid (\sone)}
  \bch
  区域$|z-3|+|z+3|\le 10 $的面积是多大?
  \ech
\end{frame}


\begin{frame}
  \chtitle{\proid 解答}
  \bch
  我们在小学曾学过:到两点之间距离之和为常数的点的集合为椭圆。所以$|z-3|+|z+3|=10$描述的是半焦距$c=3$,半长轴$a=5$的椭圆。$|z-3|+|z+3|\le 10 $对应的是该椭圆包围的区域,面积为
  $$ S = \pi a b = \pi a \sqrt{a^2-c^2} = 20\pi .$$
  \ech
\end{frame}


\stepcounter{problem}
\begin{frame}
  \chtitle{\proid (\stwo)}
  \bch
  设$z$为任意一个复数,则$f(z) = |z-1|^6+(|z|+1)^6$的最小可能的值是多少?
  \ech
\end{frame}


\begin{frame}
  \chtitle{\proid 解答}
  \bch
  $f(0) = 2$是最小值,证明如下:
  \bea
  f(z) &=& |1 - 6z+\ldots + z^6| + (|z|+1)^6 \newl
  &\ge & 1- 6|z| - \ldots - |z|^6 + 1 + 6 |z| + \ldots + |z|^6 \newl
  &=& 2
  \eea
  
  \ech
\end{frame}


\stepcounter{problem}
\begin{frame}
  \chtitle{\proid (\stwo)}
  \bch
  $z_1,z_2,z_3,z_4$是两两不同的四个复数,且$\frac{(z_1-z_2)(z_3-z_4)}{(z_2-z_3)(z_4-z_1)}$为实数。试证明:$z_1,z_2,z_3,z_4$在复平面上对应的四个点共线或共圆。
  \ech
\end{frame}

\begin{frame}
  \chtitle{\proid 解答}
  \bch
  $\frac{z_1-z_2}{z_3-z_2}$的幅角和$\frac{z_3-z_4}{z_1-z_4}$的幅角(在允许相差$2\pi$整数倍的意义下)或相等或互补,所以$z_1,z_2,z_3,z_4$共圆或共线。
  \ech
\end{frame}


\stepcounter{problem}
\begin{frame}
  \chtitle{\proid (\sthree)}
  \bch
  设解析函数$f(z)$可以拆分成实虚部$u$和$\upsilon$,即
  $$f(x+\ii y) = u(x, y) + \ii \upsilon(x, y), \ \ x,y,u,\upsilon\in \Re $$
  \bitem
\item{证明柯西-黎曼条件:
  $$\frac{\partial u}{\partial x} = \frac{\partial \upsilon}{\partial y}; \frac{\partial u}{\partial y} = -\frac{\partial \upsilon}{\partial x}.$$ }
\item{证明$u$和$\upsilon$都是调和函数$$\nabla^2 u = \nabla^2 \upsilon = 0,$$
其中算符$\nabla^2 \equiv \partial_x^2+\partial_y^2$。}
  \eitem
  \ech
\end{frame}

\begin{frame}
  \chtitle{\proid (\sthree)}
  \bch
  证明,设$f'(z) = a+\ii b$,则
  $$ df = du+\ii d\upsilon = (a+\ii b) (dx + \ii dy).$$
  分离实虚部得到两个全微分表达式:
  $$ du = a dx - b dy;\ \  d\upsilon = bdx + ady. $$
  显然
  $$ \frac{\partial u}{\partial x} = a = \frac{\partial \upsilon}{\partial y}; \frac{\partial u}{\partial y} = -b =  -\frac{\partial \upsilon}{\partial x}. $$
  再根据两个全微分条件:
  $$\frac{\partial a}{\partial y} = - \frac{\partial b}{\partial x};\ \
  \frac{\partial b}{\partial y} = \frac{\partial a}{\partial x}.$$
  即可得到$u$,$\upsilon$为调和函数的结论。
  \ech
\end{frame}

\stepcounter{problem}
\begin{frame}
  \chtitle{\proid (\sthree)}
  \bch
  证明:一元$n$次复系数多项式 ($n\in Z^+$)
  $$P(z) = z^n + c_{n-1}z^{n-1}+ c_{n-2}z^{n-2}+\ldots + c_1 z + c_0 $$
  一定可以分解为一次多项式的乘积:
  $$P(z) = (z-z_1)(z-z_2)\ldots(z-z_n).$$
  \ech
\end{frame}

\begin{frame}
  \chtitle{\proid 解答}
  \bch
  证明:用归纳法。命题对$n=1$显然。对$n>1$,假设命题对$n-1$成立。

  在作业题中我们已经用最大模原理证明了$P(z)$至少有一个复数根,不妨设为$z_1$,用$z-z_1$去除$P(z)$,一定可以得到如下结果:
  $$P(z) = (z-z_1) Q(z) + c$$
  其中 $Q(z)$为$n-1$次多项式,$c$为常数。两边令$z=z_1$并利用$z_1$是$P(z)$的根这一事实,就得到$c=0$。也就是说
  $$P(z) = (z-z_1)Q(z).$$
  再利用归纳假设,$Q(z)$可以分解为一次多项式的乘积。证毕。

  
  \ech
\end{frame}

\begin{frame}
\chtitle{初中知识回顾:多项式的根和系数关系}
\bch
设多项式$P(z) = z^n + c_{n-1}z^{n-1}+ c_{n-2}z^{n-2}+\ldots + c_1 z + c_0$的所有根为
$z_1, z_2, \ldots, z_n$,则
$$(z-z_1)(z-z_2)\ldots(z-z_n) = z^n + c_{n-1}z^{n-1}+ c_{n-2}z^{n-2}+\ldots + c_1 z + c_0$$
比较两边的同次项系数,可以得到
\bea
z_1+z_2+\ldots + z_n &=& -c_{n-1} \newl
\sum_{1\le i<j\le n} z_i z_j &=& c_{n-2} \newl
\sum_{1\le i<j<k\le n} z_i z_jz_k &=& -c_{n-3} \newl
\ldots && \newl
z_1z_2\ldots z_n\left(\frac{1}{z_1}+\frac{1}{z_2}+\ldots+\frac{1}{z_n}\right) &=& (-1)^{n-1} c_1 \newl
z_1z_2\ldots z_n &=& (-1)^n c_0
\eea
\ech
\end{frame}

\stepcounter{problem}
\begin{frame}
\chtitle{\proid (\stwo)}
\bch
方程$z^5+z^4+5z^3+1=0$的所有复数根的平方和等于多少?
\ech
\end{frame}

\begin{frame}
\chtitle{\proid 解答}
\bch
设五个根为$z_1, z_2, \ldots, z_5$,根据根和系数的关系
\begin{eqnarray}
\sum_{i=1}^5 z_i &=& -1,  \label{eq:1-1-1} \\
\sum_{1\le i<j\le 5} z_iz_j &=& 5. \label{eq:1-1-2}
\end{eqnarray}
\eqref{eq:1-1-1} $- 2\times $ \eqref{eq:1-1-2} 得到
$$\sum_i z_i^2 = -9. $$
\ech
\end{frame}

\stepcounter{problem}
\begin{frame}
\chtitle{\proid (\sthree)}
\bch
证明对任意正整数$n$,
$$\sin\frac{\pi}{n} \sin\frac{2\pi}{n}\ldots \sin\frac{(n-1)\pi}{n} = \frac{n}{2^{n-1}} \, . $$
\ech
\end{frame}

\begin{frame}
\chtitle{\proid  解答}
\bch
考虑$\frac{1-(1-z)^n}{z} = 0$的$n-1$个根$ z_k = 1- e^{\frac{2k \pi \ii}{n}} $ ($k=1,\ldots, n-1$),利用$|z_k| = 2\sin\frac{k\pi}{n}$,以及根与系数关系
$$|z_1z_2\ldots z_{n-1}| = n\,. $$
立刻可以得到结论。
\ech
\end{frame}



\stepcounter{problem}
\begin{frame}
\chtitle{\proid (\stwo)}
\bch
设$f(z)$在单位圆$|z|\le 1$内部解析,边界上连续;而且已知当$|z|=1$时,$|f(z)-z|\le 1$。试证明:
$$\left\vert\frac{1}{n!}f^{(n)}\left(\frac{1}{2}\right)\right\vert \le 2^{n+2}.$$
\ech
\end{frame}



\begin{frame}
\chtitle{\proid 解答}
\bch
\bea
\left\vert\frac{1}{n!}f^{(n)}\left(\frac{1}{2}\right) \right\vert &=& \left\vert\frac{1}{2\pi \ii}\oint_{|z|=1}\frac{f(z)}{\left(z-\frac{1}{2}\right)^{n+1}} dz \right\vert \newl
&\le & \frac{1}{2\pi} \oint_{|z|=1} \frac{|f(z)-z|+|z|}{\left\vert z-\frac{1}{2}\right\vert^{n+1}} |dz| \newl
&\le & \frac{1}{2\pi} \oint_{|z|=1} \frac{2}{\left(\frac{1}{2}\right)^{n+1}} |dz| \newl
&=& 2^{n+2}
\eea
\ech
\end{frame}



\begin{frame}
\chtitle{无穷远点是个伪概念\bigbye}
\bch
在很多复变函数的教材中有无穷远点和无穷远点处的留数的概念。

\skiplines

我们并不想这样自寻烦恼:有两种很简单的方法可以解决问题:
\bitem
\item[(1)]{画一个半径$\rightarrow \infty$的大圆,考虑大圆和已有围道之间的区域。}
\item[(2)]{换元$u=\frac{1}{z}$。}
\eitem
\ech
\end{frame}

\stepcounter{problem}
\begin{frame}
\chtitle{\proid (\stwo)}
\bch
\addfig{2}{problem1-3.png}
在如图的围道上计算积分
$$\oint_{|z|=2}\,\frac{z^{60}}{(z-3)\left(z^{32}-1\right)^2} dz\,. $$
\ech
\end{frame}

\begin{frame}
\chtitle{\proid 解法一}
\bch
\addfig{1.9}{problem1-3-s.png}

令区域$T: 3<|z|<R$。所考虑的函数在$T$内只有一个孤立奇点:$z=3$。根据留数定理:
$$\left(\oint_{C_R} - \oint_C\right) f(z) dz = 2\pi\ii\res{f}{3} $$
\ech
\end{frame}

\begin{frame}
\chtitle{\proid 解法一(续)}
\bch
\addfig{1.8}{problem1-3-s.png}
容易看出当$R\rightarrow \infty$时,在$C_R$上有$|f(z)|\sim \frac{1}{R^5}$,沿$C_R$积分后最多$\sim \frac{1}{R^4} \rightarrow 0$。因此得到
$$ \oint_C f(z) dz  = -2\pi\ii \res{f}{3} = -2\pi\ii\frac{3^{60}}{(3^{32}-1)^2} $$
\ech
\end{frame}


\begin{frame}
\chtitle{\proid 解法二}
\bch
令$u = \frac{1}{z}$,则
$$\oint_{|z|=2} \frac{z^{60}}{(z-3)\left(z^{32}-1\right)^2} dz = \int_{|u|=\frac{1}{2}} \frac{u^3}{(1-3u)\left(1-u^{32}\right)^2} du $$
{\scriptsize 注意映射$u=1/z$使逆时针方向的围道变为顺时针方向,换回到(默认的)逆时针方向后又多了个负号。}

在围道内仅有$u=1/3$一个孤立奇点,容易用留数定理算出
$$ \int_{|u|=\frac{1}{2}} \frac{u^3}{(1-3u)\left(1-u^{32}\right)^2} du = -\frac{2\pi\ii}{3} \frac{1}{3^3\left(1-\frac{1}{3^{32}}\right)^2} = -2\pi\ii \frac{3^{60}}{\left(3^{32}-1\right)^2} $$

\ech
\end{frame}


\stepcounter{problem}

\begin{frame}
\chtitle{\proid (\stwo)}
\bch
\addfig{2}{problem1-3.png}
在如图的围道上计算积分
$$\oint_{|z|=2}\,\frac{z^3e^{\frac{1}{z}}}{1+z} dz .$$
\ech
\end{frame}

\begin{frame}
\chtitle{\proid 解法一}
\bch
{\small
$f(z) = \frac{z^3e^{\frac{1}{z}}}{1+z}$在围道内部有两个孤立奇点:$z=0$和$z=-1$。
\be
\res{f}{-1} = - e^{-1}
\ee
在$z=0$邻域则可以直接进行Laurent展开:
$$f(z) = (z^3-z^4+z^5-\ldots)\left(1+\frac{1}{z}+\frac{1}{2! z^2}+ \ldots\right)$$
\bea
\res{f}{0} &=& \frac{1}{4!} - \frac{1}{5!} + \frac{1}{6!} - \ldots \newl
&=& e^{-1} - \left(1-\frac{1}{1!}+\frac{1}{2!} - \frac{1}{3!}\right) \newl
&=&  e^{-1} - \frac{1}{3}.
\eea
所以所求积分为
$$2\pi \ii\left(e^{-1}-\frac{1}{3}-e^{-1}\right) = -\frac{2\pi \ii}{3}.$$
}
\ech
\end{frame}

\begin{frame}
\chtitle{\proid 解法二}
\bch
令$u=\frac{1}{z}$,则所求积分为
$$\oint_{|u|=\frac{1}{2}} \frac{e^u}{u^4(u+1)} du ,$$
在围道内仅有一孤立奇点$u = 0$,故所求积分为
$$2\pi\ii \frac{1}{3!}\left.\left(\frac{d^3}{du^3} \frac{e^u}{1+u}\right)\right\vert_{u=0} = -\frac{2\pi\ii}{3}.$$
\ech
\end{frame}

\end{document}
