\documentclass[CJK]{beamer}
\usepackage{CJKutf8}
\usepackage{beamerthemesplit}
\usetheme{Malmoe}
\useoutertheme[footline=authortitle]{miniframes}
\usepackage{amsmath}
\usepackage{amssymb}
\usepackage{graphicx}
\usepackage{eufrak}
\usepackage{color}
\usepackage{slashed}
\usepackage{simplewick}
\usepackage{tikz}
\usepackage{tcolorbox}
\graphicspath{{../figures/}}
%%figures
\def\lfig#1#2{\includegraphics[width=#1 in]{#2}}
\def\addfig#1#2{\begin{center}\includegraphics[width=#1 in]{#2}\end{center}}
\def\wulian{\includegraphics[width=0.18in]{emoji_wulian.jpg}}
\def\bigwulian{\includegraphics[width=0.35in]{emoji_wulian.jpg}}
\def\bye{\includegraphics[width=0.18in]{emoji_bye.jpg}}
\def\bigbye{\includegraphics[width=0.35in]{emoji_bye.jpg}}
\def\huaixiao{\includegraphics[width=0.18in]{emoji_huaixiao.jpg}}
\def\bighuaixiao{\includegraphics[width=0.35in]{emoji_huaixiao.jpg}}
\def\jianxiao{\includegraphics[width=0.18in]{emoji_jianxiao.jpg}}
\def\bigjianxiao{\includegraphics[width=0.35in]{emoji_jianxiao.jpg}}
%% colors
\def\blacktext#1{{\color{black}#1}}
\def\bluetext#1{{\color{blue}#1}}
\def\redtext#1{{\color{red}#1}}
\def\darkbluetext#1{{\color[rgb]{0,0.2,0.6}#1}}
\def\skybluetext#1{{\color[rgb]{0.2,0.7,1.}#1}}
\def\cyantext#1{{\color[rgb]{0.,0.5,0.5}#1}}
\def\greentext#1{{\color[rgb]{0,0.7,0.1}#1}}
\def\darkgray{\color[rgb]{0.2,0.2,0.2}}
\def\lightgray{\color[rgb]{0.6,0.6,0.6}}
\def\gray{\color[rgb]{0.4,0.4,0.4}}
\def\blue{\color{blue}}
\def\red{\color{red}}
\def\green{\color{green}}
\def\darkgreen{\color[rgb]{0,0.4,0.1}}
\def\darkblue{\color[rgb]{0,0.2,0.6}}
\def\skyblue{\color[rgb]{0.2,0.7,1.}}
%%control
\def\be{\begin{equation}}
\def\ee{\nonumber\end{equation}}
\def\bea{\begin{eqnarray}}
\def\eea{\nonumber\end{eqnarray}}
\def\bch{\begin{CJK}{UTF8}{gbsn}}
\def\ech{\end{CJK}}
\def\bitem{\begin{itemize}}
\def\eitem{\end{itemize}}
\def\bcenter{\begin{center}}
\def\ecenter{\end{center}}
\def\bex{\begin{minipage}{0.2\textwidth}\includegraphics[width=0.6in]{jugelizi.png}\end{minipage}\begin{minipage}{0.76\textwidth}}
\def\eex{\end{minipage}}
\def\chtitle#1{\frametitle{\bch#1\ech}}
\def\bmat#1{\left(\begin{array}{#1}}
\def\emat{\end{array}\right)}
\def\bcase#1{\left\{\begin{array}{#1}}
\def\ecase{\end{array}\right.}
\def\bmini#1{\begin{minipage}{#1\textwidth}}
\def\emini{\end{minipage}}
\def\tbox#1{\begin{tcolorbox}#1\end{tcolorbox}}
\def\pfrac#1#2#3{\left(\frac{\partial #1}{\partial #2}\right)_{#3}}
%%symbols
\def\bropt{\,(\ \ \ )}
\def\sone{$\star$}
\def\stwo{$\star\star$}
\def\sthree{$\star\star\star$}
\def\sfour{$\star\star\star\star$}
\def\sfive{$\star\star\star\star\star$}
\def\rint{{\int_\leftrightarrow}}
\def\roint{{\oint_\leftrightarrow}}
\def\stdHf{{\textit{\r H}_f}}
\def\deltaH{{\Delta \textit{\r H}}}
\def\ii{{\dot{\imath}}}
\def\skipline{{\vskip0.1in}}
\def\skiplines{{\vskip0.2in}}
\def\lagr{{\mathcal{L}}}
\def\hamil{{\mathcal{H}}}
\def\vecv{{\mathbf{v}}}
\def\vecx{{\mathbf{x}}}
\def\vecy{{\mathbf{y}}}
\def\veck{{\mathbf{k}}}
\def\vecp{{\mathbf{p}}}
\def\vecn{{\mathbf{n}}}
\def\vecA{{\mathbf{A}}}
\def\vecP{{\mathbf{P}}}
\def\vecsigma{{\mathbf{\sigma}}}
\def\hatJn{{\hat{J_\vecn}}}
\def\hatJx{{\hat{J_x}}}
\def\hatJy{{\hat{J_y}}}
\def\hatJz{{\hat{J_z}}}
\def\hatj#1{\hat{J_{#1}}}
\def\hatphi{{\hat{\phi}}}
\def\hatq{{\hat{q}}}
\def\hatpi{{\hat{\pi}}}
\def\vel{\upsilon}
\def\Dint{{\mathcal{D}}}
\def\adag{{\hat{a}^\dagger}}
\def\bdag{{\hat{b}^\dagger}}
\def\cdag{{\hat{c}^\dagger}}
\def\ddag{{\hat{d}^\dagger}}
\def\hata{{\hat{a}}}
\def\hatb{{\hat{b}}}
\def\hatc{{\hat{c}}}
\def\hatd{{\hat{d}}}
\def\hatN{{\hat{N}}}
\def\hatH{{\hat{H}}}
\def\hatp{{\hat{p}}}
\def\Fup{{F^{\mu\nu}}}
\def\Fdown{{F_{\mu\nu}}}
\def\newl{\nonumber \\}
\def\vece{\mathrm{e}}
\def\calM{{\mathcal{M}}}
\def\calT{{\mathcal{T}}}
\def\calR{{\mathcal{R}}}
\def\barpsi{\bar{\psi}}
\def\baru{\bar{u}}
\def\barv{\bar{\upsilon}}
\def\qeq{\stackrel{?}{=}}
\def\torder#1{\mathcal{T}\left(#1\right)}
\def\rorder#1{\mathcal{R}\left(#1\right)}
\def\contr#1#2{\contraction{}{#1}{}{#2}#1#2}
\def\trof#1{\mathrm{Tr}\left(#1\right)}
\def\trace{\mathrm{Tr}}
\def\comm#1{\ \ \ \left(\mathrm{used}\ #1\right)}
\def\tcomm#1{\ \ \ (\text{#1})}
\def\slp{\slashed{p}}
\def\slk{\slashed{k}}
\def\calp{{\mathfrak{p}}}
\def\veccalp{\mathbf{\mathfrak{p}}}
\def\Tthree{T_{\tiny \textcircled{3}}}
\def\pthree{p_{\tiny \textcircled{3}}}
\def\dbar{{\,\mathchar'26\mkern-12mu d}}
\def\erf{\mathrm{erf}}
\def\const{\mathrm{constant}}
\def\pheat{\pfrac p{\ln T}V}
\def\vheat{\pfrac V{\ln T}p}
%%units
\def\fdeg{{^\circ \mathrm{F}}}
\def\cdeg{^\circ \mathrm{C}}
\def\atm{\,\mathrm{atm}}
\def\angstrom{\,\text{\AA}}
\def\SIL{\,\mathrm{L}}
\def\SIkm{\,\mathrm{km}}
\def\SIyr{\,\mathrm{yr}}
\def\SIGyr{\,\mathrm{Gyr}}
\def\SIV{\,\mathrm{V}}
\def\SImV{\,\mathrm{mV}}
\def\SIeV{\,\mathrm{eV}}
\def\SIkeV{\,\mathrm{keV}}
\def\SIMeV{\,\mathrm{MeV}}
\def\SIGeV{\,\mathrm{GeV}}
\def\SIcal{\,\mathrm{cal}}
\def\SIkcal{\,\mathrm{kcal}}
\def\SImol{\,\mathrm{mol}}
\def\SIN{\,\mathrm{N}}
\def\SIHz{\,\mathrm{Hz}}
\def\SIm{\,\mathrm{m}}
\def\SIcm{\,\mathrm{cm}}
\def\SIfm{\,\mathrm{fm}}
\def\SImm{\,\mathrm{mm}}
\def\SInm{\,\mathrm{nm}}
\def\SImum{\,\mathrm{\mu m}}
\def\SIJ{\,\mathrm{J}}
\def\SIW{\,\mathrm{W}}
\def\SIkJ{\,\mathrm{kJ}}
\def\SIs{\,\mathrm{s}}
\def\SIkg{\,\mathrm{kg}}
\def\SIg{\,\mathrm{g}}
\def\SIK{\,\mathrm{K}}
\def\SImmHg{\,\mathrm{mmHg}}
\def\SIPa{\,\mathrm{Pa}}

\def\courseurl{https://github.com/zqhuang/SYSU\_TD}

\def\tpage#1#2{
\begin{frame}
\begin{center}
\begin{Large}
\bch
热学 \\
第#1讲 #2

{\vskip 0.3in}

黄志琦

\ech
\end{Large}
\end{center}

\vskip 0.2in

\bch
教材:《热学》第二版,赵凯华,罗蔚茵,高等教育出版社
\ech

\bch
课件下载
\ech
\courseurl
\end{frame}
}

\def\bfr#1{
\begin{frame}
\chtitle{#1} 
\bch
}

\def\efr{
\ech 
\end{frame}
}

  \date{}
\begin{document}
\tpage{10}{Problem Set II}

\newcounter{chap}
\newcounter{problem}[chap]
\def\proid{{Problem \thechap.\theproblem}\ }

\begin{frame}
  \chtitle{本讲内容: 复变积分实战技巧}
  \bch
  \bitem
\item{复变积分的思考次序}
\item{复变积分不等式的证明}
  \eitem
  \ech
\end{frame}

\setcounter{chap}{2}
\setcounter{problem}{0}

\begin{frame}
\chtitle{复变积分的思考次序}
\bch
复变积分的思考次序
\bitem
\item{是否可以一眼看出原函数}
\item{是否可以直接用留数定理}
\item{是否可以构造围道或变量替换间接地用留数定理}
\item{最后一招:写成实积分硬算}
\eitem
  \ech
\end{frame}


\stepcounter{problem}
\begin{frame}
\chtitle{\proid (\sone)}
\bch
\addfig{1.5}{problem2-1.png}

如图,计算从$1$到$-1$的一条形状为抛物线的曲线$C$计算积分
$$\int_C \frac{dz}{\sqrt{z}},$$
约定$z=1$时幅角为零且幅角连续变化。

\ech
\end{frame}



\begin{frame}
\chtitle{\proid 解答}
\bch
\addfig{1.5}{problem2-1.png}

一眼可以看出原函数,但注意要弄清积分起点和终点的幅角。

\be
\int_C \frac{dz}{\sqrt{z}} = 2 \sqrt{z}|_{1}^{e^{\ii\pi}} = 2(e^{\ii\pi/2} - 1) = 2(\ii -1)
\ee

\ech
\end{frame}

\stepcounter{problem}

\begin{frame}
\chtitle{\proid (\stwo)}
\bch
\addfig{1.5}{problem2-1.png}

沿如图的抛物线形状路径$C$计算积分
$$\oint_C \frac{dz}{1+z^2}. $$
\ech
\end{frame}


\begin{frame}
\chtitle{\proid 解答}
\bch
根据
$$\frac{1}{1+z^2} = \frac{1}{2\ii}\left(\frac{1}{z-\ii} - \frac{1}{z+\ii}\right)$$
一眼可以看出原函数为$$F(z) = \frac{1}{2\ii} \ln \frac{z-\ii}{z+\ii}$$ (注意不要凭借高数的经验把$\frac{1}{1+z^2}$的原函数写成$\arctan z$这样随幅角变化规律不明显的形式)。

从起点到终点$\frac{z-\ii}{z+\ii}$的模不变,幅角增大了$\pi$(这点可能需要仔细画个图分析下才能明白),所以$\ln \frac{z-\ii}{z+\ii}$的变化为$\ii \pi$。积分结果为
  $$\frac{1}{2\ii} \ii \pi = \frac{\pi}{2}. $$



\ech
\end{frame}


\stepcounter{problem}

\begin{frame}
\chtitle{\proid (\stwo)}
\bch
计算沿逆时针方向的围道积分:
$$\oint_{|z|=4}\frac{\sin z}{(z-\pi)^4}dz$$
\ech
\end{frame}

\begin{frame}
\chtitle{\proid 解答}
\bch
一眼看不出原函数。

围道内只有$z=\pi$一个孤立奇点,可以直接使用留数定理。虽然这是个三阶极点,但是把它当作四阶极点,用去极点法来求留数更方便:
$$\res{f}{\pi} = \left.\frac{1}{3!}\left[\frac{d^3}{dz^3} \sin z\right] \right\vert_{z=\pi} = \left.\frac{1}{3!}\left[-\cos z\right] \right\vert_{z=\pi} = \frac{1}{6} $$
于是所求积分
$$\oint_{|z|=4}f(z) dz = 2\pi\ii \res{f}{\pi} = \frac{\pi \ii}{3}$$

\ech
\end{frame}

\stepcounter{problem}
\begin{frame}
\chtitle{\proid (\sthree)}
\bch
\addfig{1.5}{problem2-1.png}

沿如图的抛物线形状路径$C$计算积分
$$\oint_C (1-z^2)^{-\frac{3}{4}} dz. $$
约定$1-z^2$的幅角在路径的中间(即$z=1.5\ii$)时为零,且幅角连续变化。
\ech
\end{frame}


\begin{frame}
\chtitle{\proid 解答}
\bch

看不出原函数。也不能直接用留数定理。

\addfig{1.8}{problem2-2.png}

如图,可以通过构造围道间接使用留数定理。

请自行用“绕一圈”的方法判断:可取适当单值分枝使这个围道内被积函数处处解析。另,请验证左右两段小圆弧上的积分(当小圆弧半径趋向于零时)趋向于零。


\ech
\end{frame}

\begin{frame}
\chtitle{\proid 解答}
\bch
于是
$$\oint_C (1-z^2)^{-\frac{3}{4}} dz + \int_{-1}^1(1-x^2)^{-\frac{3}{4}} dx = 0.$$
{\small (请验证$1-x^2$的幅角按照题中所给约定确实为零。)}

令$x=2u-1$
$$\int_{-1}^1(1-x^2)^{-\frac{3}{4}} dx= \frac{1}{\sqrt{2}}\int_0^1u^{-\frac{3}{4}}(1-u)^{-\frac{3}{4}} du = \frac{1}{\sqrt{2\pi}} \left[\sfgamma{\frac{1}{4}}\right]^2 $$
所以
$$\oint_C (1-z^2)^{-\frac{3}{4}} dz = -\frac{1}{\sqrt{2\pi}} \left[\sfgamma{\frac{1}{4}}\right]^2$$
\ech
\end{frame}


\begin{frame}
\chtitle{无穷远点是个伪概念}
\bch
在很多复变函数的教材中有无穷远点和无穷远点处的留数的概念:这些都是为了多折腾几个公式让你背\bye。

\skiplines

实际上并不需要把无穷远点当作一个点,用构造围道和变量替换间接使用留数定理的方法完全可以解决。典型的技巧有:
\bitem
\item[(1)]{画一个半径$\rightarrow \infty$的大圆,考虑大圆和已有围道之间的区域。}
\item[(2)]{换元$u=\frac{1}{z}$。}
\eitem
\ech
\end{frame}

\stepcounter{problem}
\begin{frame}
\chtitle{\proid (\stwo)}
\bch
\addfig{2}{problem1-3.png}
在如图的围道上计算积分
$$\oint_{|z|=2}\,\frac{z^{60}}{(z-3)\left(z^{32}-1\right)^2} dz\,. $$
\ech
\end{frame}

\begin{frame}
\chtitle{\proid 解法一}
\bch
\addfig{1.9}{problem1-3-s.png}

令区域$T: 2<|z|<R$。所考虑的函数在$T$内只有一个孤立奇点:$z=3$。根据留数定理:
$$\left(\oint_{C_R} - \oint_C\right) f(z) dz = 2\pi\ii\res{f}{3} $$
\ech
\end{frame}

\begin{frame}
\chtitle{\proid 解法一(续)}
\bch
\addfig{1.8}{problem1-3-s.png}
容易看出当$R\rightarrow \infty$时,在$C_R$上有$|f(z)|\sim \frac{1}{R^5}$,沿$C_R$积分后最多$\sim \frac{1}{R^4} \rightarrow 0$。因此得到
$$ \oint_C f(z) dz  = -2\pi\ii \res{f}{3} = -2\pi\ii\frac{3^{60}}{(3^{32}-1)^2} $$
\ech
\end{frame}


\begin{frame}
\chtitle{\proid 解法二}
\bch
令$u = \frac{1}{z}$,则
$$\oint_{|z|=2} \frac{z^{60}}{(z-3)\left(z^{32}-1\right)^2} dz = \int_{|u|=\frac{1}{2}} \frac{u^3}{(1-3u)\left(1-u^{32}\right)^2} du $$
{\scriptsize 注意映射$u=1/z$使逆时针方向的围道变为顺时针方向,换回到(默认的)逆时针方向后又多了个负号。}

在围道内仅有$u=1/3$一个孤立奇点,容易用留数定理算出
$$ \int_{|u|=\frac{1}{2}} \frac{u^3}{(1-3u)\left(1-u^{32}\right)^2} du = -\frac{2\pi\ii}{3} \frac{1}{3^3\left(1-\frac{1}{3^{32}}\right)^2} = -2\pi\ii \frac{3^{60}}{\left(3^{32}-1\right)^2} $$

\ech
\end{frame}


\stepcounter{problem}

\begin{frame}
\chtitle{\proid (\stwo)}
\bch
\addfig{2}{problem1-3.png}
在如图的围道上计算积分
$$\oint_{|z|=2}\,\frac{z^3e^{\frac{1}{z}}}{1+z} dz .$$
\ech
\end{frame}

\begin{frame}
\chtitle{\proid 解法一}
\bch
{\small
$f(z) = \frac{z^3e^{\frac{1}{z}}}{1+z}$在围道内部有两个孤立奇点:$z=0$和$z=-1$。
\be
\res{f}{-1} = - e^{-1}
\ee
在$z=0$邻域则可以直接进行Laurent展开:
$$f(z) = (z^3-z^4+z^5-\ldots)\left(1+\frac{1}{z}+\frac{1}{2! z^2}+ \ldots\right)$$
\bea
\res{f}{0} &=& \frac{1}{4!} - \frac{1}{5!} + \frac{1}{6!} - \ldots \newl
&=& e^{-1} - \left(1-\frac{1}{1!}+\frac{1}{2!} - \frac{1}{3!}\right) \newl
&=&  e^{-1} - \frac{1}{3}.
\eea
所以所求积分为
$$2\pi \ii\left(e^{-1}-\frac{1}{3}-e^{-1}\right) = -\frac{2\pi \ii}{3}.$$
}
\ech
\end{frame}

\begin{frame}
\chtitle{\proid 解法二}
\bch
令$u=\frac{1}{z}$,则所求积分为
$$\oint_{|u|=\frac{1}{2}} \frac{e^u}{u^4(u+1)} du ,$$
在围道内仅有一孤立奇点$u = 0$,故所求积分为
\bea
&& 2\pi\ii \frac{1}{3!}\left.\left(\frac{d^3}{du^3} \frac{e^u}{1+u}\right)\right\vert_{u=0} \newl
&=& \frac{\pi\ii}{3}\frac{e^u}{1+u}\left(1-\frac{3}{1+u}+\frac{6}{(1+u)^2}-\frac{6}{(1+u)^3}\right)_{u=0} \newl
&=& -\frac{2\pi\ii}{3}.
\eea
\ech
\end{frame}


\stepcounter{problem}
\begin{frame}
\chtitle{\proid (\sthree)}
\bch
\addfig{1.5}{problem2-1.png}

在如图的抛物线路径上计算积分
$$\int_C \left(z^*dz - zdz^*\right) $$
其中$z^*$表示$z$的共轭复数。
\ech
\end{frame}

\begin{frame}
\chtitle{\proid 解答}
\bch
{\small 依赖于$z^*$的函数一般都不是解析函数,原函数还是留数定理都希望不大。无奈只能写成实积分。先尝试极坐标,令$z = re^{\ii\theta}$:
\bea
\int_C \left(z^*dz - zdz^*\right) &=& \int_C re^{-\ii\theta}(dr + \ii rd\theta)e^{\ii\theta} - re^{\ii\theta} (dr - \ii r d\theta)e^{-\ii\theta} \newl
&=& 2\ii r^2d\theta 
\eea
而$\frac{1}{2}r^2d\theta$恰巧是图中扫过的面积元。

\addfig{1.5}{problem2-3.png}

因此原积分等于抛物线下的面积乘以$4\ii$,即结果为$8\ii$。}
\ech
\end{frame}


\secpage{复变积分不等式的证明}{$$\left\vert\int f(z) dz \right\vert\le \int |f(z)| |dz| $$}

\stepcounter{problem}
\begin{frame}
\chtitle{\proid (\stwo)}
\bch
设$f(z)$在单位圆$|z|\le 1$内部解析,边界上连续;而且已知当$|z|=1$时,$|f(z)-z|\le 1$。试证明:
$$\left\vert f^{(n)}\left(\frac{1}{2}\right)\right\vert \le  2^{n+2} n!.$$
\ech
\end{frame}



\begin{frame}
\chtitle{\proid 解答}
\bch
\bea
\left\vert f^{(n)}\left(\frac{1}{2}\right) \right\vert &=& \left\vert\frac{n!}{2\pi \ii}\oint_{|z|=1}\frac{f(z)}{\left(z-\frac{1}{2}\right)^{n+1}} dz \right\vert \newl
&\le & \frac{n!}{2\pi} \oint_{|z|=1} \frac{|f(z)|}{\left\vert z-\frac{1}{2}\right\vert^{n+1}} |dz| \newl
&\le & \frac{n!}{2\pi} \oint_{|z|=1} \frac{|f(z)-z|+|z|}{\left\vert z-\frac{1}{2}\right\vert^{n+1}} |dz| \newl
&\le & \frac{n!}{2\pi} \oint_{|z|=1} \frac{2}{\left(\frac{1}{2}\right)^{n+1}} |dz| \newl
&=& 2^{n+2} n! 
\eea
\ech
\end{frame}



\end{document}
