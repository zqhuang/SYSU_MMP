\documentclass[CJK,13pt]{beamer}
\usepackage{CJKutf8}
\usepackage{beamerthemesplit}
\usetheme{Malmoe}
\useoutertheme[footline=authortitle]{miniframes}
\usepackage{amsmath}
\usepackage{amssymb}
\usepackage{graphicx}
\usepackage{eufrak}
\usepackage{color}
\usepackage{slashed}
\usepackage{simplewick}
\usepackage{tikz}
\usepackage{tcolorbox}
\graphicspath{{../figures/}}
%%figures
\def\lfig#1#2{\includegraphics[width=#1 in]{#2}}
\def\addfig#1#2{\begin{center}\includegraphics[width=#1 in]{#2}\end{center}}
\def\wulian{\includegraphics[width=0.18in]{emoji_wulian.jpg}}
\def\bigwulian{\includegraphics[width=0.35in]{emoji_wulian.jpg}}
\def\bye{\includegraphics[width=0.18in]{emoji_bye.jpg}}
\def\bigbye{\includegraphics[width=0.35in]{emoji_bye.jpg}}
\def\huaixiao{\includegraphics[width=0.18in]{emoji_huaixiao.jpg}}
\def\bighuaixiao{\includegraphics[width=0.35in]{emoji_huaixiao.jpg}}
\def\jianxiao{\includegraphics[width=0.18in]{emoji_jianxiao.jpg}}
\def\bigjianxiao{\includegraphics[width=0.35in]{emoji_jianxiao.jpg}}
%% colors
\def\blacktext#1{{\color{black}#1}}
\def\bluetext#1{{\color{blue}#1}}
\def\redtext#1{{\color{red}#1}}
\def\darkbluetext#1{{\color[rgb]{0,0.2,0.6}#1}}
\def\skybluetext#1{{\color[rgb]{0.2,0.7,1.}#1}}
\def\cyantext#1{{\color[rgb]{0.,0.5,0.5}#1}}
\def\greentext#1{{\color[rgb]{0,0.7,0.1}#1}}
\def\darkgray{\color[rgb]{0.2,0.2,0.2}}
\def\lightgray{\color[rgb]{0.6,0.6,0.6}}
\def\gray{\color[rgb]{0.4,0.4,0.4}}
\def\blue{\color{blue}}
\def\red{\color{red}}
\def\green{\color{green}}
\def\darkgreen{\color[rgb]{0,0.4,0.1}}
\def\darkblue{\color[rgb]{0,0.2,0.6}}
\def\skyblue{\color[rgb]{0.2,0.7,1.}}
%%control
\def\be{\begin{equation}}
\def\ee{\nonumber\end{equation}}
\def\bea{\begin{eqnarray}}
\def\eea{\nonumber\end{eqnarray}}
\def\bch{\begin{CJK}{UTF8}{gbsn}}
\def\ech{\end{CJK}}
\def\bitem{\begin{itemize}}
\def\eitem{\end{itemize}}
\def\bcenter{\begin{center}}
\def\ecenter{\end{center}}
\def\bex{\begin{minipage}{0.2\textwidth}\includegraphics[width=0.6in]{jugelizi.png}\end{minipage}\begin{minipage}{0.76\textwidth}}
\def\eex{\end{minipage}}
\def\chtitle#1{\frametitle{\bch#1\ech}}
\def\bmat#1{\left(\begin{array}{#1}}
\def\emat{\end{array}\right)}
\def\bcase#1{\left\{\begin{array}{#1}}
\def\ecase{\end{array}\right.}
\def\bmini#1{\begin{minipage}{#1\textwidth}}
\def\emini{\end{minipage}}
\def\tbox#1{\begin{tcolorbox}#1\end{tcolorbox}}
\def\pfrac#1#2#3{\left(\frac{\partial #1}{\partial #2}\right)_{#3}}
%%symbols
\def\bropt{\,(\ \ \ )}
\def\sone{$\star$}
\def\stwo{$\star\star$}
\def\sthree{$\star\star\star$}
\def\sfour{$\star\star\star\star$}
\def\sfive{$\star\star\star\star\star$}
\def\rint{{\int_\leftrightarrow}}
\def\roint{{\oint_\leftrightarrow}}
\def\stdHf{{\textit{\r H}_f}}
\def\deltaH{{\Delta \textit{\r H}}}
\def\ii{{\dot{\imath}}}
\def\skipline{{\vskip0.1in}}
\def\skiplines{{\vskip0.2in}}
\def\lagr{{\mathcal{L}}}
\def\hamil{{\mathcal{H}}}
\def\vecv{{\mathbf{v}}}
\def\vecx{{\mathbf{x}}}
\def\vecy{{\mathbf{y}}}
\def\veck{{\mathbf{k}}}
\def\vecp{{\mathbf{p}}}
\def\vecn{{\mathbf{n}}}
\def\vecA{{\mathbf{A}}}
\def\vecP{{\mathbf{P}}}
\def\vecsigma{{\mathbf{\sigma}}}
\def\hatJn{{\hat{J_\vecn}}}
\def\hatJx{{\hat{J_x}}}
\def\hatJy{{\hat{J_y}}}
\def\hatJz{{\hat{J_z}}}
\def\hatj#1{\hat{J_{#1}}}
\def\hatphi{{\hat{\phi}}}
\def\hatq{{\hat{q}}}
\def\hatpi{{\hat{\pi}}}
\def\vel{\upsilon}
\def\Dint{{\mathcal{D}}}
\def\adag{{\hat{a}^\dagger}}
\def\bdag{{\hat{b}^\dagger}}
\def\cdag{{\hat{c}^\dagger}}
\def\ddag{{\hat{d}^\dagger}}
\def\hata{{\hat{a}}}
\def\hatb{{\hat{b}}}
\def\hatc{{\hat{c}}}
\def\hatd{{\hat{d}}}
\def\hatN{{\hat{N}}}
\def\hatH{{\hat{H}}}
\def\hatp{{\hat{p}}}
\def\Fup{{F^{\mu\nu}}}
\def\Fdown{{F_{\mu\nu}}}
\def\newl{\nonumber \\}
\def\vece{\mathrm{e}}
\def\calM{{\mathcal{M}}}
\def\calT{{\mathcal{T}}}
\def\calR{{\mathcal{R}}}
\def\barpsi{\bar{\psi}}
\def\baru{\bar{u}}
\def\barv{\bar{\upsilon}}
\def\qeq{\stackrel{?}{=}}
\def\torder#1{\mathcal{T}\left(#1\right)}
\def\rorder#1{\mathcal{R}\left(#1\right)}
\def\contr#1#2{\contraction{}{#1}{}{#2}#1#2}
\def\trof#1{\mathrm{Tr}\left(#1\right)}
\def\trace{\mathrm{Tr}}
\def\comm#1{\ \ \ \left(\mathrm{used}\ #1\right)}
\def\tcomm#1{\ \ \ (\text{#1})}
\def\slp{\slashed{p}}
\def\slk{\slashed{k}}
\def\calp{{\mathfrak{p}}}
\def\veccalp{\mathbf{\mathfrak{p}}}
\def\Tthree{T_{\tiny \textcircled{3}}}
\def\pthree{p_{\tiny \textcircled{3}}}
\def\dbar{{\,\mathchar'26\mkern-12mu d}}
\def\erf{\mathrm{erf}}
\def\const{\mathrm{constant}}
\def\pheat{\pfrac p{\ln T}V}
\def\vheat{\pfrac V{\ln T}p}
%%units
\def\fdeg{{^\circ \mathrm{F}}}
\def\cdeg{^\circ \mathrm{C}}
\def\atm{\,\mathrm{atm}}
\def\angstrom{\,\text{\AA}}
\def\SIL{\,\mathrm{L}}
\def\SIkm{\,\mathrm{km}}
\def\SIyr{\,\mathrm{yr}}
\def\SIGyr{\,\mathrm{Gyr}}
\def\SIV{\,\mathrm{V}}
\def\SImV{\,\mathrm{mV}}
\def\SIeV{\,\mathrm{eV}}
\def\SIkeV{\,\mathrm{keV}}
\def\SIMeV{\,\mathrm{MeV}}
\def\SIGeV{\,\mathrm{GeV}}
\def\SIcal{\,\mathrm{cal}}
\def\SIkcal{\,\mathrm{kcal}}
\def\SImol{\,\mathrm{mol}}
\def\SIN{\,\mathrm{N}}
\def\SIHz{\,\mathrm{Hz}}
\def\SIm{\,\mathrm{m}}
\def\SIcm{\,\mathrm{cm}}
\def\SIfm{\,\mathrm{fm}}
\def\SImm{\,\mathrm{mm}}
\def\SInm{\,\mathrm{nm}}
\def\SImum{\,\mathrm{\mu m}}
\def\SIJ{\,\mathrm{J}}
\def\SIW{\,\mathrm{W}}
\def\SIkJ{\,\mathrm{kJ}}
\def\SIs{\,\mathrm{s}}
\def\SIkg{\,\mathrm{kg}}
\def\SIg{\,\mathrm{g}}
\def\SIK{\,\mathrm{K}}
\def\SImmHg{\,\mathrm{mmHg}}
\def\SIPa{\,\mathrm{Pa}}

\def\courseurl{https://github.com/zqhuang/SYSU\_TD}

\def\tpage#1#2{
\begin{frame}
\begin{center}
\begin{Large}
\bch
热学 \\
第#1讲 #2

{\vskip 0.3in}

黄志琦

\ech
\end{Large}
\end{center}

\vskip 0.2in

\bch
教材:《热学》第二版,赵凯华,罗蔚茵,高等教育出版社
\ech

\bch
课件下载
\ech
\courseurl
\end{frame}
}

\def\bfr#1{
\begin{frame}
\chtitle{#1} 
\bch
}

\def\efr{
\ech 
\end{frame}
}

  \date{}
  \begin{document}
  \bch
\tpage{2}{Complex Numbers}

\begin{frame}
\frametitle{Outline}
\tableofcontents
\end{frame}

\section{Euler's Identity}

\secpage{欧拉公式}{$$e^{\ii \theta}=\cos\theta+\ii \sin\theta$$}

\begin{frame}
  \question{你知道欧拉公式吗?}

  \answer{$e^{\ii \theta}=\cos\theta+\ii \sin\theta$}

  \question{这是不是可以看作对$e^{\ii\theta}$的定义?}

   \ \ \ (言下之意:欧拉公式不是等于什么都没说吗!)
  
  \answer{年轻人,你想得太简单了!}
\end{frame}

\begin{frame}
  \frametitle{从实变函数到复变函数的推广}

  \bex
  实变函数
  $$f(x) = x^2, \ \ (x\in \Re)$$
  可以推广为复变函数
  $$f(z) = z^2. \ \ (z\in C) $$
  \eex

  
\end{frame}


\begin{frame}
  \frametitle{从实变函数到复变函数的推广}
  
  

  \bex
  实变函数
  $$f(x) = \frac{x}{1+x^2},\ \ (x\in \Re)$$
  可以推广为复变函数
  $$f(z) = \frac{z}{1+z^2}. \ \ (z\in C)$$
  \eex

  
\end{frame}



\begin{frame}
  

  \addfig{1.}{kaisen.jpg}
  \bcenter
      {\bf \Large 这种推广简直弱爆了!}

      \ecenter
  
\end{frame}


\begin{frame}
  \frametitle{那我们继续}
  
  实变函数
  $$f(x) = e^x,\ \ (x\in \Re)$$
  推广为复变函数
  $$f(z) = \ ? \ \ (z\in C)$$
  
\end{frame}

\begin{frame}
  
  \addfig{1.}{kaisen.jpg}
  \bcenter
  {\bf \Large 当然是$e^z$啊!}
  \ecenter
  
\end{frame}


\begin{frame}
  \frametitle{但,$e^z$到底是什么意思?}
  
  \addfig{1}{wocaidaan.png}
  \bcenter
  {\Large \bf  $e^{2+3\ii}$是什么意思?}
  \ecenter
  
\end{frame}


\begin{frame}
  \frametitle{转化为加减乘除的问题}
  
  实变的指数函数可以用“加减乘除”来替代:
  $$ e^x  = \sum_{n=0}^{\infty} \frac{x^n}{n!}$$
  由此自然而然地定义复变的指数函数
  $$ e^z = \sum_{n=0}^{\infty} \frac{z^n}{n!}$$
  
\end{frame}

\begin{frame}

  \frametitle{思考题}
  \addfig{1.2}{nimashinima.jpg}
  
  对推广到复数域的指数函数,证明``它是指数函数‘’。
  即对任意$z_1, z_2\in C$,有
  $$e^{z_1+z_2} = e^{z_1}e^{z_2}$$
  
\end{frame}



\begin{frame}
  \frametitle{Euler公式}
  
  对实数$\theta$,按定义
  $$e^{\ii\theta} = \sum_{n=0}^\infty \frac{\ii^n\theta^n}{n!} $$
  上式右边$n$为偶数的项是实数项,$n$为奇数的项是虚数项。分离实虚部即得:
  $$e^{\ii\theta} = \sum_{n=0}^\infty \frac{(-1)^n\theta^{2n}}{(2n)!} + \ii  \sum_{n=0}^\infty \frac{(-1)^n\theta^{2n+1}}{(2n+1)!}$$
  实部和虚部恰好分别是三角函数$\cos\theta$和$\sin\theta$的级数展开式,于是
  \tbox{\blue $$e^{\ii\theta} = \cos\theta + \ii\sin\theta$$}
  这就是著名的{\bf\blue 欧拉公式}。
  
\end{frame}

\begin{frame}
  \frametitle{复数的指数表示和复数的乘法规则}
  
      {\blue 模为$r$,幅角为$\theta$的复数可以写成$re^{\ii\theta}$。}
      \addfig{1.6}{exp_expression.png}
      
  对两个复数$r_1e^{\ii \theta_1}$和$r_2e^{\ii\theta_2}$,利用刚刚证明的$e^{z_1}e^{z_2} =e^{z_1+z_2}$,就有
  $$\left(r_1e^{\ii \theta_1}\right)\left(r_2e^{\ii\theta_2}\right) = (r_1r_2)\left(e^{\ii\theta_1}e^{\ii\theta_2}\right)= (r_1r_2)e^{\ii(\theta_1+\theta_2)}$$
      这就是模相乘,幅角相加的复数乘法法则。  
  
  
\end{frame}


\thinka{我们采用从零开始``C语言''计数习惯,定义一个$N\times N$的方阵$E$的第$m$行第$n$列($m,n$遍历$0,1,\ldots, N-1$)元素为
  $$E_{mn} = \frac{1}{\sqrt{N}}e^{- \frac{2\pi mn \ii }{N}}$$
    证明$E$是个酉矩阵。
}


\begin{frame}
  \frametitle{对数函数 $\ln z$ 是多值函数}
  
  设$z$的模为$r$,幅角为$\theta$,则 
  $$e^{\ln r + \ii \theta} = r e^{\ii\theta} = z$$
  那么,是否可以定义推广的对数函数:
  $${\blue \ln z := \ln |z| + \ii \arg z}$$
  ($|\cdot|$表示取模,$\arg$表示取幅角)
  
  不幸地是,因为$\arg z$可以随意加上$2\pi$的整数倍,所以上式是一对多的映射,不是普通意义上的函数。在复变函数论里把这样的映射称为{\bf\blue ``多值函数''}。

  \skipline
  
  {\blue $\ln z$的多值性直接导致了柯西积分公式、留数定理等}。我们稍后就会接触到。
\end{frame}

\section{Delta Function}

\secpage{又高又瘦的$\delta$函数}{$$\int_{-\infty}^{\infty} \delta(x-x_0)f(x) \,dx = f(x_0)$$}

\begin{frame}
  \frametitle{物理里的理想化模型}
  
  物理里有很多“无穷大$\times$无穷小 = 有限“的模型:
  \bitem
\item{瞬时冲量: 力无限大,作用时间无穷短,但两者的乘积(冲量)是有限的。}
\item{质点:质量密度无穷大,体积无穷小,但两者的乘积(总质量)是有限的。}
\item{点电荷:电荷密度无穷大,体积无穷小,但两者的乘积(总电荷)是有限的。}
  \eitem

  \skiplines
  
  在数学上这些表述都是不合法的,需要搞很多事情才能把这些模型说清楚。 不喜欢搞事情的物理学家们于是发明了{\blue $\delta$函数}。
  
\end{frame}

\begin{frame}
  \frametitle{重量级dalao —— Paul Dirac (狄拉克)}
  
  \bcenter
  \lfig{1.2}{Dirac.jpg}

  你们不要搞事情
  \ecenter
  
\end{frame}
  
\begin{frame}
  \frametitle{Dirac $\delta$ function}
  
  \bmini{0.4}
  \lfig{1.5}{DiracDelta.png}
  \emini
  \bmini{0.55}
  Dirac $\delta$ function 不是传统意义上的函数。它可以通过下面的{\blue 单位脉冲函数取脉冲时间为零的极限}得到:
  $$\delta_D(x) = \left\{\begin{array}{ll} \frac{1}{\epsilon}, & \text{ if } -\frac{\epsilon}{2}<x<\frac{\epsilon}{2} \\  0, & \text{ else}\end{array}\right. $$
  其中$\epsilon\rightarrow 0^+$。  
  \emini

  在本课程中,我们{\blue 简称Dirac $\delta$ function为$\delta$函数,并简写为$\delta(x)$}。
  
\end{frame}


\begin{frame}
  \frametitle{$\delta$函数的另一种逼近方式}
  
  有时候需要计算$\delta$函数的导数甚至高阶导数,这时可以考虑用高斯函数逼近方式:

  $$\delta(x) = \frac{1}{\sqrt{2\pi\epsilon}}e^{-\frac{x^2}{2\epsilon}},$$
  其中$\epsilon\rightarrow 0^+$。

%  {\scriptsize 注:这一般只是为了帮助理解$\delta$函数的导数的图像,并非为了计算。如果要用具体的逼近方式来进行计算,$\delta$函数的便捷性就大打折扣了。}
  
\end{frame}


\begin{frame}
  \frametitle{$\delta$函数的抽象定义}
  
  借助上述两种逼近方式的辅助,我们归纳出$\delta$函数的下述抽象定义:
  {\blue
    \be
    \delta(x) = \left\{
    \begin{array}{ll}
      0, & \text{ if } x\ne 0; \\
      +\infty, & \text{ if } x = 0;
    \end{array}\right.
    \ee
    $$ \int_{-\infty}^\infty \delta(x)\,dx = 1.$$ 
  }
  从上述抽象定义中可以看出$\delta$函数是偶函数:
  $$\delta(-x) = \delta(x)$$
  此外,积分的范围可以限定在$0$的任意小领域。
    $$ \int_{0^-}^{0^+} \delta(x)\,dx = 1.$$   
  
\end{frame}

\begin{frame}
  \frametitle{$\delta$函数最重要的性质}
  
  从被使用频率上来讲,下式至关重要:
  \tbox{
    $$\int_{-\infty}^\infty \delta(x-x_0) f(x)\, dx = f(x_0).$$}

  (请自行用物理图像理解上式)
  
\end{frame}

\begin{frame}
  \frametitle{$n$维空间的$\delta$函数}
  
  $n$维$\delta$函数标记为{\blue $\delta^{(n)}(\vecx)$},既可以理解为
      {\blue
        $$\delta^{(n)}(\vecx) = \delta(x_1)\delta(x_2)\ldots\delta(x_n) $$
      }
      (其中$(x_1,x_2,\ldots, x_n)$是$\vecx$的坐标分量),
        
      也可以直接抽象地理解为{\blue 在原点附近体积为$\epsilon \rightarrow 0^+$的邻域内,函数值为$\frac{1}{\epsilon}\rightarrow \infty $,而在其余位置函数值均为零的函数。}

      \skiplines
      
\bex
三维空间某点$\vecx'$处的点电荷的电荷密度可以写成
$$\rho(\vecx) = Q\, \delta^{(3)}(\vecx - \vecx')$$
\eex

\end{frame}

\begin{frame}
  \frametitle{$n$维空间的$\delta$函数的性质}
  
  和一维空间类似,
  \tbox{
   $$ \int \delta^{(n)}(\vecx - \vecx_0) f(\vecx)\, d^n\vecx = f(\vecx_0)$$
   }
  这里的$\int d^n\vecx$是$n$重积分$\int_{-\infty}^\infty dx_1\int_{-\infty}^\infty dx_2\ldots\int_{-\infty}^\infty dx_n$的简写,物理文献中经常碰到这种写法。
\end{frame}


\begin{frame}
  \frametitle{思考题}
  原点处的静电势可以用积分
  $$U = \frac{1}{4\pi\epsilon_0} \int \frac{\rho(\vecx)}{|\vecx|} d^3\vecx, $$
  来计算。

  \skipline
  
  由此计算在$\vecx_0$处点电荷$Q$在原点产生的电势。
\end{frame}



\section{Fourier Transform}

\secpage{傅立叶变换}{$$\widetilde{f}(k) = \frac{1}{\sqrt{2\pi}}\int_{-\infty}^{\infty}e^{-\ii kx}f(x) dx$$}


\begin{frame}
  \frametitle{一维傅立叶变换的定义}
傅立叶变换把``$x$空间''的函数$f(x)$变换到``k空间''的函数:
  \tbox{  $$ \widetilde{f}(k) \equiv \frac{1}{\sqrt{2\pi}} \int_{-\infty}^\infty e^{-ikx} f(x) dx $$}

  \bitem
  \item{注意$x$和$k$都是实参量;但$f(x)$和$\widetilde{f}(k)$的取值可以为复数。}
\item{在实际物理问题中,$x$空间和$k$空间(也叫傅立叶空间)往往对应明确的物理空间。例如在量子力学中,$x$空间对应坐标空间,$k$空间对应动量空间;在频谱分析中,$x$空间对应时间轴,$k$空间对应频率空间。}
\item{注意在有些文献中没有$\frac{1}{\sqrt{2\pi}}$这个因子,很多公式的系数会有所区别。}
  \eitem
\end{frame}

\thinka{设$f(x)$在区间$[-1,1]$上取值为$1$,其余处处为零。计算$f(x)$的傅立叶变换。}

\begin{frame}
  \frametitle{傅立叶变换的逆变换}
  请阅读附录B理解傅立叶变换本质上是一个酉变换(复向量的旋转)。

  附录B的离散傅立叶变换的逆变换过渡到连续情况就是:
  \tbox{  $$ f(x) = \frac{1}{\sqrt{2\pi}} \int_{-\infty}^\infty e^{ikx} \widetilde{f}(k) dk $$}
\end{frame}

\begin{frame}
  \frametitle{$n$维空间的傅立叶变换}
  
  $n$维空间的函数$f(\vecx)$的傅立叶变换就是对每个维度都进行一维傅立叶变换:
  
  $$ \widetilde{f}(\veck) = \frac{1}{(2\pi)^{n/2}}\int f(\vecx) e^{-\ii \veck \cdot \vecx} d^n\vecx $$

  这里的$\veck\cdot\vecx = k_1x_1+k_2x_2+\ldots+k_nx_n$.
  
  
  显然,其逆变换为
  
  $$ f(\vecx) = \frac{1}{(2\pi)^{n/2}}\int \widetilde{f}(\veck) e^{\ii \veck \cdot \vecx} d^n\veck. $$

\end{frame}



\begin{frame}
  \frametitle{$\delta(x)$的积分表示}
  通过把$\delta(x)$傅立叶变换再求逆变换,可以得到:
  
  \tbox{$$\frac{1}{2\pi}\int_{-\infty}^\infty e^{ikx}dk = \delta(x).$$}

  这个结论连续运用$n$次,可以把结果推广到$n$维空间:
  \tbox{
   $$ \frac{1}{(2\pi)^n} \int e^{i\veck\cdot\vecx} d^n\veck =  \delta^{(n)}(\vecx)$$
   }
  
  在大量的物理问题中会用到它。
\end{frame}

\begin{frame}
  \frametitle{傅立叶变换保内积不变}

  设$f(x)$, $g(x)$的傅立叶变换分别为$\widetilde{f}(k)$和$\widetilde{g}(k)$,则

  $$\int_{-\infty}^\infty f^\dagger(x) g(x) dx =  \int_{-\infty}^\infty \widetilde{f}^\dagger(k) \widetilde{g}(k) dk$$
  
  证明:由于傅立叶变换是酉变换(见附录B),所以保内积不变。
  
\end{frame}

\thinka{计算积分$$\int_0^\infty \left(\frac{\sin x}{x}\right)^2dx.$$}

\section{Homework}

\begin{frame}
  \frametitle{Homework}
  \bitem
  \item{仿照复数域上指数函数$e^z$的定义方法,用全复平面收敛的幂级数把正弦和余弦函数推广到复数域:
  $$\sin z \equiv \sum_{n=0}^\infty \frac{(-1)^nz^{2n+1}}{(2n+1)!};\ \ \ \cos z \equiv \sum_{n=0}^\infty \frac{(-1)^nz^{2n}}{(2n)!}.$$
  \bitem
  \item[(1)]{ 用指数函数来表示正弦和余弦函数;}
  \item[(2)]{ 求出满足$\sin z = 0$的全部复数解$z$。}
    \eitem}
\item{课堂上用傅立叶变换的逆变换证明了$\delta$函数的积分表示,请反过来用$\delta$函数的积分表示证明傅立叶变换的逆变换。}
\item{用$\delta$函数的积分表示证明傅立叶变换的保内积性。}
\item{计算积分$$\int_0^\infty \frac{\sin x}{x} dx.$$}    
  \eitem
\end{frame}


\section{Appendices}

\append{A}{``指数函数是指数函数''的证明}

\begin{frame}
  \frametitle{$e^{z_1+z_2} = e^{z_1}e^{z_2}$的证明}
  
  按照$e^z$的定义,命题等价于
  $$\sum_{n=0}^\infty \frac{(z_1+z_2)^n}{n!} = \sum_{n_1, n_2\ge 0}\frac{z_1^{n_1}z_2^{n_2}}{n_1!n_2!}$$

  对任意$n_1,n_2\ge0$,我们来比较两边$z_1^{n_1}z_2^{n_2}$的项的系数:
  
  左边$z_1^{n_1}z_2^{n_2}$的项只能来自于$n=n_1+n_2$的项的展开,在展开$(z_1+z_2)^n$时在$n_1$个括号内取$z_1$,$n_2$个括号内取$z_2$,总共有$\frac{n!}{n_1!n_2!}$种取法,即左边$z_1^{n_1}z_2^{n_2}$的系数为
  $$\frac{1}{n!} \frac{n!}{n_1!n_2!} = \frac{1}{n_1!n_2!}$$
  和右边相同。
  
\end{frame}


\append{B}{傅立叶变换的本质是复向量的旋转}

\begin{frame}
  \frametitle{一维有限格点世界}
  $\delta(x-x_0)$函数在$x_0$处的取值是$\infty$,这件事多多少少有点让人不爽。有没有更好的描述$\delta$函数的办法呢?

  具体的物理问题总是局限在有限的范围内,对位置的测量精度也总是有限的。因此我们假想$x$在一根长度为$L$($L$非常非常大)的线段上,且$x$的位置的测量精度为$dx=L/N$(这里我们稍稍改变了$dx$的含义,把它看成一个很小很小,但是有限的长度)。$x$的取值局限为$x_0=0, x_1=dx, x_2=2dx, \ldots, x_{N-1}=(N-1)dx$中的一个。

  {\blue 函数$f(x)$退化为一个$N$维空间矢量
    $$ \vecf = \left( f(x_0) \sqrt{dx}, f(x_1) \sqrt{dx},\ldots, f(x_{N-1})\sqrt{dx} \right).$$}
    这里的$\sqrt{dx}$的因子看起来有些奇怪,其实去掉也无妨,留着它的好处是可以把很多函数的术语和线性代数的术语统一起来。

\end{frame}


\begin{frame}
  \frametitle{积分的离散化形式}

  在有限格点空间里,积分也被离散化了:

  $$\int_{-\infty}^{\infty} f(x) dx \rightarrow \sum_{i=0}^{N-1} f(x_i) dx = \sqrt{dx} \sum_{i=0}^{N-1} \vecf_{i} $$
\end{frame}


\begin{frame}
  \frametitle{内积}
  两个函数$f(x)$, $g(x)$的内积就可以和矢量内积统一起来:
  $$ \int_{-\infty}^\infty f^\dagger(x) g(x) dx \rightarrow \sum_{i=0}^{N-1} f^\dagger(x_i)g(x_i) dx  = \vecf^\dagger\vecg. $$
  注意我们把$dx$拆成两个$\sqrt{dx}$并分别吸收到矢量$\vecf$, $\vecg$的定义中去了。
\end{frame}

\begin{frame}
  \frametitle{$\delta$函数}
  在任意一点$x_i$处的$\delta$脉冲也变得非常简明:
  $$\delta(x-x_i) = \left\{\begin{array}{ll}\frac{1}{dx}, &\text{ if }x=x_i \\
  0, & \text{ else}\end{array}\right.$$
  至少,在我们的游戏规则里$\frac{1}{dx}$是有限的数!
\end{frame}

\begin{frame}
  \frametitle{旋转}
  还记得我们讨论过一个$N\times N$的酉矩阵$E$吗?
  $$E_{mn} = \frac{1}{\sqrt{N}}e^{-\frac{2\pi mn \ii }{N}}$$
  用这个酉矩阵把函数(向量)$\vecf$旋转一下会发生什么呢?
  $$ \widetilde{\vecf} \equiv E\vecf$$
  的第$m$个分量是:
  $$ \widetilde{\vecf}_m = \sum_{n=0}^{N-1} \frac{1}{\sqrt{N}}e^{-\frac{2\pi mn \ii }{N}} f(x_n)\sqrt{dx}. $$
\end{frame}


\begin{frame}
  \frametitle{对偶空间}
  把$\widetilde{\vecf}_m$看成一个和原空间互为对偶的有限格点空间里的函数(矢量);为了区别于原来的有限格点空间,我们用$k$来表示这个对偶空间里的坐标。这两个空间的格点数$N$是相同的,
  而且满足 {\blue $dk\, dx = \frac{2\pi}{N}$} (怎么看着有点像量子力学的不确定性原理)。

      {\scriptsize (如果你熟悉傅立叶分析,也许你更喜欢把它写成$dk = \frac{2\pi}{N\,dx}=\frac{2\pi}{L}$.)}

  \skipline
      
  在$k$空间里我们关注的格点为$k_0=0$, $k_1=dk = \frac{2\pi}{N\,dx}$, $k_2=2dk=\frac{4\pi}{N\,dx}$, \ldots

  容易验证$\frac{1}{\sqrt{N}} = \frac{\sqrt{dk\, dx}}{\sqrt{2\pi}}$, $\frac{2\pi mn}{N} = k_mx_n $, 于是
  $$\widetilde{\vecf}_m = \sqrt{dk} \, \left(\frac{1}{\sqrt{2\pi}}\sum_{i=0}^{N-1} e^{-\ii k_m x_n} f(x_n) dx \right)$$

\end{frame}

\begin{frame}
  \frametitle{离散傅立叶变换和逆变换}
  也就是说,矢量$\widetilde{\vecf}$对应于$k$空间的函数:
  $$ \widetilde{f}(k_m)= \frac{1}{\sqrt{2\pi}}\sum_{i=0}^{N-1} e^{-\ii k_m x_n} f(x_n) dx. $$
  利用$E^\dagger E = I$,很容易反过来得到$\vecf = E^\dagger \widetilde{\vecf}$。
  经过几乎是一模一样的推理(除了把$-\ii$换成了$\ii$),得到:
  $$f(x_n) = \frac{1}{\sqrt{2\pi}} \sum_{i=0}^{N-1}e^{\ii k_m x_n} \widetilde{f}(k_m) dk .$$
  上面的$f\rightarrow \widetilde{f}$的操作就是离散傅立叶变换;反过来的操作就是离散傅立叶逆变换。它们的本质是对复向量的一种旋转和相应的逆旋转。
\end{frame}

\ech
\end{document}
