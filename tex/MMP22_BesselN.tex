\documentclass[CJK]{beamer}
\usepackage{CJKutf8}
\usepackage{beamerthemesplit}
\usetheme{Malmoe}
\useoutertheme[footline=authortitle]{miniframes}
\usepackage{amsmath}
\usepackage{amssymb}
\usepackage{graphicx}
\usepackage{eufrak}
\usepackage{color}
\usepackage{slashed}
\usepackage{simplewick}
\usepackage{tikz}
\usepackage{tcolorbox}
\graphicspath{{../figures/}}
%%figures
\def\lfig#1#2{\includegraphics[width=#1 in]{#2}}
\def\addfig#1#2{\begin{center}\includegraphics[width=#1 in]{#2}\end{center}}
\def\wulian{\includegraphics[width=0.18in]{emoji_wulian.jpg}}
\def\bigwulian{\includegraphics[width=0.35in]{emoji_wulian.jpg}}
\def\bye{\includegraphics[width=0.18in]{emoji_bye.jpg}}
\def\bigbye{\includegraphics[width=0.35in]{emoji_bye.jpg}}
\def\huaixiao{\includegraphics[width=0.18in]{emoji_huaixiao.jpg}}
\def\bighuaixiao{\includegraphics[width=0.35in]{emoji_huaixiao.jpg}}
\def\jianxiao{\includegraphics[width=0.18in]{emoji_jianxiao.jpg}}
\def\bigjianxiao{\includegraphics[width=0.35in]{emoji_jianxiao.jpg}}
%% colors
\def\blacktext#1{{\color{black}#1}}
\def\bluetext#1{{\color{blue}#1}}
\def\redtext#1{{\color{red}#1}}
\def\darkbluetext#1{{\color[rgb]{0,0.2,0.6}#1}}
\def\skybluetext#1{{\color[rgb]{0.2,0.7,1.}#1}}
\def\cyantext#1{{\color[rgb]{0.,0.5,0.5}#1}}
\def\greentext#1{{\color[rgb]{0,0.7,0.1}#1}}
\def\darkgray{\color[rgb]{0.2,0.2,0.2}}
\def\lightgray{\color[rgb]{0.6,0.6,0.6}}
\def\gray{\color[rgb]{0.4,0.4,0.4}}
\def\blue{\color{blue}}
\def\red{\color{red}}
\def\green{\color{green}}
\def\darkgreen{\color[rgb]{0,0.4,0.1}}
\def\darkblue{\color[rgb]{0,0.2,0.6}}
\def\skyblue{\color[rgb]{0.2,0.7,1.}}
%%control
\def\be{\begin{equation}}
\def\ee{\nonumber\end{equation}}
\def\bea{\begin{eqnarray}}
\def\eea{\nonumber\end{eqnarray}}
\def\bch{\begin{CJK}{UTF8}{gbsn}}
\def\ech{\end{CJK}}
\def\bitem{\begin{itemize}}
\def\eitem{\end{itemize}}
\def\bcenter{\begin{center}}
\def\ecenter{\end{center}}
\def\bex{\begin{minipage}{0.2\textwidth}\includegraphics[width=0.6in]{jugelizi.png}\end{minipage}\begin{minipage}{0.76\textwidth}}
\def\eex{\end{minipage}}
\def\chtitle#1{\frametitle{\bch#1\ech}}
\def\bmat#1{\left(\begin{array}{#1}}
\def\emat{\end{array}\right)}
\def\bcase#1{\left\{\begin{array}{#1}}
\def\ecase{\end{array}\right.}
\def\bmini#1{\begin{minipage}{#1\textwidth}}
\def\emini{\end{minipage}}
\def\tbox#1{\begin{tcolorbox}#1\end{tcolorbox}}
\def\pfrac#1#2#3{\left(\frac{\partial #1}{\partial #2}\right)_{#3}}
%%symbols
\def\bropt{\,(\ \ \ )}
\def\sone{$\star$}
\def\stwo{$\star\star$}
\def\sthree{$\star\star\star$}
\def\sfour{$\star\star\star\star$}
\def\sfive{$\star\star\star\star\star$}
\def\rint{{\int_\leftrightarrow}}
\def\roint{{\oint_\leftrightarrow}}
\def\stdHf{{\textit{\r H}_f}}
\def\deltaH{{\Delta \textit{\r H}}}
\def\ii{{\dot{\imath}}}
\def\skipline{{\vskip0.1in}}
\def\skiplines{{\vskip0.2in}}
\def\lagr{{\mathcal{L}}}
\def\hamil{{\mathcal{H}}}
\def\vecv{{\mathbf{v}}}
\def\vecx{{\mathbf{x}}}
\def\vecy{{\mathbf{y}}}
\def\veck{{\mathbf{k}}}
\def\vecp{{\mathbf{p}}}
\def\vecn{{\mathbf{n}}}
\def\vecA{{\mathbf{A}}}
\def\vecP{{\mathbf{P}}}
\def\vecsigma{{\mathbf{\sigma}}}
\def\hatJn{{\hat{J_\vecn}}}
\def\hatJx{{\hat{J_x}}}
\def\hatJy{{\hat{J_y}}}
\def\hatJz{{\hat{J_z}}}
\def\hatj#1{\hat{J_{#1}}}
\def\hatphi{{\hat{\phi}}}
\def\hatq{{\hat{q}}}
\def\hatpi{{\hat{\pi}}}
\def\vel{\upsilon}
\def\Dint{{\mathcal{D}}}
\def\adag{{\hat{a}^\dagger}}
\def\bdag{{\hat{b}^\dagger}}
\def\cdag{{\hat{c}^\dagger}}
\def\ddag{{\hat{d}^\dagger}}
\def\hata{{\hat{a}}}
\def\hatb{{\hat{b}}}
\def\hatc{{\hat{c}}}
\def\hatd{{\hat{d}}}
\def\hatN{{\hat{N}}}
\def\hatH{{\hat{H}}}
\def\hatp{{\hat{p}}}
\def\Fup{{F^{\mu\nu}}}
\def\Fdown{{F_{\mu\nu}}}
\def\newl{\nonumber \\}
\def\vece{\mathrm{e}}
\def\calM{{\mathcal{M}}}
\def\calT{{\mathcal{T}}}
\def\calR{{\mathcal{R}}}
\def\barpsi{\bar{\psi}}
\def\baru{\bar{u}}
\def\barv{\bar{\upsilon}}
\def\qeq{\stackrel{?}{=}}
\def\torder#1{\mathcal{T}\left(#1\right)}
\def\rorder#1{\mathcal{R}\left(#1\right)}
\def\contr#1#2{\contraction{}{#1}{}{#2}#1#2}
\def\trof#1{\mathrm{Tr}\left(#1\right)}
\def\trace{\mathrm{Tr}}
\def\comm#1{\ \ \ \left(\mathrm{used}\ #1\right)}
\def\tcomm#1{\ \ \ (\text{#1})}
\def\slp{\slashed{p}}
\def\slk{\slashed{k}}
\def\calp{{\mathfrak{p}}}
\def\veccalp{\mathbf{\mathfrak{p}}}
\def\Tthree{T_{\tiny \textcircled{3}}}
\def\pthree{p_{\tiny \textcircled{3}}}
\def\dbar{{\,\mathchar'26\mkern-12mu d}}
\def\erf{\mathrm{erf}}
\def\const{\mathrm{constant}}
\def\pheat{\pfrac p{\ln T}V}
\def\vheat{\pfrac V{\ln T}p}
%%units
\def\fdeg{{^\circ \mathrm{F}}}
\def\cdeg{^\circ \mathrm{C}}
\def\atm{\,\mathrm{atm}}
\def\angstrom{\,\text{\AA}}
\def\SIL{\,\mathrm{L}}
\def\SIkm{\,\mathrm{km}}
\def\SIyr{\,\mathrm{yr}}
\def\SIGyr{\,\mathrm{Gyr}}
\def\SIV{\,\mathrm{V}}
\def\SImV{\,\mathrm{mV}}
\def\SIeV{\,\mathrm{eV}}
\def\SIkeV{\,\mathrm{keV}}
\def\SIMeV{\,\mathrm{MeV}}
\def\SIGeV{\,\mathrm{GeV}}
\def\SIcal{\,\mathrm{cal}}
\def\SIkcal{\,\mathrm{kcal}}
\def\SImol{\,\mathrm{mol}}
\def\SIN{\,\mathrm{N}}
\def\SIHz{\,\mathrm{Hz}}
\def\SIm{\,\mathrm{m}}
\def\SIcm{\,\mathrm{cm}}
\def\SIfm{\,\mathrm{fm}}
\def\SImm{\,\mathrm{mm}}
\def\SInm{\,\mathrm{nm}}
\def\SImum{\,\mathrm{\mu m}}
\def\SIJ{\,\mathrm{J}}
\def\SIW{\,\mathrm{W}}
\def\SIkJ{\,\mathrm{kJ}}
\def\SIs{\,\mathrm{s}}
\def\SIkg{\,\mathrm{kg}}
\def\SIg{\,\mathrm{g}}
\def\SIK{\,\mathrm{K}}
\def\SImmHg{\,\mathrm{mmHg}}
\def\SIPa{\,\mathrm{Pa}}

\def\courseurl{https://github.com/zqhuang/SYSU\_TD}

\def\tpage#1#2{
\begin{frame}
\begin{center}
\begin{Large}
\bch
热学 \\
第#1讲 #2

{\vskip 0.3in}

黄志琦

\ech
\end{Large}
\end{center}

\vskip 0.2in

\bch
教材:《热学》第二版,赵凯华,罗蔚茵,高等教育出版社
\ech

\bch
课件下载
\ech
\courseurl
\end{frame}
}

\def\bfr#1{
\begin{frame}
\chtitle{#1} 
\bch
}

\def\efr{
\ech 
\end{frame}
}

  \date{}
  \begin{document}
  \bch
\tpage{22}{第二类贝塞尔函数}


\begin{frame}
  \frametitle{极坐标下分离变量形式的谐函数}
对固定$k>0$,在原点不发散的极坐标下分离变量形式的谐函数为
$$ J_m(kr)\cos(m\theta), \ J_m(kr)\sin(m\theta), \ \ (m=0,1,2,\ldots) $$
它们实际上是直角坐标系下分离变量形式的谐函数,也就是一堆平面波$e^{\ii \veck \cdot \vecx}$ ($|\veck|=k$)的重新线性组合。
\end{frame}


\begin{frame}
  \frametitle{$k=0$的情形}
  我们再回头考虑之前被忽略掉的一种比较特殊的情形:$k=0$。这时候可以取$k\rightarrow 0^+$的极限:$ J_m(kr)\sim  r^m$。

  所以对$k=0$,在原点不发散的极坐标下分离变量形式的谐函数为:
  $$ r^m \cos(m\theta), \ r^m \sin(m\theta), \ \ (m=0,1,2,\ldots)$$

  {\scriptsize 注意:$m=0$的解恰好对应了直角坐标系下$k=0$的解,而$m>0$的解在无穷远处发散,是直角坐标下平面波所不能描述的。}
\end{frame}


\begin{frame}
  \frametitle{允许原点处发散的谐函数}
  如果考虑的是挖去原点的二维平面,则会有一族新的成员加入谐函数的大家庭:
  \bitem
\item{对$k>0$,
  $$N_m(kr)\cos(m\theta), \ N_m(kr)\sin(m\theta), \ \ (m=0,1,2,\ldots) $$}
\item{对$k=0$,
  $$r^{-m} \cos(m\theta), \ r^{-m} \sin(m\theta), \ \ (m=0,1,2,\ldots)$$}
  \eitem

  本讲的重点就是要介绍这里的第二类贝塞尔函数$N_m$。
\end{frame}


\begin{frame}
\frametitle{本讲内容}

\tableofcontents
\end{frame}

\section{Bessel Functions of Fractional Order}
\secpage{任意阶第一类贝塞尔函数}{非整数阶$J_{-\nu}$和$J_\nu$线性独立,整数阶满足$J_{-m}(x) = (-1)^m J_m(x)$}

\begin{frame}
\frametitle{任意阶贝塞尔函数}

直接把$m$推广到任意实数$\nu$,
\tbox
    {$$J_\nu(x) = \sum_{k=0}^\infty \frac{(-1)^k}{k!(k+\nu)!} \left(\frac{x}{2}\right)^{2k+\nu}.$$}
    定义对所有$x>0$有效。当$\nu \ge 0$,还可以取$x \rightarrow 0$的极限。


    
    \skiplines
    \bitem
  \item{当$\nu$不是整数时,$(k+\nu)!$要理解为$\Gamma(k+\nu+1)$。}
  \item{$\frac{1}{\Gamma(z)}$的零点为$0,-1,-2,\ldots$,或者说负整数的阶乘是无穷大。所以上面的级数也可以写成$k$从$-\infty$到$\infty$求和}
  \item{如果你乐意,还可以把定义解析延拓到$-\pi<\arg x<\pi$。}
    \eitem

\end{frame}


\begin{frame}
\frametitle{贝塞尔函数的性质}

递推公式对所有$\nu$成立:{\blue
  $$ \frac{d}{dx}\left[x^\nu J_\nu(x)\right] = x^\nu J_{\nu -1}(x);\ \frac{d}{dx}\left[x^{-\nu} J_\nu(x)\right] = -x^{-\nu} J_{\nu +1}(x);  $$

$$ J_{\nu-1}(x)-J_{\nu+1}(x) =2J_\nu'(x) ;\  J_{\nu-1}(x)+J_{\nu+1}(x) = \frac{2\nu}{x}J_\nu(x).$$}



渐近公式对所有$\nu$都成立:{\blue $x\gg \nu^2$时
$$ J_\nu(x)\approx \sqrt{\frac{2}{\pi x}}\cos{\left(x - \frac{\nu \pi}{2} - \frac{\pi}{4}\right)} $$}

另外,{\blue 三个正交定理对$\nu \ge 0 $都成立}。


\end{frame}



\begin{frame}
\frametitle{整数阶$J_{-m}(x)$和$J_m(x)$的线性关系}

设$m\ge 0$为整数,利用负整数的阶乘为$\infty$,有
\bea
J_{-m}(x) &=& \sum_{k=0}^\infty \frac{(-1)^k}{k!(k-m)!} \left(\frac{x}{2}\right)^{2k-m} \newl
&=&  \sum_{k=m}^\infty  \frac{(-1)^k}{k!(k-m)!} \left(\frac{x}{2}\right)^{2k-m} \newl
&=&  \sum_{n=0}^\infty  \frac{(-1)^{n+m}}{(n+m)!n!} \left(\frac{x}{2}\right)^{2(n+m)-m} \newl
&=&  (-1)^mJ_m(x)
\eea

\end{frame}

\begin{frame}
\frametitle{非整数阶$J_\nu$和$J_{-\nu}$线性独立}

当$x\rightarrow \infty$时
$$J_\nu(x) \approx \sqrt{\frac{2}{\pi x}}\cos{\left(x-\frac{\nu \pi}{2}-\frac{\pi}{4}\right)}.$$
$$J_{-\nu}(x) \approx \sqrt{\frac{2}{\pi x}}\cos{\left(x+\frac{\nu \pi}{2}-\frac{\pi}{4}\right)}.$$
两者的相位差$\nu\pi$,$\nu$不是整数时,无法通过线性叠加完全消失。

\skiplines

这个结论也能从$x\rightarrow 0^+$时的渐近行为得到,请自行研究。

\end{frame}


\section{Bessel Functions of the 2$^{\rm nd}$ Kind}
\secpage{第二类贝塞尔函数}{和第一类贝塞尔函数渐近正交,大多数性质相同,除了在$x\rightarrow 0^+$时发散}

\begin{frame}
\frametitle{贝塞尔方程的线性独立解}

设$\nu >0$为非整数,贝塞尔方程
$$f''+\frac{1}{x}f' + \left(1-\frac{\nu^2}{x^2}\right)f = 0$$
有两个线性独立的解:$J_\nu(x)$和$J_{-\nu}(x)$。

\skiplines

这无法推广到$\nu$为整数的情形。我们来思考其背后的原因。

\end{frame}

\begin{frame}
\frametitle{$J_\nu$和$J_{-\nu}$之间的“独立性”随着$\nu$逼近整数而趋向于消失}

如果考虑一下$J_\nu$和$J_{-\nu}$的渐近展开:
$$J_\nu(x) \approx \sqrt{\frac{2}{\pi x}}\cos{\left(x-\frac{\nu \pi}{2}-\frac{\pi}{4}\right)}.$$
$$J_{-\nu}(x) \approx \sqrt{\frac{2}{\pi x}}\cos{\left(x+\frac{\nu \pi}{2}-\frac{\pi}{4}\right)}.$$
当$\nu$趋向于一个整数,两者的“独立性”趋向于消失。

\skipline

因为$J_\nu$和$J_{-\nu}$之间的独立性依赖于$\nu$,取它们作为两个独立解并不合适。最合理的方法是取$J_\nu$和另一个渐近行为为
$$ \sim \sqrt{\frac{2}{\pi x}}\sin{\left(x-\frac{\nu \pi}{2}-\frac{\pi}{4}\right)}$$
的解。这样两个解就类似于余弦和正弦一样“正交”了。

\end{frame}

\begin{frame}
\frametitle{第二类贝塞尔函数}

利用
{\small
$$ \cos{\left(x+\frac{\nu \pi}{2}-\frac{\pi}{4}\right)} = \cos{\left(x-\frac{\nu \pi}{2}-\frac{\pi}{4}\right)}\cos{(\nu\pi)} -  \sin{\left(x-\frac{\nu \pi}{2}-\frac{\pi}{4}\right)}\sin{(\nu\pi)} $$}
只要定义
$$N_\nu(x) = \frac{J_\nu(x)\cos{(\nu\pi)} - J_{-\nu}(x)}{\sin{(\nu\pi)}},$$
则$N_\nu(x)$满足我们期待的渐近公式:
\tbox{
$$ N_\nu(x) \approx \sqrt{\frac{2}{\pi x}}\sin{\left(x-\frac{\nu \pi}{2}-\frac{\pi}{4}\right)},\ \ \ x\gg \nu^2$$}

{\blue 注意:在很多文献上把第二类贝塞尔函数写成$Y_\nu$,仅是符号的差别。}
\end{frame}

\begin{frame}
\frametitle{整数阶$N_m(x)$}

为了对整数阶也适用,我们把定义修改为极限\tbox{
  $$N_\nu(x) := \lim_{\alpha \rightarrow \nu } \frac{J_\alpha(x)\cos{(\alpha\pi)} - J_{-\alpha}(x)}{\sin{(\alpha\pi)}}$$}

\skiplines

因为$N_\nu$和$J_\nu$渐近正交,不可能线性相关。所以
{\blue
对任意$\nu $,$J_\nu$和$N_\nu$都是贝塞尔方程的两个线性独立解。}

\end{frame}


\begin{frame}
  \frametitle{贝塞尔方程的推广:类贝塞尔方程}
  \tbox{定理:类贝塞尔方程
  $$ y''+\frac{1-2\alpha}{x} y' + \left(\beta^2\gamma^2x^{2\gamma-2}+\frac{\alpha^2-\nu^2\gamma^2}{x^2}\right)y=0 $$
  在 $x>0$ 范围内有两个线性无关解: $x^{\alpha}J_\nu(\beta x^\gamma)$和$x^{\alpha}N_\nu(\beta x^\gamma)$.}

  显然,上述定理$\alpha=0$, $\beta = 1$, $\gamma=1$时就退化为贝塞尔方程的问题。
\end{frame}


\begin{frame}
  \frametitle{思考题}
  \addfig{0.5}{think.jpg}
  
  你能想办法证明前述关于类贝塞尔方程的解的定理吗?
\end{frame}


\begin{frame}
\frametitle{$N_0, N_1, N_2$}

\addfig{3}{Y012.png}

\end{frame}


\begin{frame}
\frametitle{$N_\nu$的定性图像}

对$\nu\ge 0$, $N_\nu$和$J_\nu$都是振幅以$\sim \frac{1}{\sqrt{x}}$方式衰减,周期近似为$2\pi$的振荡函数。区别是$x\rightarrow 0^+$时的行为。

如果$\nu>0$,当$x\rightarrow 0^+$时
$$J_\nu(x)\approx \frac{1}{\nu !}\left(\frac{x}{2}\right)^\nu,\ \ N_\nu(x) \approx -\frac{\Gamma(\nu)}{\pi}\left(\frac{x}{2}\right)^{-\nu}. $$

\skiplines

如果$\nu = 0$,当$x\rightarrow 0^+$时
$$J_0(x)\approx 1 , \ \ N_0(x) \approx \frac{2}{\pi}\ln x $$

证明非常简单,留为练习

{\blue 因为$N_\nu(kr)$在$r\rightarrow 0$时发散,当且仅当考虑的区域为去心的环形区域时才需要考虑谐函数: $N_m(kr)e^{\pm \ii m\theta}$.}

\end{frame}

\begin{frame}
\frametitle{$N_\nu$的递推关系}

$N_\nu$和$J_\nu$一样满足递推关系:
{\blue
  $$ \frac{d}{dx}\left[x^\nu N_\nu(x)\right] = x^\nu N_{\nu -1}(x);\ \frac{d}{dx}\left[x^{-\nu} N_\nu(x)\right] = -x^{-\nu} N_{\nu +1}(x);  $$

  $$ N_{\nu-1}(x)-N_{\nu+1}(x) =2N_\nu'(x) ;\  N_{\nu-1}(x)+N_{\nu+1}(x) = \frac{2\nu}{x}N_\nu(x).$$}

证明非常简单,留为练习

\end{frame}

\section{sample problem}
\secpage{环形区域上的数理方程}{当圆心被挖掉时,$N_m(kr)e^{\pm \ii m\theta}$ 也是合法解}

\begin{frame}
\frametitle{例题}

\bmini{0.45}
\lfig{1.8}{ring_problem.png}
\emini
\bmini{0.5}
如图,一根外半径为$R_1$,内半径为$R_2$的无限长的均匀材质空心圆柱。外表面和温度为$T_0$的热库接触,保持温度为$T_0$。在$t=0$时刻空心圆柱处于热平衡,温度为$T_0$。在$t>0$时,从内表面处处都注入热流$j$。已知材质的导热系数为$\lambda$,质量密度为$\rho$,单位质量比热为$c$。计算该空心圆柱各处的温度。
\emini

\end{frame}


\begin{frame}
\frametitle{解答思路}

当达到温度梯度不再变化的稳恒状态时,因为外表面温度已经固定,所有位置的温度将保持恒定。那么流入的热流将不再积累,直接穿过空心圆柱流出外表面。这时,距离中心轴为$r$处的热流为
$$  \frac{j(2\pi R_2)}{2\pi r} ,$$
即
$$ \frac{\partial T}{\partial r} = -\frac{jR_2}{\lambda r}. $$
解出稳恒解:
$$ \left. T \right\vert_{t\rightarrow \infty}= T_0  -\frac{jR_2}{\lambda }\ln\frac{r}{R_1}. $$

\end{frame}

\begin{frame}
\frametitle{解答思路(续)}

设严格解为
$$  T = T_0  -\frac{jR_2}{\lambda }\ln\frac{r}{R_1} + u(r,t). $$
则$u$满足
\bea
\frac{\partial u}{\partial t} - a \nabla^2 u &=& 0 ,\newl
\left. u\right\vert_{r=R_1} &=& 0, \newl
\left.\frac{\partial u}{\partial r} \right\vert_{r=R_2} &=& 0, \newl
\left. u\right\vert_{t=0} &=& \frac{jR_2}{\lambda}\ln\frac{r}{R_1} .
\eea
其中$a =\frac{\lambda}{\rho c}$.

\end{frame}


\begin{frame}
\frametitle{解答思路(续)}

把解按满足边界条件的谐函数展开:
$$ u(r, t) = \sum_{i=1}^\infty c_i\left[N_0(k_iR_1) J_0(k_ir) - J_0(k_iR_1) N_0(k_ir)\right]e^{-ak_i^2t} .$$
其中$k_i$是满足
$$ N_0(k_iR_1) J_0'(k_iR_2) - J_0(k_iR_1) N_0’(k_iR_2) = 0 $$
的第$i$个解。

\skiplines

剩下的用正交关系求系数的工作略去。

\end{frame}





\section{Homework}

\begin{frame}
\frametitle{Homework}

\bitem
\item{对$0<\nu<1$,当$x\rightarrow 0^+$时,取$J_\nu$和$J_{-\nu}$的级数展开的最低次近似,证明:
  $$N_\nu(x) \approx -\frac{\Gamma(-\nu)\cos{\nu\pi}}{\pi}\left(\frac{x}{2}\right)^\nu - \frac{\Gamma(\nu)}{\pi}\left(\frac{x}{2}\right)^{-\nu}. $$
  在上式中对$x$求导,然后令$\nu\rightarrow 0$求极限,再对$x$积分,导出
  $$ N_0(x)\approx \frac{2}{\pi}\ln x + c,$$
  其中$c$为常数,当$x\rightarrow 0^+$时可以忽略它的贡献。

  \skiplines
      {\small \darkgreen 提示:要用到$\Gamma$函数的递推关系和互余宗量关系
        $\sfgamma{z}\sfgamma{1-z}=\frac{\pi}{\sin{(\pi z)}}$(推导见第5讲)。}}
  \eitem

\end{frame}


\begin{frame}
\frametitle{Homework for Quizphobia (cont.)}
\bitem
\item{在上一题中,我们无耻地交换了求极限和求导/积分的次序。请保持这种无耻,用$N_\nu$的极限定义证明它的两个递推关系:
  $$ \frac{d}{dx}\left[x^\nu N_\nu(x)\right] = x^\nu N_{\nu -1}(x);$$
  $$ \frac{d}{dx}\left[x^{-\nu} N_\nu(x)\right] = -x^{-\nu} N_{\nu +1}(x).  $$
}
\item{把例题的解答补充完整,写出系数$c_i$的积分表达式,并使出洪荒之力进行化简。}  \eitem
\end{frame}

\ech
\end{document}
