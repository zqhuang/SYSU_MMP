\documentclass[CJK]{beamer}
\usepackage{CJKutf8}
\usepackage{beamerthemesplit}
\usetheme{Malmoe}
\useoutertheme[footline=authortitle]{miniframes}
\usepackage{amsmath}
\usepackage{amssymb}
\usepackage{graphicx}
\usepackage{eufrak}
\usepackage{color}
\usepackage{slashed}
\usepackage{simplewick}
\usepackage{tikz}
\usepackage{tcolorbox}
\graphicspath{{../figures/}}
%%figures
\def\lfig#1#2{\includegraphics[width=#1 in]{#2}}
\def\addfig#1#2{\begin{center}\includegraphics[width=#1 in]{#2}\end{center}}
\def\wulian{\includegraphics[width=0.18in]{emoji_wulian.jpg}}
\def\bigwulian{\includegraphics[width=0.35in]{emoji_wulian.jpg}}
\def\bye{\includegraphics[width=0.18in]{emoji_bye.jpg}}
\def\bigbye{\includegraphics[width=0.35in]{emoji_bye.jpg}}
\def\huaixiao{\includegraphics[width=0.18in]{emoji_huaixiao.jpg}}
\def\bighuaixiao{\includegraphics[width=0.35in]{emoji_huaixiao.jpg}}
\def\jianxiao{\includegraphics[width=0.18in]{emoji_jianxiao.jpg}}
\def\bigjianxiao{\includegraphics[width=0.35in]{emoji_jianxiao.jpg}}
%% colors
\def\blacktext#1{{\color{black}#1}}
\def\bluetext#1{{\color{blue}#1}}
\def\redtext#1{{\color{red}#1}}
\def\darkbluetext#1{{\color[rgb]{0,0.2,0.6}#1}}
\def\skybluetext#1{{\color[rgb]{0.2,0.7,1.}#1}}
\def\cyantext#1{{\color[rgb]{0.,0.5,0.5}#1}}
\def\greentext#1{{\color[rgb]{0,0.7,0.1}#1}}
\def\darkgray{\color[rgb]{0.2,0.2,0.2}}
\def\lightgray{\color[rgb]{0.6,0.6,0.6}}
\def\gray{\color[rgb]{0.4,0.4,0.4}}
\def\blue{\color{blue}}
\def\red{\color{red}}
\def\green{\color{green}}
\def\darkgreen{\color[rgb]{0,0.4,0.1}}
\def\darkblue{\color[rgb]{0,0.2,0.6}}
\def\skyblue{\color[rgb]{0.2,0.7,1.}}
%%control
\def\be{\begin{equation}}
\def\ee{\nonumber\end{equation}}
\def\bea{\begin{eqnarray}}
\def\eea{\nonumber\end{eqnarray}}
\def\bch{\begin{CJK}{UTF8}{gbsn}}
\def\ech{\end{CJK}}
\def\bitem{\begin{itemize}}
\def\eitem{\end{itemize}}
\def\bcenter{\begin{center}}
\def\ecenter{\end{center}}
\def\bex{\begin{minipage}{0.2\textwidth}\includegraphics[width=0.6in]{jugelizi.png}\end{minipage}\begin{minipage}{0.76\textwidth}}
\def\eex{\end{minipage}}
\def\chtitle#1{\frametitle{\bch#1\ech}}
\def\bmat#1{\left(\begin{array}{#1}}
\def\emat{\end{array}\right)}
\def\bcase#1{\left\{\begin{array}{#1}}
\def\ecase{\end{array}\right.}
\def\bmini#1{\begin{minipage}{#1\textwidth}}
\def\emini{\end{minipage}}
\def\tbox#1{\begin{tcolorbox}#1\end{tcolorbox}}
\def\pfrac#1#2#3{\left(\frac{\partial #1}{\partial #2}\right)_{#3}}
%%symbols
\def\bropt{\,(\ \ \ )}
\def\sone{$\star$}
\def\stwo{$\star\star$}
\def\sthree{$\star\star\star$}
\def\sfour{$\star\star\star\star$}
\def\sfive{$\star\star\star\star\star$}
\def\rint{{\int_\leftrightarrow}}
\def\roint{{\oint_\leftrightarrow}}
\def\stdHf{{\textit{\r H}_f}}
\def\deltaH{{\Delta \textit{\r H}}}
\def\ii{{\dot{\imath}}}
\def\skipline{{\vskip0.1in}}
\def\skiplines{{\vskip0.2in}}
\def\lagr{{\mathcal{L}}}
\def\hamil{{\mathcal{H}}}
\def\vecv{{\mathbf{v}}}
\def\vecx{{\mathbf{x}}}
\def\vecy{{\mathbf{y}}}
\def\veck{{\mathbf{k}}}
\def\vecp{{\mathbf{p}}}
\def\vecn{{\mathbf{n}}}
\def\vecA{{\mathbf{A}}}
\def\vecP{{\mathbf{P}}}
\def\vecsigma{{\mathbf{\sigma}}}
\def\hatJn{{\hat{J_\vecn}}}
\def\hatJx{{\hat{J_x}}}
\def\hatJy{{\hat{J_y}}}
\def\hatJz{{\hat{J_z}}}
\def\hatj#1{\hat{J_{#1}}}
\def\hatphi{{\hat{\phi}}}
\def\hatq{{\hat{q}}}
\def\hatpi{{\hat{\pi}}}
\def\vel{\upsilon}
\def\Dint{{\mathcal{D}}}
\def\adag{{\hat{a}^\dagger}}
\def\bdag{{\hat{b}^\dagger}}
\def\cdag{{\hat{c}^\dagger}}
\def\ddag{{\hat{d}^\dagger}}
\def\hata{{\hat{a}}}
\def\hatb{{\hat{b}}}
\def\hatc{{\hat{c}}}
\def\hatd{{\hat{d}}}
\def\hatN{{\hat{N}}}
\def\hatH{{\hat{H}}}
\def\hatp{{\hat{p}}}
\def\Fup{{F^{\mu\nu}}}
\def\Fdown{{F_{\mu\nu}}}
\def\newl{\nonumber \\}
\def\vece{\mathrm{e}}
\def\calM{{\mathcal{M}}}
\def\calT{{\mathcal{T}}}
\def\calR{{\mathcal{R}}}
\def\barpsi{\bar{\psi}}
\def\baru{\bar{u}}
\def\barv{\bar{\upsilon}}
\def\qeq{\stackrel{?}{=}}
\def\torder#1{\mathcal{T}\left(#1\right)}
\def\rorder#1{\mathcal{R}\left(#1\right)}
\def\contr#1#2{\contraction{}{#1}{}{#2}#1#2}
\def\trof#1{\mathrm{Tr}\left(#1\right)}
\def\trace{\mathrm{Tr}}
\def\comm#1{\ \ \ \left(\mathrm{used}\ #1\right)}
\def\tcomm#1{\ \ \ (\text{#1})}
\def\slp{\slashed{p}}
\def\slk{\slashed{k}}
\def\calp{{\mathfrak{p}}}
\def\veccalp{\mathbf{\mathfrak{p}}}
\def\Tthree{T_{\tiny \textcircled{3}}}
\def\pthree{p_{\tiny \textcircled{3}}}
\def\dbar{{\,\mathchar'26\mkern-12mu d}}
\def\erf{\mathrm{erf}}
\def\const{\mathrm{constant}}
\def\pheat{\pfrac p{\ln T}V}
\def\vheat{\pfrac V{\ln T}p}
%%units
\def\fdeg{{^\circ \mathrm{F}}}
\def\cdeg{^\circ \mathrm{C}}
\def\atm{\,\mathrm{atm}}
\def\angstrom{\,\text{\AA}}
\def\SIL{\,\mathrm{L}}
\def\SIkm{\,\mathrm{km}}
\def\SIyr{\,\mathrm{yr}}
\def\SIGyr{\,\mathrm{Gyr}}
\def\SIV{\,\mathrm{V}}
\def\SImV{\,\mathrm{mV}}
\def\SIeV{\,\mathrm{eV}}
\def\SIkeV{\,\mathrm{keV}}
\def\SIMeV{\,\mathrm{MeV}}
\def\SIGeV{\,\mathrm{GeV}}
\def\SIcal{\,\mathrm{cal}}
\def\SIkcal{\,\mathrm{kcal}}
\def\SImol{\,\mathrm{mol}}
\def\SIN{\,\mathrm{N}}
\def\SIHz{\,\mathrm{Hz}}
\def\SIm{\,\mathrm{m}}
\def\SIcm{\,\mathrm{cm}}
\def\SIfm{\,\mathrm{fm}}
\def\SImm{\,\mathrm{mm}}
\def\SInm{\,\mathrm{nm}}
\def\SImum{\,\mathrm{\mu m}}
\def\SIJ{\,\mathrm{J}}
\def\SIW{\,\mathrm{W}}
\def\SIkJ{\,\mathrm{kJ}}
\def\SIs{\,\mathrm{s}}
\def\SIkg{\,\mathrm{kg}}
\def\SIg{\,\mathrm{g}}
\def\SIK{\,\mathrm{K}}
\def\SImmHg{\,\mathrm{mmHg}}
\def\SIPa{\,\mathrm{Pa}}

\def\courseurl{https://github.com/zqhuang/SYSU\_TD}

\def\tpage#1#2{
\begin{frame}
\begin{center}
\begin{Large}
\bch
热学 \\
第#1讲 #2

{\vskip 0.3in}

黄志琦

\ech
\end{Large}
\end{center}

\vskip 0.2in

\bch
教材:《热学》第二版,赵凯华,罗蔚茵,高等教育出版社
\ech

\bch
课件下载
\ech
\courseurl
\end{frame}
}

\def\bfr#1{
\begin{frame}
\chtitle{#1} 
\bch
}

\def\efr{
\ech 
\end{frame}
}

  \date{}
\begin{document}
\tpage{6}{Delta Function and Fourier Transform}


\begin{frame}
\chtitle{本讲内容}
\bch
\bitem
\item{狄拉克$\delta$函数}
\item{傅立叶变换}
\eitem
\ech
\end{frame}

\section{Dirac $\delta$ function}
\secpage{又高又瘦的$\delta$函数}{$$\int_{-\infty}^{\infty} \delta(x-x_0)f(x) \,dx = f(x_0)$$}

\begin{frame}
  \chtitle{物理里的理想化模型}
  \bch
  物理里有很多“无穷大$\times$无穷小 = 有限“的模型:
  \bitem
\item{瞬时冲量: 力无限大,作用时间无穷短,但两者的乘积(冲量)是有限的。}
\item{质点:质量密度无穷大,体积无穷小,但两者的乘积(总质量)是有限的。}
\item{点电荷:电荷密度无穷大,体积无穷小,但两者的乘积(总电荷)是有限的。}
  \eitem

  \skiplines
  
  在数学上这些表述都是不合法的,需要搞很多事情才能把这些模型说清楚。 不喜欢搞事情的物理学家们于是发明了{\blue $\delta$函数}。
  \ech
\end{frame}

\begin{frame}
  \chtitle{重量级dalao —— Paul Dirac (狄拉克)}
  \bch
  \bcenter
  \lfig{1.2}{Dirac.jpg}

  你们不要搞事情
  \ecenter
  \ech
\end{frame}
  
\begin{frame}
  \chtitle{Dirac $\delta$ function}
  \bch
  \bmini{0.4}
  \lfig{1.5}{DiracDelta.png}
  \emini
  \bmini{0.55}
  Dirac $\delta$ function 不是传统意义上的函数。它可以通过下面的{\blue 单位脉冲函数取脉冲时间为零的极限}得到:
  $$\delta_D(x) = \left\{\begin{array}{ll} \frac{1}{\epsilon}, & \text{ if } -\frac{\epsilon}{2}<x<\frac{\epsilon}{2} \\  0, & \text{ else}\end{array}\right. $$
  其中$\epsilon\rightarrow 0^+$。  
  \emini

  在本课程中,我们{\blue 简称Dirac $\delta$ function为$\delta$函数,并简写为$\delta(x)$}。
  \ech
\end{frame}


\begin{frame}
  \chtitle{$\delta$函数的另一种逼近方式}
  \bch
  有时候需要计算$\delta$函数的导数甚至高阶导数,这时可以考虑用高斯函数逼近方式:

  $$\delta(x) = \frac{1}{\sqrt{2\pi}\epsilon}e^{-\frac{x^2}{2\epsilon^2}},$$
  其中$\epsilon\rightarrow 0^+$。

  \skiplines
  

  {\scriptsize 注:这一般只是为了帮助理解$\delta$函数的导数的图像,并非为了计算。如果要用具体的逼近方式来进行计算,$\delta$函数的便捷性就大打折扣了。}
  \ech
\end{frame}


\begin{frame}
  \chtitle{$\delta$函数的抽象定义}
  \bch
  借助上述两种逼近方式的辅助,我们归纳出$\delta$函数的下述抽象定义:
  {\blue
    \be
    \delta(0) = \left\{
    \begin{array}{ll}
      0, & \text{ if } x\ne 0; \\
      +\infty, & \text{ if } x = 0;
    \end{array}\right.
    \ee
    $$ \int_{-\infty}^\infty \delta(x)\,dx = 1.$$ 
  }
  从上述抽象定义中可以看出$\delta$函数是偶函数:
  $$\delta(-x) = \delta(x)$$
  此外,积分的范围可以限定在$0$的任意小领域。
    $$ \int_{0^-}^{0^+} \delta(x)\,dx = 1.$$   
  \ech
\end{frame}

\begin{frame}
  \chtitle{$\delta$函数最重要的性质}
  \bch
  从被使用频率上来讲,下式至关重要:
  \tbox{
    $$\int_{-\infty}^\infty \delta(x-x_0) f(x)\, dx = f(x_0).$$}

  (请自行用物理图像理解上式)
  \ech
\end{frame}

\begin{frame}
  \chtitle{侮辱智商的思考题(变相点名)}
  \bch
  \addfig{1.7}{wuruzhishang.jpg}
  
  计算积分
  $$\int_{-\infty}^{\infty}   \delta(x-\frac{\pi}{2})\cos x\,dx $$
  \ech
\end{frame}


\begin{frame}
  \chtitle{$\delta$函数的导函数}
  \bch
  利用$\delta$函数在两边都是零的特点,可以用分部积分的方法得到$\delta$函数的导数的性质:
  {\blue
  $$\int_{-\infty}^\infty \left[\frac{d^n}{dx^n}\delta(x-x_0)\right] f(x)\, dx = (-1)^n f^{(n)}(x_0)$$}
  \ech
\end{frame}

\begin{frame}
  \chtitle{侮辱智商的思考题(变相点名)}
  \bch
  \addfig{1.7}{wuruzhishang.jpg}
  
  计算积分
  $$\int_{-\infty}^{\infty}   \delta'(x-\frac{\pi}{2})\cos x\,dx $$
  \ech
\end{frame}



\begin{frame}
  \chtitle{$\delta$函数的终极大招}
  \bch
  用变量替换的方法可以得到:\tbox{
    $$ \delta\left(\alpha(x)\right)  = \sum_{\rm roots}  \frac{\delta(x-x_i)}{|\alpha'(x_i)|}$$}
  求和对所有$\alpha(x)$的根$x_1$, $x_2$, \ldots 进行。

   {\scriptsize (请思考当$\alpha(x)$没有根或者有重根的情形上面的等式会如何。)}

  \skiplines
  
 {\small 另一种等价的写法是:
  $$\int_{-\infty}^\infty \delta\left(\alpha(x)\right) f(x) \,dx = \sum_{\rm roots} \frac{f(x_i)}{|\alpha'(x_i)|}.$$}
  
  \ech
\end{frame}


\begin{frame}
  \chtitle{侮辱智商的思考题(变相点名)}
  \bch
  \addfig{1.3}{wuruzhishang2.jpg}
  
  计算积分
  $$\int_{-\infty}^\infty \delta(x^2-1) e^x\, dx$$
  \ech
\end{frame}



\section{Fourier Transform}
\secpage{傅立叶变换}{无穷维空间的旋转}


\begin{frame}
  \chtitle{据说你们学过线性代数}
  \bch
  \bcenter
  \lfig{1}{blackq.jpg}

  线。性。代。数。
  \ecenter
  \ech
\end{frame}
  

\begin{frame}
  \chtitle{从实数矢量的内积说起}
  \bch
  考虑$n$维平直的内积实空间的矢量
  $$\vecx = \left(\begin{array}{l}x_1 \\ x_2 \\ \ldots \\ x_n \end{array}\right),$$
  其中$x_i$ ($i=1,2,\ldots, n$)都是实数。

  \skipline
  
  两个矢量的内积定义为:
  $$ \vecx \cdot \vecy \equiv  \vecx^T \vecy = \sum_{i=1}^n x_iy_i.$$

  \skipline
  
  一个矢量的长度定义为
  $$\left\Vert\vecx\right\Vert \equiv \sqrt{\vecx\cdot\vecx}= \sqrt{\sum_{i=1}^n x_i^2}. $$

  
  \ech
\end{frame}

\begin{frame}
  \chtitle{换基:正交变换}
  \bch
  在内积空间里一个“没有拉伸和变形”的旋转,新基矢可以由老基矢正交变换得到:
  \be
  \bral \vece_1' \\ \vece_2' \\ \ldots \\ \vece_n' \brar = R  \bral \vece_1 \\ \vece_2 \\ \ldots \\ \vece_n \brar
  \ee
  其中$R$是一个$n \times n$的正交矩阵 ($R^TR = I$)。


  \skipline

  在换基的操作下,矢量的分量也会发生变化:

  $$\vecx' = R \vecx$$

  显然,两个矢量的内积不变

  $$\vecx'^T\vecy'= \vecx^T R^T R\vecy = \vecx^T \vecy$$
  
  \ech
\end{frame}


\begin{frame}
  \chtitle{复矢量的内积和长度}
  \bch
  对复矢量(即矩阵元允许为复数的$n\times 1$矩阵),我们通常只要把转置变成“共轭转置”(即转置并对每个分量取共轭):
  $$ \vecx \cdot \vecy \equiv  \vecx^\dagger \vecy = \sum_{i=1}^n x_i^*y_i.$$

  矢量长度为
  $$\left\Vert\vecx\right\Vert \equiv \sqrt{\vecx\cdot\vecx}= \sqrt{\sum_{i=1}^n |x_i|^2}. $$

  \skipline

  {\small
    注:共轭转置和转置一样,可以进行“倒序乘展开”
{\blue  $$ (A_1A_2\ldots A_{N-1} A_N)^\dagger = A_N^\dagger A_{N-1}^\dagger \ldots A_2^\dagger A_1^\dagger .$$}}
  \ech
\end{frame}



\begin{frame}
  \chtitle{换基:酉变换}
  \bch
  在复内积空间里一个“没有拉伸和变形”的旋转,新基矢可以由老基矢酉变换得到:
  \be
  \bral \vece_1' \\ \vece_2' \\ \ldots \\ \vece_n' \brar = U  \bral \vece_1 \\ \vece_2 \\ \ldots \\ \vece_n \brar
  \ee
  其中$U$是一个$n \times n$的酉矩阵 ($U^\dagger U = I$)。


  \skipline

  在换基的操作下,复矢量的分量也会发生变化:

  $$\vecx' = U^* \vecx$$

  显然,两个矢量的内积不变

  $$\vecx'^\dagger\vecy'= \vecx^\dagger U^T U^*\vecy = \vecx^\dagger \vecy$$
  \ech
\end{frame}


\begin{frame}
  \chtitle{思考题}
  \bch

  \addfig{1}{think2.jpg}
  
  对任意正整数$N$以及整数$k$,证明

  \be
  \sum_{n=0}^{N-1} e^{\frac{2\pi n k \ii}{N}} = \branchll N, & \text{ if } \mathrm{mod}(k, N) = 0 \\ 0, & \text{ else} \branchrr
  \ee
  其中$\mathrm{mod}(k, N) = 0$指$k$能被$N$整除。
  \ech
\end{frame}


\begin{frame}
  \chtitle{思考题}
  \bch

  \addfig{1}{think3.jpg}
  
  设$N\times N$矩阵E的第$m$行第$n$列元素定义为
  $$ E_{mn} = \frac{1}{\sqrt{N}}e^{\frac{2\pi  mn \ii}{N}}$$
  这里我们采用$C$语言的从零计数的习惯,$m = 0,1,2,\ldots, N-1$, $n = 0,1,2,\ldots, N-1$。

  \bitem
\item[1]{试证明$E$是酉矩阵。}
\item[2]{对变换$\vecx' = E^* \vecx$,证明其逆变换为$\vecx =E^T \vecx'$。}
\item[3]{证明变换$\vecx' = E^* \vecx$保持复矢量长度不变。}
  \eitem
  \ech
\end{frame}


\begin{frame}
  \chtitle{函数的内积}
  \bch
  实变量(但函数值允许为复数)的函数$f$和$g$的内积定义为
  $$ f\cdot g \equiv \int_{-\infty}^{\infty} f^*(x) g(x)\, dx .$$
  \ech
\end{frame}

\begin{frame}
  \chtitle{把函数离散化理解}
  \bch
  
  设想把实轴进行了间隔为$dx$的划分$x_{0}, x_{1}, x_2, \ldots$,把
  $ f_i\sqrt{dx} $ 
  当成复矢量的分量,其中$f_i = f(x_i).$

  \ech
\end{frame}


\begin{frame}
  \chtitle{傅立叶变换}
  \bch
  {\small
  考虑$N\rightarrow \infty$的情况,利用$ E_{mn} = \frac{1}{\sqrt{N}}e^{\frac{2\pi  mn \ii}{N}}$  为酉矩阵, 把复矢量$f$进行酉变换到另一个“傅立叶空间”。在傅立叶空间里,通常我们使用变量$k$。

  $$\tilde{f}_m \sqrt{dk} = \sum_{n} E^*_{mn} f_n\sqrt{dx}  = \frac{1}{\sqrt{N}} \sum_{n} e^{-\frac{2\pi  mn \ii}{N}}f_n\sqrt{dx} .$$
  令傅立叶空间里的划分宽度$dk = \frac{2\pi}{Ndx}$,则$k_mx_n = \frac{2\pi mn}{N}$,上式可以写成
  $$\tilde{f}_m =\frac{1}{\sqrt{2\pi}} \sum_{n} e^{-\ii k_m x_n}f_n  dx .$$
  写成积分形式{\blue
  $$ \tilde{f}(k) = \frac{1}{\sqrt{2\pi}} \int_{-\infty}^\infty e^{-ikx} f(x) dx $$
 } }
  \ech
\end{frame}


\begin{frame}
  \chtitle{逆变换}
  \bch
  利用酉变换的逆变换立刻可以写出傅立叶变换的逆变换:
{\blue
  $$ f(x) = \frac{1}{\sqrt{2\pi}} \int_{-\infty}^\infty e^{ikx} \tilde{f}(k) dk $$
 }
  
  \ech
\end{frame}


\begin{frame}
  \chtitle{傅立叶变换保持矢量长度不变}
  \bch
  利用酉变换保持矢量长度不变的性质,有
{\blue
 $$\int_{-\infty}^\infty \tilde{f}^*(k) \tilde{g}(k) dk = \int_{-\infty}^\infty f^*(x)g(x) dx. $$
 }
  
  \ech
\end{frame}

\begin{frame}
  \chtitle{$\delta$函数的傅立叶变换}
  \bch
  考虑$f(x) = \delta(x)$ 的傅立叶变换
  $$\tilde{f} = \frac{1}{\sqrt{2\pi}} \int_{-\infty}^{\infty} \delta(x) e^{-\ii kx}dx = \frac{1}{\sqrt{2\pi}}.$$
  再进行逆变换,我们就得到一个非常有用的恒等式:
  \tbox{$$\frac{1}{2\pi}\int_{-\infty}^\infty e^{ikx}dk = \delta(x).$$}
  \ech
\end{frame}


\begin{frame}
  \chtitle{思考题}
  \bch
  \addfig{1.}{think3.jpg}
  
  利用
    \tbox{$$\frac{1}{2\pi}\int_{-\infty}^\infty e^{ikx}dk = \delta(x),$$}
    你能直接证明
    $$\int_{-\infty}^\infty \tilde{f}^*(k) \tilde{g}(k) dk = \int_{-\infty}^\infty f^*(x)g(x) dx $$
    吗?
  \ech
\end{frame}

\begin{frame}
  \chtitle{思考题}
  \bch
  \bmini{0.48}
  \addfig{2}{stepfunction.png}
  \emini
  \bmini{0.48}
  \be
  u(x) = \branchll 1, & \text{ if } -1<x<1 \\ 0, & \text{ else} \branchrr
  \ee
  \emini

  \skipline
  
  计算如图函数的傅立叶变换,然后利用傅立叶变换保持内积不变的性质,求积分
  $$\int_0^\infty \frac{\sin x}{x} \,dx. $$
  
  \ech
\end{frame}

\section{Homework}

\begin{frame}
  \chtitle{课后作业}
  \bch
  \bitem
\item[16]{计算积分$$\int_{-\infty}^\infty \delta\left(\sin x\right) e^{-|x|} dx.$$ }
\item[17]{计算高斯函数$$f(x) = \frac{1}{\sqrt{2\pi}\sigma} e^{-\frac{x^2}{2\sigma^2}}$$
的傅立叶变换。}
\item[18]{用你喜欢的方法求积分
  $$\int_0^\infty \left(\frac{\sin x}{x}\right)^2dx . $$}
  \eitem
  \ech
\end{frame}

\end{document}
