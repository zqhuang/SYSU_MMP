\documentclass[CJK]{beamer}
\usepackage{CJKutf8}
\usepackage{beamerthemesplit}
\usetheme{Malmoe}
\useoutertheme[footline=authortitle]{miniframes}
\usepackage{amsmath}
\usepackage{amssymb}
\usepackage{graphicx}
\usepackage{eufrak}
\usepackage{color}
\usepackage{slashed}
\usepackage{simplewick}
\usepackage{tikz}
\usepackage{tcolorbox}
\graphicspath{{../figures/}}
%%figures
\def\lfig#1#2{\includegraphics[width=#1 in]{#2}}
\def\addfig#1#2{\begin{center}\includegraphics[width=#1 in]{#2}\end{center}}
\def\wulian{\includegraphics[width=0.18in]{emoji_wulian.jpg}}
\def\bigwulian{\includegraphics[width=0.35in]{emoji_wulian.jpg}}
\def\bye{\includegraphics[width=0.18in]{emoji_bye.jpg}}
\def\bigbye{\includegraphics[width=0.35in]{emoji_bye.jpg}}
\def\huaixiao{\includegraphics[width=0.18in]{emoji_huaixiao.jpg}}
\def\bighuaixiao{\includegraphics[width=0.35in]{emoji_huaixiao.jpg}}
\def\jianxiao{\includegraphics[width=0.18in]{emoji_jianxiao.jpg}}
\def\bigjianxiao{\includegraphics[width=0.35in]{emoji_jianxiao.jpg}}
%% colors
\def\blacktext#1{{\color{black}#1}}
\def\bluetext#1{{\color{blue}#1}}
\def\redtext#1{{\color{red}#1}}
\def\darkbluetext#1{{\color[rgb]{0,0.2,0.6}#1}}
\def\skybluetext#1{{\color[rgb]{0.2,0.7,1.}#1}}
\def\cyantext#1{{\color[rgb]{0.,0.5,0.5}#1}}
\def\greentext#1{{\color[rgb]{0,0.7,0.1}#1}}
\def\darkgray{\color[rgb]{0.2,0.2,0.2}}
\def\lightgray{\color[rgb]{0.6,0.6,0.6}}
\def\gray{\color[rgb]{0.4,0.4,0.4}}
\def\blue{\color{blue}}
\def\red{\color{red}}
\def\green{\color{green}}
\def\darkgreen{\color[rgb]{0,0.4,0.1}}
\def\darkblue{\color[rgb]{0,0.2,0.6}}
\def\skyblue{\color[rgb]{0.2,0.7,1.}}
%%control
\def\be{\begin{equation}}
\def\ee{\nonumber\end{equation}}
\def\bea{\begin{eqnarray}}
\def\eea{\nonumber\end{eqnarray}}
\def\bch{\begin{CJK}{UTF8}{gbsn}}
\def\ech{\end{CJK}}
\def\bitem{\begin{itemize}}
\def\eitem{\end{itemize}}
\def\bcenter{\begin{center}}
\def\ecenter{\end{center}}
\def\bex{\begin{minipage}{0.2\textwidth}\includegraphics[width=0.6in]{jugelizi.png}\end{minipage}\begin{minipage}{0.76\textwidth}}
\def\eex{\end{minipage}}
\def\chtitle#1{\frametitle{\bch#1\ech}}
\def\bmat#1{\left(\begin{array}{#1}}
\def\emat{\end{array}\right)}
\def\bcase#1{\left\{\begin{array}{#1}}
\def\ecase{\end{array}\right.}
\def\bmini#1{\begin{minipage}{#1\textwidth}}
\def\emini{\end{minipage}}
\def\tbox#1{\begin{tcolorbox}#1\end{tcolorbox}}
\def\pfrac#1#2#3{\left(\frac{\partial #1}{\partial #2}\right)_{#3}}
%%symbols
\def\bropt{\,(\ \ \ )}
\def\sone{$\star$}
\def\stwo{$\star\star$}
\def\sthree{$\star\star\star$}
\def\sfour{$\star\star\star\star$}
\def\sfive{$\star\star\star\star\star$}
\def\rint{{\int_\leftrightarrow}}
\def\roint{{\oint_\leftrightarrow}}
\def\stdHf{{\textit{\r H}_f}}
\def\deltaH{{\Delta \textit{\r H}}}
\def\ii{{\dot{\imath}}}
\def\skipline{{\vskip0.1in}}
\def\skiplines{{\vskip0.2in}}
\def\lagr{{\mathcal{L}}}
\def\hamil{{\mathcal{H}}}
\def\vecv{{\mathbf{v}}}
\def\vecx{{\mathbf{x}}}
\def\vecy{{\mathbf{y}}}
\def\veck{{\mathbf{k}}}
\def\vecp{{\mathbf{p}}}
\def\vecn{{\mathbf{n}}}
\def\vecA{{\mathbf{A}}}
\def\vecP{{\mathbf{P}}}
\def\vecsigma{{\mathbf{\sigma}}}
\def\hatJn{{\hat{J_\vecn}}}
\def\hatJx{{\hat{J_x}}}
\def\hatJy{{\hat{J_y}}}
\def\hatJz{{\hat{J_z}}}
\def\hatj#1{\hat{J_{#1}}}
\def\hatphi{{\hat{\phi}}}
\def\hatq{{\hat{q}}}
\def\hatpi{{\hat{\pi}}}
\def\vel{\upsilon}
\def\Dint{{\mathcal{D}}}
\def\adag{{\hat{a}^\dagger}}
\def\bdag{{\hat{b}^\dagger}}
\def\cdag{{\hat{c}^\dagger}}
\def\ddag{{\hat{d}^\dagger}}
\def\hata{{\hat{a}}}
\def\hatb{{\hat{b}}}
\def\hatc{{\hat{c}}}
\def\hatd{{\hat{d}}}
\def\hatN{{\hat{N}}}
\def\hatH{{\hat{H}}}
\def\hatp{{\hat{p}}}
\def\Fup{{F^{\mu\nu}}}
\def\Fdown{{F_{\mu\nu}}}
\def\newl{\nonumber \\}
\def\vece{\mathrm{e}}
\def\calM{{\mathcal{M}}}
\def\calT{{\mathcal{T}}}
\def\calR{{\mathcal{R}}}
\def\barpsi{\bar{\psi}}
\def\baru{\bar{u}}
\def\barv{\bar{\upsilon}}
\def\qeq{\stackrel{?}{=}}
\def\torder#1{\mathcal{T}\left(#1\right)}
\def\rorder#1{\mathcal{R}\left(#1\right)}
\def\contr#1#2{\contraction{}{#1}{}{#2}#1#2}
\def\trof#1{\mathrm{Tr}\left(#1\right)}
\def\trace{\mathrm{Tr}}
\def\comm#1{\ \ \ \left(\mathrm{used}\ #1\right)}
\def\tcomm#1{\ \ \ (\text{#1})}
\def\slp{\slashed{p}}
\def\slk{\slashed{k}}
\def\calp{{\mathfrak{p}}}
\def\veccalp{\mathbf{\mathfrak{p}}}
\def\Tthree{T_{\tiny \textcircled{3}}}
\def\pthree{p_{\tiny \textcircled{3}}}
\def\dbar{{\,\mathchar'26\mkern-12mu d}}
\def\erf{\mathrm{erf}}
\def\const{\mathrm{constant}}
\def\pheat{\pfrac p{\ln T}V}
\def\vheat{\pfrac V{\ln T}p}
%%units
\def\fdeg{{^\circ \mathrm{F}}}
\def\cdeg{^\circ \mathrm{C}}
\def\atm{\,\mathrm{atm}}
\def\angstrom{\,\text{\AA}}
\def\SIL{\,\mathrm{L}}
\def\SIkm{\,\mathrm{km}}
\def\SIyr{\,\mathrm{yr}}
\def\SIGyr{\,\mathrm{Gyr}}
\def\SIV{\,\mathrm{V}}
\def\SImV{\,\mathrm{mV}}
\def\SIeV{\,\mathrm{eV}}
\def\SIkeV{\,\mathrm{keV}}
\def\SIMeV{\,\mathrm{MeV}}
\def\SIGeV{\,\mathrm{GeV}}
\def\SIcal{\,\mathrm{cal}}
\def\SIkcal{\,\mathrm{kcal}}
\def\SImol{\,\mathrm{mol}}
\def\SIN{\,\mathrm{N}}
\def\SIHz{\,\mathrm{Hz}}
\def\SIm{\,\mathrm{m}}
\def\SIcm{\,\mathrm{cm}}
\def\SIfm{\,\mathrm{fm}}
\def\SImm{\,\mathrm{mm}}
\def\SInm{\,\mathrm{nm}}
\def\SImum{\,\mathrm{\mu m}}
\def\SIJ{\,\mathrm{J}}
\def\SIW{\,\mathrm{W}}
\def\SIkJ{\,\mathrm{kJ}}
\def\SIs{\,\mathrm{s}}
\def\SIkg{\,\mathrm{kg}}
\def\SIg{\,\mathrm{g}}
\def\SIK{\,\mathrm{K}}
\def\SImmHg{\,\mathrm{mmHg}}
\def\SIPa{\,\mathrm{Pa}}

\def\courseurl{https://github.com/zqhuang/SYSU\_TD}

\def\tpage#1#2{
\begin{frame}
\begin{center}
\begin{Large}
\bch
热学 \\
第#1讲 #2

{\vskip 0.3in}

黄志琦

\ech
\end{Large}
\end{center}

\vskip 0.2in

\bch
教材:《热学》第二版,赵凯华,罗蔚茵,高等教育出版社
\ech

\bch
课件下载
\ech
\courseurl
\end{frame}
}

\def\bfr#1{
\begin{frame}
\chtitle{#1} 
\bch
}

\def\efr{
\ech 
\end{frame}
}

  \date{}
  \begin{document}
  \bch
\tpage{19}{球面谐函数$Y_{\ell m}$}

\begin{frame}
\frametitle{本讲内容}

\bitem
\item{球面上的分离变量形式的谐函数}
\item{球面谐函数的微分表示}
\eitem

\end{frame}


\section{Spherical Harmonics}
\secpage{单位球面上的谐函数}{
  \addfig{3}{sphcoor.png}
}


\begin{frame}
\frametitle{单位球面上的谐函数}

在球坐标$(r,\theta,\phi)$里固定$r=1$,就得到单位球面:这是一个二维空间。

\skipline

单位球面上的谐函数满足:
$$ \nabla^2Q = -k^2Q $$
其中拉普拉斯算符用$\theta,\phi$的坐标明确写出来就是
$$ \nabla^2 = \frac{1}{\sin\theta}\frac{\partial}{\partial\theta}\left(\sin\theta\frac{\partial}{\partial\theta}\right) + \frac{1}{\sin^2\theta}\frac{\partial^2}{\partial\phi^2} $$

{\scriptsize (如对此有疑问,请回顾第15讲正交曲面坐标系的知识)}

\end{frame}



\begin{frame}
\frametitle{单位球面上的谐函数}

同样直接给出球面谐函数的求解结果:
\skiplines
\tbox{\blue 单位球面上谐函数存在的条件是$k^2=\ell(\ell+1),\ \ell = 0,1,2,\ldots$。对每个$\ell$,存在$2\ell+1$个解: $Y_{\ell m}(\theta,\phi)$ ($-\ell\le m\le \ell$)。}

在文献中常常把$Y_{\ell m}$叫做{\blue 球谐函数}。


\end{frame}




\begin{frame}
\frametitle{$\ell = 0$的情况}

$\ell = 0$的情况只有$2\ell+1=1$个球面谐函数,它是个常数:

\tbox{$$Y_{00}(\theta,\phi) = \frac{1}{\sqrt{4\pi}} $$}


\end{frame}



\begin{frame}
\frametitle{$\ell = 1$的情况}

$\ell = 1$的情况有$2\ell+1=3$个球面谐函数:

\tbox{
  \bea
  Y_{10}(\theta,\phi) &=& \sqrt{\frac{3}{8\pi}}\cos\theta, \newl
  Y_{1,\pm 1}(\theta,\phi) &=& \mp \sqrt{\frac{3}{8\pi}}\sin\theta e^{\pm \ii\phi}.
  \eea
}

\end{frame}

\begin{frame}
\frametitle{$\ell = 2$的情况}

$\ell = 2$的情况有$2\ell+1=5$个球面谐函数:

\tbox{
  \bea
  Y_{20}(\theta,\phi) &=& \sqrt{\frac{5}{16\pi}}\left(3\cos^2\theta-1\right), \newl
  Y_{2,\pm 1}(\theta,\phi) &=& \mp \sqrt{\frac{15}{8\pi}}\sin\theta \cos\theta e^{\pm \ii\phi}, \newl
  Y_{2,\pm 2}(\theta,\phi) &=&   \sqrt{\frac{15}{32\pi}}\sin^2\theta e^{\pm 2\ii\phi} .
  \eea
}

\end{frame}


\begin{frame}
\frametitle{$Y_{\ell m}$“大致长什么样”}

\bitem
\item{$Y_{\ell m}$可以写成$\theta$的一个函数$\Psi_{\ell m}(\theta)$乘以$e^{\ii m\phi}$:
  {\blue $$Y_{\ell m}(\theta,\phi) = \Psi_{\ell m}(\theta)e^{\ii m\phi}$$}
}
\item{在单位球面上按复数内积规则正交归一化:
  {\blue $$\int d\Omega\,  Y_{\ell m}^*(\theta,\phi)Y_{\ell',m'}(\theta,\phi) = \delta_{\ell\ell'}\delta_{m m'} .$$}
  其中$\int d\Omega$是球面积分$\int_0^\pi \sin\theta d\theta\int_0^{2\pi}d\phi$的简写。
}  
  \eitem


\end{frame}


\begin{frame}
\frametitle{思考题}



显然谐函数乘以任何一个非零常数仍然是谐函数,所以谐函数的归一化是人为规定的。

\addfig{1}{think1.jpg}

谐函数的正交性也是人为规定的吗?


\end{frame}


\section{Explicit Expression of $Y_{\ell m}$}
\secpage{球面谐函数的微分表示}{  {\scriptsize
    $$ Y_{\ell m}(\theta,\phi) =\frac{1}{2^\ell \ell !}\sqrt{\frac{(2\ell+1)}{4\pi} \frac{(\ell-m)!}{(\ell+m)!}}\left[\sin^m\theta \left(\frac{1}{\sin\theta}\frac{d}{d\theta}\right)^{\ell+m}\sin^{2\ell}\theta \right] e^{\ii m\phi}$$}}

\begin{frame}
  \frametitle{准备工作:乘积多重导数公式}
  
  设$f$,$g$为$x$的函数,我们都知道
  $$ (fg)' = f'g + fg'. $$
  这个公式可以推广到任意阶导数:\tbox{
  $$ (fg)^{(n)} = \sum_{k=0}^n \frac{n!}{k!(n-k)!} f^{(k)}g^{(n-k)}, $$}
  其中 $f^{(k)}$表示$f$的$k$重导数。

  \skipline
  {\small \darkgreen (和二项式定理一样,这个公式最简明有效的证明方法是用数学归纳法,请自行完成。)}
  
\end{frame}


\begin{frame}
  \frametitle{准备工作:推广的乘积多重导数公式}
  
  设$\rho$, $f$, $g$为$x$的函数,则
  \tbox{
    $$ \left(\rho\frac{d}{dx}\right)^n (fg) = \sum_{k=0}^n \frac{n!}{k!(n-k)!} \left[ \left(\rho\frac{d}{dx}\right)^k f \right]\left[  \left(\rho\frac{d}{dx}\right)^{n-k} g\right], $$}

  \skipline
  
  {\small \darkgreen 证明思路:令$y = \int \frac{dx}{\rho}$并对变量$y$应用乘积的多重导数公式。}
  
  
\end{frame}


\begin{frame}
\frametitle{思考题}

记算符$\hatD = \frac{1}{\sin\theta}\frac{d}{d\theta}$,请验证:
\bitem
\item{$\hatD\sin^2\theta = 2\cos\theta$;}
\item{$\hatD^2\sin^2\theta = -2$;}
\item{对整数$n>2$,$\hatD^n \sin^2\theta = 0$.}  
\item{$\hatD\cos\theta = -1$;}
\item{对整数$m$, $\hatD \sin^m\theta = m \sin^{m-2}\theta \cos\theta$;}
\item{对整数$m$, $\hatD (\sin^m\theta\cos\theta) =m\sin^{m-2}\theta - (m+1)\sin^m\theta$;}  
\eitem
  

\end{frame}




\begin{frame}
  \frametitle{球面谐函数的微分表达式}
  

  {\small \blue
    $$ Y_{\ell m}(\theta,\phi) =\frac{1}{2^\ell \ell !}\sqrt{\frac{(2\ell+1)}{4\pi} \frac{(\ell-m)!}{(\ell+m)!}}\left[\sin^m\theta \left(\frac{1}{\sin\theta}\frac{d}{d\theta}\right)^{\ell+m}\sin^{2\ell}\theta \right] e^{\ii m\phi}$$}

  
\end{frame}


\begin{frame}
  \frametitle{证明概要}
  
  记
  $$N_{\ell m}  := \frac{1}{2^\ell \ell !}\sqrt{\frac{(2\ell+1)}{4\pi} \frac{(\ell-m)!}{(\ell+m)!}};$$
  $$\Psi_{\ell m}(\theta)  := \sin^m\theta \left(\frac{1}{\sin\theta}\frac{d}{d\theta}\right)^{\ell+m}\sin^{2\ell}\theta. $$
  则只要证明微分方程和归一化:
  \bitem
\item{$$ \frac{1}{\sin \theta}\frac{d}{d\theta} \left[\sin\theta \frac{d}{d\theta}\Psi_{\ell m}\right] + \left[\ell(\ell+1)-\frac{m^2}{\sin^2\theta}\right]\Psi_{\ell m} = 0.$$}
\item{$$ \int_0^\pi \left[\Psi_{\ell m}(\theta)\right]^2 \sin\theta d\theta =  \frac{1}{2\pi N_{\ell m}^2} = 4^\ell (\ell!)^2\frac{(\ell+m)!}{(\ell-m)!} \frac{2}{2\ell+1}.  $$
}
  \eitem
  
\end{frame}



\begin{frame}
  \frametitle{第一部分:微分方程}
  
  令$\hatD = \frac{1}{\sin\theta}\frac{d}{d\theta}$,在恒等式
  $$\sin^2\theta \hatD\sin^{2\ell}\theta = 2\ell \cos\theta \sin^{2\ell}\theta $$
  两边作用$\hatD^{\ell+m}$,得到
  \bea
  && \sin^2\theta \hatD^{\ell+m+1}\sin^{2\ell}\theta + 2(\ell+m)\cos\theta \hatD^{\ell+m}\sin^{2\ell}\theta \newl
  && - (\ell+m)(\ell+m-1)\hatD^{\ell+m-1}\sin^{2\ell}\theta \newl
  &=& 2\ell\cos\theta\hatD^{\ell+m}\sin^{2\ell}\theta - 2\ell(\ell+m)\hatD^{\ell+m-1}\sin^{2\ell}\theta.
  \eea
  稍作整理:
  \begin{eqnarray}
  && \sin^2\theta \hatD^{\ell+m+1}\sin^{2\ell}\theta + 2m\cos\theta \hatD^{\ell+m}\sin^{2\ell}\theta \newl
  && + (\ell+m)(\ell-m+1)\hatD^{\ell+m-1}\sin^{2\ell}\theta =0 . \label{eq1}
  \end{eqnarray}

  
\end{frame}


\begin{frame}
  \frametitle{第一部分:微分方程}
  
  两边同乘以$\sin^m\theta$,
  \bea
  && \sin^{m+2}\theta \hatD^{\ell+m+1}\sin^{2\ell}\theta + 2m\sin^m\theta\cos\theta \hatD^{\ell+m}\sin^{2\ell}\theta \newl
  && + (\ell+m)(\ell-m+1)\sin^m\theta\hatD^{\ell+m-1}\sin^{2\ell}\theta =0  .
  \eea
  上式可以写成
  \bea
  && \sin^2\theta \hatD\left[\sin^m\theta \hatD^{\ell+m}\sin^{2\ell}\theta\right] + m\sin^m\theta\cos\theta \hatD^{\ell+m}\sin^{2\ell}\theta \newl 
   && + (\ell+m)(\ell-m+1)\sin^m\theta\hatD^{\ell+m-1}\sin^{2\ell}\theta =0  .
  \eea
  
\end{frame}


\begin{frame}
  \frametitle{第一部分:微分方程}
  
  两边作用$\hatD$,得到
  \bea
  && \hatD\left\{\sin^2\theta \hatD\left[\sin^m\theta \hatD^{\ell+m}\sin^{2\ell}\theta\right]\right\} + m^2\sin^{m-2}\theta \hatD^{\ell+m}\sin^{2\ell}\theta \newl
  && -m(m+1)\sin^m\theta \hatD^{\ell+m}\sin^{2\ell}\theta  + {\blue m\sin^m\theta\cos\theta \hatD^{\ell+m+1}\sin^{2\ell}\theta} \newl
  &&{\blue  + (\ell+m)(\ell-m+1)m\sin^{m-2}\cos\theta\hatD^{\ell+m-1}\sin^{2\ell}\theta} \newl
  && + (\ell+m)(\ell-m+1)\sin^m\theta\hatD^{\ell+m}\sin^{2\ell}\theta =0  .
  \eea
  利用前面的方程\eqref{eq1},把上式蓝色部分替换掉,得到:

  \bea
  && \hatD\left\{\sin^2\theta \hatD\left[\sin^m\theta \hatD^{\ell+m}\sin^{2\ell}\theta\right]\right\} + m^2\sin^{m-2}\theta \hatD^{\ell+m}\sin^{2\ell}\theta \newl
  && -m(m+1)\sin^m\theta \hatD^{\ell+m}\sin^{2\ell}\theta  {\blue - 2m^2\sin^{m-2}\theta\hatD^{\ell+m}\sin^{2\ell}\theta} \newl
  && {\blue +2m^2\sin^m\theta\hatD^{\ell+m}\sin^{2\ell}\theta} \newl
  && + (\ell+m)(\ell-m+1)\sin^m\theta\hatD^{\ell+m}\sin^{2\ell}\theta =0  .
  \eea  

  
\end{frame}


\begin{frame}
  \frametitle{第一部分:微分方程}
  
  把同类项都写到一起就得到了最后的结果
  \bea
  && \hatD\left\{\sin^2\theta \hatD\left[\sin^m\theta \hatD^{\ell+m}\sin^{2\ell}\theta\right]\right\} \newl
  && +\left[\ell(\ell+1)-\frac{m^2}{\sin^2\theta}\right]\sin^m\theta\hatD^{\ell+m}\sin^{2\ell}\theta = 0 .
  \eea
  即
  $$ \frac{1}{\sin \theta}\frac{d}{d\theta} \left[\sin\theta \frac{d}{d\theta}\Psi_{\ell m}\right] + \left[\ell(\ell+1)-\frac{m^2}{\sin^2\theta}\right]\Psi_{\ell m} = 0.$$
  第一部分证毕。
  
\end{frame}


\begin{frame}
  \frametitle{第二部分:归一化}
  
  \bea
  && \int_{0}^\pi \left[\Psi_{\ell m}(\theta)\right]^2\sin\theta d\theta \newl
  &=& \int_{0}^\pi \sin^{2m}\theta \left[\left(\frac{1}{\sin\theta}\frac{d}{d\theta}\right)^{\ell+m}\sin^{2\ell}\theta\right]\sin\theta d\theta \newl
  &=& \int_{-1}^1 (1-x^2)^m \left[\frac{d^{\ell+m}}{dx^{\ell+m}}(1-x^2)^\ell\right]^2 dx 
  \eea
  在最后一步我们做了变量替换$x=\cos\theta$。
  
\end{frame}

\begin{frame}
  \frametitle{第二部分:归一化}
  
  分部积分$\ell+m$次,注意$1-x^2$在端点总是消失,就得到
  \bea
  && \int_{0}^\pi \left[\Psi_{\ell m}(\theta)\right]^2\sin\theta d\theta \newl
  &=& (-1)^{\ell+m}\int_{-1}^1 (1-x^2)^\ell{\blue \frac{d^{\ell+m}}{dx^{\ell+m}}\left[(1-x^2)^m\frac{d^{\ell+m}}{dx^{\ell+m}}(1-x^2)^\ell\right]} dx \newl
  &=&  {\blue \frac{(2\ell)!(\ell+m)!}{(\ell-m)!}  } \int_{-1}^1 (1-x^2)^\ell dx
  \eea
  注意蓝色部分之所以为常数,是因为$(1-x^2)^m\frac{d^{\ell+m}}{dx^{\ell+m}}(1-x^2)^\ell$是$\ell+m$次多项式,求导$\ell+m$次后只有最高次幂有非零贡献。
  
\end{frame}


\begin{frame}
  \frametitle{第二部分:归一化}
  
  最后,做变量替换$ t = \frac{1+x}{2}$,得到
  \bea
  && \int_{0}^\pi \left[\Psi_{\ell m}(\theta)\right]^2\sin\theta d\theta \newl
  &=&  \frac{2^{2\ell+1}(2\ell)!(\ell+m)!}{(\ell-m)!}   \int_0^1  t^\ell (1-t)^\ell dt \newl
  &=&  \frac{2^{2\ell+1}(2\ell)!(\ell+m)! }{(\ell-m)!}  \frac{\left[\Gamma(\ell+1)\right]^2}{\Gamma(2\ell+2)} \newl
  &=&  \frac{2^{2\ell+1}(2\ell)!(\ell+m)! }{(\ell-m)!}  \frac{\left(\ell!\right)^2}{(2\ell+1)!}  \newl
  &=&  4^\ell \left(\ell!\right)^2 \frac{ (\ell+m)! }{(\ell-m)!}  \frac{2}{2\ell+1}   
  \eea
  第二部分证毕。  
  
\end{frame}



\section{Homework}

\begin{frame}
\frametitle{Homework for Quizphobias}

\bitem
\item[55]{证明球面谐函数的正交性。}
\item[56]{列出所有$\ell=3$的球面谐函数}
\item[57]{利用球面谐函数的微分表示证明:对相反的两个方向$\vecn = (\theta,\phi)$和$-\vecn = (\pi-\theta,\pi+\phi)$,有
  $$ Y_{\ell m}(\vecn) = (-1)^\ell Y_{\ell m}(-\vecn)$$}
  \eitem
\end{frame}

\ech
\end{document}
