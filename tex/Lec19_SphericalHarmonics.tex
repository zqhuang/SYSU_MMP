\documentclass[CJK]{beamer}
\usepackage{CJKutf8}
\usepackage{beamerthemesplit}
\usetheme{Malmoe}
\useoutertheme[footline=authortitle]{miniframes}
\usepackage{amsmath}
\usepackage{amssymb}
\usepackage{graphicx}
\usepackage{eufrak}
\usepackage{color}
\usepackage{slashed}
\usepackage{simplewick}
\usepackage{tikz}
\usepackage{tcolorbox}
\graphicspath{{../figures/}}
%%figures
\def\lfig#1#2{\includegraphics[width=#1 in]{#2}}
\def\addfig#1#2{\begin{center}\includegraphics[width=#1 in]{#2}\end{center}}
\def\wulian{\includegraphics[width=0.18in]{emoji_wulian.jpg}}
\def\bigwulian{\includegraphics[width=0.35in]{emoji_wulian.jpg}}
\def\bye{\includegraphics[width=0.18in]{emoji_bye.jpg}}
\def\bigbye{\includegraphics[width=0.35in]{emoji_bye.jpg}}
\def\huaixiao{\includegraphics[width=0.18in]{emoji_huaixiao.jpg}}
\def\bighuaixiao{\includegraphics[width=0.35in]{emoji_huaixiao.jpg}}
\def\jianxiao{\includegraphics[width=0.18in]{emoji_jianxiao.jpg}}
\def\bigjianxiao{\includegraphics[width=0.35in]{emoji_jianxiao.jpg}}
%% colors
\def\blacktext#1{{\color{black}#1}}
\def\bluetext#1{{\color{blue}#1}}
\def\redtext#1{{\color{red}#1}}
\def\darkbluetext#1{{\color[rgb]{0,0.2,0.6}#1}}
\def\skybluetext#1{{\color[rgb]{0.2,0.7,1.}#1}}
\def\cyantext#1{{\color[rgb]{0.,0.5,0.5}#1}}
\def\greentext#1{{\color[rgb]{0,0.7,0.1}#1}}
\def\darkgray{\color[rgb]{0.2,0.2,0.2}}
\def\lightgray{\color[rgb]{0.6,0.6,0.6}}
\def\gray{\color[rgb]{0.4,0.4,0.4}}
\def\blue{\color{blue}}
\def\red{\color{red}}
\def\green{\color{green}}
\def\darkgreen{\color[rgb]{0,0.4,0.1}}
\def\darkblue{\color[rgb]{0,0.2,0.6}}
\def\skyblue{\color[rgb]{0.2,0.7,1.}}
%%control
\def\be{\begin{equation}}
\def\ee{\nonumber\end{equation}}
\def\bea{\begin{eqnarray}}
\def\eea{\nonumber\end{eqnarray}}
\def\bch{\begin{CJK}{UTF8}{gbsn}}
\def\ech{\end{CJK}}
\def\bitem{\begin{itemize}}
\def\eitem{\end{itemize}}
\def\bcenter{\begin{center}}
\def\ecenter{\end{center}}
\def\bex{\begin{minipage}{0.2\textwidth}\includegraphics[width=0.6in]{jugelizi.png}\end{minipage}\begin{minipage}{0.76\textwidth}}
\def\eex{\end{minipage}}
\def\chtitle#1{\frametitle{\bch#1\ech}}
\def\bmat#1{\left(\begin{array}{#1}}
\def\emat{\end{array}\right)}
\def\bcase#1{\left\{\begin{array}{#1}}
\def\ecase{\end{array}\right.}
\def\bmini#1{\begin{minipage}{#1\textwidth}}
\def\emini{\end{minipage}}
\def\tbox#1{\begin{tcolorbox}#1\end{tcolorbox}}
\def\pfrac#1#2#3{\left(\frac{\partial #1}{\partial #2}\right)_{#3}}
%%symbols
\def\bropt{\,(\ \ \ )}
\def\sone{$\star$}
\def\stwo{$\star\star$}
\def\sthree{$\star\star\star$}
\def\sfour{$\star\star\star\star$}
\def\sfive{$\star\star\star\star\star$}
\def\rint{{\int_\leftrightarrow}}
\def\roint{{\oint_\leftrightarrow}}
\def\stdHf{{\textit{\r H}_f}}
\def\deltaH{{\Delta \textit{\r H}}}
\def\ii{{\dot{\imath}}}
\def\skipline{{\vskip0.1in}}
\def\skiplines{{\vskip0.2in}}
\def\lagr{{\mathcal{L}}}
\def\hamil{{\mathcal{H}}}
\def\vecv{{\mathbf{v}}}
\def\vecx{{\mathbf{x}}}
\def\vecy{{\mathbf{y}}}
\def\veck{{\mathbf{k}}}
\def\vecp{{\mathbf{p}}}
\def\vecn{{\mathbf{n}}}
\def\vecA{{\mathbf{A}}}
\def\vecP{{\mathbf{P}}}
\def\vecsigma{{\mathbf{\sigma}}}
\def\hatJn{{\hat{J_\vecn}}}
\def\hatJx{{\hat{J_x}}}
\def\hatJy{{\hat{J_y}}}
\def\hatJz{{\hat{J_z}}}
\def\hatj#1{\hat{J_{#1}}}
\def\hatphi{{\hat{\phi}}}
\def\hatq{{\hat{q}}}
\def\hatpi{{\hat{\pi}}}
\def\vel{\upsilon}
\def\Dint{{\mathcal{D}}}
\def\adag{{\hat{a}^\dagger}}
\def\bdag{{\hat{b}^\dagger}}
\def\cdag{{\hat{c}^\dagger}}
\def\ddag{{\hat{d}^\dagger}}
\def\hata{{\hat{a}}}
\def\hatb{{\hat{b}}}
\def\hatc{{\hat{c}}}
\def\hatd{{\hat{d}}}
\def\hatN{{\hat{N}}}
\def\hatH{{\hat{H}}}
\def\hatp{{\hat{p}}}
\def\Fup{{F^{\mu\nu}}}
\def\Fdown{{F_{\mu\nu}}}
\def\newl{\nonumber \\}
\def\vece{\mathrm{e}}
\def\calM{{\mathcal{M}}}
\def\calT{{\mathcal{T}}}
\def\calR{{\mathcal{R}}}
\def\barpsi{\bar{\psi}}
\def\baru{\bar{u}}
\def\barv{\bar{\upsilon}}
\def\qeq{\stackrel{?}{=}}
\def\torder#1{\mathcal{T}\left(#1\right)}
\def\rorder#1{\mathcal{R}\left(#1\right)}
\def\contr#1#2{\contraction{}{#1}{}{#2}#1#2}
\def\trof#1{\mathrm{Tr}\left(#1\right)}
\def\trace{\mathrm{Tr}}
\def\comm#1{\ \ \ \left(\mathrm{used}\ #1\right)}
\def\tcomm#1{\ \ \ (\text{#1})}
\def\slp{\slashed{p}}
\def\slk{\slashed{k}}
\def\calp{{\mathfrak{p}}}
\def\veccalp{\mathbf{\mathfrak{p}}}
\def\Tthree{T_{\tiny \textcircled{3}}}
\def\pthree{p_{\tiny \textcircled{3}}}
\def\dbar{{\,\mathchar'26\mkern-12mu d}}
\def\erf{\mathrm{erf}}
\def\const{\mathrm{constant}}
\def\pheat{\pfrac p{\ln T}V}
\def\vheat{\pfrac V{\ln T}p}
%%units
\def\fdeg{{^\circ \mathrm{F}}}
\def\cdeg{^\circ \mathrm{C}}
\def\atm{\,\mathrm{atm}}
\def\angstrom{\,\text{\AA}}
\def\SIL{\,\mathrm{L}}
\def\SIkm{\,\mathrm{km}}
\def\SIyr{\,\mathrm{yr}}
\def\SIGyr{\,\mathrm{Gyr}}
\def\SIV{\,\mathrm{V}}
\def\SImV{\,\mathrm{mV}}
\def\SIeV{\,\mathrm{eV}}
\def\SIkeV{\,\mathrm{keV}}
\def\SIMeV{\,\mathrm{MeV}}
\def\SIGeV{\,\mathrm{GeV}}
\def\SIcal{\,\mathrm{cal}}
\def\SIkcal{\,\mathrm{kcal}}
\def\SImol{\,\mathrm{mol}}
\def\SIN{\,\mathrm{N}}
\def\SIHz{\,\mathrm{Hz}}
\def\SIm{\,\mathrm{m}}
\def\SIcm{\,\mathrm{cm}}
\def\SIfm{\,\mathrm{fm}}
\def\SImm{\,\mathrm{mm}}
\def\SInm{\,\mathrm{nm}}
\def\SImum{\,\mathrm{\mu m}}
\def\SIJ{\,\mathrm{J}}
\def\SIW{\,\mathrm{W}}
\def\SIkJ{\,\mathrm{kJ}}
\def\SIs{\,\mathrm{s}}
\def\SIkg{\,\mathrm{kg}}
\def\SIg{\,\mathrm{g}}
\def\SIK{\,\mathrm{K}}
\def\SImmHg{\,\mathrm{mmHg}}
\def\SIPa{\,\mathrm{Pa}}

\def\courseurl{https://github.com/zqhuang/SYSU\_TD}

\def\tpage#1#2{
\begin{frame}
\begin{center}
\begin{Large}
\bch
热学 \\
第#1讲 #2

{\vskip 0.3in}

黄志琦

\ech
\end{Large}
\end{center}

\vskip 0.2in

\bch
教材:《热学》第二版,赵凯华,罗蔚茵,高等教育出版社
\ech

\bch
课件下载
\ech
\courseurl
\end{frame}
}

\def\bfr#1{
\begin{frame}
\chtitle{#1} 
\bch
}

\def\efr{
\ech 
\end{frame}
}

  \date{}
\begin{document}
\tpage{19}{Spherical Harmonics}

\begin{frame}
\chtitle{本讲内容}
\bch
\bitem
\item{一些简单问题的回顾}
\item{单位球面上的谐函数}
\eitem
\ech
\end{frame}


\section{Review}
\secpage{一些简单问题的回顾}{……}

\begin{frame}
  \chtitle{思考题}
  \bch
  \addfig{2}{ringarea.png}
  
  一个圆环($R_1\ge r\ge R_2$)上的满足边界上值为零的谐函数有哪些?它们满足的一般正交定理具体写出来是怎样的?
  
  \ech
\end{frame}


\begin{frame}
  \chtitle{思考题}
  \bch
  \bmini{0.45}
  单位圆周是一个一维空间,用坐标$\phi$ 来描述。

  \skiplines
  
  对单位圆周,思考如下问题:
  \emini
  \bmini{0.5}
  \addfig{1.8}{unitcircle.png}
  \emini

  \bitem
\item{$\nabla^2Q = -k^2Q$里的$k^2$取什么值时才能使谐函数有解?}
\item{不同的$k^2$对应的谐函数是正交的吗?}
\item{对一个固定的$k^2$,谐函数有哪些?它们是正交的吗?}
\item{回忆实对称矩阵的本征值理论并和上述讨论结果做比较。}
  \eitem
  \ech
\end{frame}

\section{Spherical Harmonics}
\secpage{单位球面上的谐函数}{
  \addfig{3}{sphcoor.png}
}


\begin{frame}
\chtitle{单位球面上的谐函数}
\bch
在球坐标$(r,\theta,\phi)$里固定$r=1$,就得到单位球面:这是一个二维空间。

\skipline

单位球面上的谐函数满足:
$$ \nabla^2Q = -k^2Q $$
其中拉普拉斯算符用$\theta,\phi$的坐标明确写出来就是
$$ \nabla^2 = \frac{1}{\sin\theta}\frac{\partial}{\partial\theta}\left(\sin\theta\frac{\partial}{\partial\theta}\right) + \frac{1}{\sin^2\theta}\frac{\partial^2}{\partial\phi^2} $$

{\scriptsize (如对此有疑问,请回顾第15讲正交曲面坐标系的知识)}
\ech
\end{frame}



\begin{frame}
\chtitle{单位球面上的谐函数}
\bch
因为推导比较复杂,先给出结果,把证明留到你们瞌睡的时候再讲:
\skiplines
\tbox{\blue 单位球面上谐函数存在的条件是$k^2=\ell(\ell+1),\ \ell = 0,1,2,\ldots$。对每个$\ell$,存在$2\ell+1$个解: $Y_{\ell m}(\theta,\phi)$ ($-\ell\le m\le \ell$)。}

在文献中常常省略“单位”二字,简单地把$Y_{\ell m}$叫做{\blue 球面谐函数}或{\blue 球谐函数}。

\ech
\end{frame}




\begin{frame}
\chtitle{$\ell = 0$的情况}
\bch
$\ell = 0$的情况只有$2\ell+1=1$个球面谐函数,它是个常数:

\tbox{$$Y_{00}(\theta,\phi) = \frac{1}{\sqrt{4\pi}} $$}

\ech
\end{frame}



\begin{frame}
\chtitle{$\ell = 1$的情况}
\bch
$\ell = 1$的情况有$2\ell+1=3$个球面谐函数:

\tbox{
  \bea
  Y_{10}(\theta,\phi) &=& \sqrt{\frac{3}{8\pi}}\cos\theta, \newl
  Y_{1,\pm 1}(\theta,\phi) &=& \mp \sqrt{\frac{3}{8\pi}}\sin\theta e^{\pm \ii\phi}.
  \eea
}
\ech
\end{frame}

\begin{frame}
\chtitle{$\ell = 2$的情况}
\bch
$\ell = 2$的情况有$2\ell+1=5$个球面谐函数:

\tbox{
  \bea
  Y_{20}(\theta,\phi) &=& \sqrt{\frac{5}{16\pi}}\left(3\cos^2\theta-1\right), \newl
  Y_{2,\pm 1}(\theta,\phi) &=& \mp \sqrt{\frac{15}{8\pi}}\sin\theta \cos\theta e^{\pm \ii\phi}, \newl
  Y_{2,\pm 2}(\theta,\phi) &=&   \sqrt{\frac{15}{32\pi}}\sin^2\theta e^{\pm 2\ii\phi} .
  \eea
}
\ech
\end{frame}


\begin{frame}
\chtitle{$Y_{\ell m}$“大致长什么样”}
\bch
\bitem
\item{$Y_{\ell m}$可以写成$\theta$的一个函数$\Psi_{\ell m}(\theta)$乘以$e^{\ii m\phi}$:
  {\blue $$Y_{\ell m}(\theta,\phi) = \Psi_{\ell m}(\theta)e^{\ii m\phi}$$}
  \bitem
\item{\blue 当$m$为偶数时,$\Psi_{\ell m}(\theta)$可以写成$\cos\theta$的$\ell$次多项式;}
\item{\blue 当$m$为奇数时,$\Psi_{\ell m}(\theta)$可以写成$\sin\theta$乘以$\cos\theta$的$\ell-1$次多项式。}
  \eitem}
\item{在单位球面上按复数内积规则正交归一化:
  {\blue $$\int_0^\pi d\theta\int_0^{2\pi}\sin\theta d\phi \, Y_{\ell m}^*(\theta,\phi)Y_{\ell',m'}(\theta,\phi) = \delta_{\ell\ell'}\delta_{m m'} .$$}}  
  \eitem

\ech
\end{frame}


\begin{frame}
\chtitle{思考题}
\bch


显然谐函数乘以任何一个非零常数仍然是谐函数,所以谐函数的归一化是人为规定的。

\addfig{1}{think1.jpg}

谐函数的正交性也是人为规定的吗?

\ech
\end{frame}

\secpage{谐函数是如何得到的}{不是重点}

\section{Chebyshev Polynomials}

\secpage{Chebyshev多项式}{$$ T_n(\cos\theta)=\cos{(n\theta)};\ U_n(\cos\theta) = \frac{\sin{[(n+1)\theta]}}{\sin\theta}$$ }

\begin{frame}
\chtitle{温习小学知识}
\bch

\addfig{1}{think1.jpg}

把$\cos(2\theta)$写成$\cos\theta$的多项式。
\ech
\end{frame}


\begin{frame}
\chtitle{温习小学知识}
\bch

\addfig{1}{think1.jpg}

把$\cos(3\theta)$写成$\cos\theta$的多项式。
\ech
\end{frame}


\begin{frame}
\chtitle{温习小学知识}
\bch

\addfig{1}{think1.jpg}

把$\frac{\sin(2\theta)}{\sin\theta}$写成$\cos\theta$的多项式。
\ech
\end{frame}


\begin{frame}
\chtitle{温习小学知识}
\bch

\addfig{1}{think1.jpg}

把$\frac{\sin(3\theta)}{\sin\theta}$写成$\cos\theta$的多项式。
\ech
\end{frame}


\begin{frame}
\chtitle{温习小学知识}
\bch

\addfig{1}{think1.jpg}

把$\frac{\sin(4\theta)}{\sin\theta}$写成$\cos\theta$的多项式。
\ech
\end{frame}


\begin{frame}
\chtitle{思考题}
\bch

\addfig{2}{think3.jpg}

证明:$\cos{(n\theta)}$和$\frac{\sin{[(n+1)\theta]}}{\sin \theta}$分别可以写成$\cos\theta$的$n$次多项式:

$$ \cos{(n\theta)} = T_n(\cos \theta);\ \ \frac{\sin{[(n+1)\theta]}}{\sin\theta} = U_n(\cos\theta).$$

其中的$n$次多项式$T_n$和$U_n$分别称为第一类Chebyshev多项式和第二类Chebyshev多项式。

\ech
\end{frame}


\begin{frame}
\chtitle{递推}
\bch
利用三角函数的加法公式,容易得到:
\bea
T_{n+2}(x) &=& xT_{n+1}(x)+(x^2-1)U_n(x), \newl
U_{n+1}(x) &=& xU_n(x) +T_{n+1}(x).
\eea
由此可以从$T_0=1, U_0=1, T_1=x$出发递推出所有Chebyshev多项式:

\ech
\end{frame}


\begin{frame}
\chtitle{前几个Chebyshev多项式}
\bch
\bmini{0.47}
\bea
T_0(x) &=& 1;\newl
T_1(x) &=& x;\newl
T_2(x) &=& 2x^2-1; \newl
T_3(x) &=& 4x^3-3x; \newl
T_4(x) &=& 8x^4 - 8x^2+1;  \newl
T_5(x) &=& 16x^5-20x^3+5x; \newl
\ldots
\eea
\emini
\bmini{0.47}
\bea
 U_0(x) &=& 1;  \newl
 U_1(x) &=& 2x; \newl
 U_2(x) &=& 4x^2-1; \newl
 U_3(x) &=& 8x^3-4x; \newl
 U_4(x) &=& 16x^4-12x^2+1;  \newl
 U_5(x) &=& 32x^5-32x^3+6x; \newl
 \ldots
\eea
\emini

\ech
\end{frame}



\begin{frame}
\chtitle{前几个$T_n(x)$的图}
\bch
\addfig{4}{ChebyshevTn.png}
\ech
\end{frame}


\begin{frame}
\chtitle{前几个$U_n(x)$的图}
\bch
\addfig{4}{ChebyshevUn.png}
\ech
\end{frame}


\begin{frame}
\chtitle{回到球面谐函数的计算}
\bch
$$\frac{1}{\sin\theta}\frac{\partial}{\partial \theta}\left(\sin\theta \frac{\partial Q}{\partial \theta}\right)+\frac{1}{\sin^2\theta}\frac{\partial^2Q}{\partial\phi^2} = -k^2Q  $$
分离变量,令$Q=  \Psi(\theta)\Lambda(\phi)$
$$\frac{(\sin\theta \Psi')'}{\Psi\sin\theta}+\frac{\Lambda''}{\Lambda\sin^2\theta } = -k^2. $$
显然必须有
$$ \frac{\Lambda''}{\Lambda}=\const$$
又显然$\Lambda$满足周期性边界条件$\Lambda(\phi+2\pi)=\Lambda(\phi)$,所以得到
$$\Lambda(\phi) = e^{\pm \ii m\phi},\ m=0,1,2,\ldots$$

\ech
\end{frame}


\begin{frame}
\chtitle{分离变量:$\theta$的函数}
\bch
对固定的$m$, $\Psi(\theta)$满足:
$$ \frac{(\sin\theta \Psi')'}{\Psi\sin\theta}-\frac{m^2}{\sin^2\theta } = -k^2$$


或者重新组织一下:
$$ \Psi''+\cot\theta \,\Psi' +\left(k^2-\frac{m^2}{\sin^2\theta}\right) \Psi = 0. $$

\ech
\end{frame}

\begin{frame}
\chtitle{分离变量:$\theta$的函数——自带边界条件}
\bch

在北极$\theta=0$和南极$\theta=\pi$处,$\Psi$必须满足下列边界条件:

对$m=0$,
$$\Psi' =  (- \Psi''-k^2\Psi) \tan\theta = 0; $$
对$m>0$,
$$ \Psi = \frac{\Psi'' \sin^2\theta  + \Psi' \sin\theta\cos\theta }{m^2-k^2\sin^2\theta } =  0. $$
如果解是振荡型的,上述边界条件就要求$[0,\pi]$里包含了半整数个振荡周期。这告诉我们$k^2$的允许取值很可能只能是一些离散的值。
\ech
\end{frame}


\begin{frame}
\chtitle{分离变量:$\theta$的函数——粗略估算}
\bch
在进行严格求解之前,我们先猜想一个解的大致样子。

当$\theta\ll 1$时,方程可以近似写为:
$$\Psi'' + \frac{1}{\theta}\Psi' + \left(k^2-\frac{m^2}{\theta^2}\right)\Psi = 0. $$
这个方程很眼熟,实际上,如果做变量替换$x=k\theta$,就可以化为标准的贝塞尔方程。也就是说,当$\theta\ll 1$时解大约是$J_m(k\theta)$。

\skipline

如果$k^2 \gg m^2$,$J_m(k\theta)$的第一个峰的位置远小于$1$。那么,可以粗略地把$J_m(k\theta)$的半周期当成$\frac{\pi}{k}$。要求在$[0,\pi]$内有整数个半周期,即要求$k = \ell, \ell \in Z, \ell \gg m$。

\skipline
我们对这个猜测并不是很有信心:因为$\theta$比较大时解未必还和$J_m(k\theta)$相似,$\ell \gg m$的条件不满足时我们也无法证明解并不存在。
\ech
\end{frame}


\begin{frame}
\chtitle{分离变量:$\theta$的函数——严格求解的结果}
\bch
严格求解的结果却出乎意料地和猜测非常接近:{\blue $k = \sqrt{\ell(\ell+1)},\ \ \ \ell \ge m.$}

\skipline

严格地得到这个结论的方法大致有两种:
\bitem
\item{代数方法}
\item{算符方法}
  \eitem

算符推导方法简洁优美,以后在学习量子力学时会接触到。我们这里只介绍比较繁琐的代数方法。

\ech
\end{frame}


\begin{frame}
\chtitle{分离变量:$\theta$的函数——$m=0$的情况}
\bch
在$[0,\pi]$内的函数$\Psi(\theta)$可以用$\cos{(n\theta)}$, $\sin{(n\theta)}$ ($n=0,1,2\ldots$) 进行展开。

  \skiplines

我们先考虑$m=0$,也就是解具有绕$z$轴的旋转对称性的情况。这时的方程为
$$ \frac{1}{\sin\theta}(\Psi'\sin\theta)' + k^2 \Psi = 0. $$


$\Psi$在$[0,\pi]$两端的导数为零,只有$\cos{(n\theta)}$满足条件,所以{\bf 猜测}(这种猜测其实是有bug的,下一讲我们会详细分析):
$$\Psi(\theta) = \sum_{n=0}^\infty a_n\cos{(n\theta)} $$
\ech
\end{frame}



\begin{frame}
\chtitle{分离变量:$\theta$的函数——$m=0$的情况}
\bch
因为我们已经知道$\cos{(n\theta)}$可以写成$\cos\theta$的多项式,所以还可以写成
$$\Psi(\theta) = \sum_{n=0}^\infty c_n\cos^n\theta $$
逐项求导得到
$$\Psi' \sin\theta  = \sum_{n=1}^\infty (-c_nn) \cos^{n-1}\theta\sin^2\theta $$
把$\sin^2\theta$换成$1-\cos^2\theta$,并按幂次重新整理,得到
$$\Psi' \sin\theta  = - c_1+ \sum_{n=0}^\infty \left[nc_n-(n+2)c_{n+2}\right] \cos^{n+1}\theta $$


\ech
\end{frame}


\begin{frame}
\chtitle{分离变量:$\theta$的函数——$m=0$的情况}
\bch
再次求导,得到
$$\frac{1}{\sin\theta}\left(\Psi' \sin\theta\right)'  = \sum_{n=0}^\infty (n+1)\left[(n+2)c_{n+2}-nc_n\right] \cos^n\theta $$
微分方程
$$\frac{1}{\sin\theta}\left(\Psi' \sin\theta\right)'+k^2\Psi = 0$$
就转化为
$$ \sum_{n=0}^\infty \left\{(n+1)(n+2)c_{n+2}+\left[k^2-n(n+1)\right]c_n\right\} \cos^n\theta  = 0 $$
\ech
\end{frame}


\begin{frame}
\chtitle{分离变量:$\theta$的函数——$m=0$的情况}
\bch
于是我们得到系数的递推关系:

$$ c_{n+2} = \frac{n(n+1)-k^2}{(n+1)(n+2)} c_n. $$

这个递推式在$n$很大时(即递推式里$k^2$可以忽略时)给出的渐近行为是$nc_n \rightarrow \const$。

\ech
\end{frame}



\begin{frame}
\chtitle{分离变量:$\theta$的函数——$m=0$的情况}
\bch
由一开始的定义:
$$\Psi(\theta) = \sum_{n=0}^\infty a_n\cos{(n\theta)} $$
可以推出
$$ \sum_{n=0}^\infty c_{2n} = \frac{\Psi(0)+\Psi(\pi)}{2},\ \ \sum_{n=0}^\infty c_{2n+1} = \frac{\Psi(0)-\Psi(\pi)}{2} $$
这似乎和我们掌握的$nc_n\rightarrow \const$ 有矛盾?(级数$\sum_{n=1}^\infty\frac{1}{n}$显然发散.)

\skiplines

解决矛盾的办法就是让$k^2 = \ell(\ell+1)$,这样递推公式会在$n=\ell$中断($c_{\ell+2}=c_{\ell+4}=\ldots = 0$)。奇偶性跟$\ell$不同的$n$对应的$c_n$则全部为零。


\ech
\end{frame}


\begin{frame}
\chtitle{$m\ne 0$的情况}
\bch
$m\ne 0$的情况的推导是类似的,留为作业。
\ech
\end{frame}


\section{Homework}

\begin{frame}
\chtitle{课后作业(题号44-45)}
\bch
\bitem
\item[44]{证明球面谐函数的正交性。}
\item[45]{当$m>0$时证明单位球面谐函数有解的条件是
  $$k^2=\ell(\ell+1),\ \ell = m, m+1, \ldots$$}
  \eitem
\ech
\end{frame}

\end{document}
