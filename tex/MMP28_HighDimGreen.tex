\documentclass[CJK]{beamer}
\usepackage{CJKutf8}
\usepackage{beamerthemesplit}
\usetheme{Malmoe}
\useoutertheme[footline=authortitle]{miniframes}
\usepackage{amsmath}
\usepackage{amssymb}
\usepackage{graphicx}
\usepackage{eufrak}
\usepackage{color}
\usepackage{slashed}
\usepackage{simplewick}
\usepackage{tikz}
\usepackage{tcolorbox}
\graphicspath{{../figures/}}
%%figures
\def\lfig#1#2{\includegraphics[width=#1 in]{#2}}
\def\addfig#1#2{\begin{center}\includegraphics[width=#1 in]{#2}\end{center}}
\def\wulian{\includegraphics[width=0.18in]{emoji_wulian.jpg}}
\def\bigwulian{\includegraphics[width=0.35in]{emoji_wulian.jpg}}
\def\bye{\includegraphics[width=0.18in]{emoji_bye.jpg}}
\def\bigbye{\includegraphics[width=0.35in]{emoji_bye.jpg}}
\def\huaixiao{\includegraphics[width=0.18in]{emoji_huaixiao.jpg}}
\def\bighuaixiao{\includegraphics[width=0.35in]{emoji_huaixiao.jpg}}
\def\jianxiao{\includegraphics[width=0.18in]{emoji_jianxiao.jpg}}
\def\bigjianxiao{\includegraphics[width=0.35in]{emoji_jianxiao.jpg}}
%% colors
\def\blacktext#1{{\color{black}#1}}
\def\bluetext#1{{\color{blue}#1}}
\def\redtext#1{{\color{red}#1}}
\def\darkbluetext#1{{\color[rgb]{0,0.2,0.6}#1}}
\def\skybluetext#1{{\color[rgb]{0.2,0.7,1.}#1}}
\def\cyantext#1{{\color[rgb]{0.,0.5,0.5}#1}}
\def\greentext#1{{\color[rgb]{0,0.7,0.1}#1}}
\def\darkgray{\color[rgb]{0.2,0.2,0.2}}
\def\lightgray{\color[rgb]{0.6,0.6,0.6}}
\def\gray{\color[rgb]{0.4,0.4,0.4}}
\def\blue{\color{blue}}
\def\red{\color{red}}
\def\green{\color{green}}
\def\darkgreen{\color[rgb]{0,0.4,0.1}}
\def\darkblue{\color[rgb]{0,0.2,0.6}}
\def\skyblue{\color[rgb]{0.2,0.7,1.}}
%%control
\def\be{\begin{equation}}
\def\ee{\nonumber\end{equation}}
\def\bea{\begin{eqnarray}}
\def\eea{\nonumber\end{eqnarray}}
\def\bch{\begin{CJK}{UTF8}{gbsn}}
\def\ech{\end{CJK}}
\def\bitem{\begin{itemize}}
\def\eitem{\end{itemize}}
\def\bcenter{\begin{center}}
\def\ecenter{\end{center}}
\def\bex{\begin{minipage}{0.2\textwidth}\includegraphics[width=0.6in]{jugelizi.png}\end{minipage}\begin{minipage}{0.76\textwidth}}
\def\eex{\end{minipage}}
\def\chtitle#1{\frametitle{\bch#1\ech}}
\def\bmat#1{\left(\begin{array}{#1}}
\def\emat{\end{array}\right)}
\def\bcase#1{\left\{\begin{array}{#1}}
\def\ecase{\end{array}\right.}
\def\bmini#1{\begin{minipage}{#1\textwidth}}
\def\emini{\end{minipage}}
\def\tbox#1{\begin{tcolorbox}#1\end{tcolorbox}}
\def\pfrac#1#2#3{\left(\frac{\partial #1}{\partial #2}\right)_{#3}}
%%symbols
\def\bropt{\,(\ \ \ )}
\def\sone{$\star$}
\def\stwo{$\star\star$}
\def\sthree{$\star\star\star$}
\def\sfour{$\star\star\star\star$}
\def\sfive{$\star\star\star\star\star$}
\def\rint{{\int_\leftrightarrow}}
\def\roint{{\oint_\leftrightarrow}}
\def\stdHf{{\textit{\r H}_f}}
\def\deltaH{{\Delta \textit{\r H}}}
\def\ii{{\dot{\imath}}}
\def\skipline{{\vskip0.1in}}
\def\skiplines{{\vskip0.2in}}
\def\lagr{{\mathcal{L}}}
\def\hamil{{\mathcal{H}}}
\def\vecv{{\mathbf{v}}}
\def\vecx{{\mathbf{x}}}
\def\vecy{{\mathbf{y}}}
\def\veck{{\mathbf{k}}}
\def\vecp{{\mathbf{p}}}
\def\vecn{{\mathbf{n}}}
\def\vecA{{\mathbf{A}}}
\def\vecP{{\mathbf{P}}}
\def\vecsigma{{\mathbf{\sigma}}}
\def\hatJn{{\hat{J_\vecn}}}
\def\hatJx{{\hat{J_x}}}
\def\hatJy{{\hat{J_y}}}
\def\hatJz{{\hat{J_z}}}
\def\hatj#1{\hat{J_{#1}}}
\def\hatphi{{\hat{\phi}}}
\def\hatq{{\hat{q}}}
\def\hatpi{{\hat{\pi}}}
\def\vel{\upsilon}
\def\Dint{{\mathcal{D}}}
\def\adag{{\hat{a}^\dagger}}
\def\bdag{{\hat{b}^\dagger}}
\def\cdag{{\hat{c}^\dagger}}
\def\ddag{{\hat{d}^\dagger}}
\def\hata{{\hat{a}}}
\def\hatb{{\hat{b}}}
\def\hatc{{\hat{c}}}
\def\hatd{{\hat{d}}}
\def\hatN{{\hat{N}}}
\def\hatH{{\hat{H}}}
\def\hatp{{\hat{p}}}
\def\Fup{{F^{\mu\nu}}}
\def\Fdown{{F_{\mu\nu}}}
\def\newl{\nonumber \\}
\def\vece{\mathrm{e}}
\def\calM{{\mathcal{M}}}
\def\calT{{\mathcal{T}}}
\def\calR{{\mathcal{R}}}
\def\barpsi{\bar{\psi}}
\def\baru{\bar{u}}
\def\barv{\bar{\upsilon}}
\def\qeq{\stackrel{?}{=}}
\def\torder#1{\mathcal{T}\left(#1\right)}
\def\rorder#1{\mathcal{R}\left(#1\right)}
\def\contr#1#2{\contraction{}{#1}{}{#2}#1#2}
\def\trof#1{\mathrm{Tr}\left(#1\right)}
\def\trace{\mathrm{Tr}}
\def\comm#1{\ \ \ \left(\mathrm{used}\ #1\right)}
\def\tcomm#1{\ \ \ (\text{#1})}
\def\slp{\slashed{p}}
\def\slk{\slashed{k}}
\def\calp{{\mathfrak{p}}}
\def\veccalp{\mathbf{\mathfrak{p}}}
\def\Tthree{T_{\tiny \textcircled{3}}}
\def\pthree{p_{\tiny \textcircled{3}}}
\def\dbar{{\,\mathchar'26\mkern-12mu d}}
\def\erf{\mathrm{erf}}
\def\const{\mathrm{constant}}
\def\pheat{\pfrac p{\ln T}V}
\def\vheat{\pfrac V{\ln T}p}
%%units
\def\fdeg{{^\circ \mathrm{F}}}
\def\cdeg{^\circ \mathrm{C}}
\def\atm{\,\mathrm{atm}}
\def\angstrom{\,\text{\AA}}
\def\SIL{\,\mathrm{L}}
\def\SIkm{\,\mathrm{km}}
\def\SIyr{\,\mathrm{yr}}
\def\SIGyr{\,\mathrm{Gyr}}
\def\SIV{\,\mathrm{V}}
\def\SImV{\,\mathrm{mV}}
\def\SIeV{\,\mathrm{eV}}
\def\SIkeV{\,\mathrm{keV}}
\def\SIMeV{\,\mathrm{MeV}}
\def\SIGeV{\,\mathrm{GeV}}
\def\SIcal{\,\mathrm{cal}}
\def\SIkcal{\,\mathrm{kcal}}
\def\SImol{\,\mathrm{mol}}
\def\SIN{\,\mathrm{N}}
\def\SIHz{\,\mathrm{Hz}}
\def\SIm{\,\mathrm{m}}
\def\SIcm{\,\mathrm{cm}}
\def\SIfm{\,\mathrm{fm}}
\def\SImm{\,\mathrm{mm}}
\def\SInm{\,\mathrm{nm}}
\def\SImum{\,\mathrm{\mu m}}
\def\SIJ{\,\mathrm{J}}
\def\SIW{\,\mathrm{W}}
\def\SIkJ{\,\mathrm{kJ}}
\def\SIs{\,\mathrm{s}}
\def\SIkg{\,\mathrm{kg}}
\def\SIg{\,\mathrm{g}}
\def\SIK{\,\mathrm{K}}
\def\SImmHg{\,\mathrm{mmHg}}
\def\SIPa{\,\mathrm{Pa}}

\def\courseurl{https://github.com/zqhuang/SYSU\_TD}

\def\tpage#1#2{
\begin{frame}
\begin{center}
\begin{Large}
\bch
热学 \\
第#1讲 #2

{\vskip 0.3in}

黄志琦

\ech
\end{Large}
\end{center}

\vskip 0.2in

\bch
教材:《热学》第二版,赵凯华,罗蔚茵,高等教育出版社
\ech

\bch
课件下载
\ech
\courseurl
\end{frame}
}

\def\bfr#1{
\begin{frame}
\chtitle{#1} 
\bch
}

\def\efr{
\ech 
\end{frame}
}

  \date{}
  \begin{document}
  \bch
\tpage{26}{*格林函数进阶知识}

\begin{frame}
\frametitle{本讲内容}

\tableofcontents

\end{frame}

\section{Green's Function: Heat Equation}

\begin{frame}
  \frametitle{瞬时点热源}
  
  我们讨论过一维无界区域的瞬时点热源的扩散问题:
  
  $$\frac{\partial u}{\partial t} - a\frac{\partial^2 u}{\partial x^2}=0.$$
  $$\left. u \right\vert_{t=0}= \delta(x-x_0).$$

  这是个格林函数问题,容易看出其解为:
  $$  G_{\rm 1D}(t, x; x_0) = \frac{1}{2\pi}\int_{-\infty}^\infty e^{ik(x-x_0)-ak^2t}dk = \frac{1}{\sqrt{4\pi at}} e^{-\frac{(x-x_0)^2}{4at}} . $$

  (它很容易推广到高维空间: $$G_{\rm 2D}(t, x,y;x_0,y_0) = G_{\rm 1D}(t, x; x_0)G_{\rm 1D}(t, y; y_0)$$
  $$G_{\rm 3D}(t, x,y,z;x_0,y_0,z_0) = G_{\rm 1D}(t, x; x_0)G_{\rm 1D}(t, y; y_0)G_{\rm 1D}(t, z; z_0)$$
  等)
  
\end{frame}

\begin{frame}
  \frametitle{时间平移}
  如果瞬时点热源是$t=\tau$时刻放的,
  $$\frac{\partial u}{\partial t} - a\frac{\partial^2 u}{\partial x^2}=0.$$
  $$\left. u \right\vert_{t=\tau}= \delta(x-x_0).$$
  则只需要做一下时间平移:
  $$ u  = \frac{1}{\sqrt{4\pi a(t-\tau)}} e^{-\frac{(x-x_0)^2}{4a(t-\tau)}}  $$
  ($t>\tau$.)

\end{frame}


\begin{frame}
  \frametitle{持续点热源}
  如果有持续的点热源,
  $$\frac{\partial u}{\partial t} - a\frac{\partial^2 u}{\partial x^2}=\delta(x-x_0).$$
  $$\left. u \right\vert_{t=0}=0.$$
  这可以看成 $\tau=0$ 时刻开始每隔 $d\tau$ 时间就放一个点热源 $\delta(x-x_0)d\tau$。在$t$时刻,只有$0\le \tau\le t$时刻的热源才有贡献。因此问题的解为:
  $$ u(x,t) = \int_0^t d\tau  \frac{1}{\sqrt{4\pi a(t-\tau)}} e^{-\frac{(x-x_0)^2}{4a(t-\tau)}}.$$
  我们来检验一下这个解——
\end{frame}


\begin{frame}
  \frametitle{检验解}
  把$t-\tau$换为$\tau$,
  $$ u(x,t) = \int_0^t d\tau  \frac{1}{\sqrt{4\pi a\tau}} e^{-\frac{(x-x_0)^2}{4a\tau}}.$$
  积分号下求导
  $$\frac{\partial u}{\partial t}= \frac{1}{\sqrt{4\pi at}} e^{-\frac{(x-x_0)^2}{4at}}.$$
  $$a\frac{\partial^2 u}{\partial x^2} = \int_0^t d\tau  \left[\frac{(x-x_0)^2}{4a\tau^2}-\frac{1}{2\tau} \right]\frac{1}{\sqrt{4\pi a\tau}} e^{-\frac{(x-x_0)^2}{4a\tau}} = \left.\frac{1}{\sqrt{4\pi a\tau}} e^{-\frac{(x-x_0)^2}{4a\tau}} \right\vert_0^t.$$
  于是有
  $$\frac{\partial u}{\partial t} - a\frac{\partial^2 u}{\partial x^2}= \lim_{\tau\rightarrow 0^+}\frac{1}{\sqrt{4\pi a\tau}} e^{-\frac{(x-x_0)^2}{4a\tau}} = \delta(x-x_0).$$
\end{frame}



\begin{frame}
  \frametitle{持续点热源的高维情况}
  很容易看出,二维和三维空间的解为
  $$ u(x,t) = \int_0^t d\tau  \frac{1}{\sqrt{4\pi a(t-\tau)}} e^{-\frac{(x-x_0)^2-(y-y_0)^2}{4a(t-\tau)}}.$$
  $$ u(x,t) = \int_0^t d\tau  \frac{1}{\sqrt{4\pi a(t-\tau)}} e^{-\frac{(x-x_0)^2-(y-y_0)^2-(z-z_0)^2}{4a(t-\tau)}}.$$
  
\end{frame}


\section{Green's Function: Wave Equation}

\begin{frame}
  \frametitle{一维瞬时波源(位移)}
  考虑一维波动方程的瞬时点源问题
  $$\frac{\partial^2u}{\partial t^2}-a^2\frac{\partial^2u }{\partial x^2} = 0, $$
  $$\left. u\right\vert_{t=0} = \delta(x-x_0),$$
  $$\left. \frac{\partial u}{\partial t}\right\vert_{t=0} = 0.$$  
  利用一维无边界波动方程的通解,得到
  $$ u(x,t) = \frac{1}{2}\left[\delta(x-at-x_0) + \delta(x+at-x_0)\right].$$
  也就是说脉冲会分成相等的两份以速度$a$向左右两边传播。
\end{frame}

\begin{frame}
  \frametitle{一维瞬时波源(速度)}
  如果脉冲是初始速度
  $$\frac{\partial^2u}{\partial t^2}-a^2\frac{\partial^2u }{\partial x^2} = 0, $$
  $$\left. u\right\vert_{t=0} = 0,$$
  $$\left. \frac{\partial u}{\partial t}\right\vert_{t=0} = \delta(x-x_0).$$  
  仍然利用一维无边界波动方程的通解,得到
  $$ u(x,t) = \frac{1}{2}\left[h(x+at-x_0) - h(x-at-x_0)\right].$$
  这里的 $h$ 是单位跃阶函数。


\end{frame}

\begin{frame}
  \frametitle{一维瞬时波源(速度)}
  解
  $$ u(x,t) = \frac{1}{2}\left[h(x+at-x_0) - h(x-at-x_0)\right].$$
  如下图所示:
  
  \addfig{3}{wave_green.png}

  这揭示一个很有意思的现象:瞬时速度可以导致整个因果关联区域(即波能传播到的地方)都有信号。
\end{frame}


\begin{frame}
  \frametitle{二维瞬时波源(速度)}
  考虑二维的情况
  $$\frac{\partial^2u}{\partial t^2}-a^2\nabla^2 u = 0, $$
  $$\left. u\right\vert_{t=0} = 0,$$
  $$\left. \frac{\partial u}{\partial t}\right\vert_{t=0} = \delta(\mathbf{x}-\mathbf{x}_0).$$  
  以$x_0$为原点建立极坐标 $(r,\theta)$ , 根据对称性,可以知道 $u$ 只依赖于 $r$,不失一般性:

  $$ u(r,t) = \int_0^\infty  c(k) J_0(kr)\sin(akt) k dk.  $$
  (这里我们已经利用 $u|_{t=0}=0$ 排除了 $\cos(akt)$ 的成份。)
\end{frame}


\begin{frame}
  \frametitle{二维瞬时波源(速度)}
  代入另一个初始条件,得到
  $$ \int_0^\infty  c(k) J_0(kr) ak^2 dk = \frac{\delta(r)}{2\pi r}.  $$
  利用无边界问题中贝塞尔函数的正交公式,可以看出
  $$ c(k) = \frac{1}{2\pi ak} $$
  $$ u(r,t) = \frac{1}{2\pi a} \int_0^\infty  J_0(kr) \sin(akt)  dk.  $$
  这个解还能进一步简化:
\end{frame}


\begin{frame}
  \frametitle{二维瞬时波源(速度)}
  利用 $J_0$的积分表示,可以得到
  $$ u(r,t) = \frac{1}{8\pi^2ai} \int_0^\infty  \int_{-\pi}^{\pi} \left(e^{ik[r(\sin\theta+i\epsilon)+at]}- e^{ik[r(\sin\theta+i\epsilon)-at]}\right)d\theta dk.  $$
  这里为了让积分收敛,加入了耍赖参数 $\epsilon\rightarrow 0^+$;先对$k$积分得到
  $$ u(r,t) = \frac{1}{4\pi^2}\int_{-\pi}^{\pi} \frac{t}{a^2t^2-r^2(\sin\theta + i\epsilon)^2 } d\theta.$$
\end{frame}

\begin{frame}
  \frametitle{二维瞬时波源(速度)}
  把积分转化为单位圆上的围道积分,
  $$ u(r,t) = \frac{1}{4\pi^2ar}\oint_{|z|=1}\ \left[\frac{1}{z^2 +2(i\lambda-\epsilon)z - 1}-\frac{1}{z^2 -2(i\lambda+\epsilon)z - 1 } \right]dz.$$
  这里的 $\lambda = at/r$。

\end{frame}


\begin{frame}
  \frametitle{二维瞬时波源(速度)}
  如果$\lambda< 1$,则记 $$\alpha_{\pm} = \epsilon-i\lambda \pm \sqrt{1-\lambda^2-2i\lambda \epsilon},\ \beta_{\pm} = \epsilon+i\lambda \pm \sqrt{1-\lambda^2 + 2i\lambda\epsilon}$$
  $$ u(r,t) = \frac{1}{4\pi^2ar}\oint_{|z|=1}\ \left[\frac{1}{(z-\alpha_+)(z-\alpha_-)}-\frac{1}{(z-\beta_+)(z-\beta_-)}\right]dz.$$
  孤立奇点$\alpha_-$, $\beta_-$在单位圆 $|z|=1$ 内部(现在你知道为什么我一直保留$\epsilon$了),留数之和为
  $$\frac{1}{\alpha_--\alpha_+}-\frac{1}{\beta_--\beta_+} = 0. $$
  也就是说,在区域 $r> at$ 内, $u = 0$ —— 物理上来看这是显然的。
\end{frame}

\begin{frame}
  \frametitle{二维瞬时波源(速度)}
  如果 $\lambda>1$,则记 $$\alpha_{\pm} = \left(-\lambda \pm \sqrt{\lambda^2-1}\right)i,\ \beta_{\pm} = \left(\lambda \pm \sqrt{\lambda^2-1}\right)i$$
  $$ u(r,t) = \frac{1}{4\pi^2ar}\oint_{|z|=1}\ \left[\frac{1}{(z-\alpha_+)(z-\alpha_-)}-\frac{1}{(z-\beta_+)(z-\beta_-)}\right]dz.$$
  孤立奇点$\alpha_+$, $\beta_-$在单位圆 $|z|=1$ 内部,由留数定理得
  $$ u(r,t) = \frac{2\pi i}{4\pi^2ar}\left(\frac{1}{\alpha_+-\alpha_-}-\frac{1}{\beta_--\beta_+}\right) = \frac{1}{2\pi a\sqrt{a^2t^2-r^2}}. $$
\end{frame}


\begin{frame}
  \frametitle{二维瞬时波源(速度)}
  总结起来,瞬时速度源产生的信号也是恰好覆盖到以信号源为圆心,$at$为半径的圆内:
  $$ u(r,t) = \frac{h(at-r)}{2\pi a\sqrt{a^2t^2-r^2}}.$$
  这里的 $h$ 是单位跃阶函数, $r=|\mathbf{x}-\mathbf{x}_0|$。
\end{frame}


\begin{frame}
  \frametitle{二维持续波源}
  考虑二维的持续波源问题:
  $$\frac{\partial^2u}{\partial t^2}-a^2\nabla^2 u = \delta(\mathbf{x}-\mathbf{x}_0), $$
  $$\left. u\right\vert_{t=0} = 0,$$
  $$\left. \frac{\partial u}{\partial t}\right\vert_{t=0} = 0.$$
  这可以看成从  $t =0$ 时刻开始,每隔 $dt$ 都放入瞬时(速度)波源 $\delta(\mathbf{x}-\mathbf{x}_0) dt$,因此问题的解为:
  $$u(r,t) = \left\{
  \begin{array}{ll}
    0, & \text{ if } r>at \\
    \frac{1}{2\pi a}\int_{r/a}^t \frac{d\tau}{\sqrt{a^2\tau^2-r^2}}=\frac{1}{2\pi a^2}\mathrm{acosh}\frac{at}{r}. & \text{ if } r< at.
  \end{array}
  \right.$$
  这里的 $r = |\mathbf{x}-\mathbf{x}_0|$.
\end{frame}


\ech
\end{document}
