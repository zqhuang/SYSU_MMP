\documentclass[CJK]{beamer}
\usepackage{CJKutf8}
\usepackage{beamerthemesplit}
\usetheme{Malmoe}
\useoutertheme[footline=authortitle]{miniframes}
\usepackage{amsmath}
\usepackage{amssymb}
\usepackage{graphicx}
\usepackage{eufrak}
\usepackage{color}
\usepackage{slashed}
\usepackage{simplewick}
\usepackage{tikz}
\usepackage{tcolorbox}
\graphicspath{{../figures/}}
%%figures
\def\lfig#1#2{\includegraphics[width=#1 in]{#2}}
\def\addfig#1#2{\begin{center}\includegraphics[width=#1 in]{#2}\end{center}}
\def\wulian{\includegraphics[width=0.18in]{emoji_wulian.jpg}}
\def\bigwulian{\includegraphics[width=0.35in]{emoji_wulian.jpg}}
\def\bye{\includegraphics[width=0.18in]{emoji_bye.jpg}}
\def\bigbye{\includegraphics[width=0.35in]{emoji_bye.jpg}}
\def\huaixiao{\includegraphics[width=0.18in]{emoji_huaixiao.jpg}}
\def\bighuaixiao{\includegraphics[width=0.35in]{emoji_huaixiao.jpg}}
\def\jianxiao{\includegraphics[width=0.18in]{emoji_jianxiao.jpg}}
\def\bigjianxiao{\includegraphics[width=0.35in]{emoji_jianxiao.jpg}}
%% colors
\def\blacktext#1{{\color{black}#1}}
\def\bluetext#1{{\color{blue}#1}}
\def\redtext#1{{\color{red}#1}}
\def\darkbluetext#1{{\color[rgb]{0,0.2,0.6}#1}}
\def\skybluetext#1{{\color[rgb]{0.2,0.7,1.}#1}}
\def\cyantext#1{{\color[rgb]{0.,0.5,0.5}#1}}
\def\greentext#1{{\color[rgb]{0,0.7,0.1}#1}}
\def\darkgray{\color[rgb]{0.2,0.2,0.2}}
\def\lightgray{\color[rgb]{0.6,0.6,0.6}}
\def\gray{\color[rgb]{0.4,0.4,0.4}}
\def\blue{\color{blue}}
\def\red{\color{red}}
\def\green{\color{green}}
\def\darkgreen{\color[rgb]{0,0.4,0.1}}
\def\darkblue{\color[rgb]{0,0.2,0.6}}
\def\skyblue{\color[rgb]{0.2,0.7,1.}}
%%control
\def\be{\begin{equation}}
\def\ee{\nonumber\end{equation}}
\def\bea{\begin{eqnarray}}
\def\eea{\nonumber\end{eqnarray}}
\def\bch{\begin{CJK}{UTF8}{gbsn}}
\def\ech{\end{CJK}}
\def\bitem{\begin{itemize}}
\def\eitem{\end{itemize}}
\def\bcenter{\begin{center}}
\def\ecenter{\end{center}}
\def\bex{\begin{minipage}{0.2\textwidth}\includegraphics[width=0.6in]{jugelizi.png}\end{minipage}\begin{minipage}{0.76\textwidth}}
\def\eex{\end{minipage}}
\def\chtitle#1{\frametitle{\bch#1\ech}}
\def\bmat#1{\left(\begin{array}{#1}}
\def\emat{\end{array}\right)}
\def\bcase#1{\left\{\begin{array}{#1}}
\def\ecase{\end{array}\right.}
\def\bmini#1{\begin{minipage}{#1\textwidth}}
\def\emini{\end{minipage}}
\def\tbox#1{\begin{tcolorbox}#1\end{tcolorbox}}
\def\pfrac#1#2#3{\left(\frac{\partial #1}{\partial #2}\right)_{#3}}
%%symbols
\def\bropt{\,(\ \ \ )}
\def\sone{$\star$}
\def\stwo{$\star\star$}
\def\sthree{$\star\star\star$}
\def\sfour{$\star\star\star\star$}
\def\sfive{$\star\star\star\star\star$}
\def\rint{{\int_\leftrightarrow}}
\def\roint{{\oint_\leftrightarrow}}
\def\stdHf{{\textit{\r H}_f}}
\def\deltaH{{\Delta \textit{\r H}}}
\def\ii{{\dot{\imath}}}
\def\skipline{{\vskip0.1in}}
\def\skiplines{{\vskip0.2in}}
\def\lagr{{\mathcal{L}}}
\def\hamil{{\mathcal{H}}}
\def\vecv{{\mathbf{v}}}
\def\vecx{{\mathbf{x}}}
\def\vecy{{\mathbf{y}}}
\def\veck{{\mathbf{k}}}
\def\vecp{{\mathbf{p}}}
\def\vecn{{\mathbf{n}}}
\def\vecA{{\mathbf{A}}}
\def\vecP{{\mathbf{P}}}
\def\vecsigma{{\mathbf{\sigma}}}
\def\hatJn{{\hat{J_\vecn}}}
\def\hatJx{{\hat{J_x}}}
\def\hatJy{{\hat{J_y}}}
\def\hatJz{{\hat{J_z}}}
\def\hatj#1{\hat{J_{#1}}}
\def\hatphi{{\hat{\phi}}}
\def\hatq{{\hat{q}}}
\def\hatpi{{\hat{\pi}}}
\def\vel{\upsilon}
\def\Dint{{\mathcal{D}}}
\def\adag{{\hat{a}^\dagger}}
\def\bdag{{\hat{b}^\dagger}}
\def\cdag{{\hat{c}^\dagger}}
\def\ddag{{\hat{d}^\dagger}}
\def\hata{{\hat{a}}}
\def\hatb{{\hat{b}}}
\def\hatc{{\hat{c}}}
\def\hatd{{\hat{d}}}
\def\hatN{{\hat{N}}}
\def\hatH{{\hat{H}}}
\def\hatp{{\hat{p}}}
\def\Fup{{F^{\mu\nu}}}
\def\Fdown{{F_{\mu\nu}}}
\def\newl{\nonumber \\}
\def\vece{\mathrm{e}}
\def\calM{{\mathcal{M}}}
\def\calT{{\mathcal{T}}}
\def\calR{{\mathcal{R}}}
\def\barpsi{\bar{\psi}}
\def\baru{\bar{u}}
\def\barv{\bar{\upsilon}}
\def\qeq{\stackrel{?}{=}}
\def\torder#1{\mathcal{T}\left(#1\right)}
\def\rorder#1{\mathcal{R}\left(#1\right)}
\def\contr#1#2{\contraction{}{#1}{}{#2}#1#2}
\def\trof#1{\mathrm{Tr}\left(#1\right)}
\def\trace{\mathrm{Tr}}
\def\comm#1{\ \ \ \left(\mathrm{used}\ #1\right)}
\def\tcomm#1{\ \ \ (\text{#1})}
\def\slp{\slashed{p}}
\def\slk{\slashed{k}}
\def\calp{{\mathfrak{p}}}
\def\veccalp{\mathbf{\mathfrak{p}}}
\def\Tthree{T_{\tiny \textcircled{3}}}
\def\pthree{p_{\tiny \textcircled{3}}}
\def\dbar{{\,\mathchar'26\mkern-12mu d}}
\def\erf{\mathrm{erf}}
\def\const{\mathrm{constant}}
\def\pheat{\pfrac p{\ln T}V}
\def\vheat{\pfrac V{\ln T}p}
%%units
\def\fdeg{{^\circ \mathrm{F}}}
\def\cdeg{^\circ \mathrm{C}}
\def\atm{\,\mathrm{atm}}
\def\angstrom{\,\text{\AA}}
\def\SIL{\,\mathrm{L}}
\def\SIkm{\,\mathrm{km}}
\def\SIyr{\,\mathrm{yr}}
\def\SIGyr{\,\mathrm{Gyr}}
\def\SIV{\,\mathrm{V}}
\def\SImV{\,\mathrm{mV}}
\def\SIeV{\,\mathrm{eV}}
\def\SIkeV{\,\mathrm{keV}}
\def\SIMeV{\,\mathrm{MeV}}
\def\SIGeV{\,\mathrm{GeV}}
\def\SIcal{\,\mathrm{cal}}
\def\SIkcal{\,\mathrm{kcal}}
\def\SImol{\,\mathrm{mol}}
\def\SIN{\,\mathrm{N}}
\def\SIHz{\,\mathrm{Hz}}
\def\SIm{\,\mathrm{m}}
\def\SIcm{\,\mathrm{cm}}
\def\SIfm{\,\mathrm{fm}}
\def\SImm{\,\mathrm{mm}}
\def\SInm{\,\mathrm{nm}}
\def\SImum{\,\mathrm{\mu m}}
\def\SIJ{\,\mathrm{J}}
\def\SIW{\,\mathrm{W}}
\def\SIkJ{\,\mathrm{kJ}}
\def\SIs{\,\mathrm{s}}
\def\SIkg{\,\mathrm{kg}}
\def\SIg{\,\mathrm{g}}
\def\SIK{\,\mathrm{K}}
\def\SImmHg{\,\mathrm{mmHg}}
\def\SIPa{\,\mathrm{Pa}}

\def\courseurl{https://github.com/zqhuang/SYSU\_TD}

\def\tpage#1#2{
\begin{frame}
\begin{center}
\begin{Large}
\bch
热学 \\
第#1讲 #2

{\vskip 0.3in}

黄志琦

\ech
\end{Large}
\end{center}

\vskip 0.2in

\bch
教材:《热学》第二版,赵凯华,罗蔚茵,高等教育出版社
\ech

\bch
课件下载
\ech
\courseurl
\end{frame}
}

\def\bfr#1{
\begin{frame}
\chtitle{#1} 
\bch
}

\def\efr{
\ech 
\end{frame}
}

  \date{}
  \begin{document}
  \bch
\tpage{13}{Wavelet Transform}

\begin{frame}
  \tableofcontents
\end{frame}

\section{Motivation}

\secpage{为什么需要小波分析}{不同时间段的信号特征不一样}

\begin{frame}
  \frametitle{实际世界的波}
  
  数学家喜欢研究正弦波,余弦波,各种各样画出来很漂亮的波。

  \skipline
  
  但实际世界的波往往是长这样的:信号在某时出现,过了一会儿消失。信号的频率一般偏高。
  
  \addfig{2}{wavelet_example.png}

  对了,往往还要加上一堆噪音——你很有可能用肉眼根本看不到这段信号在哪里。
  
\end{frame}


\begin{frame}
  \frametitle{怎么分析局域的波}
  
  如果知道信号在哪个时间段出现,截出该时间段进行频谱(傅立叶)分析就解决问题了。

  \addfig{2}{wavelet_example2.png}
  
  但实际信号处理往往伴随着大量的噪音,无法预判信号在哪个时间段出现,怎么寻找信号并分析其特征呢?

\end{frame}


\begin{frame}
  \frametitle{从傅立叶变换到小波分析}

  傅立叶变换:
  
  频率 $\rightarrow$ 该频率成分的大小

  \skiplines


  小波分析:
  
  (频率,时间段) $\rightarrow $ 在该时间段内该频率成分的大小

  \skiplines
  

{\bf  小波分析其实就是扫描各个时间段的频谱信息。当扫描低频信息时,时间段的间隔可以取得大些;当扫描高频信息时,时间段的间隔要取得小些。 } 
\end{frame}


\begin{frame}
  \frametitle{小波分析的本质}
  傅立叶变换的本质是把一个函数分解为一堆正交归一化的函数$\frac{e^{i\omega t}}{\sqrt{2\pi}}$的线性组合。

  \skiplines
  
  小波变换呢,当然也是把一个函数分解为一堆正交归一化的函数$\psi(t; \tau, \omega)$ 的线性组合。这里$\tau$代表时间段的位置,$\omega$代表圆频率。
\end{frame}


\begin{frame}
  \frametitle{离散小波分析}

  先不谈$\psi(t;\tau,\omega)$如何构造,讲真,这堆函数太多了。毕竟有无数个可以连续变化的$\tau$和$\omega$组合起来让人有些头疼。(好吧,数学家不头疼,物理学家头疼。)

  \skipline
  
  虽然数学家十分热衷于研究连续的小波变换;物理学家们实际用的小波变换往往是离散的:$\psi(t; \tau_n, \omega_m)$或者简单记成$\psi_{n,m}(t)$。

  \skipline
  
  当然,如何合适地选取$\tau_n,\omega_m$以及$\psi$函数的形式,使得这组展开的基“足够完备”(可以对物理信号做足够好的近似),是{\bf\large 一个需要详细讨论的复杂问题。}
  
\end{frame}


\begin{frame}
  \frametitle{但是——}

  我们这么养生的课,怎么可能讨论复杂的问题!

  \addfig{1.5}{kaisen.jpg}

  直接给出大家讨论好的结果就可以了!
\end{frame}

\section{Haar Wavelet}
\secpage{Haar小波}{简单,但有点毛糙}

\begin{frame}
  \frametitle{Haar小波的母函数}

  第一个介绍的是应用很广泛的 Haar 小波。

  不要着急,我们一个个来:先给出$\psi_{0,0}(t)$(它称为小波分析的母函数)


  \bmini{0.5}
  \be
  \psi_{0,0}(t) = \left\{
    \begin{array}{ll}
      1 & \text{, if } 0\le t < \frac{1}{2} \\
      -1 & \text{, if } \frac{1}{2} \le t < 1 \\
      0 & \text{, otherwise } \\      
    \end{array}
    \right. \nonumber
    \ee
    \emini
    \bmini{0.45}
    \addfig{1.8}{Haar.pdf}
    \emini
\end{frame}


\begin{frame}
  \frametitle{任意$m, n$}
  $\psi_{n,m}$可以直接从母函数生成

  $$\psi_{n,m} = 2^{n/2}\psi_{0,0}(2^n t - m).$$

  这其实就是把母函数进行平移和缩放。

  \skiplines
  
  因为Haar小波都是不连续的函数,可以期待近似出来的函数比较“毛糙”。如果你只在意函数数值大小,不大在意函数是否光滑,那么Haar小波是不错的选择。
\end{frame}


\thinka {证明Haar小波基是正交归一化的,即
  $$\int_{-\infty}^\infty \psi_{n,m}(t)\psi_{n',m'}(t) dt= \delta_{nn'}\delta_{mm'}.$$}



\begin{frame}
  \frametitle{截断和展开系数}

  我们可以把信号按Haar基进行展开。

  $$f(t) = \sum_{n,m}c_{n,m}\psi_{n,m}(t).$$
  你可以直接用求内积的方法进行投影,得到任意函数$f(t)$的$\psi_{n,m}$分量为:
  $$c_{n,m} = \int_{-\infty}^\infty f(t) \psi_{n,m}(t) dt. $$
  数值计算中存在更便捷的算法,这些就留到你的数值计算课上去学习吧。


  \skiplines
  
  实际问题都有一定频率和时间范围,所以可以做个$n,m$的截断,只要计算有限项的系数就可以了。
\end{frame}



\section{Daubechies Wavelet}
\secpage{Daubechies小波}{稍复杂,但更连续光滑}


\begin{frame}
  \frametitle{母函数的选取}
  由Haar母函数生成的:
  $$\psi_{n,m} = 2^{n/2}\psi_{0,0}(2^n t - m).$$
  是正交归一化的函数组。这看起来有些神奇。

  这样的“神奇函数”是唯一的吗?
\end{frame}


\begin{frame}
  \frametitle{Daubechies系列}
  寻找产生正交小波的母函数并不容易,相当长时间内大家都只知道Haar这么一个。有一天Daubechies突然搞出来一堆:

  \bmini{0.75}
  \lfig{3.}{Daubechies.pdf}
  \emini
  \bmini{0.2}
  1阶的Daubechies就是Haar,高阶的Daubechies则是连续的,越高阶光滑性越好。
  \emini
\end{frame}


\begin{frame}
  \frametitle{怎么计算Daubechies小波函数}
  高阶的Daubechies小波函数并没有解析表达式,所以只能数值计算。

  \skiplines
  
  在Mathematica里有内置函数
  
  WavePsi[DaubechiesWavelet[n],t]

  其中$n$是阶数;$t$是时间变量。

  \skiplines
  
  也可以用C或者Fortran自己写低阶的Daubechies小波函数,这些内容留到数值计算课上去学习吧。
  
\end{frame}


\begin{frame}
  \frametitle{时间序列的Daubechies小波分析}
  原则上讲,只要小波函数都是正交归一化的,各个小波成份的系数都可以通过用小波函数和待分解函数作内积(即投影)得到。

  \skiplines
  
  对于离散的时间序列,Daubechies小波分析就只要做一些简单的求和(相当于求内积的离散版本)。这些快速的算法会在时间序列分析等一些工科学科中接触到。
  
\end{frame}



\section{Homework}

\begin{frame}
  \frametitle{Homework}
  
  \bitem
\item{把函数$\frac{\sin t}{t}$分解为Haar小波的线性组合。取合适的阶段使误差处处不大于$10^{-3}$.}
  \eitem
  
\end{frame}



\ech
\end{document}
