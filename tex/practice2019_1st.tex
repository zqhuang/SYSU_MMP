\documentclass[10pt,CJK]{article}
\usepackage{geometry}
\input{reduced_macros.tex}
\geometry{tmargin=0.3in, bmargin=0.5in, lmargin=0.5in, rmargin=0.9in, nohead, nofoot}
\def\mark#1{{\color{blue} (#1分)}}
\renewcommand{\thepage}{}
\begin{document}
\bch
{\large 数学物理方法2019 课堂小测I 本卷总计100分(30签到分+70卷面分)}

{\vskip 0.06in}

姓名 ....................... {\hskip 0.5in}    学号 .......................{\hskip 0.5in}  分数 ...................

{\vskip 0.1in}

{\bf 门票区(20分)} 请认真审题,不要手滑——如果{\bf 门票区错误答案个数超过1个,整卷答题无效}(此处有微笑表情)

{\vskip 0.02in}

  \bitem
\item[(1)]{如果把一个圆锥体的底面半径和高都增加$1\%$,则圆锥体的体积大约增加了多少?   \hfill (\ \ )
  
  \optlist{$1\%$}{$2\%$}{$3\%$}{$4\%$} }
\item[(2)]{周长为1的轮子在平地上滚一圈,轮子边缘上的固定点的轨迹的长度是多少? \hfill  (\ \ )


\optlist{$\pi$}{$\frac{4}{\pi}$}{$\frac{8}{\pi}$}{$8$} }
\item[(3)]{$e^{\frac{2\pi i}{3}}=?$ \hfill  (\ \ )


\optlist{$\frac{\sqrt{3}-i}{2}$}{$\frac{1-\sqrt{3}i}{2}$}{$\frac{-\sqrt{3}+i}{2}$}{$\frac{-1+\sqrt{3}i}{2}$} }
\item[(4)]{积分 $\int_{-\infty}^\infty  \delta(x+\pi) \cos x\, dx$ 等于 \hfill (\ \ )

\optlist{$-1$}{$0$}{$1$}{$\pi$} }
\item[(5)]{两个矢量的外积的大小等于它们 \hfill (\ \ )

\optlist{长度的乘积}{长度的和}{张成的平行四边形面积}{张成的三角形的面积} }
  \eitem
  
  {\vskip 0.03in}

{\bf 景点一:柏松 (10分)}

{\vskip 0.03in}

$f(x)=e^{-|x|}$的傅立叶变换是 $F(k)=$ \uline{3}  

{\vskip 0.06in}

{\bf 景点二:送风亭 (10分)}

{\vskip 0.03in}

设某$X$-$Y$-$Z$坐标系的任意相近两点$(X,Y,Z)$和$(X+dX, Y+dY, Z+dZ)$之间的距离$ds$,如果近似到$dX,dY,dZ$的二次项,可以写为
$$ ds = \sqrt{dX^2+e^{2X}(dY^2+dZ^2)}$$
写出在$X$-$Y$-$Z$坐标系的拉普拉斯算符的具体的(对$X$,$Y$,$Z$的)微分表达式:

{\vskip 0.15in}

$\nabla^2f = $ \uline{5}

{\vskip 0.06in}

{\bf 景点三:凉亭 (10分)}

{\vskip 0.03in}

比较大小:$\int_{-1}^1dx\int_{-1}^1dy\int_{-1}^1 (1+xyz)^{1/3}dz$ \uline{0.5} $\int_{-1}^1dx\int_{-1}^1dy\int_{-1}^1 (1+xyz)^{2/3}dz$ (填写$>$, $<$ 或 $=$)

  {\vskip 0.06in}

{\bf 景点四:梦泉 (10分)}

{\vskip 0.03in}

对厄米矩阵$A$,定义“矩阵的指数函数”(其实是矩阵到矩阵的映射)为无穷级数和:
$$ e^A \equiv \sum_{n=0}^\infty \frac{1}{n!}A^n = I + A + \frac{1}{2}A^2 + \frac{1}{6}A^3+...$$
其中$I$为同阶的单位矩阵。现假设$A$是一个5阶零迹(即所有对角元素之和为零)的厄米矩阵,则矩阵$e^A$的迹的最小可能值是多少?为什么?
  

\newpage


\

{\vskip 2in}

{\bf 景点五:新苔倚栅 (10分)}

一个高为$1$的封闭盒子,上下底面均为半径为$1$的圆。盒子的表面积的最小值是多少?

\ech
\end{document}
