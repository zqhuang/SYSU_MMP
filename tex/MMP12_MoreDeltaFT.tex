\documentclass[CJK]{beamer}
\usepackage{CJKutf8}
\usepackage{beamerthemesplit}
\usetheme{Malmoe}
\useoutertheme[footline=authortitle]{miniframes}
\usepackage{amsmath}
\usepackage{amssymb}
\usepackage{graphicx}
\usepackage{eufrak}
\usepackage{color}
\usepackage{slashed}
\usepackage{simplewick}
\usepackage{tikz}
\usepackage{tcolorbox}
\graphicspath{{../figures/}}
%%figures
\def\lfig#1#2{\includegraphics[width=#1 in]{#2}}
\def\addfig#1#2{\begin{center}\includegraphics[width=#1 in]{#2}\end{center}}
\def\wulian{\includegraphics[width=0.18in]{emoji_wulian.jpg}}
\def\bigwulian{\includegraphics[width=0.35in]{emoji_wulian.jpg}}
\def\bye{\includegraphics[width=0.18in]{emoji_bye.jpg}}
\def\bigbye{\includegraphics[width=0.35in]{emoji_bye.jpg}}
\def\huaixiao{\includegraphics[width=0.18in]{emoji_huaixiao.jpg}}
\def\bighuaixiao{\includegraphics[width=0.35in]{emoji_huaixiao.jpg}}
\def\jianxiao{\includegraphics[width=0.18in]{emoji_jianxiao.jpg}}
\def\bigjianxiao{\includegraphics[width=0.35in]{emoji_jianxiao.jpg}}
%% colors
\def\blacktext#1{{\color{black}#1}}
\def\bluetext#1{{\color{blue}#1}}
\def\redtext#1{{\color{red}#1}}
\def\darkbluetext#1{{\color[rgb]{0,0.2,0.6}#1}}
\def\skybluetext#1{{\color[rgb]{0.2,0.7,1.}#1}}
\def\cyantext#1{{\color[rgb]{0.,0.5,0.5}#1}}
\def\greentext#1{{\color[rgb]{0,0.7,0.1}#1}}
\def\darkgray{\color[rgb]{0.2,0.2,0.2}}
\def\lightgray{\color[rgb]{0.6,0.6,0.6}}
\def\gray{\color[rgb]{0.4,0.4,0.4}}
\def\blue{\color{blue}}
\def\red{\color{red}}
\def\green{\color{green}}
\def\darkgreen{\color[rgb]{0,0.4,0.1}}
\def\darkblue{\color[rgb]{0,0.2,0.6}}
\def\skyblue{\color[rgb]{0.2,0.7,1.}}
%%control
\def\be{\begin{equation}}
\def\ee{\nonumber\end{equation}}
\def\bea{\begin{eqnarray}}
\def\eea{\nonumber\end{eqnarray}}
\def\bch{\begin{CJK}{UTF8}{gbsn}}
\def\ech{\end{CJK}}
\def\bitem{\begin{itemize}}
\def\eitem{\end{itemize}}
\def\bcenter{\begin{center}}
\def\ecenter{\end{center}}
\def\bex{\begin{minipage}{0.2\textwidth}\includegraphics[width=0.6in]{jugelizi.png}\end{minipage}\begin{minipage}{0.76\textwidth}}
\def\eex{\end{minipage}}
\def\chtitle#1{\frametitle{\bch#1\ech}}
\def\bmat#1{\left(\begin{array}{#1}}
\def\emat{\end{array}\right)}
\def\bcase#1{\left\{\begin{array}{#1}}
\def\ecase{\end{array}\right.}
\def\bmini#1{\begin{minipage}{#1\textwidth}}
\def\emini{\end{minipage}}
\def\tbox#1{\begin{tcolorbox}#1\end{tcolorbox}}
\def\pfrac#1#2#3{\left(\frac{\partial #1}{\partial #2}\right)_{#3}}
%%symbols
\def\bropt{\,(\ \ \ )}
\def\sone{$\star$}
\def\stwo{$\star\star$}
\def\sthree{$\star\star\star$}
\def\sfour{$\star\star\star\star$}
\def\sfive{$\star\star\star\star\star$}
\def\rint{{\int_\leftrightarrow}}
\def\roint{{\oint_\leftrightarrow}}
\def\stdHf{{\textit{\r H}_f}}
\def\deltaH{{\Delta \textit{\r H}}}
\def\ii{{\dot{\imath}}}
\def\skipline{{\vskip0.1in}}
\def\skiplines{{\vskip0.2in}}
\def\lagr{{\mathcal{L}}}
\def\hamil{{\mathcal{H}}}
\def\vecv{{\mathbf{v}}}
\def\vecx{{\mathbf{x}}}
\def\vecy{{\mathbf{y}}}
\def\veck{{\mathbf{k}}}
\def\vecp{{\mathbf{p}}}
\def\vecn{{\mathbf{n}}}
\def\vecA{{\mathbf{A}}}
\def\vecP{{\mathbf{P}}}
\def\vecsigma{{\mathbf{\sigma}}}
\def\hatJn{{\hat{J_\vecn}}}
\def\hatJx{{\hat{J_x}}}
\def\hatJy{{\hat{J_y}}}
\def\hatJz{{\hat{J_z}}}
\def\hatj#1{\hat{J_{#1}}}
\def\hatphi{{\hat{\phi}}}
\def\hatq{{\hat{q}}}
\def\hatpi{{\hat{\pi}}}
\def\vel{\upsilon}
\def\Dint{{\mathcal{D}}}
\def\adag{{\hat{a}^\dagger}}
\def\bdag{{\hat{b}^\dagger}}
\def\cdag{{\hat{c}^\dagger}}
\def\ddag{{\hat{d}^\dagger}}
\def\hata{{\hat{a}}}
\def\hatb{{\hat{b}}}
\def\hatc{{\hat{c}}}
\def\hatd{{\hat{d}}}
\def\hatN{{\hat{N}}}
\def\hatH{{\hat{H}}}
\def\hatp{{\hat{p}}}
\def\Fup{{F^{\mu\nu}}}
\def\Fdown{{F_{\mu\nu}}}
\def\newl{\nonumber \\}
\def\vece{\mathrm{e}}
\def\calM{{\mathcal{M}}}
\def\calT{{\mathcal{T}}}
\def\calR{{\mathcal{R}}}
\def\barpsi{\bar{\psi}}
\def\baru{\bar{u}}
\def\barv{\bar{\upsilon}}
\def\qeq{\stackrel{?}{=}}
\def\torder#1{\mathcal{T}\left(#1\right)}
\def\rorder#1{\mathcal{R}\left(#1\right)}
\def\contr#1#2{\contraction{}{#1}{}{#2}#1#2}
\def\trof#1{\mathrm{Tr}\left(#1\right)}
\def\trace{\mathrm{Tr}}
\def\comm#1{\ \ \ \left(\mathrm{used}\ #1\right)}
\def\tcomm#1{\ \ \ (\text{#1})}
\def\slp{\slashed{p}}
\def\slk{\slashed{k}}
\def\calp{{\mathfrak{p}}}
\def\veccalp{\mathbf{\mathfrak{p}}}
\def\Tthree{T_{\tiny \textcircled{3}}}
\def\pthree{p_{\tiny \textcircled{3}}}
\def\dbar{{\,\mathchar'26\mkern-12mu d}}
\def\erf{\mathrm{erf}}
\def\const{\mathrm{constant}}
\def\pheat{\pfrac p{\ln T}V}
\def\vheat{\pfrac V{\ln T}p}
%%units
\def\fdeg{{^\circ \mathrm{F}}}
\def\cdeg{^\circ \mathrm{C}}
\def\atm{\,\mathrm{atm}}
\def\angstrom{\,\text{\AA}}
\def\SIL{\,\mathrm{L}}
\def\SIkm{\,\mathrm{km}}
\def\SIyr{\,\mathrm{yr}}
\def\SIGyr{\,\mathrm{Gyr}}
\def\SIV{\,\mathrm{V}}
\def\SImV{\,\mathrm{mV}}
\def\SIeV{\,\mathrm{eV}}
\def\SIkeV{\,\mathrm{keV}}
\def\SIMeV{\,\mathrm{MeV}}
\def\SIGeV{\,\mathrm{GeV}}
\def\SIcal{\,\mathrm{cal}}
\def\SIkcal{\,\mathrm{kcal}}
\def\SImol{\,\mathrm{mol}}
\def\SIN{\,\mathrm{N}}
\def\SIHz{\,\mathrm{Hz}}
\def\SIm{\,\mathrm{m}}
\def\SIcm{\,\mathrm{cm}}
\def\SIfm{\,\mathrm{fm}}
\def\SImm{\,\mathrm{mm}}
\def\SInm{\,\mathrm{nm}}
\def\SImum{\,\mathrm{\mu m}}
\def\SIJ{\,\mathrm{J}}
\def\SIW{\,\mathrm{W}}
\def\SIkJ{\,\mathrm{kJ}}
\def\SIs{\,\mathrm{s}}
\def\SIkg{\,\mathrm{kg}}
\def\SIg{\,\mathrm{g}}
\def\SIK{\,\mathrm{K}}
\def\SImmHg{\,\mathrm{mmHg}}
\def\SIPa{\,\mathrm{Pa}}

\def\courseurl{https://github.com/zqhuang/SYSU\_TD}

\def\tpage#1#2{
\begin{frame}
\begin{center}
\begin{Large}
\bch
热学 \\
第#1讲 #2

{\vskip 0.3in}

黄志琦

\ech
\end{Large}
\end{center}

\vskip 0.2in

\bch
教材:《热学》第二版,赵凯华,罗蔚茵,高等教育出版社
\ech

\bch
课件下载
\ech
\courseurl
\end{frame}
}

\def\bfr#1{
\begin{frame}
\chtitle{#1} 
\bch
}

\def\efr{
\ech 
\end{frame}
}

  \date{}
  \begin{document}
  \bch
\tpage{12}{Advanced Topics on $\delta$ Function and Fourier Transform}

\begin{frame}
  \tableofcontents
\end{frame}

\section{Delta function}
\secpage{$\delta$函数的高级性质}{$$ \delta\left(\alpha(x)\right)  = \sum_{\rm roots}  \frac{\delta(x-x_i)}{|\alpha'(x_i)|}$$}


\begin{frame}
  \frametitle{再复习一下$\delta$函数的积分表示}

  \tbox{$$\frac{1}{2\pi}\int_{-\infty}^\infty e^{ikx}dk = \delta(x).$$}

  在大量的物理问题中会用到它。
\end{frame}


\begin{frame}
  \frametitle{用围道积分的证明方法}
  \bea
  && \frac{1}{2\pi}\int_{-\infty}^\infty e^{ikx}dk \newl  
  &=& \lim_{\epsilon\rightarrow 0^+} \frac{1}{2\pi}\int_{-\infty}^\infty e^{ikx-\frac{\epsilon}{2}k^2}dk \newl
  &=& \lim_{\epsilon\rightarrow 0^+}\frac{1}{2\pi}\int_{-\infty}^\infty e^{-\frac{\epsilon}{2}(k-\frac{ix}{\epsilon})^2-\frac{x^2}{2\epsilon }}dk \newl
  &=&  \lim_{\epsilon\rightarrow 0^+}\frac{1}{\sqrt{2\pi\epsilon}}e^{-\frac{ x^2}{2\epsilon}} \newl  
  &=& \delta(x)
  \eea
\end{frame}

\begin{frame}
  \frametitle{$n$维空间的$\delta$函数的积分表示}
  
  \tbox{
   $$ \frac{1}{(2\pi)^n} \int e^{i\veck\cdot\vecx} d^n\veck =  \delta^{(n)}(\vecx)$$
   }

\end{frame}


\begin{frame}
  \frametitle{$\delta$函数的导函数}
  
  利用$\delta$函数在两边都是零的特点,可以用分部积分的方法得到$\delta$函数的导数的性质:
  {\blue
  $$\int_{-\infty}^\infty \left[\frac{d^n}{dx^n}\delta(x-x_0)\right] f(x)\, dx = (-1)^n f^{(n)}(x_0)$$}
  
\end{frame}

\begin{frame}
  \frametitle{思考题}
  
  计算积分
  $$\int_{-\infty}^{\infty}   \delta'(x-\frac{\pi}{2})\cos x\,dx $$
  
\end{frame}



\begin{frame}
  \frametitle{$\delta$函数的终极大招}
  
  用变量替换的方法可以得到:\tbox{
    $$ \delta\left(\alpha(x)\right)  = \sum_{\rm roots}  \frac{\delta(x-x_i)}{|\alpha'(x_i)|}$$}
  求和对所有$\alpha(x)$的根$x_1$, $x_2$, \ldots 进行。

   {\scriptsize (请思考当$\alpha(x)$没有根或者有重根的情形上面的等式会如何。)}

  \skiplines
  
 {\small 另一种等价的写法是:
  $$\int_{-\infty}^\infty \delta\left(\alpha(x)\right) f(x) \,dx = \sum_{\rm roots} \frac{f(x_i)}{|\alpha'(x_i)|}.$$}
  
  
\end{frame}


\begin{frame}
  \frametitle{思考题}
  计算积分
  $$\int_{-\infty}^\infty \delta(x^2-1) e^x\, dx$$
\end{frame}




\section{Fourier Transform}
\secpage{傅立叶变换的高级性质}{$$ (f\star g)(\vecx) \equiv \frac{1}{(2\pi)^{n/2}} \int f(\vecy)\,g(\vecx - \vecy)\, d^n\vecy  $$}

\begin{frame}
  \frametitle{$\nabla \rightarrow \ii \veck$}
  
  对傅立叶逆变换式
  $$ f(\vecx) = \frac{1}{(2\pi)^{n/2}}\int \widetilde{f}(\veck) e^{\ii \veck \cdot \vecx} d^n\veck. $$
  两边作用$\partial_j$
  $$ \partial_j f(\vecx) =  \frac{1}{(2\pi)^{n/2}}\int (\ii k_j \widetilde{f}(\veck)) e^{\ii \veck \cdot \vecx} d^n\veck. $$
  当$j$取遍$1,2,\ldots, n$,上式可以写成矢量形式:
  {\blue  $$ \widetilde{\ \nabla f\ } = \ii \veck \widetilde{f} $$}
  在量子力学里,我们常常说:位置空间的算符$-\ii\nabla$对应于(动量空间的)动量$\veck$。
\end{frame}


\begin{frame}
  \frametitle{思考题}
      位置空间的拉普拉斯算符 $\nabla^2$ 在傅立叶空间对应什么?

\end{frame}


\begin{frame}
  \frametitle{卷积}
  
  
  两个函数$f(\vecx)$和$g(\vecx)$的卷积定义为
  \tbox{$$ (f\star g)(\vecx) \equiv \frac{1}{(2\pi)^{n/2}} \int f(\vecy)\,g(\vecx - \vecy)\, d^n\vecy  $$}

  显然,卷积满足交换律: {\blue $f\star g = g\star f$}。

  更重要的是卷积定理: 设$f,g$的傅立叶变换依次为$\widetilde{f},\widetilde{g}$,则$f\star g$的傅立叶变换为$\widetilde{f}\widetilde{g}$,即
  \tbox{卷积的傅立叶变换等于傅立叶变换的乘积}
  
 \end{frame}


\begin{frame}
  \frametitle{思考题}
  
  请用666的操作证明卷积定理。
  
\end{frame}


\begin{frame}
  \frametitle{思考题}

  定义函数
  $$ f(x) = \int_{x-1}^{x+1}e^{-t^2}dt $$
  求$f(x)$ 的傅立叶变换。
  
 \end{frame}



\begin{frame}
  \frametitle{思考题}

  如果在傅立叶空间也定义卷积:
  \tbox{$$ (\widetilde{f}\star \widetilde{g})(\veck) \equiv \frac{1}{(2\pi)^{n/2}} \int \widetilde{f}(\vecp)\,\widetilde{g}(\veck - \vecp)\, d^n\vecp  $$}

  那么是否有
  $$\widetilde{\,fg\,} = \widetilde{f}\star \widetilde{g} $$
  即乘积的傅立叶变换是否等于傅立叶变换的卷积?
  
\end{frame}




\section{Homework}

\begin{frame}
  \frametitle{Homework}
  
  \bitem
\item{计算积分$$\int_{-\infty}^\infty \delta\left(\sin x\right) e^{-|x|} dx.$$ }
\item{定义函数
  $$ F(k) = \int_0^\infty e^{-x+ik(x^2-1)} dx, $$
  计算积分
  $$ \int_{-\infty}^\infty F(k) dk. $$}
\item{把二维空间的$\delta$函数写成积分表示,然后换到极坐标下重新写一遍这个积分表示。}  
  \eitem
  
\end{frame}



\ech
\end{document}
