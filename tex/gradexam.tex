\documentclass[12pt,CJK]{article}
\usepackage{geometry}
\input{reduced_macros.tex}
\geometry{tmargin=0.6in, bmargin=0.5in, lmargin=0.5in, rmargin=0.9in, nohead, nofoot}
\def\mark#1{{\color{blue} (#1分)}}
\renewcommand{\thepage}{}
\begin{document}
\bch
{\large 物理与天文学院2019年研究生硕转博试题 (总计100分)}

{\vskip 0.2in}

姓名 ....................... {\hskip 0.5in}    学号 .......................{\hskip 0.5in}  分数 ...................

{\vskip 0.1in}

\bitem
\item[(一)]{简答题,请简要回答每个小题(每题答案不得超过50字)。本大题共42分。
    \bitem
  \item[(1)]{请举出一个能量形式发生变化,但总能量守恒的例子?\mark{7}
      
        \vspace{1.2in}
      }
    \item[(2)]{请举出一个角动量守恒的例子?\mark{7}
        
        \vspace{1.2in}
      }
    \item[(3)]{在真空中自由悬浮的水滴是什么形状的?为什么?\mark{7}
        
        \vspace{1.2in}
      }
    \item[(4)]{请简要解释为什么小碗里的汤比大碗里的汤凉得快。\mark{7}

        
        \vspace{1.2in}
      }
    \item[(5)]{金属达到静电平衡时,其内部的电场分布有什么特点?\mark{7}
        
        \vspace{1.2in}
      }
    \item[(6)]{为什么水中的物体看起来比实际要浅?\mark{7}
        
        \vspace{1.2in}
      }
      \eitem
    }
\item[(二)]{某运动员往斜上方扔垒球。垒球出手时,离地面的高度为 $2\mathrm{m}$,速度大小为 $18\mathrm{m/s}$,方向和水平面成 $30$ 度角。忽略空气阻力,计算垒球落地点和出手点的水平距离。重力加速度取 $g=10\mathrm{m/s^2}$. \mark{15}

    \vspace{4.4in}
  }
\item[(三)]{
    初始压强为 $10^5\mathrm{Pa}$ 的理想气体,经历等温过程,体积缓慢地从 $1\mathrm{m^3}$ 膨胀到 $2\mathrm{m^3}$,问这个过程中理想气体对外做功多少? \mark{13}

    \vspace{4.4in}    
  }
\item[(四)]{一根均匀细杆底部铰接于水平地面上的固定点。杆可以绕铰接点无摩擦转动。让杆从静止的竖直直立状态开始,自由地倒下,倒地时角速度为 $\omega$。如果砍掉杆的上半部分,让剩下的只有原先一半长的杆重复上述过程,杆倒地时的角速度是多少? \mark{10}

    \vspace{5in}
  }
\item[(五)]{
    如图,地面(视为水平无限大半平面)的电势为零,墙(视为竖直的无限大半平面,和地面接触处绝缘)的电势为 $12\mathrm{V}$。
    
    在离墙角水平距离为 $1\mathrm{m}$,竖直距离为 $\sqrt{3}\mathrm{m}$ 的点$A$处,电势为多少$\mathrm{V}$? \mark{10}

    \includegraphics[width=1.5in]{problem_Lpot.png}    
    
    \vspace{2in}
    
  }
\item[(六)]{

    如图,用两根长度均为 $\ell$ 的理想刚性轻杆和两个质量均为 $m$ 的质点连接成一个复摆。设所有连接处均可无摩擦转动,且整个复摆只能在图示平面内活动。本地重力加速度$g$的方向竖直向下。
    
    一开始缓慢拉中间的质点,使得上面的杆和竖直线成 $\theta\ll 1$ 的小角度,下面的杆保持竖直(如图所示)。在 $t=0$ 时刻由静止释放该复摆。在 $t=1\mathrm{s}$ 时,上面的杆第一次经过竖直线(即瞬时状态为竖直方向);在 $t=2\mathrm{s}$ 时,上面的杆第二次经过竖直线。
    
    问:$t$ 为多少时,上面的杆第三次经过竖直线? \mark{10}    

    \includegraphics[width=1.2in]{pendulum.png}

   
    
  }
\eitem    


\ech
\end{document}
