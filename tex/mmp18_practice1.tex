\documentclass[12pt,CJK]{article}
\usepackage{geometry}
\input{reduced_macros.tex}
\geometry{tmargin=0.3in, bmargin=0.5in, lmargin=0.5in, rmargin=0.9in, nohead, nofoot}
\def\mark#1{{\color{blue} (#1 分)}}
\renewcommand{\thepage}{}
\begin{document}
\bch

 《数学物理方法》课堂小测 I \hspace{0.2in}  姓名 \uline{1.}  学号 \uline{1.}  分数 \uline{0.5}


\bitem
\item[1. ]{ 在复平面上的圆盘 $\{z:\ |z|\le 2\}$ 内随机取一个点$z$, 计算$|z|+|z-1|>3$的概率。\mark{10} {\vskip 3in} }  
\item[2. ]{ 复变函数 $f(z) = e^{|z|}-|z|$ 在复平面上哪里是可导的?导数为多少?\mark{15} {\vskip 3in}}  
\item[3. ]{已知 $f(z)$ 和 $g(z)$ 都在全复平面上解析, 且对任何整数 $n$ 都有 $f(n)=g(n)$,则 $f$ 和 $g$ 在整个复平面上一定处处相等吗? 如果一定,请给出证明,否则请举出一个反例。\mark{15} {\vskip 2.9in}}
\item[4. ]{计算复变函数 $f(z) = \frac{1}{z^{100}+1}$ 在孤立奇点 $z=e^{\frac{\pi\ii}{100}}$ 处的留数。\mark{20} {\vskip 3in}}         
\item[5. ]{计算积分$\int_0^\infty \frac{dx}{1+x^\pi}$. \mark{20}{\vskip 3in}}
\item[6. ]{定义函数 $F(k) = \int_0^\infty e^{-x}\cos{\left[k(x^4-1)\right]}\, dx$,计算 $\int_{-\infty}^\infty F(k)\,dk$. \mark{20} {\vskip 2.8in}}           
\eitem    


\ech
\end{document}
