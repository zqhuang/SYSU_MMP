\documentclass[CJK]{beamer}
\usepackage{CJKutf8}
\usepackage{beamerthemesplit}
\usetheme{Malmoe}
\useoutertheme[footline=authortitle]{miniframes}
\usepackage{amsmath}
\usepackage{amssymb}
\usepackage{graphicx}
\usepackage{eufrak}
\usepackage{color}
\usepackage{slashed}
\usepackage{simplewick}
\usepackage{tikz}
\usepackage{tcolorbox}
\graphicspath{{../figures/}}
%%figures
\def\lfig#1#2{\includegraphics[width=#1 in]{#2}}
\def\addfig#1#2{\begin{center}\includegraphics[width=#1 in]{#2}\end{center}}
\def\wulian{\includegraphics[width=0.18in]{emoji_wulian.jpg}}
\def\bigwulian{\includegraphics[width=0.35in]{emoji_wulian.jpg}}
\def\bye{\includegraphics[width=0.18in]{emoji_bye.jpg}}
\def\bigbye{\includegraphics[width=0.35in]{emoji_bye.jpg}}
\def\huaixiao{\includegraphics[width=0.18in]{emoji_huaixiao.jpg}}
\def\bighuaixiao{\includegraphics[width=0.35in]{emoji_huaixiao.jpg}}
\def\jianxiao{\includegraphics[width=0.18in]{emoji_jianxiao.jpg}}
\def\bigjianxiao{\includegraphics[width=0.35in]{emoji_jianxiao.jpg}}
%% colors
\def\blacktext#1{{\color{black}#1}}
\def\bluetext#1{{\color{blue}#1}}
\def\redtext#1{{\color{red}#1}}
\def\darkbluetext#1{{\color[rgb]{0,0.2,0.6}#1}}
\def\skybluetext#1{{\color[rgb]{0.2,0.7,1.}#1}}
\def\cyantext#1{{\color[rgb]{0.,0.5,0.5}#1}}
\def\greentext#1{{\color[rgb]{0,0.7,0.1}#1}}
\def\darkgray{\color[rgb]{0.2,0.2,0.2}}
\def\lightgray{\color[rgb]{0.6,0.6,0.6}}
\def\gray{\color[rgb]{0.4,0.4,0.4}}
\def\blue{\color{blue}}
\def\red{\color{red}}
\def\green{\color{green}}
\def\darkgreen{\color[rgb]{0,0.4,0.1}}
\def\darkblue{\color[rgb]{0,0.2,0.6}}
\def\skyblue{\color[rgb]{0.2,0.7,1.}}
%%control
\def\be{\begin{equation}}
\def\ee{\nonumber\end{equation}}
\def\bea{\begin{eqnarray}}
\def\eea{\nonumber\end{eqnarray}}
\def\bch{\begin{CJK}{UTF8}{gbsn}}
\def\ech{\end{CJK}}
\def\bitem{\begin{itemize}}
\def\eitem{\end{itemize}}
\def\bcenter{\begin{center}}
\def\ecenter{\end{center}}
\def\bex{\begin{minipage}{0.2\textwidth}\includegraphics[width=0.6in]{jugelizi.png}\end{minipage}\begin{minipage}{0.76\textwidth}}
\def\eex{\end{minipage}}
\def\chtitle#1{\frametitle{\bch#1\ech}}
\def\bmat#1{\left(\begin{array}{#1}}
\def\emat{\end{array}\right)}
\def\bcase#1{\left\{\begin{array}{#1}}
\def\ecase{\end{array}\right.}
\def\bmini#1{\begin{minipage}{#1\textwidth}}
\def\emini{\end{minipage}}
\def\tbox#1{\begin{tcolorbox}#1\end{tcolorbox}}
\def\pfrac#1#2#3{\left(\frac{\partial #1}{\partial #2}\right)_{#3}}
%%symbols
\def\bropt{\,(\ \ \ )}
\def\sone{$\star$}
\def\stwo{$\star\star$}
\def\sthree{$\star\star\star$}
\def\sfour{$\star\star\star\star$}
\def\sfive{$\star\star\star\star\star$}
\def\rint{{\int_\leftrightarrow}}
\def\roint{{\oint_\leftrightarrow}}
\def\stdHf{{\textit{\r H}_f}}
\def\deltaH{{\Delta \textit{\r H}}}
\def\ii{{\dot{\imath}}}
\def\skipline{{\vskip0.1in}}
\def\skiplines{{\vskip0.2in}}
\def\lagr{{\mathcal{L}}}
\def\hamil{{\mathcal{H}}}
\def\vecv{{\mathbf{v}}}
\def\vecx{{\mathbf{x}}}
\def\vecy{{\mathbf{y}}}
\def\veck{{\mathbf{k}}}
\def\vecp{{\mathbf{p}}}
\def\vecn{{\mathbf{n}}}
\def\vecA{{\mathbf{A}}}
\def\vecP{{\mathbf{P}}}
\def\vecsigma{{\mathbf{\sigma}}}
\def\hatJn{{\hat{J_\vecn}}}
\def\hatJx{{\hat{J_x}}}
\def\hatJy{{\hat{J_y}}}
\def\hatJz{{\hat{J_z}}}
\def\hatj#1{\hat{J_{#1}}}
\def\hatphi{{\hat{\phi}}}
\def\hatq{{\hat{q}}}
\def\hatpi{{\hat{\pi}}}
\def\vel{\upsilon}
\def\Dint{{\mathcal{D}}}
\def\adag{{\hat{a}^\dagger}}
\def\bdag{{\hat{b}^\dagger}}
\def\cdag{{\hat{c}^\dagger}}
\def\ddag{{\hat{d}^\dagger}}
\def\hata{{\hat{a}}}
\def\hatb{{\hat{b}}}
\def\hatc{{\hat{c}}}
\def\hatd{{\hat{d}}}
\def\hatN{{\hat{N}}}
\def\hatH{{\hat{H}}}
\def\hatp{{\hat{p}}}
\def\Fup{{F^{\mu\nu}}}
\def\Fdown{{F_{\mu\nu}}}
\def\newl{\nonumber \\}
\def\vece{\mathrm{e}}
\def\calM{{\mathcal{M}}}
\def\calT{{\mathcal{T}}}
\def\calR{{\mathcal{R}}}
\def\barpsi{\bar{\psi}}
\def\baru{\bar{u}}
\def\barv{\bar{\upsilon}}
\def\qeq{\stackrel{?}{=}}
\def\torder#1{\mathcal{T}\left(#1\right)}
\def\rorder#1{\mathcal{R}\left(#1\right)}
\def\contr#1#2{\contraction{}{#1}{}{#2}#1#2}
\def\trof#1{\mathrm{Tr}\left(#1\right)}
\def\trace{\mathrm{Tr}}
\def\comm#1{\ \ \ \left(\mathrm{used}\ #1\right)}
\def\tcomm#1{\ \ \ (\text{#1})}
\def\slp{\slashed{p}}
\def\slk{\slashed{k}}
\def\calp{{\mathfrak{p}}}
\def\veccalp{\mathbf{\mathfrak{p}}}
\def\Tthree{T_{\tiny \textcircled{3}}}
\def\pthree{p_{\tiny \textcircled{3}}}
\def\dbar{{\,\mathchar'26\mkern-12mu d}}
\def\erf{\mathrm{erf}}
\def\const{\mathrm{constant}}
\def\pheat{\pfrac p{\ln T}V}
\def\vheat{\pfrac V{\ln T}p}
%%units
\def\fdeg{{^\circ \mathrm{F}}}
\def\cdeg{^\circ \mathrm{C}}
\def\atm{\,\mathrm{atm}}
\def\angstrom{\,\text{\AA}}
\def\SIL{\,\mathrm{L}}
\def\SIkm{\,\mathrm{km}}
\def\SIyr{\,\mathrm{yr}}
\def\SIGyr{\,\mathrm{Gyr}}
\def\SIV{\,\mathrm{V}}
\def\SImV{\,\mathrm{mV}}
\def\SIeV{\,\mathrm{eV}}
\def\SIkeV{\,\mathrm{keV}}
\def\SIMeV{\,\mathrm{MeV}}
\def\SIGeV{\,\mathrm{GeV}}
\def\SIcal{\,\mathrm{cal}}
\def\SIkcal{\,\mathrm{kcal}}
\def\SImol{\,\mathrm{mol}}
\def\SIN{\,\mathrm{N}}
\def\SIHz{\,\mathrm{Hz}}
\def\SIm{\,\mathrm{m}}
\def\SIcm{\,\mathrm{cm}}
\def\SIfm{\,\mathrm{fm}}
\def\SImm{\,\mathrm{mm}}
\def\SInm{\,\mathrm{nm}}
\def\SImum{\,\mathrm{\mu m}}
\def\SIJ{\,\mathrm{J}}
\def\SIW{\,\mathrm{W}}
\def\SIkJ{\,\mathrm{kJ}}
\def\SIs{\,\mathrm{s}}
\def\SIkg{\,\mathrm{kg}}
\def\SIg{\,\mathrm{g}}
\def\SIK{\,\mathrm{K}}
\def\SImmHg{\,\mathrm{mmHg}}
\def\SIPa{\,\mathrm{Pa}}

\def\courseurl{https://github.com/zqhuang/SYSU\_TD}

\def\tpage#1#2{
\begin{frame}
\begin{center}
\begin{Large}
\bch
热学 \\
第#1讲 #2

{\vskip 0.3in}

黄志琦

\ech
\end{Large}
\end{center}

\vskip 0.2in

\bch
教材:《热学》第二版,赵凯华,罗蔚茵,高等教育出版社
\ech

\bch
课件下载
\ech
\courseurl
\end{frame}
}

\def\bfr#1{
\begin{frame}
\chtitle{#1} 
\bch
}

\def\efr{
\ech 
\end{frame}
}

  \date{}
\begin{document}
\tpage{17}{Bessel Functions of the First Kind II}

\begin{frame}
\chtitle{本讲内容}
\bch
\bitem
\item{回顾:圆形薄膜振动问题}
\item{推广:圆盘上的热传导问题}  
\item{ $J_m(x)$的积分表示}
\item{$J_m(x)$在无穷远处的渐进展开}
\item{无限大板上的热传导问题}      
\eitem
\ech
\end{frame}


\section{Review}

\secpage{圆形薄膜振动问题的回顾}{ 记$J_m(x)$第$i$个正实数根为$\mu_i$,则$$\int_0^1 \,x\,J_m(\mu_ix)J_m(\mu_j x) \,dx = \delta_{ij} \frac{\left[J_{m+1}(\mu_i)\right]^2}{2} .$$}

\begin{frame}
  \chtitle{例题1:圆形薄膜振动问题的一般初始条件}
  \bch
  \addfig{1.}{drum.jpg}
  考虑边界固定,半径为$R$的圆形薄膜的横向小振动问题。在$t=0$时刻的初始位移为$f(r,\theta)$ ,初始速度为$g(r,\theta)$。其中$(r,\theta)$是以圆盘中心为原点建立的极坐标。$f$和$g$都满足边界条件:$f(r,\theta+2\pi) = f(r,\theta)$,$g(r,\theta+2\pi)=g(r,\theta)$,$f(R,\theta) = g(R,\theta) = 0$。求解之后薄膜的振动。
    \ech
\end{frame}

\begin{frame}
  \chtitle{写出方程和边界条件}
  \bch
  设位移为$u(r,\theta,t)$。

  \bea
  \frac{\partial^2u}{\partial t^2} - a^2\nabla^2 u = 0 , \newl
  \left.u\right\vert_{r=R} = 0,\newl
  \left.u\right\vert_{t = 0} = f(r,\theta) , \newl
  \left.\frac{\partial u}{\partial t}\right\vert_{t = 0} = g(r,\theta) .
  \eea
  \ech
\end{frame}


\begin{frame}
  \chtitle{分离变量}
  \bch
  符合边界条件的分离变量形式解为:
  $$ J_m\left(\frac{\mu_{m,i}r}{R}\right)\cos{(m\theta)} \cos{\left(\frac{\mu_{m,i}at}{R}\right)},\ J_m\left(\frac{\mu_{m,i}r}{R}\right)\cos{(m\theta)} \sin{\left(\frac{\mu_{m,i}at}{R}\right)}, $$
  其中$m\in Z, \ \mu_{m,i}$是$J_m(x)$的第$i$个正实数根。 当$m>0$时还有
  $$ J_m\left(\frac{\mu_{m,i}r}{R}\right)\sin{(m\theta)} \cos{\left(\frac{\mu_{m,i}at}{R}\right)},\ J_m\left(\frac{\mu_{m,i}r}{R}\right)\sin{(m\theta)} \sin{\left(\frac{\mu_{m,i}at}{R}\right)}. $$
  
  \ech
\end{frame}


\begin{frame}
  \chtitle{级数展开}
  \bch
  完整的级数形式解为
  \bea
  u &=& \sum_{i=1}^\infty \left\{\frac{A_{0,i}}{2} J_0\left(\frac{\mu_{0,i}r}{R}\right) \cos{\left(\frac{\mu_{0,i}at}{R}\right)} + \frac{B_{0,i}}{2} J_0\left(\frac{\mu_{0,i}r}{R}\right) \sin{\left(\frac{\mu_{0,i}at}{R}\right)}\right. \newl
  && +\sum_{m=1}^\infty \left[ A_{m,i} J_m\left(\frac{\mu_{m,i}r}{R}\right)\cos{(m\theta)} \cos{\left(\frac{\mu_{m,i}at}{R}\right)} \right. \newl
  && + B_{m,i} J_m\left(\frac{\mu_{m,i}r}{R}\right)\cos{(m\theta)} \sin{\left(\frac{\mu_{m,i}at}{R}\right)} \newl
    && + C_{m,i} J_m\left(\frac{\mu_{m,i}r}{R}\right)\sin{(m\theta)} \cos{\left(\frac{\mu_{m,i}at}{R}\right)} \newl
      &&  +\left.\left. D_{m,i} J_m\left(\frac{\mu_{m,i}r}{R}\right)\sin{(m\theta)} \sin{\left(\frac{\mu_{m,i}at}{R}\right)} \right]\right\}
  \eea
  (在$A_{0,i}$和$B_{0,i}$里的$\frac{1}{2}$因子是为了和$m\ne 0$的结果看起来更一致。)
  \ech
\end{frame}



\begin{frame}
  \chtitle{初始位移条件}
  \bch
  利用初始位移条件得到
  {\small
  \bea
  f(Rx,\theta)  &=& \sum_{i=1}^\infty \frac{A_{0,i}}{2} J_0\left(\mu_{0,i}x\right)  \newl
  &+& \sum_{m=1}^\infty\sum_{i=1}^\infty \left[ A_{m,i} J_m\left(\mu_{m,i}x\right)\cos{(m\theta)} + C_{m,i} J_m\left(\mu_{m,i}x\right)\sin{(m\theta)}  \right]
  \eea }
  其中$x = \frac{r}{R}$。

  由谐函数的正交性得到
  $$ A_{m,i} = \frac{2}{\pi\left[J_{m+1}(\mu_{m,i})\right]^2} \int_0^1 xJ_m(\mu_{m,i} x) dx\int_0^{2\pi} \cos{(m\theta)}   f(Rx, \theta) d\theta $$
  $$C_{m,i} = \frac{2}{\pi\left[J_{m+1}(\mu_{m,i})\right]^2} \int_0^1 xJ_m(\mu_{m,i} x) dx\int_0^{2\pi} \sin{(m\theta)}   f(Rx, \theta) d\theta $$
  \ech
\end{frame}


\begin{frame}
  \chtitle{初始速度条件}
  \bch
  利用初始速度条件得到
  {\small
  \bea
  g(Rx,\theta)  &=& \sum_{i=1}^\infty \frac{B_{0,i}\mu_{0,i}a}{2R} J_0\left(\mu_{0,i}x\right)  + \sum_{m=1}^\infty\sum_{i=1}^\infty \left[ \frac{B_{m,i}\mu_{m,i}a}{R} J_m\left(\mu_{m,i}x\right)\cos{(m\theta)} \right. \newl
    && \left. + \frac{D_{m,i}\mu_{m,i}a}{R} J_m\left(\mu_{m,i}x\right)\sin{(m\theta)}  \right]
  \eea }
  其中$x = \frac{r}{R}$。

  由谐函数的正交性得到
  $$ B_{m,i} = \frac{2R}{\pi \mu_{m,i}a\left[J_{m+1}(\mu_{m,i})\right]^2} \int_0^1 xJ_m(\mu_{m,i} x) dx\int_0^{2\pi} \cos{(m\theta)}   g(Rx, \theta) d\theta $$
  $$D_{m,i} = \frac{2R}{\pi\mu_{m,i}a\left[J_{m+1}(\mu_{m,i})\right]^2} \int_0^1 xJ_m(\mu_{m,i} x) dx\int_0^{2\pi} \sin{(m\theta)}   g(Rx, \theta) d\theta $$
  \ech
\end{frame}


\section{Heat Equation on a Disk}

\secpage{圆盘上的热传导问题}{记$J_m'(x)$的第$i$个正实数根为$\lambda_i$,则
    $$\int_0^1 x\, J_m(\lambda_i x)J_m(\lambda_j x) dx = \delta_{ij}\frac{\left(1-\frac{m^2}{\lambda_i^2}\right)\left[J_m(\lambda_i)\right]^2}{2} $$}

\begin{frame}
  \chtitle{例题2:圆盘上的热传导问题}
  \bch

  有孤立的,半径为$R$的均匀圆形金属薄片。初态$t=0$时刻金属片上距离中心$r$处的温度为
$$T(r) =T_0\left(1+\frac{r^2}{R^2}\right).$$
已知金属片质量密度为$\rho$,导热系数为$\lambda$,单位质量比热为$c$。求$t\ge 0$时圆盘上各点的温度。

  \ech
\end{frame}


\begin{frame}
  \chtitle{写出方程}
  \bch
\bea
\frac{\partial T}{\partial t}-a\nabla^2 T &=& 0, \newl
\left.\frac{\partial T}{\partial r}\right\vert_{r=R} &=& 0, \newl
\left.T\right\vert_{t=0} &=& T_0\left(1+\frac{r^2}{R^2}\right).
\eea
其中$a = \frac{\lambda}{\rho c}.$
  \ech
\end{frame}

\begin{frame}
  \chtitle{按谐函数展开}
  \bch
  一般的分离变量解为$J_m(kr)e^{\pm \ii m\theta}e^{-ak^2t}$。但因为初始条件旋转对称(不依赖于$\theta$),所以解只包含$m=0$,也就是$J_0(kr)e^{-ak^2t}$的形式。

    \skipline
    
    根据边界条件知道$k = \frac{\lambda_i}{R}$,其中$\lambda_i$是$J_0'(x)$的第$i$个正实数根。

    $\lambda_0 = 0,\ \lambda_1 = 3.8317,\ \lambda_2 = 7.0156,\ \lambda_3 = 10.1735, \ \lambda_4 = 13.3237, \lambda_5 = 16.4706 ,\ldots$
    我们把$\lambda_0=0$这个特殊情形包括了进来,代表的是常数项。
    {\small   (思考:以前解固定边界的圆形薄膜振动问题时为什么没有这样做?)}
    
    \skipline
    
    于是,按谐函数展开
    $$ T = \sum_{i=0}^\infty c_iJ_0\left(\frac{\lambda_ir}{R}\right)e^{-\frac{a\lambda_i^2t}{R^2}} .$$
  \ech
\end{frame}




\begin{frame}
  \chtitle{按谐函数展开}
  \bch
  从初始条件可以得到,
  $$ \sum_{i=0}^\infty c_iJ_0\left(\lambda_i x\right) = T_0\left(1+x^2\right), $$
  其中$x = \frac{r}{R}.$

  \skiplines
  
  看起来要求出$c_i$就要研究下$J_0(\lambda_ix)$的正交性。我们来考虑更一般的$J_m(\lambda_ix)$($\lambda_i$为$J'_m(x)$的正实数根)的正交性。
  \ech
\end{frame}


\begin{frame}
  \chtitle{第一类贝塞尔函数的第二个正交定理}
  \bch
  设$\lambda_i$为$J_m'(x)$的第$i$个正实数根,则
  \tbox{
    $$\int_0^1 x\, J_m(\lambda_i x)J_m(\lambda_j x) dx = \delta_{ij}\frac{\left(1-\frac{m^2}{\lambda_i^2}\right)\left[J_m(\lambda_i)\right]^2}{2} $$}
  当$m = 0$时,允许取$\lambda_0= 0$这个特殊情形,这时需要约定等式右边$\frac{m^2}{\lambda_0^2} = 0$。
  \ech
\end{frame}




\begin{frame}
  \chtitle{证明}
  \bch
贝塞尔函数满足贝塞尔方程
  $$ \frac{1}{x}\frac{d}{dx}\left[x\frac{d}{dx}J_m(x)\right] +\left(1-\frac{m^2}{x^2}\right)J_m(x) =0 .$$
  把$x$换成$\lambda_i x$,得到
  \be
    \frac{1}{x}\frac{d}{dx}\left[x\frac{d}{dx} J_m(\lambda_i x)\right] + \left(\lambda_i^2-\frac{m^2}{x^2}\right)J_m(\lambda_ix) = 0.
    \ee
    注意符号$\frac{d}{dx}J_m(\lambda_ix)$和$J_m'(\lambda_i x)$的不同:前者多$\lambda_i$的因子。

    \skipline
    
    {\small
    (如果你还没弄清楚,请看例子:由$f(x) = x^2$,可以得到$f'(x) = 2x$;把$x$换为$2x$可以得到$f(2x) = 4x^2$,$f'(2x) = 4x$;显然$\frac{d}{dx}f(2x) = 8x = 2 f'(2x)$。)
  }
  
  \ech
\end{frame}

\begin{frame}
  \chtitle{证明}
  \bch

  两边同乘以$xJ_m(\lambda_j x)$,得到
   \begin{equation}
     J_m(\lambda_jx)\frac{d}{dx}\left[x\frac{d}{dx} J_m(\lambda_i x)\right] + \left(\lambda_i^2-\frac{m^2}{x^2}\right)xJ_m(\lambda_ix)J_m(\lambda_jx) = 0. \label{eq1}
   \end{equation}
   交换$i$和$j$,得到
   \begin{equation}
     J_m(\lambda_ix)\frac{d}{dx}\left[x\frac{d}{dx} J_m(\lambda_j x)\right] + \left(\lambda_j^2-\frac{m^2}{x^2}\right)xJ_m(\lambda_ix)J_m(\lambda_jx) = 0. \label{eq2}
   \end{equation}
  \ech
\end{frame}


\begin{frame}
  \chtitle{证明(续)}
  \bch
  \eqref{eq1}减去\eqref{eq2},得到
  \begin{eqnarray}
 && \frac{d}{dx}\left[ xJ_m(\lambda_jx)\frac{d}{dx} J_m(\lambda_i x) - xJ_m(\lambda_ix)\frac{d}{dx} J_m(\lambda_j x) \right] \newl
  && + (\lambda_i^2-\lambda_j^2)xJ_m(\lambda_ix)J_m(\lambda_jx) = 0. \label{eq3}
  \end{eqnarray}
  如果$i\ne j$,两边从$0$到$1$积分即得到
  $$\int_0^1 xJ_m(\lambda_ix)J_m(\lambda_jx) dx= 0. $$
  \ech
\end{frame}


\begin{frame}
  \chtitle{证明(续)}
  \bch
  事实上,在推导\eqref{eq3}时,$\lambda_i$,$\lambda_j$可以为任何正实数,所以我们还可以取$\lambda_i$为$J_m'$的零点,而$\lambda_j = \lambda_i +\epsilon$。
    同样从$0$到$1$积分,得到
    $$ -(\lambda_i+\epsilon)J_m(\lambda_i) J_m'(\lambda_i+\epsilon)  -(2\lambda_i+\epsilon)\epsilon\int_0^1xJ_m(\lambda_ix)J_m[(\lambda_i+\epsilon)x]dx = 0 $$
    两边除以$\epsilon$并令 $\epsilon\rightarrow 0^+$
    $$ \int_0^1x\left[J_m(\lambda_ix)\right]^2dx = -\frac{J_m(\lambda_i)J_m''(\lambda_i)}{2} $$
    
  \ech
\end{frame}

\begin{frame}
  \chtitle{证明(续)}
  \bch
  在贝塞尔方程
  $$J_m''(x) + \frac{1}{x}J_m'(x) + \left(1-\frac{m^2}{x^2}\right)J_m(x) = 0$$
  中令$x = \lambda_i$,即得到
  $$J_m''(\lambda_i) = - \left(1-\frac{m^2}{\lambda_i^2}\right)J_m(\lambda_i).$$
  代回前面得到的结论就完成了$\lambda_i,\lambda_j \ne 0$情况下的证明。
  \ech
\end{frame}


\begin{frame}
  \chtitle{证明(续)}
  \bch
  最后我们考虑$m=0$,$\lambda_i=0$的情况。这时如果$\lambda_j\ne \lambda_i$,则
  

$$  \int_0^1 xJ_m(\lambda_ix)J_m(\lambda_jx) dx = \int_0^1xJ_0(\lambda_j x) dx
 = \left. \frac{1}{\lambda_j} xJ_1(\lambda_j x) \right\vert_0^1 =0 $$
我们用到了$J_0(0)=1$。在最后一步利用了$m=0$时的第二个递推关系: $J_0'(x) = - J_1(x)$,以及$J_0'(\lambda_j)=0$。

\skipline

  如果$\lambda_j=\lambda_i=0$,则可以直截了当地进行计算
  $$  \int_0^1 xJ_m(\lambda_ix)J_m(\lambda_jx) dx = \int_0^1 x dx  = \frac{1}{2} .$$
  这样我们对$m=0$, $\lambda_i=0$的特殊情况完成了证明。
  \ech
\end{frame}


\begin{frame}
  \chtitle{思考题}
  \bch
  \addfig{1}{think3.jpg}
  
  之前得到的结论
  $$ \int_0^1x\left[J_m(\lambda_ix)\right]^2dx = -\frac{J_m(\lambda_i)J_m''(\lambda_i)}{2} $$
  对$m=0$, $\lambda_0=0$的特殊情形成立吗?
  
  \ech
\end{frame}

\begin{frame}
  \chtitle{回到原问题}
  \bch
  在
  $$ \sum_i c_iJ_0\left(\lambda_i x\right) = T_0\left(1+x^2\right), $$
  两边同乘以$xJ_0(\lambda_j x)$并从$0$到$1$积分,得到
  $$ \frac{c_j\left[J_0(\lambda_j)\right]^2}{2} = T_0 \int_0^1(1+x^2)xJ_0(\lambda_jx)dx $$
  对$j=0$ (即$\lambda_j =0$)的特殊情况,
  $$\int_0^1(1+x^2)xJ_0(\lambda_j x) = \int_0^1x(1+x^2)dx  =\frac{3}{4}$$
  即
  $$c_0 = \frac{3}{2}T_0$$
  \ech
\end{frame}

\begin{frame}
  \chtitle{回到原问题}
  \bch
  若$j>0$,利用$ \lambda_jxJ_0(\lambda_jx) = \frac{d}{dx} \left[x J_1(\lambda_jx)\right]$进行分部积分:
  \bea
  \int_0^1(1+x^2)xJ_0(\lambda_jx)dx &=& \left.\frac{1+x^2}{\lambda_j}xJ_1(\lambda_jx)\right\vert_0^1 -\frac{2}{\lambda_j} \int_0^1x^2J_1(\lambda_jx) dx\newl
  &=& -\frac{2}{\lambda_j} \int_0^1x^2J_1(\lambda_jx) dx \newl
  &=& \left.-\frac{2}{\lambda_j^2} x^2J_2(\lambda_jx)\right\vert_0^1\newl
  &=& -\frac{2}{\lambda_j^2}J_2(\lambda_j) \newl
  &=& \frac{2}{\lambda_j^2}J_0(\lambda_j)
  \eea
  最后一步利用了$J_0(\lambda_j) + J_2(\lambda_j) = \frac{2}{\lambda_j} J_1(\lambda_j) = -\frac{2}{\lambda_j}J_0'(\lambda_j) = 0 $
  
  \ech
\end{frame}


\begin{frame}
  \chtitle{最终的解}
  \bch
  得到
  $$c_j = \frac{4T_0}{\lambda_j^2 J_0(\lambda_j)}$$
  即
  $$T = \frac{3}{2}T_0 + 4T_0\sum_{j=1}^\infty \frac{J_0\left(\frac{\lambda_j r}{R}\right)e^{-\frac{a\lambda_j^2t}{R^2}}}{\lambda_j^2 J_0(\lambda_j)}$$

  \skipline

 有趣的是,令$r=R$,$t=0$,可以得到:
  $$\sum_{i=1}^\infty \frac{1}{\lambda_i^2} = \frac{1}{8}$$
  
  \ech
\end{frame}


\section{Integral form}
\secpage{贝塞尔函数的积分表示}{$$J_m(x) = \frac{1}{2\pi}\int_{-\pi}^{\pi} e^{\ii (x\sin\theta - m \theta) } d\theta$$}

\begin{frame}
  \chtitle{贝塞尔函数的积分表示}
  \bch
  除了递推公式,整数阶贝塞尔函数最为常用的性质是它的积分表示:
  \tbox{
    $$J_m(x) = \frac{1}{2\pi}\int_{-\pi}^{\pi} e^{\ii (x\sin\theta - m \theta) } d\theta$$}
  被积函数的周期为$2\pi$,上面的积分区间可以取任何一个完整周期。
  \ech
\end{frame}


\begin{frame}
  \chtitle{证明 (心累请跳过)}
  \bch
  把$f(x)=\frac{1}{2\pi}\int_{-\pi}^{\pi} e^{\ii (x\sin\theta - m \theta) } d\theta$进行泰勒展开,其$n$次系数为
  $$c_n = \frac{1}{2\pi n!}\int_{-\pi}^{\pi} (\ii\sin\theta)^n e^{-\ii m \theta } d\theta$$
  转化为单位圆上的围道积分:
  \bea
  c_n &=& \frac{1}{2\pi 2^nn!}\oint (z-\frac{1}{z})^n \frac{1}{z^m} \frac{dz}{\ii z} \newl
  &=&  \frac{1}{2\pi \ii}\oint\sum_{k=0}^n \frac{1}{2^nk!(n-k)!} (-1)^kz^{n-2k-m-1} dz
  \eea
  根据留数定理,仅有的非零项为$c_{2k+m} = \frac{(-1)^k}{k!(m+k)!2^{2k+m}}$,和$J_m$的级数定义相同。证毕。
  \ech
\end{frame}


\begin{frame}
  \chtitle{贝塞尔函数的积分表示}
  \bch
  把积分式右边用欧拉公式展开,发现虚部为奇函数,积分为零;而实部为偶函数。所以
  \tbox{
  $$J_m(x) = \frac{1}{\pi}\int_{0}^{\pi} \cos{ (x\sin\theta - m \theta) } d\theta$$}
  \ech
\end{frame}

\section{Asymptotic Behavior}
\secpage{无穷远处的渐近行为}{当$x\gg m^2$时,$$J_m(x)\approx \sqrt{\frac{2}{\pi x}}\cos\left(x-\frac{m\pi}{2}-\frac{\pi}{4}\right) $$}

\begin{frame}
  \chtitle{无穷远处的渐近行为}
  \bch
    $$J_m(x) = \frac{1}{\pi}\int_{0}^{\pi} \cos{ (x\sin\theta - m \theta) } d\theta$$
  当$x$很大时,$\cos(x\sin\theta - m\theta)$快速振荡,对积分无贡献。唯一的例外是$\theta=\frac{\pi}{2}$附近,$\sin\theta$变化停滞,积分贡献并不消失。令$\theta = \frac{\pi}{2}+\epsilon$,把$\sin\theta$作二阶近似展开:
  \bea
  J_m(x) &\approx & \frac{1}{\pi}\int_{-\infty}^{\infty} \cos{\left[x(1-\frac{1}{2}\epsilon^2) - \frac{m}{2}\pi\right]} d\epsilon \newl
  &= & \frac{1}{\pi}\mathrm{Re}\left[e^{-\ii\phi}\int_{-\infty}^{\infty}e^{\frac{ \ii x}{2}\epsilon^2} d\epsilon\right]  \newl
  &= & \frac{2}{\pi}\mathrm{Re}\left[e^{-\ii\phi}\int_0^{\infty}e^{\frac{ \ii x}{2}\epsilon^2} d\epsilon\right]  \newl  
  \eea
  其中$\phi = x-\frac{m}{2}\pi$
  \ech
\end{frame}


\begin{frame}
  \chtitle{无穷远处的渐近行为(续)}
  \bch
  \bmini{0.41}
  \addfig{1.5}{contour04.png}
  \emini
  \bmini{0.53}
  在如图的围道上积分$$\oint e^{\frac{\ii x}{2}\epsilon^2}d\epsilon$$
  容易得到
  \emini
  \be
  \int_0^{\infty}e^{\frac{ \ii x}{2}\epsilon^2} d\epsilon = e^{\frac{\pi\ii}{4}}\int_0^\infty e^{-\frac{x}{2}\epsilon^2}d\epsilon= e^{\frac{\pi\ii}{4}}\sqrt{\frac{\pi}{2x}}
  \ee
  最后得到
  \tbox{
    $$J_m(x)\approx \sqrt{\frac{2}{\pi x}}\cos\left(x-\frac{m\pi}{2}-\frac{\pi}{4}\right) $$}

  
  \ech
\end{frame}


\section{Infinite Disk}

\secpage{无限大板上的热传导问题}{    $$ \int_0^\infty J_m(k_1r)J_m(k_2r) r dr = \frac{\delta(k_1-k_2)}{k_1}. $$}

\begin{frame}
  \chtitle{例题3: 无限大板上的热传导问题}
  \bch
  一块很大(可以视为无限大)的均匀薄板的温度为$T_0$,在$t=0$时刻在中心点注入热量$Q$,求之后板上的温度变化。设板的单位质量比热为$c$,质量密度为$\rho$,厚度为$h$,导热系数为$\lambda$。
  \ech
\end{frame}


\begin{frame}
  \chtitle{写出方程}
  \bch
  假设一开始的热量被注入到面积为$\epsilon$的小区域内,则该区域内的温度为$T_0+\frac{Q}{\rho c \epsilon h} $。令$u(r,\theta) = \frac{\rho c h }{Q}(T - T_0)$,写出方程和边界条件:

  \bea
  \frac{\partial u}{\partial t} - a\nabla^2 u &=& 0,\newl
  \left. u\right\vert_{r=\infty} &=& 0 ,\newl
  \left. u \right\vert_{t=0} &=& \delta(\vecx).
  \eea
  其中$a = \frac{\lambda}{\rho c}$。

  \skipline
  我们认出这是一个典型的求解格林函数问题。
  \ech
\end{frame}

\begin{frame}
  \chtitle{直角坐标的解}
  \bch
  先看直角坐标系下能否求解。把$u$按谐函数分解:
  $$u(\vecx, t) =\frac{1}{2\pi} \int d^2\veck\, c(\veck)e^{\ii \veck\cdot\vecx}e^{-ak^2t},  $$
  注意这里的$\veck = (k_x, k_y)$; $\vecx = (x, y)$; $\veck\cdot\vecx = k_xx+k_yy$; $k = |\veck|=\sqrt{k_x^2+k_y^2}$; $\int d^2\veck$是全平面积分$\int_{-\infty}^\infty dk_x \int_{-\infty}^\infty dk_y $的简写。

  令$t=0$,
  $$\delta(\vecx) = \frac{1}{2\pi} \int d^2\veck\, c(\veck)e^{\ii \veck\cdot\vecx},  $$
  这说明$c(\veck)$是$\delta(\vecx)$的二维傅立叶变换。
  \ech
\end{frame}


\begin{frame}
  \chtitle{直角坐标的解(续)}
  \bch
  $$ c(\veck) = \frac{1}{2\pi}\int d^2\vecx\, \delta(\vecx)e^{-\ii \veck\cdot\vecx } = \frac{1}{2\pi} $$
  即解为
  $$u(\vecx, t) =\frac{1}{4\pi^2} \int d^2\veck\, e^{\ii \veck\cdot\vecx}e^{-ak^2t} = \frac{1}{4\pi at}e^{-\frac{x^2+y^2}{4at}},  $$

  (很容易认出来这其实是两个一维格林函数的乘积。)
  \ech
\end{frame}


\begin{frame}
  \chtitle{极坐标的解}
  \bch
  在极坐标里,我们一开始就可以看出问题具有旋转对称性,写出
  $$  u(r, t) = \int_0^\infty f(k) J_0(kr)e^{-ak^2t} k dk $$
  取$t=0$,得到
  $$  \delta(\vecx) = \int_0^\infty f(k) J_0(kr) k dk $$  
  为了求出$f(k)$,我们来研究第一类贝塞尔函数在无穷区间上的正交关系。
  \ech
\end{frame}


\begin{frame}
  \chtitle{第一类贝塞尔函数在无穷区间上的正交关系}
  \bch
  \tbox{
    $$ \int_0^\infty J_m(k_1r)J_m(k_2r) r dr = \frac{\delta(k_1-k_2)}{k_1}. $$
  }
  证明留为作业。
  \ech
\end{frame}


\begin{frame}
  \chtitle{极坐标系求解(续)}
  \bch
  $$  \delta(\vecx) = \int_0^\infty f(k) J_0(kr)k dk $$
  对任意给定的$k'$,两边乘以$rJ_0(k'r)$并对$r$从$0$到$\infty$积分,得到
  $$ \frac{1}{2\pi} = \int_0^\infty f(k)\frac{\delta(k-k')}{k} k dk =f(k')$$
  所以解为
  $$  u(r, t) = \frac{1}{2\pi} \int_0^\infty J_0(kr)e^{-ak^2t} k dk $$  
  \ech
\end{frame}



\section{Homework}

\begin{frame}
  \chtitle{课后作业(题号38-40)}
  \bch
  \bitem
\item[38]{  有孤立的,半径为$R$的均匀圆形金属薄片,以中心为原点建立极坐标$(r,\theta)$。初态$t=0$时刻,金属片上的温度为$f(r,\theta)$。已知金属片质量密度为$\rho$,导热系数为$\lambda$,单位质量比热为$c$。求$t\ge 0$时刻金属片上各点的温度。}
\item[39]{利用$J_m$在有限区间上的正交定理和$J_m$的渐近表达式证明
  $$ \int_0^\infty J_m(k_1r)J_m(k_2r) r dr = \frac{\delta(k_1-k_2)}{k_1}. $$ }
\item[40]{  
  我们在两种不同的坐标系中求解例题3,分别得到
  $$\frac{1}{4\pi at}e^{-\frac{r^2}{4at}}$$
  和
  $$ \frac{1}{2\pi} \int_0^\infty J_0(kr)e^{-ak^2t} k dk.$$
  试直接证明这两个解恒等。}
  \eitem
  \ech
\end{frame}


\end{document}
