\documentclass[CJK,13pt]{beamer}
\usepackage{CJKutf8}
\usepackage{beamerthemesplit}
\usetheme{Malmoe}
\useoutertheme[footline=authortitle]{miniframes}
\usepackage{amsmath}
\usepackage{amssymb}
\usepackage{graphicx}
\usepackage{eufrak}
\usepackage{color}
\usepackage{slashed}
\usepackage{simplewick}
\usepackage{tikz}
\usepackage{tcolorbox}
\graphicspath{{../figures/}}
%%figures
\def\lfig#1#2{\includegraphics[width=#1 in]{#2}}
\def\addfig#1#2{\begin{center}\includegraphics[width=#1 in]{#2}\end{center}}
\def\wulian{\includegraphics[width=0.18in]{emoji_wulian.jpg}}
\def\bigwulian{\includegraphics[width=0.35in]{emoji_wulian.jpg}}
\def\bye{\includegraphics[width=0.18in]{emoji_bye.jpg}}
\def\bigbye{\includegraphics[width=0.35in]{emoji_bye.jpg}}
\def\huaixiao{\includegraphics[width=0.18in]{emoji_huaixiao.jpg}}
\def\bighuaixiao{\includegraphics[width=0.35in]{emoji_huaixiao.jpg}}
\def\jianxiao{\includegraphics[width=0.18in]{emoji_jianxiao.jpg}}
\def\bigjianxiao{\includegraphics[width=0.35in]{emoji_jianxiao.jpg}}
%% colors
\def\blacktext#1{{\color{black}#1}}
\def\bluetext#1{{\color{blue}#1}}
\def\redtext#1{{\color{red}#1}}
\def\darkbluetext#1{{\color[rgb]{0,0.2,0.6}#1}}
\def\skybluetext#1{{\color[rgb]{0.2,0.7,1.}#1}}
\def\cyantext#1{{\color[rgb]{0.,0.5,0.5}#1}}
\def\greentext#1{{\color[rgb]{0,0.7,0.1}#1}}
\def\darkgray{\color[rgb]{0.2,0.2,0.2}}
\def\lightgray{\color[rgb]{0.6,0.6,0.6}}
\def\gray{\color[rgb]{0.4,0.4,0.4}}
\def\blue{\color{blue}}
\def\red{\color{red}}
\def\green{\color{green}}
\def\darkgreen{\color[rgb]{0,0.4,0.1}}
\def\darkblue{\color[rgb]{0,0.2,0.6}}
\def\skyblue{\color[rgb]{0.2,0.7,1.}}
%%control
\def\be{\begin{equation}}
\def\ee{\nonumber\end{equation}}
\def\bea{\begin{eqnarray}}
\def\eea{\nonumber\end{eqnarray}}
\def\bch{\begin{CJK}{UTF8}{gbsn}}
\def\ech{\end{CJK}}
\def\bitem{\begin{itemize}}
\def\eitem{\end{itemize}}
\def\bcenter{\begin{center}}
\def\ecenter{\end{center}}
\def\bex{\begin{minipage}{0.2\textwidth}\includegraphics[width=0.6in]{jugelizi.png}\end{minipage}\begin{minipage}{0.76\textwidth}}
\def\eex{\end{minipage}}
\def\chtitle#1{\frametitle{\bch#1\ech}}
\def\bmat#1{\left(\begin{array}{#1}}
\def\emat{\end{array}\right)}
\def\bcase#1{\left\{\begin{array}{#1}}
\def\ecase{\end{array}\right.}
\def\bmini#1{\begin{minipage}{#1\textwidth}}
\def\emini{\end{minipage}}
\def\tbox#1{\begin{tcolorbox}#1\end{tcolorbox}}
\def\pfrac#1#2#3{\left(\frac{\partial #1}{\partial #2}\right)_{#3}}
%%symbols
\def\bropt{\,(\ \ \ )}
\def\sone{$\star$}
\def\stwo{$\star\star$}
\def\sthree{$\star\star\star$}
\def\sfour{$\star\star\star\star$}
\def\sfive{$\star\star\star\star\star$}
\def\rint{{\int_\leftrightarrow}}
\def\roint{{\oint_\leftrightarrow}}
\def\stdHf{{\textit{\r H}_f}}
\def\deltaH{{\Delta \textit{\r H}}}
\def\ii{{\dot{\imath}}}
\def\skipline{{\vskip0.1in}}
\def\skiplines{{\vskip0.2in}}
\def\lagr{{\mathcal{L}}}
\def\hamil{{\mathcal{H}}}
\def\vecv{{\mathbf{v}}}
\def\vecx{{\mathbf{x}}}
\def\vecy{{\mathbf{y}}}
\def\veck{{\mathbf{k}}}
\def\vecp{{\mathbf{p}}}
\def\vecn{{\mathbf{n}}}
\def\vecA{{\mathbf{A}}}
\def\vecP{{\mathbf{P}}}
\def\vecsigma{{\mathbf{\sigma}}}
\def\hatJn{{\hat{J_\vecn}}}
\def\hatJx{{\hat{J_x}}}
\def\hatJy{{\hat{J_y}}}
\def\hatJz{{\hat{J_z}}}
\def\hatj#1{\hat{J_{#1}}}
\def\hatphi{{\hat{\phi}}}
\def\hatq{{\hat{q}}}
\def\hatpi{{\hat{\pi}}}
\def\vel{\upsilon}
\def\Dint{{\mathcal{D}}}
\def\adag{{\hat{a}^\dagger}}
\def\bdag{{\hat{b}^\dagger}}
\def\cdag{{\hat{c}^\dagger}}
\def\ddag{{\hat{d}^\dagger}}
\def\hata{{\hat{a}}}
\def\hatb{{\hat{b}}}
\def\hatc{{\hat{c}}}
\def\hatd{{\hat{d}}}
\def\hatN{{\hat{N}}}
\def\hatH{{\hat{H}}}
\def\hatp{{\hat{p}}}
\def\Fup{{F^{\mu\nu}}}
\def\Fdown{{F_{\mu\nu}}}
\def\newl{\nonumber \\}
\def\vece{\mathrm{e}}
\def\calM{{\mathcal{M}}}
\def\calT{{\mathcal{T}}}
\def\calR{{\mathcal{R}}}
\def\barpsi{\bar{\psi}}
\def\baru{\bar{u}}
\def\barv{\bar{\upsilon}}
\def\qeq{\stackrel{?}{=}}
\def\torder#1{\mathcal{T}\left(#1\right)}
\def\rorder#1{\mathcal{R}\left(#1\right)}
\def\contr#1#2{\contraction{}{#1}{}{#2}#1#2}
\def\trof#1{\mathrm{Tr}\left(#1\right)}
\def\trace{\mathrm{Tr}}
\def\comm#1{\ \ \ \left(\mathrm{used}\ #1\right)}
\def\tcomm#1{\ \ \ (\text{#1})}
\def\slp{\slashed{p}}
\def\slk{\slashed{k}}
\def\calp{{\mathfrak{p}}}
\def\veccalp{\mathbf{\mathfrak{p}}}
\def\Tthree{T_{\tiny \textcircled{3}}}
\def\pthree{p_{\tiny \textcircled{3}}}
\def\dbar{{\,\mathchar'26\mkern-12mu d}}
\def\erf{\mathrm{erf}}
\def\const{\mathrm{constant}}
\def\pheat{\pfrac p{\ln T}V}
\def\vheat{\pfrac V{\ln T}p}
%%units
\def\fdeg{{^\circ \mathrm{F}}}
\def\cdeg{^\circ \mathrm{C}}
\def\atm{\,\mathrm{atm}}
\def\angstrom{\,\text{\AA}}
\def\SIL{\,\mathrm{L}}
\def\SIkm{\,\mathrm{km}}
\def\SIyr{\,\mathrm{yr}}
\def\SIGyr{\,\mathrm{Gyr}}
\def\SIV{\,\mathrm{V}}
\def\SImV{\,\mathrm{mV}}
\def\SIeV{\,\mathrm{eV}}
\def\SIkeV{\,\mathrm{keV}}
\def\SIMeV{\,\mathrm{MeV}}
\def\SIGeV{\,\mathrm{GeV}}
\def\SIcal{\,\mathrm{cal}}
\def\SIkcal{\,\mathrm{kcal}}
\def\SImol{\,\mathrm{mol}}
\def\SIN{\,\mathrm{N}}
\def\SIHz{\,\mathrm{Hz}}
\def\SIm{\,\mathrm{m}}
\def\SIcm{\,\mathrm{cm}}
\def\SIfm{\,\mathrm{fm}}
\def\SImm{\,\mathrm{mm}}
\def\SInm{\,\mathrm{nm}}
\def\SImum{\,\mathrm{\mu m}}
\def\SIJ{\,\mathrm{J}}
\def\SIW{\,\mathrm{W}}
\def\SIkJ{\,\mathrm{kJ}}
\def\SIs{\,\mathrm{s}}
\def\SIkg{\,\mathrm{kg}}
\def\SIg{\,\mathrm{g}}
\def\SIK{\,\mathrm{K}}
\def\SImmHg{\,\mathrm{mmHg}}
\def\SIPa{\,\mathrm{Pa}}

\def\courseurl{https://github.com/zqhuang/SYSU\_TD}

\def\tpage#1#2{
\begin{frame}
\begin{center}
\begin{Large}
\bch
热学 \\
第#1讲 #2

{\vskip 0.3in}

黄志琦

\ech
\end{Large}
\end{center}

\vskip 0.2in

\bch
教材:《热学》第二版,赵凯华,罗蔚茵,高等教育出版社
\ech

\bch
课件下载
\ech
\courseurl
\end{frame}
}

\def\bfr#1{
\begin{frame}
\chtitle{#1} 
\bch
}

\def\efr{
\ech 
\end{frame}
}

  \date{}
  \begin{document}
  \bch
\tpage{7}{Fourier Transform}


\begin{frame}
\frametitle{本讲内容}

\bitem
\item{傅立叶变换和逆变换,保内积性}
\item{梯度算符和拉普拉斯算符}
\item{卷积定理}
\eitem

\end{frame}

\section{Fourier Transform}
\begin{frame}
  \frametitle{傅立叶变换}
  实参量复值函数 $f(x)$ (即$x$必须取实数,$f$则不一定是实数)的傅立叶变换定义为:
  $$ \widetilde{f}(k) = \frac{1}{\sqrt{2\pi}} \int_{-\infty}^\infty e^{-ikx} f(x) dx $$
  同样,我们约定$k$必须取实数,而$\widetilde{f}$显然未必为实数。

\end{frame}


\begin{frame}
  \frametitle{逆变换}
  $$ f(x) = \frac{1}{\sqrt{2\pi}} \int_{-\infty}^\infty e^{ikx} \widetilde{f}(k) dk $$
  证明很简单:只要把$\widetilde{f}(k)$的定义代入,交换积分次序,然后用$\delta$函数的积分表示。(一系列不合法的操作玩得非常溜!)
  

\end{frame}


\begin{frame}
  \frametitle{傅立叶变换的保内积性}

 $$\int_{-\infty}^\infty \widetilde{f}^*(k) \widetilde{g}(k) dk = \int_{-\infty}^\infty f^*(x)g(x) dx. $$
  证明同样可以用直接代入并用$\delta$函数的积分表示。
  
\end{frame}


\begin{frame}
  \frametitle{思考题}
  
  \bmini{0.48}
  \addfig{2}{stepfunction.png}
  \emini
  \bmini{0.48}
  \be
  u(x) = \branchll 1, & \text{ if } -1<x<1 \\ 0, & \text{ else} \branchrr
  \ee
  \emini

  \skipline
  
  计算如图函数的傅立叶变换,然后利用傅立叶变换保持内积不变的性质,求积分
  $$\int_0^\infty \frac{\sin x}{x} \,dx. $$
  
  
\end{frame}


\begin{frame}
  \frametitle{思考题}
  
  求$f(x) = \frac{\sin x}{x}$的傅立叶变换。
  
\end{frame}


\begin{frame}
  \frametitle{$n$维空间的傅立叶变换}
  
  $n$维内积空间的函数$f(\vecx)$的傅立叶变换就是对每个维度都进行傅立叶变换:
  $$ \widetilde{f}(\veck) = \frac{1}{(2\pi)^{n/2}}\int f(\vecx) e^{-\ii \veck \cdot \vecx} d^n\vecx $$

  其逆变换为
  
  $$ f(\vecx) = \frac{1}{(2\pi)^{n/2}}\int \widetilde{f}(\veck) e^{\ii \veck \cdot \vecx} d^n\veck. $$

  在物理问题中,我们经常把$\vecx$所在的$n$维空间称为位置空间,把$\veck$所在的空间叫傅立叶空间或动量空间。

  {\scriptsize 当然上述叫法不是绝对的:在一维傅立叶变换中,有时我们考虑的是时间的函数的傅立叶变换。这时对应的两个空间常常被叫做时间域和频率域。}
\end{frame}



\begin{frame}
  \frametitle{思考题}
  
  试证明:  高维空间傅立叶变换保持内积不变的结论也依然成立,设$\widetilde{f}, \widetilde{g}$分别为$f,g$的傅立叶变换,则
{\blue  $$\int  f^*(\vecx)g(\vecx)\,d^n\vecx = \int \widetilde{f}^*(\veck) \widetilde{g}(\veck)\, d^n\veck . $$}

\end{frame}



\section{Gradient Operator}
\secpage{梯度算符和拉普拉斯算符}{引而不发才是高境界}

\begin{frame}
  \frametitle{算符}
  
      {\blue 算符是把一种对象映射到另一种对象的操作}。


      \bcenter
      \lfig{1}{blackq.jpg}
      
      感觉等于什么都没说.jpg
      \ecenter
      
\end{frame}


\begin{frame}
  \frametitle{梯度算符}
  
  \bex
  $n$维空间梯度算符$\nabla$把$n$维空间的函数$f(\vecx)$映射为该函数的梯度(矢量)
    $$ \nabla f \equiv (\partial_1 f, \partial_2 f,\ldots, \partial_n f) .$$
    
    (为了简单起见,我们把第$i$个方向的偏导算符$\frac{\partial}{\partial_{x_i}}$简写为了$\partial_i$。)
  \eex

  
\end{frame}


\begin{frame}
  \frametitle{广义矢量}
  
  $n$维空间的矢量$\vecx = (x_1, x_2, \ldots, x_n)$, 可以看成$n$个基的线性组合:

  $$\vecx = x_1\vece_1 + x_2\vece_2 + \ldots + x_n\vece_n. $$

  对广义矢量,{\blue 基的线性组合可以不仅仅用数字作为系数,还能用任何对象。}
  例如,在二维内积空间,

  \skipline
  
  (苹果,橘子) = 苹果$\times \vece_1$ + 橘子$\times\vece_2$.

  \skipline
  
  (尚小嚎,肖凉) = 尚小嚎$\times \vece_1$ + 肖凉$\times\vece_2$.  
  
  
\end{frame}


\begin{frame}
  \frametitle{思考题}
  
  计算矢量$(3,2)$和(苹果,橘子) 的内积。
  
\end{frame}

\begin{frame}
  \frametitle{梯度算符}
  
  梯度算符$\nabla$又可以看成一个广义矢量,它的每个分量是偏导算符:
  $$\nabla \equiv (\partial_1, \partial_2,\ldots, \partial_n). $$

  这就是数学中省略作用对象的“引而不发”的写法。如果在两边补充写个$f$,则又回到熟悉的小学生可以理解的形式:
  $$\nabla f \equiv (\partial_1f, \partial_2f,\ldots, \partial_nf). $$
  
\end{frame}

\begin{frame}
  \frametitle{散度}
  
  梯度算符$\nabla$和矢量函数$\mathbf{E}(\vecx) = \left(E_1(\vecx), E_2(\vecx), \ldots, E_n(\vecx)\right)$的内积为
  $$\nabla\cdot \mathbf{E} \equiv \partial_1 E_1 + \partial_2 E_2 + \ldots + \partial_n E_n $$
  这通常称为$\mathbf{E}$的{\blue 散度}。
  \skipline
  
  
\end{frame}

\begin{frame}
  \frametitle{拉普拉斯算符}
  
  梯度算符作用到一个标量函数$f(\vecx)$上,就得到一个矢量函数$\nabla f = (\partial_1 f, \partial_2 f, \ldots, \partial_n f)$。然后$\nabla$和$\nabla f$的内积就是
  
  $$ \nabla \cdot \nabla f \equiv \partial_1^2 f + \partial_2^2 f + \ldots + \partial_n^2 f.$$
  上式左边的$\nabla\cdot \nabla$称为{\blue 拉普拉斯算符},通常简写为{\blue $\nabla^2$},重写一遍就是:\tbox{
    $$ \nabla^2 f \equiv \partial_1^2 f + \partial_2^2 f + \ldots + \partial_n^2 f.$$}
  
  {
    \small
  在静电磁学里,电场强度$\mathbf{E}$正比于电势$\varphi$的梯度: $\mathbf{E} = -\nabla \varphi$。高斯定律(电场的散度正比于电荷密度)就可以写成
  $$ \nabla^2 \varphi = -\frac{\rho}{\epsilon_0} $$}
  
  
\end{frame}


\begin{frame}
  \frametitle{$\nabla \rightarrow \ii \veck$}
  
  对傅立叶逆变换式
  $$ f(\vecx) = \frac{1}{(2\pi)^{n/2}}\int \widetilde{f}(\veck) e^{\ii \veck \cdot \vecx} d^n\veck. $$
  两边作用$\partial_j$
  $$ \partial_j f(\vecx) =  \frac{1}{(2\pi)^{n/2}}\int (\ii k_j \widetilde{f}(\veck)) e^{\ii \veck \cdot \vecx} d^n\veck. $$
  当$j$取遍$1,2,\ldots, n$,上式可以写成矢量形式:
  {\blue  $$ \widetilde{\ \nabla f\ } = \ii \veck \widetilde{f} $$}
  在量子力学里,我们常常说:位置空间的算符$-\ii\nabla$对应于(动量空间的)动量$\veck$。
\end{frame}


\begin{frame}
  \frametitle{思考题}
      位置空间的拉普拉斯算符 $\nabla^2$ 在傅立叶空间对应什么?

\end{frame}


\begin{frame}
  \frametitle{第一个数理方程的例子}
  
  用高斯定律来计算在原点的点电荷$Q$造成的电势$\varphi$
  $$\nabla^2 \varphi(\vecx) = -\frac{Q}{\epsilon_0} \,\delta^{(3)}(\vecx)$$
  
  对这个方程两边进行傅立叶变换,
  $$ -k^2\widetilde{\varphi}(\veck) = -\frac{1}{(2\pi)^{3/2}} \frac{Q}{\epsilon_0} ,$$
  即
  $$\widetilde{\varphi}(\veck) = \frac{1}{(2\pi)^{3/2}\epsilon_0} \frac{Q}{k^2}.$$
  再进行逆变换
 $$\varphi(\vecx) =\frac{Q}{(2\pi)^3\epsilon_0} \int \frac{e^{\ii \veck\cdot\vecx}}{k^2}d^3\veck$$
  
\end{frame}

\begin{frame}
  \frametitle{第一个数理方程的例子}
  
  取$\vecx$方向为北极方向在傅立叶空间建立球坐标$(k, \theta,\phi)$,记$r=|\vecx|$
  \bea
  \int \frac{e^{\ii \veck\cdot\vecx}}{k^2}d^3\veck &=& \int_0^\infty k^2 dk \int_{-1}^1 d(\cos\theta)\int_0^{2\pi} \frac{e^{\ii kr cos\theta}}{k^2} d\phi \newl
  &=& 2\pi \int_0^\infty dk \int_{-1}^1 d(\cos\theta) e^{\ii kr cos\theta} \newl
  &=& 2\pi  \int_0^\infty dk  \frac{2\sin(kr)}{kr} \newl
  &=& \frac{2\pi^2}{r}
  \eea
  由此我们求出
  $$\varphi(\vecx) = \frac{Q}{4\pi \epsilon_0r}$$
  
\end{frame}

\begin{frame}
  \frametitle{思考题}
  
  电势可以随便取零点,为什么解出来的电势不带积分常数?
  
\end{frame}

\section{Convolution}
\secpage{卷积定理}{$$ \widetilde{f\star g} = \widetilde{f}\, \widetilde{g}  $$}


\begin{frame}
  \frametitle{卷积}
  
  
  两个函数$f(\vecx)$和$g(\vecx)$的卷积定义为
  \tbox{$$ (f\star g)(\vecx) \equiv \frac{1}{(2\pi)^{n/2}} \int f(\vecy)\,g(\vecx - \vecy)\, d^n\vecy  $$}

  显然,卷积满足交换律: {\blue $f\star g = g\star f$}。

  更重要的是卷积定理: 设$f,g$的傅立叶变换依次为$\widetilde{f},\widetilde{g}$,则$f\star g$的傅立叶变换为$\widetilde{f}\widetilde{g}$,即
  \tbox{卷积的傅立叶变换等于傅立叶变换的乘积}
  
 \end{frame}


\begin{frame}
  \frametitle{思考题}
  
  请用666的操作证明卷积定理。
  
 \end{frame}




\section{Homework for Quizphobias}

\begin{frame}
  \frametitle{课后作业}
  
  \bitem
\item[19]{计算高斯函数$$f(x) = \frac{1}{\sqrt{2\pi}\sigma} e^{-\frac{x^2}{2\sigma^2}}$$
的傅立叶变换。}
\item[20]{用你喜欢的方法求积分
  $$\int_0^\infty \left(\frac{\sin x}{x}\right)^2dx . $$}
\item[21]{计算三维top-hat函数
  \begin{equation}
    h(x,y,z) = \left\{
    \begin{array}{ll}
      1, & \text{ if }\ x^2+y^2+z^2 \le 1 \\
      0, & \text{ else.}
    \end{array}
    \right. \nonumber
  \end{equation}
  的傅立叶变换。}  
  \eitem
  
\end{frame}


\ech
\end{document}
