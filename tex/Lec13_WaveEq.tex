\documentclass[CJK]{beamer}
\usepackage{CJKutf8}
\usepackage{beamerthemesplit}
\usetheme{Malmoe}
\useoutertheme[footline=authortitle]{miniframes}
\usepackage{amsmath}
\usepackage{amssymb}
\usepackage{graphicx}
\usepackage{eufrak}
\usepackage{color}
\usepackage{slashed}
\usepackage{simplewick}
\usepackage{tikz}
\usepackage{tcolorbox}
\graphicspath{{../figures/}}
%%figures
\def\lfig#1#2{\includegraphics[width=#1 in]{#2}}
\def\addfig#1#2{\begin{center}\includegraphics[width=#1 in]{#2}\end{center}}
\def\wulian{\includegraphics[width=0.18in]{emoji_wulian.jpg}}
\def\bigwulian{\includegraphics[width=0.35in]{emoji_wulian.jpg}}
\def\bye{\includegraphics[width=0.18in]{emoji_bye.jpg}}
\def\bigbye{\includegraphics[width=0.35in]{emoji_bye.jpg}}
\def\huaixiao{\includegraphics[width=0.18in]{emoji_huaixiao.jpg}}
\def\bighuaixiao{\includegraphics[width=0.35in]{emoji_huaixiao.jpg}}
\def\jianxiao{\includegraphics[width=0.18in]{emoji_jianxiao.jpg}}
\def\bigjianxiao{\includegraphics[width=0.35in]{emoji_jianxiao.jpg}}
%% colors
\def\blacktext#1{{\color{black}#1}}
\def\bluetext#1{{\color{blue}#1}}
\def\redtext#1{{\color{red}#1}}
\def\darkbluetext#1{{\color[rgb]{0,0.2,0.6}#1}}
\def\skybluetext#1{{\color[rgb]{0.2,0.7,1.}#1}}
\def\cyantext#1{{\color[rgb]{0.,0.5,0.5}#1}}
\def\greentext#1{{\color[rgb]{0,0.7,0.1}#1}}
\def\darkgray{\color[rgb]{0.2,0.2,0.2}}
\def\lightgray{\color[rgb]{0.6,0.6,0.6}}
\def\gray{\color[rgb]{0.4,0.4,0.4}}
\def\blue{\color{blue}}
\def\red{\color{red}}
\def\green{\color{green}}
\def\darkgreen{\color[rgb]{0,0.4,0.1}}
\def\darkblue{\color[rgb]{0,0.2,0.6}}
\def\skyblue{\color[rgb]{0.2,0.7,1.}}
%%control
\def\be{\begin{equation}}
\def\ee{\nonumber\end{equation}}
\def\bea{\begin{eqnarray}}
\def\eea{\nonumber\end{eqnarray}}
\def\bch{\begin{CJK}{UTF8}{gbsn}}
\def\ech{\end{CJK}}
\def\bitem{\begin{itemize}}
\def\eitem{\end{itemize}}
\def\bcenter{\begin{center}}
\def\ecenter{\end{center}}
\def\bex{\begin{minipage}{0.2\textwidth}\includegraphics[width=0.6in]{jugelizi.png}\end{minipage}\begin{minipage}{0.76\textwidth}}
\def\eex{\end{minipage}}
\def\chtitle#1{\frametitle{\bch#1\ech}}
\def\bmat#1{\left(\begin{array}{#1}}
\def\emat{\end{array}\right)}
\def\bcase#1{\left\{\begin{array}{#1}}
\def\ecase{\end{array}\right.}
\def\bmini#1{\begin{minipage}{#1\textwidth}}
\def\emini{\end{minipage}}
\def\tbox#1{\begin{tcolorbox}#1\end{tcolorbox}}
\def\pfrac#1#2#3{\left(\frac{\partial #1}{\partial #2}\right)_{#3}}
%%symbols
\def\bropt{\,(\ \ \ )}
\def\sone{$\star$}
\def\stwo{$\star\star$}
\def\sthree{$\star\star\star$}
\def\sfour{$\star\star\star\star$}
\def\sfive{$\star\star\star\star\star$}
\def\rint{{\int_\leftrightarrow}}
\def\roint{{\oint_\leftrightarrow}}
\def\stdHf{{\textit{\r H}_f}}
\def\deltaH{{\Delta \textit{\r H}}}
\def\ii{{\dot{\imath}}}
\def\skipline{{\vskip0.1in}}
\def\skiplines{{\vskip0.2in}}
\def\lagr{{\mathcal{L}}}
\def\hamil{{\mathcal{H}}}
\def\vecv{{\mathbf{v}}}
\def\vecx{{\mathbf{x}}}
\def\vecy{{\mathbf{y}}}
\def\veck{{\mathbf{k}}}
\def\vecp{{\mathbf{p}}}
\def\vecn{{\mathbf{n}}}
\def\vecA{{\mathbf{A}}}
\def\vecP{{\mathbf{P}}}
\def\vecsigma{{\mathbf{\sigma}}}
\def\hatJn{{\hat{J_\vecn}}}
\def\hatJx{{\hat{J_x}}}
\def\hatJy{{\hat{J_y}}}
\def\hatJz{{\hat{J_z}}}
\def\hatj#1{\hat{J_{#1}}}
\def\hatphi{{\hat{\phi}}}
\def\hatq{{\hat{q}}}
\def\hatpi{{\hat{\pi}}}
\def\vel{\upsilon}
\def\Dint{{\mathcal{D}}}
\def\adag{{\hat{a}^\dagger}}
\def\bdag{{\hat{b}^\dagger}}
\def\cdag{{\hat{c}^\dagger}}
\def\ddag{{\hat{d}^\dagger}}
\def\hata{{\hat{a}}}
\def\hatb{{\hat{b}}}
\def\hatc{{\hat{c}}}
\def\hatd{{\hat{d}}}
\def\hatN{{\hat{N}}}
\def\hatH{{\hat{H}}}
\def\hatp{{\hat{p}}}
\def\Fup{{F^{\mu\nu}}}
\def\Fdown{{F_{\mu\nu}}}
\def\newl{\nonumber \\}
\def\vece{\mathrm{e}}
\def\calM{{\mathcal{M}}}
\def\calT{{\mathcal{T}}}
\def\calR{{\mathcal{R}}}
\def\barpsi{\bar{\psi}}
\def\baru{\bar{u}}
\def\barv{\bar{\upsilon}}
\def\qeq{\stackrel{?}{=}}
\def\torder#1{\mathcal{T}\left(#1\right)}
\def\rorder#1{\mathcal{R}\left(#1\right)}
\def\contr#1#2{\contraction{}{#1}{}{#2}#1#2}
\def\trof#1{\mathrm{Tr}\left(#1\right)}
\def\trace{\mathrm{Tr}}
\def\comm#1{\ \ \ \left(\mathrm{used}\ #1\right)}
\def\tcomm#1{\ \ \ (\text{#1})}
\def\slp{\slashed{p}}
\def\slk{\slashed{k}}
\def\calp{{\mathfrak{p}}}
\def\veccalp{\mathbf{\mathfrak{p}}}
\def\Tthree{T_{\tiny \textcircled{3}}}
\def\pthree{p_{\tiny \textcircled{3}}}
\def\dbar{{\,\mathchar'26\mkern-12mu d}}
\def\erf{\mathrm{erf}}
\def\const{\mathrm{constant}}
\def\pheat{\pfrac p{\ln T}V}
\def\vheat{\pfrac V{\ln T}p}
%%units
\def\fdeg{{^\circ \mathrm{F}}}
\def\cdeg{^\circ \mathrm{C}}
\def\atm{\,\mathrm{atm}}
\def\angstrom{\,\text{\AA}}
\def\SIL{\,\mathrm{L}}
\def\SIkm{\,\mathrm{km}}
\def\SIyr{\,\mathrm{yr}}
\def\SIGyr{\,\mathrm{Gyr}}
\def\SIV{\,\mathrm{V}}
\def\SImV{\,\mathrm{mV}}
\def\SIeV{\,\mathrm{eV}}
\def\SIkeV{\,\mathrm{keV}}
\def\SIMeV{\,\mathrm{MeV}}
\def\SIGeV{\,\mathrm{GeV}}
\def\SIcal{\,\mathrm{cal}}
\def\SIkcal{\,\mathrm{kcal}}
\def\SImol{\,\mathrm{mol}}
\def\SIN{\,\mathrm{N}}
\def\SIHz{\,\mathrm{Hz}}
\def\SIm{\,\mathrm{m}}
\def\SIcm{\,\mathrm{cm}}
\def\SIfm{\,\mathrm{fm}}
\def\SImm{\,\mathrm{mm}}
\def\SInm{\,\mathrm{nm}}
\def\SImum{\,\mathrm{\mu m}}
\def\SIJ{\,\mathrm{J}}
\def\SIW{\,\mathrm{W}}
\def\SIkJ{\,\mathrm{kJ}}
\def\SIs{\,\mathrm{s}}
\def\SIkg{\,\mathrm{kg}}
\def\SIg{\,\mathrm{g}}
\def\SIK{\,\mathrm{K}}
\def\SImmHg{\,\mathrm{mmHg}}
\def\SIPa{\,\mathrm{Pa}}

\def\courseurl{https://github.com/zqhuang/SYSU\_TD}

\def\tpage#1#2{
\begin{frame}
\begin{center}
\begin{Large}
\bch
热学 \\
第#1讲 #2

{\vskip 0.3in}

黄志琦

\ech
\end{Large}
\end{center}

\vskip 0.2in

\bch
教材:《热学》第二版,赵凯华,罗蔚茵,高等教育出版社
\ech

\bch
课件下载
\ech
\courseurl
\end{frame}
}

\def\bfr#1{
\begin{frame}
\chtitle{#1} 
\bch
}

\def\efr{
\ech 
\end{frame}
}

  \date{}
\begin{document}
\tpage{13}{Wave Equation}

\begin{frame}
\chtitle{本讲内容}
\bch
\bitem
\item{波动方程的例子:
  \bitem
\item{弦的横振动}
\item{杆的纵振动}
\item{电磁波和引力波}
  \eitem}
\item{齐次边界条件下的解法}
\item{非齐次边界条件}  
\eitem
\ech
\end{frame}


\section{Wave Equation}
\secpage{波动方程}{$$\frac{\partial^2 u}{\partial t^2}-\nabla^2u = 0$$}

\begin{frame}
\chtitle{弦的横振动方程}
\bch
完全均匀沿水平方向绷直的的弦,线密度为$\lambda$,张力为$f$。弦的各部分沿垂直方向稍稍偏离平衡位置,偏离量$u$是水平位置$x$的函数。

\addfig{2}{string_force.png}

如图考虑弦的一小段,弦的纵向(沿$x$向)净受力为 $$ f \cos\theta_2 - f\cos\theta_1 \approx \frac{1}{2}\left(\theta_1^2-\theta_2^2\right) f $$
这是$\theta_1$和$\theta_2$的高阶小量,我们将它忽略掉(即认为横向是受力平衡的)。

\ech
\end{frame}



\begin{frame}
\chtitle{弦的横振动方程}
\bch
\addfig{2.}{string_force.png}

弦的横向净受力为 $$ f \sin\theta_2 - f\sin\theta_1 \approx \left(\theta_2-\theta_1\right) f $$
$\theta_2$和$\theta_1$可以近似用弦的斜率来代替,设这一小段的坐标从$x$到$x+dx$
$$\theta_2-\theta_1 \approx \left.\frac{\partial u}{\partial x}\right\vert_{x+dx}-\left.\frac{\partial u}{\partial x}\right\vert_{x} \approx \frac{\partial^2u}{\partial x^2} dx $$

\ech
\end{frame}


\begin{frame}
\chtitle{弦的横振动方程}
\bch
\addfig{2.}{string_force.png}
根据$F=ma$,得到弦的这一段沿垂直方向的加速度
$$\frac{\partial ^2u}{\partial t^2} = \frac{f\frac{\partial^2u}{\partial x^2} dx}{\rho dx} = a^2 \frac{\partial^2u}{\partial x^2} ,$$
其中$\rho$为质量线密度, $a$定义为$\sqrt{\frac{f}{\rho}}$,具有速度的量纲。

\ech
\end{frame}



\begin{frame}
\chtitle{弦的横振动方程}
\bch
最后,得到弦的横振动方程为波动方程:
\tbox{\blue $$ \frac{\partial ^2u}{\partial t^2}  -  a^2 \frac{\partial^2u}{\partial x^2} = 0 $$}
波动方程和热传导方程不同:
\bitem
\item{热传导方程分离变量之后得到的时间因子方程是$\Psi'/\Psi =$常数,解的变化规律是指数式衰减(指数式增长模式因一般不符合物理图像而被无情地抛弃);}
\item{
  波动方程分离变量后得到的时间因子方程是$\Psi''/\Psi =$常数,解的变化规律是无衰减的波动(正弦和余弦)。}
  \eitem

\ech
\end{frame}


\begin{frame}
\chtitle{思考题}
\bch
\addfig{0.8}{think1.jpg}

弦一振动不是就被拉长了吗?为什么在前面的推导过程中把张力$f$当成常量?
\ech
\end{frame}


\begin{frame}
\chtitle{杆的纵振动方程}
\bch
设一根均匀弹性杆,一开始处于静止并受力平衡,各个切面的位置可以用$x$来标记。

然后考虑沿杆的方向发生压缩-拉伸的变化: 每个初始坐标为$x$的切面沿着杆的方向有小小的偏离平衡位置的位移$u(x)$。

\addfig{3.2}{longi_osc.png}

在线性近似下,应力$P$(横截面上单位面积受力)和$\frac{\partial u}{\partial x}$成正比 (可以对一小段杆运用胡克定律导出这个结论):

$$ P = E\frac{\partial u}{\partial x},$$
其中$E$称为杨氏模量 (Young's modulus).

\ech
\end{frame}



\begin{frame}
\chtitle{杆的纵振动方程}
\bch

按照套路,对$x$和$x+dx$之间的一小段杆运用强大的$F=ma$,得到
$$ \frac{\partial^2u}{\partial t^2} = \frac{(P(x+dx)-P(x))S}{\rho S dx} = \frac{E}{\rho} \frac{\partial^2u}{\partial x^2},$$
其中 $S$为截面积,$\rho$为质量密度。

\skipline

令$a = \sqrt{\frac{E}{\rho}}$,则又得到了波动方程。

\tbox{\blue $$ \frac{\partial ^2u}{\partial t^2}  -  a^2 \frac{\partial^2u}{\partial x^2} = 0 $$}

\ech
\end{frame}


\begin{frame}
\chtitle{电磁波和引力波}
\bch
上述两个例子里空间是一维的,一般的波动方程为
\tbox{\blue $$\frac{\partial ^2u}{\partial t^2}  -  a^2 \nabla^2 u = 0 $$}
我们马上会看到,$a$的物理意义为波速。

\skiplines


电磁波和引力波都满足上述方程($a$为光速)。

电磁波的情况,$u$是垂直电磁波的传播方向的电场强度或磁场强度;

引力波的情况,$u$是度规二阶张量的分量(此处有黑人问号.jpg)。

\ech
\end{frame}

\section{BD1}
\secpage{齐次边界条件下的解法}{按套路来就行}

\begin{frame}
\chtitle{分离变量法}
\bch
仍然回到一维空间的情况:

令 $u = \Phi(x)\Psi(t)$,代入波动方程,得到

$$ a^2\frac{\Phi'' }{\Phi} =  \frac{\Psi''}{\Psi}, $$
等式两边分别是$x$和$t$的函数,所以只能是常数。由此得到两个解

$$ e^{\ii k(x\mp at)}, $$

其中$k$是任意的常数。

如果我们追踪相位为零的点,则得到$x = \pm at$;即 $e^{\ii k(x - at)}$描述的是沿$x$轴正向传播的波,$e^{\ii k(x + at)}$ 描述的是沿$x$轴负向传播的波。

\ech
\end{frame}


\begin{frame}
\chtitle{分离变量法}
\bch
虽然
$$ e^{\ii k(x\mp at)}, $$
具有很容易看出波速的优点,但这并不是唯一的写法。

\skiplines


当我们需要实数解时,可以把解写成:
$$ \cos (kx)\cos(akt),\ \sin(kx)\cos(akt),\ \cos(kx)\sin(akt),\ \sin(kx)\sin(akt)$$
的线性组合。
\ech
\end{frame}


\begin{frame}
\chtitle{无界的情形}
\bch
在空间无边界的情况下,可以按照套路写出:
$$ u(x,t) = \frac{1}{\sqrt{2\pi}} \int c(k)e^{\ii k(x-at)} dk + \frac{1}{\sqrt{2\pi}} \int d(k)e^{\ii k(x+at)} dk $$
设$c(k)$和$d(k)$的傅立叶逆变换为$C(x)$和$D(x)$。上式只是把傅立叶逆变换中的$x$换成了$x\mp at$,结果应为:
$$ u(x, t) = C(x-at) + D(x+at).$$
这就是无边界波动问题的最一般解。$C$和$D$的形式可以通过初始条件来确定。

{\scriptsize $u(x,t)$显然除了依赖于初始的位移$u$,还依赖于初始的速度$\frac{\partial u}{\partial t}$,因此需要两个初始条件,相应可以确定两个未知函数$C$和$D$。}

\ech
\end{frame}

\begin{frame}
\chtitle{例题}
\bch
求解满足下列初始条件的无边界波动方程:
\bea
\frac{\partial ^2u}{\partial t^2}  -  a^2 \frac{\partial^2u}{\partial x^2} &=& 0, \newl
\left.u\right\vert_{t=0} &=& A e^{-\frac{x^2}{2\sigma^2}} , \newl
\left.\frac{\partial u}{\partial t}\right\vert_{t=0} &=&  0.  
\eea
其中$A,\sigma$均为已知常量。
\ech
\end{frame}

\begin{frame}
\chtitle{解答}
\bch
令 $u(x,t) = C(x-at)+D(x+at)$,则根据初始条件,有
$$ C(x) + D(x) = Ae^{-\frac{x^2}{2\sigma^2}} $$
以及
$$ C'(x) = D'(x)  $$
由此易解出
$$ u(x,t) = \frac{A}{2}\left(e^{-\frac{(x-at)^2}{2\sigma^2}}  + e^{-\frac{(x+at)^2}{2\sigma^2}}\right).$$

\ech
\end{frame}

\begin{frame}
\chtitle{解答}
\chtitle{思考题}
\bch

\addfig{1}{think2.jpg}

把初始条件换成一般的
\bea
\left.u\right\vert_{t=0} &=& \phi(x) , \newl
\left.\frac{\partial u}{\partial t}\right\vert_{t=0} &=&  \psi(x).
\eea
其中$\phi(x)$和$\psi(x)$为给定的函数,你还能求解吗?
\ech
\end{frame}


\begin{frame}
\chtitle{思考题}
\bch

\addfig{1}{think3.jpg}

波动方程也是线性方程,可以用积分变换和格林函数的方法来求解无边界问题吗?
\ech
\end{frame}


\begin{frame}
\chtitle{有边界的情况}
\bch
设弦的两端固定于$x=0$和$x=L$处,则

\bea
\frac{\partial ^2u}{\partial t^2}  -  a^2 \frac{\partial^2u}{\partial x^2} &=& 0, \newl
\left.u\right\vert_{x=0} &=& 0,\newl
\left.u\right\vert_{x=L} &=& 0,\newl
\left.u\right\vert_{t=0} &=& \phi(x) , \newl
\left.\frac{\partial u}{\partial t}\right\vert_{t=0} &=&  \psi(x).
\eea
其中$\phi$和$\psi$均为给定的函数。
\ech
\end{frame}



\begin{frame}
\chtitle{套路}
\bch
    (空间)边界条件已经是齐次的了,可以直接看出解为$\sin(\frac{n\pi}{L} x)\cos{\frac{n\pi a}{L}t}$和$\sin(\frac{n\pi}{L} x)\sin{\frac{n\pi a}{L}t}$的线性组合。不妨设:

    $$u(x,t) = \sum_{n=0}^\infty \sin(\frac{n\pi}{L} x)\left(c_n\cos{\frac{n\pi a}{L}t} + s_n \sin{\frac{n\pi a}{L}t} \right). $$
    利用第一个初始条件:
    $$ \sum_{n=0}^\infty c_n \sin(\frac{n\pi}{L} x) = \phi(x) $$
    可以确定$c_n$。
    再利用第二个初始条件:
    $$ \sum_{n=0}^\infty \frac{n\pi a}{L} s_n \sin(\frac{n\pi}{L} x) = \psi(x) $$   可以确定 $s_n$。
\ech
\end{frame}

\section{BD2}
\secpage{非齐次边界条件}{凑/蒙特解:能否成功看人品…}

\begin{frame}
\chtitle{非齐次边界条件}
\bch
考虑固定在$x=0$和$x=L$之间的弦的横振动问题。设一开始$t=0$时刻弦处于平衡位置静止,在$t>0$时强迫弦的$x=L$端以振幅$A$,频率$\omega$振动。

\bea
\frac{\partial ^2u}{\partial t^2}  -  a^2 \frac{\partial^2u}{\partial x^2} &=& 0, \newl
\left.u\right\vert_{x=0} &=& 0,\newl
\left.u\right\vert_{x=L} &=& A\sin (\omega t),\newl
\left.u\right\vert_{t=0} &=& 0 , \newl
\left.\frac{\partial u}{\partial t}\right\vert_{t=0} &=&  0.
\eea
\ech
\end{frame}


\begin{frame}
\chtitle{求(蒙)特解化为齐次边界条件问题}
\bch
考虑如下的{\blue 满足边界条件但不满足初始条件的特解}:
$$ u_0(x,t) = A \sin(\omega t) \frac{\sin\frac{\omega x}{a}}{\sin\frac{\omega L}{a}}  $$
令$u(x,t) = u_0(x,t) + \upsilon(x,t)$,则$\upsilon(x,t)$满足:


\bea
\frac{\partial ^2\upsilon}{\partial t^2}  -  a^2 \frac{\partial^2\upsilon}{\partial x^2} &=& 0, \newl
\left.\upsilon\right\vert_{x=0} &=& 0, \newl
\left.\upsilon \right\vert_{x=L} &=& 0,\newl
\left.\upsilon \right\vert_{t=0} &=& 0 , \newl
\left.\frac{\partial \upsilon}{\partial t}\right\vert_{t=0} &=&  -A\omega \frac{\sin\frac{\omega x}{a}}{\sin\frac{\omega L}{a}}
\eea

求解$\upsilon(x,t)$的问题我们前面已经讨论过。
\ech
\end{frame}


\section{Homework}

\begin{frame}
\chtitle{课后作业 (题号 27-29)}
\bch
\bitem
\item[27]{求解无边界的定解问题:
\bea
\frac{\partial ^2u}{\partial t^2}  -  a^2 \frac{\partial^2u}{\partial x^2} &=& 0, \newl
\left.u\right\vert_{t=0} &=& A e^{-\frac{x^2}{2\sigma^2}} , \newl
\left.\frac{\partial u}{\partial t}\right\vert_{t=0} &=&  Bxe^{-\frac{x^2}{2\sigma^2}} .  
\eea
其中$a, A,B,\sigma$均为已知常量。
}
\eitem
\ech
\end{frame}


\begin{frame}
\chtitle{课后作业 (题号 27-29)}
\bch
\bitem
\item[28]{求解$0\le x\le L$上的定解问题:
\bea
\frac{\partial ^2u}{\partial t^2}  -  a^2 \frac{\partial^2u}{\partial x^2} &=& 0, \newl
\left. u\right\vert_{x=0} &=& 0,\newl
\left.\frac{\partial u}{\partial x}\right\vert_{x=L} &=& 0,\newl
\left.u\right\vert_{t=0} &=& \phi(x) , \newl
\left.\frac{\partial u}{\partial t}\right\vert_{t=0} &=& \psi(x) 
\eea
其中 $a$为常量, $\phi$和$\psi$为已知函数。
}
\eitem
\ech
\end{frame}


\begin{frame}
\chtitle{课后作业 (题号 27-29)}
\bch
\bitem
\item[29]{
我们在课上讨论了$0\le x\le L$上的定解问题:
\bea
\frac{\partial ^2u}{\partial t^2}  -  a^2 \frac{\partial^2u}{\partial x^2} &=& 0, \newl
\left. u\right\vert_{x=0} &=& 0,\newl
\left. u\right\vert_{x=L} &=& A \sin(\omega t),\newl
\left.u \right\vert_{t=0} &=& 0 , \newl
\left.\frac{\partial u}{\partial t}\right\vert_{t=0} &=& 0 
\eea
其中$a$, $A$, $\omega$均为已知常量。把求解过程补充完整,求出最终的解,并讨论:当$\frac{\omega L}{\pi a}$为整数时,是否有可能维持这种振动?
}
\eitem
\ech
\end{frame}


\end{document}
