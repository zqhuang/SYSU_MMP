\documentclass[CJK]{article}
\usepackage{geometry}
\input{reduced_macros.tex}
\geometry{tmargin=0.5in, bmargin=0.5in, lmargin=0.7in, rmargin=0.8in, nohead, nofoot}
\begin{document}
\bch
\bcenter
中山大学物理与天文学院 《数学物理方法》 2018学年 期中小测 (共4页,满分100分) 

{\vskip 0.04in}

姓名 \uline{1.2} {\hskip 0.5in}    学号 \uline{1.2}{\hskip 0.5in} 得分 \uline{0.5}

\ecenter

{\vskip 0.02in}
    
{\bf \noindent(一)选择题,每小题4分,共60分。请集中把答案写在下面的答案区:}

{\vskip 0.04in}

{(1) \uline{0.5}\  (2) \uline{0.5}\ (3) \uline{0.5}\  (4) \uline{0.5}\ (5) \uline{0.5}}

{\vskip 0.04in}

{(6) \uline{0.5}\ (7) \uline{0.5}\ (8) \uline{0.5}\  (9) \uline{0.5}\ (10) \uline{0.43}}

{\vskip 0.04in}

{(11) \uline{0.43}\  (12) \uline{0.43}\ (13) \uline{0.43}\ (14) \uline{0.43}\ (15) \uline{0.43}}

\bitem
\item[(1)]{$e^{\frac{3\pi i}{2}}$等于 

  \opts{$-i$}{$i$}{$-1$}{$1$}
}
\item[(2)]{满足下列哪一项条件的所有复数$z$构成的点集是复平面上的开区域?

  \opts{$1\le |z|\le 2$}{$z$是实数且$|z|<1$}{$z=0$}{$0<|z|<1$}
}
\item[(3)]{方程$z^3=1+i$的所有复数解为 

  \opts{$2^{\frac{1}{6}}e^{\frac{(4n+1)\pi i}{12}}$, $n=0,1,2$}{$2^{\frac{1}{6}}e^{\frac{(8n+1)\pi i}{12}}$, $n=0,1,2$}{$2^{\frac{1}{6}}e^{\frac{(8n-1)\pi i}{12}}$, $n=0,1,2$}{$e^{\frac{(4n-1)\pi i}{12}}$, $n=0,1,2$}
}
\item[(4)]{复变函数$z\cos z$的导函数是:
  
  \opts{$z\sin z$}{$\cos z$}{$\cos z + z\sin z$}{$\cos z - z\sin z$}}
\item[(5)]{复变函数 $z\sin z$ 的不定积分(忽略不写积分常数)是 
  
  \opts{$z\cos z$}{$ \sin z$}{$ \sin z+z\cos z$}{$\sin z - z\cos z$} }
\item[(6)]{$\frac{1}{(1+e^z)\sin z}$ 在区域$|z|< 5 $内有多少个孤立奇点? 
  
  \opts{$3$}{$4$}{$5$}{$6$}}
\item[(7)]{$\frac{1}{z^2-3z+2}$ 在 $z=2$ 处的留数等于 

    \opts{$2$}{$1$}{$0$}{$-1$}}  
\item[(8)]{积分$\int_{-1}^1 \delta(x-2) \cos x\, dx =$

    \opts{$0$}{$\tan 2$}{$\cos 2$}{$\sin 2$}}  
\item[(9)]{下列哪个多值函数在区域 $1<|z|<2$ 内可以规定适当的幅角范围成为解析函数? 

  \opts{$\ln(z-1)$}{$\ln (z+1)$}{$\ln [(z-1)(z+1)]$}{$\ln \frac{z-1}{z+1}$}}
\item[(10)]{$\Gamma(- \frac{1}{2}) = $ 

  \opts{$\frac{\sqrt{\pi}}{2}$}{$-2\sqrt{\pi}$}{$\sqrt{\pi}$}{$-\frac{\sqrt{\pi}}{2}$}}
\item[(11)]{函数 $\frac{2z^4}{z^{5}+z+1}$ 的所有孤立奇点处的留数之和为 

    \opts{$0$}{$1$}{$2$}{$-1$}}  
\item[(12)]{函数 $f(t) = \delta(t^2-1)$ 的拉普拉斯变换$F(p)=$ 

  \opts{$e^{-p}$}{$e^p+e^{-p}$}{$\frac{1}{2} e^{-p}$}{$\frac{1}{2}\left(e^p+e^{-p}\right)$}}
\item[(13)]{设$f(x) = e^{-\frac{x^2}{2}}$的傅立叶变换为$F(k)$,则积分$\int_{-\infty}^\infty \left\vert F(k)\right\vert^2\,dk$等于

  \opts{$\sqrt{\pi}$}{$\frac{\pi}{2}$}{$\pi$}{$\sqrt{\frac{2}{\pi}}$}
}
\item[(14)]{积分 $\int_0^\infty e^{-x^2}\cos{(2x)}\,dx$ 等于

  \opts{$e\sqrt{\pi}$}{$\frac{\sqrt{\pi}}{2e}$}{$\frac{\sqrt{\pi}}{e}$}{$\frac{e\sqrt{\pi}}{2}$}
}
\item[(15)]{某个正交曲面坐标系$(x,y,z)$的相近两点之间的距离平方可以写为:$ ds^2 = dx^2 + e^{2x}\left(dy^2 + dz^2\right).$
  该坐标系的拉普拉斯算符$\nabla^2$的显式微分表达式为:

  \opts{$e^{2x}\left[\frac{\partial(e^{-2x}\frac{\partial }{\partial x})}{\partial x}+\frac{\partial^2}{\partial y^2}+\frac{\partial^2}{\partial z^2}\right]$}{$e^{-2x}\left[\frac{\partial(e^{2x}\frac{\partial }{\partial x})}{\partial x}+\frac{\partial^2}{\partial y^2}+\frac{\partial^2}{\partial z^2}\right]$}{$\frac{\partial(e^{2x}\frac{\partial }{\partial x})}{\partial x}+\frac{\partial^2}{\partial y^2}+\frac{\partial^2}{\partial z^2}$}{$\frac{\partial^2}{\partial x^2}+\frac{\partial^2}{\partial y^2}+\frac{\partial^2}{\partial z^2}$}
}
  
  
\eitem

\newpage
{\bf \noindent (二)问答题,每小题10分,共30分。}

\bitem
\item[(1)]{解析函数的定义是什么?请用不超过30字的一句话简洁地叙述(关于解析函数积分的)柯西定理。
{\vskip 2.5in}
  }
\item[(2)]{把函数$f(z) = \frac{1}{(z-2)(z-3)}$在环形区域$2<|z|<3$内展开成洛浪级数。
\newpage
\bcenter

姓名 \uline{1.2} {\hskip 0.5in}    学号 \uline{1.2}{\hskip 0.5in}

\ecenter

{\vskip 0.1in}
}
\item[(3)]{

  计算沿逆时针方向的围道积分
  $$\frac{1}{2\pi i}\oint_{|z|=1} \,\frac{\cos z}{z^{273}}\,dz,$$
  并估算结果的数量级(大概是10的多少次方)。
}  
\eitem

\newpage
{\bf \noindent(三)选答题,10分。请在下列若干个问题中勾选并回答一个问题(请不要多选,否则不计分)。}

\bitem
\item[{\bf $\Box$}]{请举出一个在整个复平面上有定义,在无穷多个点可导,却处处不解析的复变函数的例子。}
\item[{\bf $\Box$}]{计算函数$f(z) = \ln \Gamma(z)$在$z=1$处的二阶导数$f''(1)$。}
\item[{\bf $\Box$}]{三维直角坐标系中曲面$x^4+y^4+z^4=1$包围的体积是多少?}
\item[{\bf $\Box$}]{求解$f(t)$的初值问题: $f'' + f + 2\sin t = 0$, $f(0)= 0$,  $f'(0) = 1$.}
  \eitem



  {\vskip 6.2in}
{\noindent \bf 公式表}
\bitem
  \item[(1)]{$e^{\beta t} t^\alpha$的拉普拉斯变换为$\frac{\Gamma(\alpha+1)}{(p-\beta)^{\alpha+1}}$;$\cos(\omega t)$的拉普拉斯变换为$\frac{p}{p^2+\omega^2}$;$\sin(\omega t)$的拉普拉斯变换为$\frac{\omega}{p^2+\omega^2}$.}
  \item[(2)]{$\Gamma$函数互余宗量关系$\Gamma(z)\Gamma(1-z) = \frac{\pi}{\sin{(\pi z)}}.$}
  \item[(3)]{$n$维限和积分公式:
 $$\int_{\Omega_n} x_1^{\alpha_1-1}x_2^{\alpha_2-1}\ldots x_n^{\alpha_n-1} f(x_1+x_2+\ldots+x_n)dx_1dx_2\ldots dx_n= \frac{\Gamma(\alpha_1)\Gamma(\alpha_2)\ldots \Gamma(\alpha_n)}{\Gamma(\alpha_1+\alpha_2+\ldots + \alpha_n)}\int_0^1f(u)u^{\alpha_1+\alpha_2+\ldots + \alpha_n-1} du ,$$
  其中等式左边的积分区域$\Omega_n = \{(x_1,x_2,\ldots,x_n): x_1,x_2,\ldots, x_n\ge 0; x_1+x_2+\ldots+x_n\le 1 \}$.
  }
    \eitem

{\vskip 0.02in}

\bcenter
    {\scriptsize 试卷印刷批号: \version}
\ecenter
\ech
\end{document}
