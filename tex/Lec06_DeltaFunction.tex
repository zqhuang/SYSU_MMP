\documentclass[CJK]{beamer}
\usepackage{CJKutf8}
\usepackage{beamerthemesplit}
\usetheme{Malmoe}
\useoutertheme[footline=authortitle]{miniframes}
\usepackage{amsmath}
\usepackage{amssymb}
\usepackage{graphicx}
\usepackage{eufrak}
\usepackage{color}
\usepackage{slashed}
\usepackage{simplewick}
\usepackage{tikz}
\usepackage{tcolorbox}
\graphicspath{{../figures/}}
%%figures
\def\lfig#1#2{\includegraphics[width=#1 in]{#2}}
\def\addfig#1#2{\begin{center}\includegraphics[width=#1 in]{#2}\end{center}}
\def\wulian{\includegraphics[width=0.18in]{emoji_wulian.jpg}}
\def\bigwulian{\includegraphics[width=0.35in]{emoji_wulian.jpg}}
\def\bye{\includegraphics[width=0.18in]{emoji_bye.jpg}}
\def\bigbye{\includegraphics[width=0.35in]{emoji_bye.jpg}}
\def\huaixiao{\includegraphics[width=0.18in]{emoji_huaixiao.jpg}}
\def\bighuaixiao{\includegraphics[width=0.35in]{emoji_huaixiao.jpg}}
\def\jianxiao{\includegraphics[width=0.18in]{emoji_jianxiao.jpg}}
\def\bigjianxiao{\includegraphics[width=0.35in]{emoji_jianxiao.jpg}}
%% colors
\def\blacktext#1{{\color{black}#1}}
\def\bluetext#1{{\color{blue}#1}}
\def\redtext#1{{\color{red}#1}}
\def\darkbluetext#1{{\color[rgb]{0,0.2,0.6}#1}}
\def\skybluetext#1{{\color[rgb]{0.2,0.7,1.}#1}}
\def\cyantext#1{{\color[rgb]{0.,0.5,0.5}#1}}
\def\greentext#1{{\color[rgb]{0,0.7,0.1}#1}}
\def\darkgray{\color[rgb]{0.2,0.2,0.2}}
\def\lightgray{\color[rgb]{0.6,0.6,0.6}}
\def\gray{\color[rgb]{0.4,0.4,0.4}}
\def\blue{\color{blue}}
\def\red{\color{red}}
\def\green{\color{green}}
\def\darkgreen{\color[rgb]{0,0.4,0.1}}
\def\darkblue{\color[rgb]{0,0.2,0.6}}
\def\skyblue{\color[rgb]{0.2,0.7,1.}}
%%control
\def\be{\begin{equation}}
\def\ee{\nonumber\end{equation}}
\def\bea{\begin{eqnarray}}
\def\eea{\nonumber\end{eqnarray}}
\def\bch{\begin{CJK}{UTF8}{gbsn}}
\def\ech{\end{CJK}}
\def\bitem{\begin{itemize}}
\def\eitem{\end{itemize}}
\def\bcenter{\begin{center}}
\def\ecenter{\end{center}}
\def\bex{\begin{minipage}{0.2\textwidth}\includegraphics[width=0.6in]{jugelizi.png}\end{minipage}\begin{minipage}{0.76\textwidth}}
\def\eex{\end{minipage}}
\def\chtitle#1{\frametitle{\bch#1\ech}}
\def\bmat#1{\left(\begin{array}{#1}}
\def\emat{\end{array}\right)}
\def\bcase#1{\left\{\begin{array}{#1}}
\def\ecase{\end{array}\right.}
\def\bmini#1{\begin{minipage}{#1\textwidth}}
\def\emini{\end{minipage}}
\def\tbox#1{\begin{tcolorbox}#1\end{tcolorbox}}
\def\pfrac#1#2#3{\left(\frac{\partial #1}{\partial #2}\right)_{#3}}
%%symbols
\def\bropt{\,(\ \ \ )}
\def\sone{$\star$}
\def\stwo{$\star\star$}
\def\sthree{$\star\star\star$}
\def\sfour{$\star\star\star\star$}
\def\sfive{$\star\star\star\star\star$}
\def\rint{{\int_\leftrightarrow}}
\def\roint{{\oint_\leftrightarrow}}
\def\stdHf{{\textit{\r H}_f}}
\def\deltaH{{\Delta \textit{\r H}}}
\def\ii{{\dot{\imath}}}
\def\skipline{{\vskip0.1in}}
\def\skiplines{{\vskip0.2in}}
\def\lagr{{\mathcal{L}}}
\def\hamil{{\mathcal{H}}}
\def\vecv{{\mathbf{v}}}
\def\vecx{{\mathbf{x}}}
\def\vecy{{\mathbf{y}}}
\def\veck{{\mathbf{k}}}
\def\vecp{{\mathbf{p}}}
\def\vecn{{\mathbf{n}}}
\def\vecA{{\mathbf{A}}}
\def\vecP{{\mathbf{P}}}
\def\vecsigma{{\mathbf{\sigma}}}
\def\hatJn{{\hat{J_\vecn}}}
\def\hatJx{{\hat{J_x}}}
\def\hatJy{{\hat{J_y}}}
\def\hatJz{{\hat{J_z}}}
\def\hatj#1{\hat{J_{#1}}}
\def\hatphi{{\hat{\phi}}}
\def\hatq{{\hat{q}}}
\def\hatpi{{\hat{\pi}}}
\def\vel{\upsilon}
\def\Dint{{\mathcal{D}}}
\def\adag{{\hat{a}^\dagger}}
\def\bdag{{\hat{b}^\dagger}}
\def\cdag{{\hat{c}^\dagger}}
\def\ddag{{\hat{d}^\dagger}}
\def\hata{{\hat{a}}}
\def\hatb{{\hat{b}}}
\def\hatc{{\hat{c}}}
\def\hatd{{\hat{d}}}
\def\hatN{{\hat{N}}}
\def\hatH{{\hat{H}}}
\def\hatp{{\hat{p}}}
\def\Fup{{F^{\mu\nu}}}
\def\Fdown{{F_{\mu\nu}}}
\def\newl{\nonumber \\}
\def\vece{\mathrm{e}}
\def\calM{{\mathcal{M}}}
\def\calT{{\mathcal{T}}}
\def\calR{{\mathcal{R}}}
\def\barpsi{\bar{\psi}}
\def\baru{\bar{u}}
\def\barv{\bar{\upsilon}}
\def\qeq{\stackrel{?}{=}}
\def\torder#1{\mathcal{T}\left(#1\right)}
\def\rorder#1{\mathcal{R}\left(#1\right)}
\def\contr#1#2{\contraction{}{#1}{}{#2}#1#2}
\def\trof#1{\mathrm{Tr}\left(#1\right)}
\def\trace{\mathrm{Tr}}
\def\comm#1{\ \ \ \left(\mathrm{used}\ #1\right)}
\def\tcomm#1{\ \ \ (\text{#1})}
\def\slp{\slashed{p}}
\def\slk{\slashed{k}}
\def\calp{{\mathfrak{p}}}
\def\veccalp{\mathbf{\mathfrak{p}}}
\def\Tthree{T_{\tiny \textcircled{3}}}
\def\pthree{p_{\tiny \textcircled{3}}}
\def\dbar{{\,\mathchar'26\mkern-12mu d}}
\def\erf{\mathrm{erf}}
\def\const{\mathrm{constant}}
\def\pheat{\pfrac p{\ln T}V}
\def\vheat{\pfrac V{\ln T}p}
%%units
\def\fdeg{{^\circ \mathrm{F}}}
\def\cdeg{^\circ \mathrm{C}}
\def\atm{\,\mathrm{atm}}
\def\angstrom{\,\text{\AA}}
\def\SIL{\,\mathrm{L}}
\def\SIkm{\,\mathrm{km}}
\def\SIyr{\,\mathrm{yr}}
\def\SIGyr{\,\mathrm{Gyr}}
\def\SIV{\,\mathrm{V}}
\def\SImV{\,\mathrm{mV}}
\def\SIeV{\,\mathrm{eV}}
\def\SIkeV{\,\mathrm{keV}}
\def\SIMeV{\,\mathrm{MeV}}
\def\SIGeV{\,\mathrm{GeV}}
\def\SIcal{\,\mathrm{cal}}
\def\SIkcal{\,\mathrm{kcal}}
\def\SImol{\,\mathrm{mol}}
\def\SIN{\,\mathrm{N}}
\def\SIHz{\,\mathrm{Hz}}
\def\SIm{\,\mathrm{m}}
\def\SIcm{\,\mathrm{cm}}
\def\SIfm{\,\mathrm{fm}}
\def\SImm{\,\mathrm{mm}}
\def\SInm{\,\mathrm{nm}}
\def\SImum{\,\mathrm{\mu m}}
\def\SIJ{\,\mathrm{J}}
\def\SIW{\,\mathrm{W}}
\def\SIkJ{\,\mathrm{kJ}}
\def\SIs{\,\mathrm{s}}
\def\SIkg{\,\mathrm{kg}}
\def\SIg{\,\mathrm{g}}
\def\SIK{\,\mathrm{K}}
\def\SImmHg{\,\mathrm{mmHg}}
\def\SIPa{\,\mathrm{Pa}}

\def\courseurl{https://github.com/zqhuang/SYSU\_TD}

\def\tpage#1#2{
\begin{frame}
\begin{center}
\begin{Large}
\bch
热学 \\
第#1讲 #2

{\vskip 0.3in}

黄志琦

\ech
\end{Large}
\end{center}

\vskip 0.2in

\bch
教材:《热学》第二版,赵凯华,罗蔚茵,高等教育出版社
\ech

\bch
课件下载
\ech
\courseurl
\end{frame}
}

\def\bfr#1{
\begin{frame}
\chtitle{#1} 
\bch
}

\def\efr{
\ech 
\end{frame}
}

  \date{}
  \begin{document}
  \bch
\tpage{6}{Delta Function}

\begin{frame}
  \tableofcontents
\end{frame}

\section{Delta function}
\secpage{又高又瘦的$\delta$函数}{$$\int_{-\infty}^{\infty} \delta(x-x_0)f(x) \,dx = f(x_0)$$}

\begin{frame}
  \frametitle{物理里的理想化模型}
  
  物理里有很多“无穷大$\times$无穷小 = 有限“的模型:
  \bitem
\item{瞬时冲量: 力无限大,作用时间无穷短,但两者的乘积(冲量)是有限的。}
\item{质点:质量密度无穷大,体积无穷小,但两者的乘积(总质量)是有限的。}
\item{点电荷:电荷密度无穷大,体积无穷小,但两者的乘积(总电荷)是有限的。}
  \eitem

  \skiplines
  
  在数学上这些表述都是不合法的,需要搞很多事情才能把这些模型说清楚。 不喜欢搞事情的物理学家们于是发明了{\blue $\delta$函数}。
  
\end{frame}

\begin{frame}
  \frametitle{重量级dalao —— Paul Dirac (狄拉克)}
  
  \bcenter
  \lfig{1.2}{Dirac.jpg}

  你们不要搞事情
  \ecenter
  
\end{frame}
  
\begin{frame}
  \frametitle{Dirac $\delta$ function}
  
  \bmini{0.4}
  \lfig{1.5}{DiracDelta.png}
  \emini
  \bmini{0.55}
  Dirac $\delta$ function 不是传统意义上的函数。它可以通过下面的{\blue 单位脉冲函数取脉冲时间为零的极限}得到:
  $$\delta_D(x) = \left\{\begin{array}{ll} \frac{1}{\epsilon}, & \text{ if } -\frac{\epsilon}{2}<x<\frac{\epsilon}{2} \\  0, & \text{ else}\end{array}\right. $$
  其中$\epsilon\rightarrow 0^+$。  
  \emini

  在本课程中,我们{\blue 简称Dirac $\delta$ function为$\delta$函数,并简写为$\delta(x)$}。
  
\end{frame}


\begin{frame}
  \frametitle{$\delta$函数的另一种逼近方式}
  
  有时候需要计算$\delta$函数的导数甚至高阶导数,这时可以考虑用高斯函数逼近方式:

  $$\delta(x) = \frac{1}{\sqrt{2\pi\epsilon}}e^{-\frac{x^2}{2\epsilon}},$$
  其中$\epsilon\rightarrow 0^+$。

%  {\scriptsize 注:这一般只是为了帮助理解$\delta$函数的导数的图像,并非为了计算。如果要用具体的逼近方式来进行计算,$\delta$函数的便捷性就大打折扣了。}
  
\end{frame}


\begin{frame}
  \frametitle{$\delta$函数的抽象定义}
  
  借助上述两种逼近方式的辅助,我们归纳出$\delta$函数的下述抽象定义:
  {\blue
    \be
    \delta(x) = \left\{
    \begin{array}{ll}
      0, & \text{ if } x\ne 0; \\
      +\infty, & \text{ if } x = 0;
    \end{array}\right.
    \ee
    $$ \int_{-\infty}^\infty \delta(x)\,dx = 1.$$ 
  }
  从上述抽象定义中可以看出$\delta$函数是偶函数:
  $$\delta(-x) = \delta(x)$$
  此外,积分的范围可以限定在$0$的任意小领域。
    $$ \int_{0^-}^{0^+} \delta(x)\,dx = 1.$$   
  
\end{frame}

\begin{frame}
  \frametitle{$\delta$函数最重要的性质}
  
  从被使用频率上来讲,下式至关重要:
  \tbox{
    $$\int_{-\infty}^\infty \delta(x-x_0) f(x)\, dx = f(x_0).$$}

  (请自行用物理图像理解上式)
  
\end{frame}

\begin{frame}
  \frametitle{思考题}
  
  计算积分
  $$\int_{-\infty}^{\infty}   \delta(x-\frac{\pi}{2})\cos x\,dx $$
  
\end{frame}


\begin{frame}
  \frametitle{$\delta$函数的导函数}
  
  利用$\delta$函数在两边都是零的特点,可以用分部积分的方法得到$\delta$函数的导数的性质:
  {\blue
  $$\int_{-\infty}^\infty \left[\frac{d^n}{dx^n}\delta(x-x_0)\right] f(x)\, dx = (-1)^n f^{(n)}(x_0)$$}
  
\end{frame}

\begin{frame}
  \frametitle{思考题}
  
  计算积分
  $$\int_{-\infty}^{\infty}   \delta'(x-\frac{\pi}{2})\cos x\,dx $$
  
\end{frame}



\begin{frame}
  \frametitle{$\delta$函数的终极大招}
  
  用变量替换的方法可以得到:\tbox{
    $$ \delta\left(\alpha(x)\right)  = \sum_{\rm roots}  \frac{\delta(x-x_i)}{|\alpha'(x_i)|}$$}
  求和对所有$\alpha(x)$的根$x_1$, $x_2$, \ldots 进行。

   {\scriptsize (请思考当$\alpha(x)$没有根或者有重根的情形上面的等式会如何。)}

  \skiplines
  
 {\small 另一种等价的写法是:
  $$\int_{-\infty}^\infty \delta\left(\alpha(x)\right) f(x) \,dx = \sum_{\rm roots} \frac{f(x_i)}{|\alpha'(x_i)|}.$$}
  
  
\end{frame}


\begin{frame}
  \frametitle{思考题}
  计算积分
  $$\int_{-\infty}^\infty \delta(x^2-1) e^x\, dx$$
\end{frame}


\begin{frame}
  \frametitle{思考题}
  计算积分
  $$\int_0^\infty \delta(x^2-1) e^x\, dx$$
\end{frame}



\section{Integral Expression of Delta Function}
\secpage{$\delta$函数的积分表示}{$$\frac{1}{2\pi}\int_{-\infty}^\infty e^{ikx}dk = \delta(x).$$}


\begin{frame}
  \frametitle{$\delta$函数的积分表示}
  \tbox{$$\frac{1}{2\pi}\int_{-\infty}^\infty e^{ikx}dk = \delta(x).$$}

  在大量的物理问题中会用到它。
\end{frame}


\begin{frame}
  \frametitle{证明大意}
  \bea
  &&  \frac{1}{2\pi}\int_{-\infty}^\infty e^{ikx-\frac{\epsilon}{2}k^2}dk \newl
  &=& \lim_{\epsilon\rightarrow 0^+}\frac{1}{2\pi}\int_{-\infty}^\infty e^{-\frac{\epsilon}{2}(k-\frac{ix}{\epsilon})^2-\frac{x^2}{2\epsilon }}dk \newl
  &=&  \lim_{\epsilon\rightarrow 0^+}\frac{1}{\sqrt{2\pi\epsilon}}e^{-\frac{ x^2}{2\epsilon}} \newl  
  &=& \delta(x)
  \eea
\end{frame}



\begin{frame}
  \frametitle{思考题}
  定义函数
  $$ F(k) = \int_0^\infty e^{-x+ikx} dx, $$
  计算积分
  $$ \int_{-\infty}^\infty F(k) dk. $$
\end{frame}


\section{Multi-dimensional Delta Function}
\secpage{高维空间$\delta$函数}{$$\int \delta^{(n)}(\vecx - \vecx_0) f(\vecx)\, d^n\vecx = f(\vecx_0).$$}


\begin{frame}
  \frametitle{$n$维内积空间}
  
  $n$维内积空间的点可以记作$\vecx = (x_1, x_2, \ldots, x_n)$,多元函数$f(x_1, x_2, \ldots, x_n)$可以简单写成$f(\vecx)$。

  \skipline
  
  两个矢量$\vecx, \vecy$的内积为
 {\blue $$ \vecx\cdot\vecy = x_1y_1+x_2y_2+\ldots + x_n y_n .$$}

  \skipline

  函数$f(\vecx)$在全空间的积分可以简单写作
 {\blue $$\int f(\vecx)\, d^n\vecx $$}
  用你们熟悉的语言写出来就是
  $$\int_{-\infty}^\infty d x_1 \int_{-\infty}^\infty dx_2 \ldots \int_{-\infty}^\infty d x_n\, f(x_1, x_2, \ldots, x_n). $$


  \end{frame}

\begin{frame}
  \frametitle{$n$维空间的$\delta$函数}
  
  $n$维$\delta$函数标记为{\blue $\delta^{(n)}(\vecx)$},既可以理解为
      {\blue
        $$\delta^{(n)}(\vecx) = \delta(x_1)\delta(x_2)\ldots\delta(x_n), $$
      }
      也可以直接抽象地理解为{\blue 在原点附近体积为$\epsilon \rightarrow 0^+$的邻域内,函数值为$\frac{1}{\epsilon}\rightarrow \infty $,而在其余位置函数值均为零的函数。}

      \skiplines
      
\bex
三维空间某点$\vecx‘$处的点电荷的电荷密度可以写成
$$\rho(\vecx) = Q\, \delta^{(3)}(\vecx - \vecx')$$
\eex

\end{frame}

\begin{frame}
  \frametitle{$n$维空间的$\delta$函数的性质}
  
  和一维空间类似,
  \tbox{
   $$ \int \delta^{(n)}(\vecx - \vecx_0) f(\vecx)\, d^n\vecx = f(\vecx_0)$$
   }

\end{frame}


\begin{frame}
  \frametitle{思考题}
  静电势可以用积分
  $$U = \frac{1}{4\pi\epsilon_0} \int \frac{\rho(\vecx)}{|\vecx|} d^3\vecx, $$
  来计算。

  \skipline
  
  由此计算在$\vecx_0$处点电荷$Q$产生的电势。
\end{frame}


\begin{frame}
  \frametitle{$n$维空间的$\delta$函数的积分表示}
  
  \tbox{
   $$ \frac{1}{(2\pi)^n} \int e^{i\veck\cdot\vecx} d^n\veck =  \delta^{(n)}(\vecx)$$
   }

\end{frame}



\section{Homework}

\begin{frame}
  \frametitle{Homework for Quizphobias}
  
  \bitem
\item[16]{计算积分$$\int_{-\infty}^\infty \delta\left(\sin x\right) e^{-|x|} dx.$$ }
\item[17]{定义函数
  $$ F(k) = \int_0^\infty e^{-x+ik(x^2-1)} dx, $$
  计算积分
  $$ \int_{-\infty}^\infty F(k) dk. $$}
\item[18]{把二维空间的$\delta$函数写成积分表示,然后换到极坐标下重新写一遍这个积分表示。}  
  \eitem
  
\end{frame}



\ech
\end{document}
