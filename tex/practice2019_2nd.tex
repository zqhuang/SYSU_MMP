\documentclass[12pt,CJK]{article}
\usepackage{geometry}
\usepackage{ulem}
\input{reduced_macros.tex}
\geometry{tmargin=0.5in, bmargin=0.5in, lmargin=0.6in, rmargin=0.9in, nohead, nofoot}
\def\mark#1{{\ \color{blue}(#1分)}}
\renewcommand{\thepage}{}
\begin{document}
\bch
\begin{center}
数学物理方法2019 课堂小测II

{\vskip 0.1in}

  共100分: 30分签到 + 70分 piece of cake 

\end{center}
{\vskip 0.1in}

{\bf 请在答题纸上任意选做下列题目(标明题号),\sout{凑够70分就行}\ 能凑几分算几分吧……}

{\vskip 0.1in}

\bitem
\item[(1)]{(为了检验你期中以前内容是否都量子波动速读了)请计算沿着逆时针方向的围道的积分:
    $$\int_{|z|=1}\, \frac{1}{\sin z - z\cos z} dz.$$
  注意使用留数定理时不要漏掉孤立奇点。\mark{15}}
\item[(2)]{设函数 $f(x,y), g(x,y)$ 都定义在直角坐标系的椭圆 $\Omega = \{(x,y): \frac{x^2}{4}+y^2\le 1\} $ 内,且处处满足方程
    $$  (x^2y^2 + \nabla^2)f(x,y) = (1-x^2y^2-\nabla^2)g(x,y)  = 0.$$
    在$\Omega$的边界上, $f$ 和 $g$ 满足:
    $$\left.x\frac{\partial f}{\partial x}+4y\frac{\partial f}{\partial y}\right\vert_{\frac{x^2}{4}+y^2=1} =  \left.x\frac{\partial g}{\partial x}+4y\frac{\partial g}{\partial y}\right\vert_{\frac{x^2}{4}+y^2=1} = 0.$$
    证明: $$\int_\Omega f(x,y)g(x,y) dx dy = 0.$$ \mark{20}
  }  
\item[(3)]{设函数 $f(x) = x^\alpha J_m(x)$ (这里的$\alpha$为实数, $m$为非负整数)在 $x=\mu$ 和 $x=\nu$ 均取到极值,且 $\mu>\nu>0$。请证明 $\int_0^1 x J_m(\mu x)J_m(\nu x) dx  = 0.$ \mark{15}}
\item[(4)]{两端固定在同一水平线上,线质量密度为$\rho$,长度为 $L$ 的绷直的均匀弦上的张力为$F$。把弦的中点竖直往上拉$\epsilon$的小距离,然后从静止释放。忽略重力、空气阻力的影响,弦之后如何振动? \mark{25}}
\item[(5)]{一个半径为 $R$ 的均匀孤立圆盘,其材质的热传导方程参数为 $a$。一开始圆盘的温度分布为 $$\left.T\right\vert_{t=0} = T_0\left(2-\frac{r^2}{R^2}\right)\cos^2\theta,$$这里的 $(r,\theta)$ ($0\le r\le R$) 是以圆盘中心为原点建立的极坐标,$T_0$ 为常量。请计算之后圆盘上的温度变化 $T(r,\theta, t)$。\mark{30}}
\item[(6)]{一个半径为 $R$,周边固定的均匀圆形薄膜,其横向小振动的波速参数为 $a$。一开始薄膜各处均处于静止,横向小位移为 $$\left.u\right\vert_{t=0} = A\frac{r}{R}\left(1-\frac{r^2}{R^2}\right)\cos\theta,$$这里的 $(r,\theta)$ ($0\le r\le R$) 是以薄膜中心为原点建立的极坐标,$A$为常量。请计算之后薄膜的横向小振动 $u(r,\theta, t)$。\mark{30}}
\item[(7)]{从前有一个神秘的星球叫浮士德星,这个星球的每年有一万天。星球上的人出生时都会和魔鬼签署了一份协议:每个人每天起床都要投掷 $600$ 次骰子。如果 $1,2,3,4,5,6$ 每个数字均恰好出现 $100$ 次,魔鬼就要带走这个人的灵魂,否则魔鬼会保护这个人一天平安无事。\\
    有一天赌王来到了浮士德星,传授给了浮士德星人一个秘术——666大法。使用秘术之后,每次掷骰子出现 $6$ 的概率会稍稍增加,但增加的幅度非常小以致于魔鬼无法察觉。从此浮士德星人的平均寿命延长了5年。请由此推断: 秘术能使单次掷骰子出现 $6$ 的概率增加多少? (结果请保留至少2位有效数) \mark{30}}
\item[(8)]{一个质量为$m$,固有圆频率为 $\omega$ 的谐振子,其三个空间坐标 $x,y,z$ 和三个动量分量 $p_x, p_y, p_z$ 构成了六维的 ``相空间''。它在任意时刻的状态(即位置和运动情况)可以用相空间的一个点来表示。从经典物理的角度看,谐振子的状态允许是相空间中任意一个点。但如果要求谐振子的能量不超过某个阈值 $E_0$ (常量$E_0>0$),即
    $$\frac{1}{2}m\omega^2 \left(x^2+y^2+z^2\right) + \frac{1}{2m}\left(p_x^2+p_y^2+p_z^2\right) \le E_0,$$      
    谐振子就被限制在相空间一个有限大小的区域内。这个区域的(六维)体积是多少?\mark{15}}
\item[(9)]{假设你MMP考试挂科,未能毕业,流落江湖摆摊卖糖葫芦为生。六一儿童节到了,糖葫芦摊抽奖送礼大酬宾。你在一个大袋子里装了很多红、黄、蓝色的小球,每次摸出每种颜色小球的概率都是1/3。你请光临摊位的小孩随机摸出 12 个小球,不计颜色的次序,摸出的三种颜色小球的个数有18种可能性: (0,0,12), (0,1,11), (0,2,10), (0,3,9), (0,4,8), (0,5,7), (0,6,6), (1,1,10), (1,2,9), (1,3,8), (1,4,7), (1,5,6), (2,2,8), (2,3,7), (2,4,6), (2,5,5), (3,4,5), (4,4,4)。只要小孩摸出的不是 (3,4,5),就白送一根糖葫芦;否则需要按日常价 10 元购买一根糖葫芦。问:为了在大酬宾活动中不亏本,平均一根糖葫芦的成本价不能高于多少元?(精确到0.01元) \mark{10}}
\item[(10)]{已知 $x$ 为正实数,且 $\sum_{n=0}^\infty J_{2n}(x) = 0.98$,估算 $x$ 的值(至少精确到2位有效数)。 \mark{20}}
\item[(11)]{传古代印度与中国之间有一萨桑国,国王山鲁亚尔生性残暴嫉妒,因王后行为不端,将其杀死,此后每日娶一少女,翌日晨即杀掉,以示报复。宰相的女儿山鲁佐德为拯救无辜的女子,自愿嫁给国王,用让国王刷quiz的方法吸引国王,每夜刷到最精彩处,天刚好亮了,使国王爱不忍杀。quiz一直刷了一千零一夜,国王终于被感动,与她白首偕老。 \\
山鲁佐德给国王传授了自然对数的算法。国王深深迷醉于这个超越时代近千年的技术。每天刷quiz前,国王都要练习自然对数的计算。\\
为了让国王每天有新的挑战,山鲁佐德使用了满足下列递推式的数列:
$$a_n = \frac{4}{3+a_{n-1}^2}.$$
数列的起始点设为$a_0=0$,也就是第一天给国王的数字是$a_1=\frac{4}{3}$,之后依次递推。\\
国王在第$n$天($n=1,2,...,1001$)刷quiz前的热身运动就是计算$a_n$的对数的绝对值的对数,也就是$\ln|\ln a_n|$。在当时可没有计算机,全靠手算,山鲁佐德只要求国王算到一位有效数就可以了。\\
那么问题来了:在最后第一千零一夜,国王算出来的$\ln|\ln a_{1001}|$等于多少? \mark{20} }
\item[(12)]{计算 $J_2(x)+ xJ_2'(x)=0$ 的所有正实数解的四次方的倒数之和。 \mark{20} }
\item[(13)]{有一个半径为 $R=0.1\mathrm{m}$,材质的热传导方程参数为 $a=10^{-4}\mathrm{m^2s^{-1}}$ 的孤立、均匀薄圆盘。初始时刻上半个圆盘 (极坐标 $0\le \theta < \pi$) 的温度为 $400\mathrm{K}$,下半个圆盘 (极坐标 $\pi \le\theta < 2\pi$)的温度为 $300\mathrm{K}$。经过一分钟之后,上半圆盘的最高边缘,即 $r=R, \theta=\frac{\pi}{2}$ 处,温度是多少?(请计算至少精确到$1\mathrm{K}$) \mark{70}}  
  \eitem

  
  
\ech
\end{document}
