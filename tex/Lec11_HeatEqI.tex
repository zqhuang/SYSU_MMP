\documentclass[CJK]{beamer}
\usepackage{CJKutf8}
\usepackage{beamerthemesplit}
\usetheme{Malmoe}
\useoutertheme[footline=authortitle]{miniframes}
\usepackage{amsmath}
\usepackage{amssymb}
\usepackage{graphicx}
\usepackage{eufrak}
\usepackage{color}
\usepackage{slashed}
\usepackage{simplewick}
\usepackage{tikz}
\usepackage{tcolorbox}
\graphicspath{{../figures/}}
%%figures
\def\lfig#1#2{\includegraphics[width=#1 in]{#2}}
\def\addfig#1#2{\begin{center}\includegraphics[width=#1 in]{#2}\end{center}}
\def\wulian{\includegraphics[width=0.18in]{emoji_wulian.jpg}}
\def\bigwulian{\includegraphics[width=0.35in]{emoji_wulian.jpg}}
\def\bye{\includegraphics[width=0.18in]{emoji_bye.jpg}}
\def\bigbye{\includegraphics[width=0.35in]{emoji_bye.jpg}}
\def\huaixiao{\includegraphics[width=0.18in]{emoji_huaixiao.jpg}}
\def\bighuaixiao{\includegraphics[width=0.35in]{emoji_huaixiao.jpg}}
\def\jianxiao{\includegraphics[width=0.18in]{emoji_jianxiao.jpg}}
\def\bigjianxiao{\includegraphics[width=0.35in]{emoji_jianxiao.jpg}}
%% colors
\def\blacktext#1{{\color{black}#1}}
\def\bluetext#1{{\color{blue}#1}}
\def\redtext#1{{\color{red}#1}}
\def\darkbluetext#1{{\color[rgb]{0,0.2,0.6}#1}}
\def\skybluetext#1{{\color[rgb]{0.2,0.7,1.}#1}}
\def\cyantext#1{{\color[rgb]{0.,0.5,0.5}#1}}
\def\greentext#1{{\color[rgb]{0,0.7,0.1}#1}}
\def\darkgray{\color[rgb]{0.2,0.2,0.2}}
\def\lightgray{\color[rgb]{0.6,0.6,0.6}}
\def\gray{\color[rgb]{0.4,0.4,0.4}}
\def\blue{\color{blue}}
\def\red{\color{red}}
\def\green{\color{green}}
\def\darkgreen{\color[rgb]{0,0.4,0.1}}
\def\darkblue{\color[rgb]{0,0.2,0.6}}
\def\skyblue{\color[rgb]{0.2,0.7,1.}}
%%control
\def\be{\begin{equation}}
\def\ee{\nonumber\end{equation}}
\def\bea{\begin{eqnarray}}
\def\eea{\nonumber\end{eqnarray}}
\def\bch{\begin{CJK}{UTF8}{gbsn}}
\def\ech{\end{CJK}}
\def\bitem{\begin{itemize}}
\def\eitem{\end{itemize}}
\def\bcenter{\begin{center}}
\def\ecenter{\end{center}}
\def\bex{\begin{minipage}{0.2\textwidth}\includegraphics[width=0.6in]{jugelizi.png}\end{minipage}\begin{minipage}{0.76\textwidth}}
\def\eex{\end{minipage}}
\def\chtitle#1{\frametitle{\bch#1\ech}}
\def\bmat#1{\left(\begin{array}{#1}}
\def\emat{\end{array}\right)}
\def\bcase#1{\left\{\begin{array}{#1}}
\def\ecase{\end{array}\right.}
\def\bmini#1{\begin{minipage}{#1\textwidth}}
\def\emini{\end{minipage}}
\def\tbox#1{\begin{tcolorbox}#1\end{tcolorbox}}
\def\pfrac#1#2#3{\left(\frac{\partial #1}{\partial #2}\right)_{#3}}
%%symbols
\def\bropt{\,(\ \ \ )}
\def\sone{$\star$}
\def\stwo{$\star\star$}
\def\sthree{$\star\star\star$}
\def\sfour{$\star\star\star\star$}
\def\sfive{$\star\star\star\star\star$}
\def\rint{{\int_\leftrightarrow}}
\def\roint{{\oint_\leftrightarrow}}
\def\stdHf{{\textit{\r H}_f}}
\def\deltaH{{\Delta \textit{\r H}}}
\def\ii{{\dot{\imath}}}
\def\skipline{{\vskip0.1in}}
\def\skiplines{{\vskip0.2in}}
\def\lagr{{\mathcal{L}}}
\def\hamil{{\mathcal{H}}}
\def\vecv{{\mathbf{v}}}
\def\vecx{{\mathbf{x}}}
\def\vecy{{\mathbf{y}}}
\def\veck{{\mathbf{k}}}
\def\vecp{{\mathbf{p}}}
\def\vecn{{\mathbf{n}}}
\def\vecA{{\mathbf{A}}}
\def\vecP{{\mathbf{P}}}
\def\vecsigma{{\mathbf{\sigma}}}
\def\hatJn{{\hat{J_\vecn}}}
\def\hatJx{{\hat{J_x}}}
\def\hatJy{{\hat{J_y}}}
\def\hatJz{{\hat{J_z}}}
\def\hatj#1{\hat{J_{#1}}}
\def\hatphi{{\hat{\phi}}}
\def\hatq{{\hat{q}}}
\def\hatpi{{\hat{\pi}}}
\def\vel{\upsilon}
\def\Dint{{\mathcal{D}}}
\def\adag{{\hat{a}^\dagger}}
\def\bdag{{\hat{b}^\dagger}}
\def\cdag{{\hat{c}^\dagger}}
\def\ddag{{\hat{d}^\dagger}}
\def\hata{{\hat{a}}}
\def\hatb{{\hat{b}}}
\def\hatc{{\hat{c}}}
\def\hatd{{\hat{d}}}
\def\hatN{{\hat{N}}}
\def\hatH{{\hat{H}}}
\def\hatp{{\hat{p}}}
\def\Fup{{F^{\mu\nu}}}
\def\Fdown{{F_{\mu\nu}}}
\def\newl{\nonumber \\}
\def\vece{\mathrm{e}}
\def\calM{{\mathcal{M}}}
\def\calT{{\mathcal{T}}}
\def\calR{{\mathcal{R}}}
\def\barpsi{\bar{\psi}}
\def\baru{\bar{u}}
\def\barv{\bar{\upsilon}}
\def\qeq{\stackrel{?}{=}}
\def\torder#1{\mathcal{T}\left(#1\right)}
\def\rorder#1{\mathcal{R}\left(#1\right)}
\def\contr#1#2{\contraction{}{#1}{}{#2}#1#2}
\def\trof#1{\mathrm{Tr}\left(#1\right)}
\def\trace{\mathrm{Tr}}
\def\comm#1{\ \ \ \left(\mathrm{used}\ #1\right)}
\def\tcomm#1{\ \ \ (\text{#1})}
\def\slp{\slashed{p}}
\def\slk{\slashed{k}}
\def\calp{{\mathfrak{p}}}
\def\veccalp{\mathbf{\mathfrak{p}}}
\def\Tthree{T_{\tiny \textcircled{3}}}
\def\pthree{p_{\tiny \textcircled{3}}}
\def\dbar{{\,\mathchar'26\mkern-12mu d}}
\def\erf{\mathrm{erf}}
\def\const{\mathrm{constant}}
\def\pheat{\pfrac p{\ln T}V}
\def\vheat{\pfrac V{\ln T}p}
%%units
\def\fdeg{{^\circ \mathrm{F}}}
\def\cdeg{^\circ \mathrm{C}}
\def\atm{\,\mathrm{atm}}
\def\angstrom{\,\text{\AA}}
\def\SIL{\,\mathrm{L}}
\def\SIkm{\,\mathrm{km}}
\def\SIyr{\,\mathrm{yr}}
\def\SIGyr{\,\mathrm{Gyr}}
\def\SIV{\,\mathrm{V}}
\def\SImV{\,\mathrm{mV}}
\def\SIeV{\,\mathrm{eV}}
\def\SIkeV{\,\mathrm{keV}}
\def\SIMeV{\,\mathrm{MeV}}
\def\SIGeV{\,\mathrm{GeV}}
\def\SIcal{\,\mathrm{cal}}
\def\SIkcal{\,\mathrm{kcal}}
\def\SImol{\,\mathrm{mol}}
\def\SIN{\,\mathrm{N}}
\def\SIHz{\,\mathrm{Hz}}
\def\SIm{\,\mathrm{m}}
\def\SIcm{\,\mathrm{cm}}
\def\SIfm{\,\mathrm{fm}}
\def\SImm{\,\mathrm{mm}}
\def\SInm{\,\mathrm{nm}}
\def\SImum{\,\mathrm{\mu m}}
\def\SIJ{\,\mathrm{J}}
\def\SIW{\,\mathrm{W}}
\def\SIkJ{\,\mathrm{kJ}}
\def\SIs{\,\mathrm{s}}
\def\SIkg{\,\mathrm{kg}}
\def\SIg{\,\mathrm{g}}
\def\SIK{\,\mathrm{K}}
\def\SImmHg{\,\mathrm{mmHg}}
\def\SIPa{\,\mathrm{Pa}}

\def\courseurl{https://github.com/zqhuang/SYSU\_TD}

\def\tpage#1#2{
\begin{frame}
\begin{center}
\begin{Large}
\bch
热学 \\
第#1讲 #2

{\vskip 0.3in}

黄志琦

\ech
\end{Large}
\end{center}

\vskip 0.2in

\bch
教材:《热学》第二版,赵凯华,罗蔚茵,高等教育出版社
\ech

\bch
课件下载
\ech
\courseurl
\end{frame}
}

\def\bfr#1{
\begin{frame}
\chtitle{#1} 
\bch
}

\def\efr{
\ech 
\end{frame}
}

  \date{}
  \begin{document}
  \bch
\tpage{11}{Heat Equation (I)}

\begin{frame}
  \frametitle{本讲内容}
  
  \bitem
\item{一维热传导问题}  
\item{热传导方程的直观理解}
\item{分离变量法求解热传导方程}
  \eitem
  
\end{frame}

\section{Introduction}

\begin{frame}
  \frametitle{一维热传导方程}
  
  考虑一根很长的不良导体棒(这里指热的传导)的温度$T(x, t)$,其中$x$是在导体棒上的位置,$t$是时间。
  在温度梯度不大的情况下,可以用线性近似:{\blue 热流正比于温度梯度}。
  显然,这个比例系数是负的,
  $$ j =  -\lambda \frac{\partial T}{\partial x} $$
  其中$\lambda >0$为常量,称为{\blue 导热系数}。

  \addfig{2.2}{heatflux.png}

  {\small 注:热流是指单位时间通过单位面积的热量。}
\end{frame}


\begin{frame}
  \frametitle{一维热传导方程}
  

  \addfig{2.6}{heateq.png}
  考虑长为$dx$的一小段不良导体棒,进入和出去的热流差为
  $$j(x)-j(x+dx) = -\lambda \left[\left.\frac{\partial T}{\partial x}\right\vert_{x}-\left.\frac{\partial T}{\partial x}\right\vert_{x+dx}\right]\approx \lambda \frac{\partial^2 T}{\partial x^2} dx. $$
\end{frame}
\begin{frame}
  设材料单位质量的比热为$c$,质量密度为$\rho$,横截面积为$S$,则
  $$ (c \rho S dx) dT = \dbar Q = [j(x)-j(x+dx)] S dt =  \lambda \frac{\partial^2 T}{\partial x^2} dx  (S dt) $$
  这里的$dT$和$dt$都是对固定$x$而言,两者之比为$\frac{\partial T}{\partial t}$。令$a = \frac{\lambda}{\rho c}$,则
  \tbox{
  $$ \frac{\partial T}{\partial t} - a \frac{\partial^2T}{\partial x^2} = 0$$
  }
  我们有时把{\blue $a=\frac{\lambda}{\rho c}$}叫做{\blue 热传导方程的参数}。
  
\end{frame}


\begin{frame}
  \frametitle{例题}
  
  \addfig{2.5}{heatflux2.png}
  
  在一根长为$2L$的不良导体棒在$t=0$时刻温度为$T_0$。在$t>0$时刻,不良导体棒两端均有强度为$j$的热流进入。设材料的导热系数$\lambda$,质量密度$\rho$,单位质量的比热$c$均已知,试计算$t\ge 0$时刻不良导体棒各处的温度$T(x,t)$。
  
\end{frame}

\begin{frame}
  \frametitle{ 解答}
  
  根据对称性,在棒中间处热流和温度梯度均为零。写出如下的方程和边界条件:
  \bea
  \frac{\partial T}{\partial t} - a\frac{\partial^2 T}{\partial x^2} &=& 0 \newl
  \left.\frac{\partial T}{\partial x}\right\vert_{x=0} &=& 0 \newl
  \left.\frac{\partial T}{\partial x}\right\vert_{x=L} &=& \frac{j}{\lambda}  \newl
  \left.T\right\vert_{t=0} &=&  T_0 
  \eea
  其中$a = \frac{\lambda}{\rho c} $。
  
  
\end{frame}


\begin{frame}
  \frametitle{ 解答}
  
  先分析主要图像。

  \skiplines

  在$t$时刻,累计流入的热量为$Q =  2 j St$ (其中$S$为横截面积)。棒子热容为$C =  c \rho (2SL)$。所以$t$时刻棒子的平均温度为
  $$ \bar{T} =   T_0  + \frac{Q}{C} = T_0 + \frac{j}{\rho cL}t $$
  
  
\end{frame}


\begin{frame}
  \frametitle{ 解答 (续)}
  
  把平均温度去掉,研究各处温度起伏:$\Delta T(x, t) = T(x, t) - \left(T_0+\frac{j}{\rho cL} t\right)$。显然$\Delta T$满足方程:
  \bea
  \frac{\partial \Delta T }{\partial t} - a \frac{\partial^2 \Delta T}{\partial x^2} &=&  -\frac{j}{\rho cL} \newl
  \left.\frac{\partial \Delta T}{\partial x}\right\vert_{x=0} &=& 0 \newl
  \left.\frac{\partial \Delta  T}{\partial x}\right\vert_{x=L} &=& \frac{j}{\lambda}  \newl
  \left.\Delta T\right\vert_{t=0} = 0
  \eea
  因为$\Delta T$描述的是温度起伏,还有一个额外条件:
  $$\int_0^L \Delta T(x, t) dx = 0 $$
  
\end{frame}

\begin{frame}
  \frametitle{解答 (续)}
  
  当$t$很大时,棒上的温度梯度趋于稳定,即$\Delta T$仅仅依赖于$x$,满足
  \bea
  - a \frac{\partial^2 \Delta T}{\partial x^2} &=&  -\frac{j}{\rho cL} \newl
  \left.\frac{\partial \Delta T}{\partial x}\right\vert_{x=0} &=& 0 \newl
  \left.\frac{\partial \Delta  T}{\partial x}\right\vert_{x=L} &=& \frac{j}{\lambda}  \newl
  \int_0^L \Delta T(x, t) dx &=& 0  
  \eea
  由此不难解出
  $$\Delta T = \frac{j}{2\lambda} \left(\frac{x^2}{L} - \frac{L}{3}\right) $$
  
\end{frame}

\begin{frame}
  \frametitle{ 解答 (续)}
  
  也就是说,当$t${\bf 很大}时
  $$ T = \left(T_0+\frac{j}{\rho cL} t\right) + \frac{j}{2\lambda} \left(\frac{x^2}{L} - \frac{L}{3}\right)  $$
  现在还留下两个问题:
  \bitem
\item{$t$多大时算“很大”?}
\item{$t$不大(即刚开始加热)时,$T(x,t)$的严格解是什么?}
  \eitem
 第一个问题很容易估算:根据热传导方程,$T$的典型“变化时间”$\Delta t$和“变化尺度”$L$之间满足$ \frac{1}{\Delta t} \sim a\frac{1}{L^2} $。我们所说的$t$很大,即指$t\gg \Delta t \sim \frac{L^2}{a}$。

  第二个问题涉及一些数学技巧(“二流物理学家的研究对象”\huaixiao),我们下面再来详细讨论。
  
\end{frame}



\section{Understanding Heat Eq.}
\secpage{热传导方程的直观理解}{散度是单位体积流出率}


\begin{frame}
  \frametitle{流密度}
  
  流密度$ \mathbf{j} = (j_x, j_y, j_z) $: 单位时间单位面积流过的某种物理量。
  
  \bex
  热流密度 (或简称热流)

  \eex  

  
\end{frame}


\begin{frame}
  \frametitle{散度等于单位体积流出率}
  {\small
  考虑以$(x, y, z)$和$(x+dx, y+dy, z+dz)$连线为对角线的小长方体。

  \addfig{1}{box.png}
  
  如果只有$x$方向有热流,则如图示,在坐标为$x$的横截面上,单位时间进入的热量为
  $$j_x(x, y, z) dy dz$$
  而在对面坐标为$x+dx$的横截面上,单位时间流出的热量为
  $$ j_x(x+dx,y,z)dydz$$
  单位时间内小长方体内积累的热量为
  $$ \frac{\partial Q}{\partial t} = (-j_x(x+dx, y, z) + j_x(y, z))dy dz \approx  - \frac{\partial j_x}{\partial x} dxdydz $$
  
  }
\end{frame}




\begin{frame}
  \frametitle{散度等于单位体积流出率 (续)}
  
{\small
  考虑三个方向都有这种不平衡的热流,就有

  $$ \frac{\partial Q}{\partial t} = - \left(\frac{\partial j_x}{\partial x} + \frac{\partial j_y}{\partial y} + \frac{\partial j_z}{\partial z}\right)  dxdydz = -\nabla\cdot \mathbf{j} \, dx dy dz $$
  把热量写成热量密度乘以体积,$Q = \rho dxdydz$ ,上式可以写成
  \tbox{
  $$\frac{\partial \rho}{\partial t} + \nabla\cdot \mathbf{j} = 0$$ }
}

这就是守恒流方程。

也就是说:{\blue \bf 对守恒量X,X流的散度可以看成X的单位体积流出率。} 这里的X可以是电荷,热量,质量等。

  
\end{frame}


\begin{frame}
  \frametitle{据说你们理论力学学过守恒流}
  
  \addfig{2.2}{Noether.jpg}
  
\end{frame}

\begin{frame}
  \frametitle{热传导方程的直观理解}
  
  {\small
  根据导热系数的定义
  $$j = -\lambda\nabla T$$
  两边取散度得到
  $$\nabla \cdot j = -\lambda \nabla^2T$$
  根据散度等于流出率
  $$\nabla\cdot j = -\frac{\partial (c\rho T)}{\partial t}$$
  (注意我们直接把热量密度写成了$c\rho T$)

    结合上面两个式子得到\tbox{\blue
    $$ \frac{\partial  T}{\partial t} - a\nabla^2T = 0,$$}
    其中$a = \frac{\lambda}{\rho c}$。这就是我们之前在一维情况下推导过的热传导方程。
  }  
  
\end{frame}

\section{SV method}
\secpage{分离变量法求解热传导方程}{$$f(x,t) =\sum_{i,j} \phi_i(x)\psi_j(t)$$}


\begin{frame}
  \frametitle{猜一猜}
  

  \addfig{0.6}{think2.jpg}
  
  先不管边界条件,对热传导方程
  $$ \frac{\partial T}{\partial t} - a\frac{\partial^2T}{\partial x^2} = 0,$$
  你能猜出一个非常数解吗?
  
  
\end{frame}


\begin{frame}
  \frametitle{分离变量法的大致想法}
  
\tbox{ \Large 寻找形如$\phi(x)\psi(t)$的解。}
\tbox{\Large 把解拆分成一堆$\phi(x)\psi(t)$的和。}
  
\end{frame}


\begin{frame}
  \frametitle{寻找形如$\phi(x)\psi(t)$的解}
  
  对热传导方程
  $$ \frac{\partial T}{\partial t} - a\frac{\partial^2T}{\partial x^2} = 0,$$
  设某个解为$\phi(x)\psi(t)$,代入上式,得到
  $$\frac{\psi'}{\psi} = a\frac{\phi''}{\phi} $$
  因为等式左边只是$t$的函数,右边只是$x$的函数,两者要恒等就必须是常数。记这个常数为$\lambda $,则有
  $$ \psi = e^{\lambda t},\ \phi = e^{ \sqrt{\frac{\lambda}{a}}x}.$$
  也就是我们找到了一种满足热传导方程的解
 $$e^{\lambda t + \sqrt{\frac{\lambda}{a}}x}$$
  
\end{frame}


\begin{frame}
  \frametitle{寻找形如$\phi(x)\psi(t)$的解}
  
  $$e^{\lambda t + \sqrt{\frac{\lambda}{a}}x}$$
  根据$\lambda$的正负,有
  \bitem
  \item{
  指数增长模式:
  $$ e^{ak^2t}e^{kx} $$}
\item{指数衰减模式:
  $$ e^{-ak^2t}e^{\ii kx} .$$ }
  \eitem
  {\small 如果需要实数解,可以把衰减模式对应$k$和$-k$的解进行线性组合,写成$e^{-ak^2t}\cos{(kx)}$和$e^{-ak^2t}\sin{(kx)}$。}
  
\end{frame}

\begin{frame}
  \frametitle{物理分析}
  
  物理上我们知道温度会自发地趋向于均匀。

  \skiplines

  \bitem

  \item{\blue 衰减模式$e^{-ak^2t}e^{\ii kx}$可以描述热自发地趋向于均匀这一过程,对大多数有限边界条件都适用。}

  \item{增长模式$e^{ak^2t}e^{kx}$需要外界强加一个指数增长的热流输入,自然中很难有这样的边界条件,即使在实验室要得到一个指数增长的热流也非易事。}


    \eitem

  
  
\end{frame}



\begin{frame}
  \frametitle{衰减模式的典型衰减时间}
  
      {\blue  当$t\gg \frac{1}{ak^2}$时,显然衰减模式$e^{-ak^2t}e^{ikx}$会变得非常小(跟初始条件比)。}


      \skiplines
      
  之前我们考虑过的不良导体棒的问题,因棒长有限,空间频率只可能取到$k\gtrsim \frac{1}{L}$。所以一开始对稳恒态的偏差(可以看作很多不同$k$的衰减模式的线性叠加)需要的时间$\frac{1}{ak^2}\lesssim\frac{L^2}{a}$进行衰减。
  
\end{frame}

\begin{frame}
  \frametitle{热求导方程求解举例}
  
  \addfig{3}{HeatEq1.png}
  以我们讲过的如下边界条件为例:
  \bea
  \frac{\partial T}{\partial t} - a\frac{\partial^2 T}{\partial x^2} &=& 0 \newl
  \left.\frac{\partial T}{\partial x}\right\vert_{x=0} &=& 0 \newl
  \left.\frac{\partial T}{\partial x}\right\vert_{x=L} &=& \frac{j}{\lambda}  \newl
  \left.T\right\vert_{t=0} &=&  T_0 
  \eea
  
  
\end{frame}


\begin{frame}
  \frametitle{不想蛮干}
  
  我们并不准备蛮干,先分析下解的渐近行为总不会吃亏。

  \skipline
  
  题目所给的边界条件比较简单,所以猜测解有稳恒的渐近行为。按照第9讲的分析方法可以得到:当$t\gg L^2/a$时,
    $$ T \rightarrow \left(T_0+\frac{j}{\rho cL} t\right) + \frac{j}{2\lambda} \left(\frac{x^2}{L} - \frac{L}{3}\right)  $$
  
\end{frame}


\begin{frame}
  \frametitle{不想蛮干}
  
  那么,令
  $$ T = \left(T_0+\frac{j}{\rho cL} t\right) + \frac{j}{2\lambda} \left(\frac{x^2}{L} - \frac{L}{3}\right) +\delta T(x, t), $$

  这里的$\delta T(x,t)$描述了解对稳恒态的偏差如何衰减。把$T(x,t)$直接代入初始的方程和边界条件,易见$\delta T$也满足热传导方程,并满足如下的边界条件:
  \bea
  \left.\frac{\partial \delta T}{\partial x}\right\vert_{x=0} &=& 0 \newl
  \left.\frac{\partial \delta T}{\partial x}\right\vert_{x=L} &=& 0 \newl
  \left.\delta T\right\vert_{t=0} &=&  -\frac{j}{2\lambda} \left(\frac{x^2}{L} - \frac{L}{3}\right) 
  \eea
  
\end{frame}


\begin{frame}
  \frametitle{$0+0=0$}
  
  通过上述操作,我们得到了每一项都是$T$或$T$的导数的一次幂的“齐次边界条件”。
  \bea
  \left.\frac{\partial \delta T}{\partial x}\right\vert_{x=0} &=& 0 \newl
  \left.\frac{\partial \delta T}{\partial x}\right\vert_{x=L} &=& 0 \newl
  \eea
  容易看出,所有形如$e^{-a\frac{n^2\pi^2}{L^2}t}\cos{\left(\frac{n\pi }{L}x\right)}$ ($n=0,1,2,\ldots$)的衰减模式的{\bf 线性叠加}都符合要求。因此我们不妨假设
  $$\delta T(x, t) = \sum_{n=0}^\infty c_n e^{-a\frac{n^2\pi^2}{L^2}t}\cos{\left(\frac{n\pi }{L}x\right)}.$$
  
\end{frame}


\begin{frame}
  \frametitle{还得回顾下高数知识}
  
  最后,利用$t=0$时的初始条件,有
  $$ -\frac{j}{2\lambda} \left(\frac{x^2}{L} - \frac{L}{3}\right) =\sum_{n=0}^\infty c_n\cos{\left(\frac{n\pi}{L}x\right)} .$$
  这是个标准的正交级数展开问题。两边乘以$\cos{\left(\frac{n\pi}{L}x\right)}$ 并对$x$从 $0$ 到 $L$进行积分,利用余弦函数的正交性得到:
  \bea
  c_0&=& 0,\newl
  c_n &=& (-1)^{n+1}\frac{2jL}{\lambda n^2\pi^2}
  \eea
  
\end{frame}


\begin{frame}
  \frametitle{大功告成}
  
  最后,完整的解为:
  {\small
  $$ T = \left(T_0+\frac{j}{\rho cL} t\right) + \frac{j}{2\lambda} \left(\frac{x^2}{L} - \frac{L}{3}\right) - \frac{2jL}{\lambda \pi^2}\sum_{n=1}^\infty \frac{(-1)^n}{n^2}e^{-a\frac{n^2\pi^2}{L^2}t}\cos{\left(\frac{n\pi }{L}x\right)}$$}

  这个解的第一个括号内是平均温度的变化,第二部分描述稳恒态的形状,第三部分描述初始时对稳恒态的偏离是如何衰减掉的。

  \skiplines
  
  现在我们可以清楚地看到,当$t\gg L^2/a$时,所有衰减项确实都被指数式压缩(终于放心了。)
  
  
\end{frame}


\begin{frame}
  \frametitle{总结}
  
  上述问题的求解过程包含了三步:

  \bitem
\item[(1)]{分离变量找到所有形如$\phi(x)\psi(t)$的解。}  
\item[(2)]{通过物理分析求出解的稳恒渐近行为。}
\item[(3)]{令解 = 稳恒渐近项 + 衰减项。根据衰减项满足的齐次边界条件进行级数展开。}
  \eitem
  
  
\end{frame}

\begin{frame}
  \frametitle{Practice I的那道送分题}
  
  \addfig{3}{HeatEq2.png}
  
  长度为$L$的不良导体棒一端和温度为$T_0$的热库接触,并在$t=0$时刻和热库处于热平衡。从$t=0$时刻开始,在不良导体棒的另一端注入恒定大小为$j$的热流。设不良导体棒的导热系数$\lambda$,单位质量的比热$c$和质量密度$\rho$均已知。求解不良导体棒上温度$T(x, t)$ ($0\le x\le L, t\ge 0$)。
  
\end{frame}

\begin{frame}
  \frametitle{边界条件}
  
    方程为
    \be
    \frac{\partial T}{\partial t} -a\frac{\partial^2T}{\partial x^2} = 0 \, .
    \ee
    其中$a= \frac{\lambda}{\rho c}$。

    边界条件为
    \bea
    \left.T\right\vert_{t=0} &=& T_0 \newl
    \left.T\right\vert_{x=0} &=& T_0 \newl
    \left.\frac{\partial T}{\partial x}\right\vert_{x=L} &=& \frac{j}{\lambda}
    \eea

    
\end{frame}

\begin{frame}
  \frametitle{渐近行为分析}
  
  假设当$t$远大于典型的变化时间$L^2/a$时,系统处于稳恒状态(温度梯度不再变化)。因为一端温度是固定的,要得到稳恒状态的必要条件是热量不在不良导体棒上积累,也就是说进来的热流$j$必须保持不变地通过整个不良导体棒,最后从另一端进入热库。这说明稳恒状态下$\frac{\partial T}{\partial x}$处处等于$\frac{j}{\lambda}$。由此得出:
    $$T(x, t) \rightarrow T_0+\frac{j}{\lambda} x$$
    
\end{frame}

\begin{frame}
  \frametitle{分离变量法}
  
  虽然很快如愿得到了解的渐近行为,但刚学习了分离变量法,必须求出严格解炫耀一下。令
  $$T(x,t) = T_0+\frac{j}{\lambda}x + \delta T(x,t)$$
  易见$\delta T$也满足热传导方程,且
  \bea
  \left.\delta T\right\vert_{x=0} &=& 0, \newl
  \left.\frac{\partial \delta T}{\partial x}\right\vert_{x=L} &=& 0, \newl
  \left.\delta T\right\vert_{t=0} &=& -\frac{j}{\lambda}x.
  \eea
  
\end{frame}


\begin{frame}
  \frametitle{级数展开}
  
  熟练地利用零边界条件进行衰减模式的分解:
  $$ \delta T = \sum_{n=0}^\infty c_ne^{-a\frac{(n+\frac{1}{2})^2\pi^2}{L^2}t}\sin{\left(\frac{(n+\frac{1}{2})\pi}{L}x\right)}. $$
  利用$t=0$时的初始条件
  $$ -\frac{j}{\lambda}x = \sum_{n=0}^\infty c_n\sin{\left(\frac{(n+\frac{1}{2})\pi}{L}x\right)} .$$
  最后,上式两边乘以$\sin{\left(\frac{(n+\frac{1}{2})\pi}{L}x\right)}$并两边对$x$从$0$积分到$L$,得到:
  $$c_n = (-1)^{n+1}\frac{2jL}{\lambda \left(n+\frac{1}{2}\right)^2\pi^2}.$$
  
\end{frame}


\begin{frame}
  \frametitle{再次大功告成}
  
  最后的结果是
  $$T(x,t) = T_0+\frac{j}{\lambda}x -\frac{2jL}{\lambda \pi^2}\sum_{n=0}^\infty \frac{(-1)^n}{\left(n+\frac{1}{2}\right)^2}e^{-a\frac{(n+\frac{1}{2})^2\pi^2}{L^2}t}\sin{\left(\frac{(n+\frac{1}{2})\pi}{L}x\right)}.$$
  
\end{frame}

\section{Homework}

\begin{frame}
  \frametitle{Homework for quizphobia}
  
  
  \bitem
\item[31]{长度为$L$的不良导体棒一端 ($x=0$处) 和温度为$T_0$的热库接触,并在$t=0$时刻和热库处于热平衡。从$t=0$时刻开始,把不良导体棒的另一端 ($x=L$处) 和温度为$T_1$的热库保持接触。设不良导体棒的导热系数$\lambda$,单位质量的比热$c$和质量密度$\rho$均已知。求解不良导体棒上温度$T(x, t)$ ($0\le x\le L, t\ge 0$)。

  \addfig{3}{HeatEq3.png}
}
    \eitem
%{\scriptsize 参考答案   $$T(x,t) = T_0+\frac{T_1-T_0}{L}x + \frac{T_1-T_0}{\pi}\sum_{n=1}^\infty \frac{(-1)^n}{n}e^{-a\frac{n^2\pi^2}{L^2}t}\sin{\left(\frac{n\pi}{L}x\right)},$$}

  
\end{frame}

\ech
\end{document}
