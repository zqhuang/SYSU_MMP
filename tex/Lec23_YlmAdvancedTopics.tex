\documentclass[CJK]{beamer}
\usepackage{CJKutf8}
\usepackage{beamerthemesplit}
\usetheme{Malmoe}
\useoutertheme[footline=authortitle]{miniframes}
\usepackage{amsmath}
\usepackage{amssymb}
\usepackage{graphicx}
\usepackage{eufrak}
\usepackage{color}
\usepackage{slashed}
\usepackage{simplewick}
\usepackage{tikz}
\usepackage{tcolorbox}
\graphicspath{{../figures/}}
%%figures
\def\lfig#1#2{\includegraphics[width=#1 in]{#2}}
\def\addfig#1#2{\begin{center}\includegraphics[width=#1 in]{#2}\end{center}}
\def\wulian{\includegraphics[width=0.18in]{emoji_wulian.jpg}}
\def\bigwulian{\includegraphics[width=0.35in]{emoji_wulian.jpg}}
\def\bye{\includegraphics[width=0.18in]{emoji_bye.jpg}}
\def\bigbye{\includegraphics[width=0.35in]{emoji_bye.jpg}}
\def\huaixiao{\includegraphics[width=0.18in]{emoji_huaixiao.jpg}}
\def\bighuaixiao{\includegraphics[width=0.35in]{emoji_huaixiao.jpg}}
\def\jianxiao{\includegraphics[width=0.18in]{emoji_jianxiao.jpg}}
\def\bigjianxiao{\includegraphics[width=0.35in]{emoji_jianxiao.jpg}}
%% colors
\def\blacktext#1{{\color{black}#1}}
\def\bluetext#1{{\color{blue}#1}}
\def\redtext#1{{\color{red}#1}}
\def\darkbluetext#1{{\color[rgb]{0,0.2,0.6}#1}}
\def\skybluetext#1{{\color[rgb]{0.2,0.7,1.}#1}}
\def\cyantext#1{{\color[rgb]{0.,0.5,0.5}#1}}
\def\greentext#1{{\color[rgb]{0,0.7,0.1}#1}}
\def\darkgray{\color[rgb]{0.2,0.2,0.2}}
\def\lightgray{\color[rgb]{0.6,0.6,0.6}}
\def\gray{\color[rgb]{0.4,0.4,0.4}}
\def\blue{\color{blue}}
\def\red{\color{red}}
\def\green{\color{green}}
\def\darkgreen{\color[rgb]{0,0.4,0.1}}
\def\darkblue{\color[rgb]{0,0.2,0.6}}
\def\skyblue{\color[rgb]{0.2,0.7,1.}}
%%control
\def\be{\begin{equation}}
\def\ee{\nonumber\end{equation}}
\def\bea{\begin{eqnarray}}
\def\eea{\nonumber\end{eqnarray}}
\def\bch{\begin{CJK}{UTF8}{gbsn}}
\def\ech{\end{CJK}}
\def\bitem{\begin{itemize}}
\def\eitem{\end{itemize}}
\def\bcenter{\begin{center}}
\def\ecenter{\end{center}}
\def\bex{\begin{minipage}{0.2\textwidth}\includegraphics[width=0.6in]{jugelizi.png}\end{minipage}\begin{minipage}{0.76\textwidth}}
\def\eex{\end{minipage}}
\def\chtitle#1{\frametitle{\bch#1\ech}}
\def\bmat#1{\left(\begin{array}{#1}}
\def\emat{\end{array}\right)}
\def\bcase#1{\left\{\begin{array}{#1}}
\def\ecase{\end{array}\right.}
\def\bmini#1{\begin{minipage}{#1\textwidth}}
\def\emini{\end{minipage}}
\def\tbox#1{\begin{tcolorbox}#1\end{tcolorbox}}
\def\pfrac#1#2#3{\left(\frac{\partial #1}{\partial #2}\right)_{#3}}
%%symbols
\def\bropt{\,(\ \ \ )}
\def\sone{$\star$}
\def\stwo{$\star\star$}
\def\sthree{$\star\star\star$}
\def\sfour{$\star\star\star\star$}
\def\sfive{$\star\star\star\star\star$}
\def\rint{{\int_\leftrightarrow}}
\def\roint{{\oint_\leftrightarrow}}
\def\stdHf{{\textit{\r H}_f}}
\def\deltaH{{\Delta \textit{\r H}}}
\def\ii{{\dot{\imath}}}
\def\skipline{{\vskip0.1in}}
\def\skiplines{{\vskip0.2in}}
\def\lagr{{\mathcal{L}}}
\def\hamil{{\mathcal{H}}}
\def\vecv{{\mathbf{v}}}
\def\vecx{{\mathbf{x}}}
\def\vecy{{\mathbf{y}}}
\def\veck{{\mathbf{k}}}
\def\vecp{{\mathbf{p}}}
\def\vecn{{\mathbf{n}}}
\def\vecA{{\mathbf{A}}}
\def\vecP{{\mathbf{P}}}
\def\vecsigma{{\mathbf{\sigma}}}
\def\hatJn{{\hat{J_\vecn}}}
\def\hatJx{{\hat{J_x}}}
\def\hatJy{{\hat{J_y}}}
\def\hatJz{{\hat{J_z}}}
\def\hatj#1{\hat{J_{#1}}}
\def\hatphi{{\hat{\phi}}}
\def\hatq{{\hat{q}}}
\def\hatpi{{\hat{\pi}}}
\def\vel{\upsilon}
\def\Dint{{\mathcal{D}}}
\def\adag{{\hat{a}^\dagger}}
\def\bdag{{\hat{b}^\dagger}}
\def\cdag{{\hat{c}^\dagger}}
\def\ddag{{\hat{d}^\dagger}}
\def\hata{{\hat{a}}}
\def\hatb{{\hat{b}}}
\def\hatc{{\hat{c}}}
\def\hatd{{\hat{d}}}
\def\hatN{{\hat{N}}}
\def\hatH{{\hat{H}}}
\def\hatp{{\hat{p}}}
\def\Fup{{F^{\mu\nu}}}
\def\Fdown{{F_{\mu\nu}}}
\def\newl{\nonumber \\}
\def\vece{\mathrm{e}}
\def\calM{{\mathcal{M}}}
\def\calT{{\mathcal{T}}}
\def\calR{{\mathcal{R}}}
\def\barpsi{\bar{\psi}}
\def\baru{\bar{u}}
\def\barv{\bar{\upsilon}}
\def\qeq{\stackrel{?}{=}}
\def\torder#1{\mathcal{T}\left(#1\right)}
\def\rorder#1{\mathcal{R}\left(#1\right)}
\def\contr#1#2{\contraction{}{#1}{}{#2}#1#2}
\def\trof#1{\mathrm{Tr}\left(#1\right)}
\def\trace{\mathrm{Tr}}
\def\comm#1{\ \ \ \left(\mathrm{used}\ #1\right)}
\def\tcomm#1{\ \ \ (\text{#1})}
\def\slp{\slashed{p}}
\def\slk{\slashed{k}}
\def\calp{{\mathfrak{p}}}
\def\veccalp{\mathbf{\mathfrak{p}}}
\def\Tthree{T_{\tiny \textcircled{3}}}
\def\pthree{p_{\tiny \textcircled{3}}}
\def\dbar{{\,\mathchar'26\mkern-12mu d}}
\def\erf{\mathrm{erf}}
\def\const{\mathrm{constant}}
\def\pheat{\pfrac p{\ln T}V}
\def\vheat{\pfrac V{\ln T}p}
%%units
\def\fdeg{{^\circ \mathrm{F}}}
\def\cdeg{^\circ \mathrm{C}}
\def\atm{\,\mathrm{atm}}
\def\angstrom{\,\text{\AA}}
\def\SIL{\,\mathrm{L}}
\def\SIkm{\,\mathrm{km}}
\def\SIyr{\,\mathrm{yr}}
\def\SIGyr{\,\mathrm{Gyr}}
\def\SIV{\,\mathrm{V}}
\def\SImV{\,\mathrm{mV}}
\def\SIeV{\,\mathrm{eV}}
\def\SIkeV{\,\mathrm{keV}}
\def\SIMeV{\,\mathrm{MeV}}
\def\SIGeV{\,\mathrm{GeV}}
\def\SIcal{\,\mathrm{cal}}
\def\SIkcal{\,\mathrm{kcal}}
\def\SImol{\,\mathrm{mol}}
\def\SIN{\,\mathrm{N}}
\def\SIHz{\,\mathrm{Hz}}
\def\SIm{\,\mathrm{m}}
\def\SIcm{\,\mathrm{cm}}
\def\SIfm{\,\mathrm{fm}}
\def\SImm{\,\mathrm{mm}}
\def\SInm{\,\mathrm{nm}}
\def\SImum{\,\mathrm{\mu m}}
\def\SIJ{\,\mathrm{J}}
\def\SIW{\,\mathrm{W}}
\def\SIkJ{\,\mathrm{kJ}}
\def\SIs{\,\mathrm{s}}
\def\SIkg{\,\mathrm{kg}}
\def\SIg{\,\mathrm{g}}
\def\SIK{\,\mathrm{K}}
\def\SImmHg{\,\mathrm{mmHg}}
\def\SIPa{\,\mathrm{Pa}}

\def\courseurl{https://github.com/zqhuang/SYSU\_TD}

\def\tpage#1#2{
\begin{frame}
\begin{center}
\begin{Large}
\bch
热学 \\
第#1讲 #2

{\vskip 0.3in}

黄志琦

\ech
\end{Large}
\end{center}

\vskip 0.2in

\bch
教材:《热学》第二版,赵凯华,罗蔚茵,高等教育出版社
\ech

\bch
课件下载
\ech
\courseurl
\end{frame}
}

\def\bfr#1{
\begin{frame}
\chtitle{#1} 
\bch
}

\def\efr{
\ech 
\end{frame}
}

  \date{}
\begin{document}
\tpage{23}{Advanced Topics on $Y_{\ell m}$'s}

\begin{frame}
\chtitle{本讲内容}
\bch
\bitem
\item{勒让德多项式的各种性质}  
\item{球谐函数的微分表达式和连带勒让德函数}
\item{球谐函数的加法公式}
\item{物理问题举例}  
\eitem
\ech
\end{frame}

\section{Legendre Polynomials}
\secpage{勒让德多项式的各种性质}{学完后很快会忘,所以有个印象就可以}

\begin{frame}
  \chtitle{例题1: 罗巨格公式}
  \bch
  
  上一讲我们定义了{\blue 勒让德多项式
    $$ P_{\ell}(x) := \sum_{k=0}^\ell \frac{(\ell + k)!}{(k!)^2(\ell-k)!}\left(\frac{x-1}{2}\right)^k. $$}


  
  试证明{\blue 罗巨格公式(Rodrigues' Formula):
  $$ P_\ell(x) = \frac{1}{2^\ell \ell!} \frac{d^\ell}{dx^\ell} \left[(x^2-1)^\ell\right]. $$}
  \ech
\end{frame}

\begin{frame}
  \chtitle{证明}
  \bch
  令$t = x-1$,证明直截了当:

  \bea
  \frac{1}{2^\ell \ell!} \frac{d^\ell}{dx^\ell} \left[(x^2-1)^\ell\right] &=&  \frac{1}{\ell!} \frac{d^\ell}{dt^\ell} \left[t^\ell(1+\frac{t}{2})^\ell\right]  \newl
  &=&  \sum_{k=0}^\ell \frac{1}{2^kk!(\ell-k)!} \frac{d^\ell}{dt^\ell} \left[t^{\ell+k}\right]  \newl
  &=&  \sum_{k=0}^\ell \frac{(\ell+k)!}{(k!)^2(\ell-k)!} \left(\frac{t}{2}\right)^k  \newl
  &=& P_\ell(x)
  \eea
  \ech
\end{frame}


\begin{frame}
  \chtitle{例题2: 其他递推公式}
  \bch
  上一讲我们借助母函数{\blue
  $$ \frac{1}{\sqrt{1-2xt +t^2}} = \sum_{\ell=0}^\infty P_\ell(x) t^\ell, $$}
  证明了递推公式:{\blue
  $$ (2\ell+1)xP_\ell(x) = (\ell+1)P_{\ell+1}(x) + \ell P_\ell(x) .$$
  }
  
  试证明其他两个递推公式:
  {\blue
    $$P_{\ell+1}'(x) = xP_\ell'(x) +(\ell+1)P_\ell(x);$$
    $$P_{\ell-1}'(x) = xP_\ell'(x) - \ell P_\ell(x) . $$
  }
  当然,由这两个递推公式还能得到:
  {\blue $$ (2\ell+1)P_\ell(x) = P_{\ell+1}'(x) - P_{\ell-1}'(x). $$}
  \ech
\end{frame}

\begin{frame}
  \chtitle{证明}
  \bch
  \eea
  \ech
\end{frame}



\begin{frame}
  \chtitle{题外话,高中知识:乘积多重导数公式}
  \bch
  设$f$,$g$为$x$的函数,我们都知道
  $$ (fg)' = f'g + fg'. $$
  这个公式可以推广到任意阶导数:\tbox{
  $$ (fg)^{(n)} = \sum_{k=0}^n \frac{n!}{k!(n-k)!} f^{(k)}g^{(n-k)}, $$}
  其中 $f^{(k)}$表示$f$的$k$重导数。

  \skipline
  {\small \darkgreen (和二项式定理一样,这个公式最简明有效的证明方法是用数学归纳法,请自行完成。)}
  \ech
\end{frame}


\begin{frame}
  \chtitle{题外话,高中知识:推广的乘积多重导数公式}
  \bch
  设$\rho$, $f$, $g$为$x$的函数,则
  \tbox{
    $$ \left(\rho\frac{d}{dx}\right)^n (fg) = \sum_{k=0}^n \frac{n!}{k!(n-k)!} \left[ \left(\rho\frac{d}{dx}\right)^k f \right]\left[  \left(\rho\frac{d}{dx}\right)^{n-k} g\right], $$}

  \skipline
  
  {\small \darkgreen 证明思路:令$y = \int \frac{dx}{\rho}$并对变量$y$应用乘积的多重导数公式。}

  
  \ech
\end{frame}



\begin{frame}
  \chtitle{罗巨格公式应用例题1}
  \bch
  \addfig{1}{think3.jpg}
  
  利用罗巨格公式证明$P_\ell(x)$满足
  $$  \frac{d}{dx}\left[(1-x^2)\frac{d}{dx}P_\ell(x)\right]  + \ell(\ell+1)P_\ell(x) = 0. $$
  \ech
\end{frame}


\begin{frame}
  \chtitle{证明}
  \bch
  $\ell=0$情况显然,假设$\ell\ge 1$,对恒等式:
  $$(x^2-1) \frac{d}{dx} \left[(x^2-1)^\ell\right] = 2\ell x (x^2-1)^\ell $$
  两边同求导$\ell$次得到
  $$(x^2-1)\frac{d^{\ell+1}}{dx^{\ell+1}} \left[(x^2-1)^\ell\right] =\ell(\ell+1) \frac{d^{\ell-1}}{dx^{\ell-1}} \left[(x^2-1)^\ell\right]. $$
  再次两边求导,并除以$2^\ell \ell !$即完成证明。

  
  \ech
\end{frame}



\begin{frame}
  \chtitle{罗巨格公式应用例题2}
  \bch
  \addfig{1}{think3.jpg}
  
  利用罗巨格公式证明$P_\ell(x)$满足
  $$\int_{-1}^1 \left[P_\ell(x)\right]^2 dx = \frac{2}{2\ell+1}. $$
  \ech
\end{frame}


\begin{frame}
  \chtitle{证明}
  \bch
{\small
  \bea
  \int_{-1}^1 \left[P_\ell(x)\right]^2 dx  &=& \frac{1}{4^\ell(\ell !)^2}  \int_{-1}^1 \left\{\frac{d^\ell}{dx^\ell}\left[(x^2-1)^\ell\right]\right\}^2 dx \newl
  &=& -\frac{1}{4^\ell(\ell !)^2}\int_{-1}^1 \frac{d^{\ell-1}}{dx^{\ell-1}}\left[(x^2-1)^\ell\right]\frac{d^{\ell+1}}{dx^{\ell+1}} \left[(x^2-1)^\ell\right] dx \newl
  &=& \ldots \newl
  &=& \frac{(-1)^\ell}{4^\ell(\ell !)^2}\int_{-1}^1 (x^2-1)^\ell\frac{d^{2\ell}}{dx^{2\ell}} \left[(x^2-1)^\ell\right] dx \newl
  &=& \frac{(2\ell)!}{(\ell !)^2}\int_{-1}^1 \left(\frac{1-x^2}{4}\right)^\ell dx   \newl
  &=& \frac{2 (2\ell)!}{(\ell !)^2}\int_0^1 t^\ell(1-t)^\ell dt  \newl
  &=& \frac{2}{2\ell+1}
  \eea
  上面我们做了变量替换:$t= \frac{x-1}{2}$并使用了$B(p,q) = \frac{\sfgamma{p}\sfgamma{q}}{\sfgamma{p+q}}$。
 }
  \ech
\end{frame}



\section{Explicit Expression of $Y_{\ell m}$}
\secpage{球谐函数的微分表示}{  {\scriptsize
    $$ Y_{\ell m}(\theta,\phi) =\frac{1}{2^\ell \ell !}\sqrt{\frac{(2\ell+1)}{4\pi} \frac{(\ell+m)!}{(\ell-m)!}}\left[\sin^m\theta \left(\frac{1}{\sin\theta}\frac{d}{d\theta}\right)^{\ell+m}\sin^{2\ell}\theta \right] e^{\ii m\phi}$$}}

\begin{frame}
  \chtitle{球谐函数的微分表达式}
  \bch

  {\small \blue
    $$ Y_{\ell m}(\theta,\phi) =\frac{1}{2^\ell \ell !}\sqrt{\frac{(2\ell+1)}{4\pi} \frac{(\ell+m)!}{(\ell-m)!}}\left[\sin^m\theta \left(\frac{1}{\sin\theta}\frac{d}{d\theta}\right)^{\ell+m}\sin^{2\ell}\theta \right] e^{\ii m\phi}$$}
  \ech
\end{frame}


\begin{frame}
  \chtitle{球谐函数的微分表达式}
  \bch
 对$Y_{\ell m}$,我们要证明
  \ech
\end{frame}

\section{Homework}

\begin{frame}
\chtitle{课后作业(题号54-56)}
\bch
\bitem
\item[54]{证明对任意小于$\ell$的非负整数$n$,
  $$\int_{-1}^1 x^n P_\ell(x) dx  = 0. $$}
\item[55]{}
\item[56]{}
\eitem
\ech
\end{frame}

\end{document}
