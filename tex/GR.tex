\section{Spacetime}
\secpage{附录A:时空}{时空是客观存在,和坐标系无关}

\begin{frame}
  \frametitle{平面}
  \addfig{1.5}{euclideands.png}
  
  欧氏(Euclidean)平面上,勾股定理处处成立。可以取直角坐标系$(x,y)$,使得
  $$ ds^2 = dx^2+dy^2.$$
  上面的表达式中的$ds$表示:点$(x,y)$和点$(x+dx, y+dy)$的距离。
\end{frame}


\begin{frame}
  \frametitle{三维欧氏空间}
  三维欧氏空间,可以取直角坐标系$(x,y,z)$,使得
  $$ ds^2 = dx^2+dy^2+dz^2.$$
  上面的表达式中的$ds$表示:点$(x,y,z)$和点$(x+dx, y+dy,z+dz)$的距离。
\end{frame}


\begin{frame}
  \frametitle{任意维欧氏空间}
  任意$n$维欧氏空间,可以取直角坐标系$(x^1, x^2,\ldots, x^n)$,使得
  $$ ds^2 = \sum_{i=1}^n (dx^i)^2.$$

  \skipline
  
  这里发生了一件让你很不舒服的事情:我把代表空间维度的指标$i$放到了右上角。这和$x$的$i$次方怎么区分?

\end{frame}


\begin{frame}
  \frametitle{$x^i$是$x$的$i$次方吗?}
  如果我事先告诉你$x^1, x^2,\ldots,x^n$代表$n$维空间的坐标,并垄断了符号$x$(你不允许再定义一个变量叫$x$),那么``$x$的$i$次方''这个说法是没有意义的。

    \addfig{1}{meimaobing.jpg}

    (但是你为啥要这样zuo呢?)

    (先别管,继续往下看)
\end{frame}


\begin{frame}
  \frametitle{矩阵$g_{ij}$}
  我们再考虑一个单位矩阵$g$,它的第$i$行第$j$列元素当然是
  $$g_{ij} = \delta_{ij} $$
  这里用到了克罗内克(Kroneck)$\delta$符号:$\delta_{ij}$当$i=j$时为$1$,当$i\ne j$时为零。

  \skiplines

  (弱弱问下:为什么现在你又把$i$, $j$写成下标了?)

  (先别管,继续往下看)
\end{frame}


\begin{frame}
  \frametitle{重新写一下$ds^2$}
  有了单位矩阵$g$以及
  $$ds^2 = \sum_{i=1}^n\sum_{j=1}^n g_{ij} x^i x^j .$$
\end{frame}


\begin{frame}
  \frametitle{爱因斯坦求和规则}
  {\bf 爱因斯坦求和规则:当一个指标在上面和下面同时各出现一次时,默认对它求和。}
             
  于是$ds^2$又变样了:

  $$ds^2=g_{ij}dx^idx^j.$$

\end{frame}


\begin{frame}
  \frametitle{非欧空间}
  如果空间是非欧几里得的(比如,一个球面),勾股定理就并不处处成立,那么,$(x^1,x^2,\ldots,x^n)$和$(x^1+dx^1, x^2+dx^2,\ldots, x^n+dx^n)$之间的``距离''就会是$(x^1,x^2,\ldots,x^n)$和$dx^1,dx^2,\ldots, dx^n$的非常复杂的函数。

\end{frame}

\begin{frame}
  \frametitle{无穷近邻点的距离平方是坐标微分的二次型}
  如果$dx^1, dx^2,\ldots,dx^n$都是小量的话,把$ds^2$做($dx^1, dx^2, \ldots, dx^n$视为变量,$x^1,x^2,\ldots,x^n$视为参数的)小量展开:首先,如果$dx^1,dx^2,\ldots,dx^n$均为零,则距离$ds$显然严格为零;所以最低阶近似不包含零次项。又因为距离平方$ds^2$不具有方向性,所以主导项不能是一次项。所以{\bf 主导项至少是二次项。}

  \skipline

  距离平方$ds^2$的主导项为坐标微分的二次型时,
  $$ ds^2 = g_{ij}dx^idx^j,$$
  这种空间的各种几何性质和空间中的物质的联系,是广义相对论的主题。
\end{frame}

