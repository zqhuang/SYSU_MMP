\documentclass[CJK]{beamer}
\usepackage{CJKutf8}
\usepackage{beamerthemesplit}
\usetheme{Malmoe}
\useoutertheme[footline=authortitle]{miniframes}
\usepackage{amsmath}
\usepackage{amssymb}
\usepackage{graphicx}
\usepackage{eufrak}
\usepackage{color}
\usepackage{slashed}
\usepackage{simplewick}
\usepackage{tikz}
\usepackage{tcolorbox}
\graphicspath{{../figures/}}
%%figures
\def\lfig#1#2{\includegraphics[width=#1 in]{#2}}
\def\addfig#1#2{\begin{center}\includegraphics[width=#1 in]{#2}\end{center}}
\def\wulian{\includegraphics[width=0.18in]{emoji_wulian.jpg}}
\def\bigwulian{\includegraphics[width=0.35in]{emoji_wulian.jpg}}
\def\bye{\includegraphics[width=0.18in]{emoji_bye.jpg}}
\def\bigbye{\includegraphics[width=0.35in]{emoji_bye.jpg}}
\def\huaixiao{\includegraphics[width=0.18in]{emoji_huaixiao.jpg}}
\def\bighuaixiao{\includegraphics[width=0.35in]{emoji_huaixiao.jpg}}
\def\jianxiao{\includegraphics[width=0.18in]{emoji_jianxiao.jpg}}
\def\bigjianxiao{\includegraphics[width=0.35in]{emoji_jianxiao.jpg}}
%% colors
\def\blacktext#1{{\color{black}#1}}
\def\bluetext#1{{\color{blue}#1}}
\def\redtext#1{{\color{red}#1}}
\def\darkbluetext#1{{\color[rgb]{0,0.2,0.6}#1}}
\def\skybluetext#1{{\color[rgb]{0.2,0.7,1.}#1}}
\def\cyantext#1{{\color[rgb]{0.,0.5,0.5}#1}}
\def\greentext#1{{\color[rgb]{0,0.7,0.1}#1}}
\def\darkgray{\color[rgb]{0.2,0.2,0.2}}
\def\lightgray{\color[rgb]{0.6,0.6,0.6}}
\def\gray{\color[rgb]{0.4,0.4,0.4}}
\def\blue{\color{blue}}
\def\red{\color{red}}
\def\green{\color{green}}
\def\darkgreen{\color[rgb]{0,0.4,0.1}}
\def\darkblue{\color[rgb]{0,0.2,0.6}}
\def\skyblue{\color[rgb]{0.2,0.7,1.}}
%%control
\def\be{\begin{equation}}
\def\ee{\nonumber\end{equation}}
\def\bea{\begin{eqnarray}}
\def\eea{\nonumber\end{eqnarray}}
\def\bch{\begin{CJK}{UTF8}{gbsn}}
\def\ech{\end{CJK}}
\def\bitem{\begin{itemize}}
\def\eitem{\end{itemize}}
\def\bcenter{\begin{center}}
\def\ecenter{\end{center}}
\def\bex{\begin{minipage}{0.2\textwidth}\includegraphics[width=0.6in]{jugelizi.png}\end{minipage}\begin{minipage}{0.76\textwidth}}
\def\eex{\end{minipage}}
\def\chtitle#1{\frametitle{\bch#1\ech}}
\def\bmat#1{\left(\begin{array}{#1}}
\def\emat{\end{array}\right)}
\def\bcase#1{\left\{\begin{array}{#1}}
\def\ecase{\end{array}\right.}
\def\bmini#1{\begin{minipage}{#1\textwidth}}
\def\emini{\end{minipage}}
\def\tbox#1{\begin{tcolorbox}#1\end{tcolorbox}}
\def\pfrac#1#2#3{\left(\frac{\partial #1}{\partial #2}\right)_{#3}}
%%symbols
\def\bropt{\,(\ \ \ )}
\def\sone{$\star$}
\def\stwo{$\star\star$}
\def\sthree{$\star\star\star$}
\def\sfour{$\star\star\star\star$}
\def\sfive{$\star\star\star\star\star$}
\def\rint{{\int_\leftrightarrow}}
\def\roint{{\oint_\leftrightarrow}}
\def\stdHf{{\textit{\r H}_f}}
\def\deltaH{{\Delta \textit{\r H}}}
\def\ii{{\dot{\imath}}}
\def\skipline{{\vskip0.1in}}
\def\skiplines{{\vskip0.2in}}
\def\lagr{{\mathcal{L}}}
\def\hamil{{\mathcal{H}}}
\def\vecv{{\mathbf{v}}}
\def\vecx{{\mathbf{x}}}
\def\vecy{{\mathbf{y}}}
\def\veck{{\mathbf{k}}}
\def\vecp{{\mathbf{p}}}
\def\vecn{{\mathbf{n}}}
\def\vecA{{\mathbf{A}}}
\def\vecP{{\mathbf{P}}}
\def\vecsigma{{\mathbf{\sigma}}}
\def\hatJn{{\hat{J_\vecn}}}
\def\hatJx{{\hat{J_x}}}
\def\hatJy{{\hat{J_y}}}
\def\hatJz{{\hat{J_z}}}
\def\hatj#1{\hat{J_{#1}}}
\def\hatphi{{\hat{\phi}}}
\def\hatq{{\hat{q}}}
\def\hatpi{{\hat{\pi}}}
\def\vel{\upsilon}
\def\Dint{{\mathcal{D}}}
\def\adag{{\hat{a}^\dagger}}
\def\bdag{{\hat{b}^\dagger}}
\def\cdag{{\hat{c}^\dagger}}
\def\ddag{{\hat{d}^\dagger}}
\def\hata{{\hat{a}}}
\def\hatb{{\hat{b}}}
\def\hatc{{\hat{c}}}
\def\hatd{{\hat{d}}}
\def\hatN{{\hat{N}}}
\def\hatH{{\hat{H}}}
\def\hatp{{\hat{p}}}
\def\Fup{{F^{\mu\nu}}}
\def\Fdown{{F_{\mu\nu}}}
\def\newl{\nonumber \\}
\def\vece{\mathrm{e}}
\def\calM{{\mathcal{M}}}
\def\calT{{\mathcal{T}}}
\def\calR{{\mathcal{R}}}
\def\barpsi{\bar{\psi}}
\def\baru{\bar{u}}
\def\barv{\bar{\upsilon}}
\def\qeq{\stackrel{?}{=}}
\def\torder#1{\mathcal{T}\left(#1\right)}
\def\rorder#1{\mathcal{R}\left(#1\right)}
\def\contr#1#2{\contraction{}{#1}{}{#2}#1#2}
\def\trof#1{\mathrm{Tr}\left(#1\right)}
\def\trace{\mathrm{Tr}}
\def\comm#1{\ \ \ \left(\mathrm{used}\ #1\right)}
\def\tcomm#1{\ \ \ (\text{#1})}
\def\slp{\slashed{p}}
\def\slk{\slashed{k}}
\def\calp{{\mathfrak{p}}}
\def\veccalp{\mathbf{\mathfrak{p}}}
\def\Tthree{T_{\tiny \textcircled{3}}}
\def\pthree{p_{\tiny \textcircled{3}}}
\def\dbar{{\,\mathchar'26\mkern-12mu d}}
\def\erf{\mathrm{erf}}
\def\const{\mathrm{constant}}
\def\pheat{\pfrac p{\ln T}V}
\def\vheat{\pfrac V{\ln T}p}
%%units
\def\fdeg{{^\circ \mathrm{F}}}
\def\cdeg{^\circ \mathrm{C}}
\def\atm{\,\mathrm{atm}}
\def\angstrom{\,\text{\AA}}
\def\SIL{\,\mathrm{L}}
\def\SIkm{\,\mathrm{km}}
\def\SIyr{\,\mathrm{yr}}
\def\SIGyr{\,\mathrm{Gyr}}
\def\SIV{\,\mathrm{V}}
\def\SImV{\,\mathrm{mV}}
\def\SIeV{\,\mathrm{eV}}
\def\SIkeV{\,\mathrm{keV}}
\def\SIMeV{\,\mathrm{MeV}}
\def\SIGeV{\,\mathrm{GeV}}
\def\SIcal{\,\mathrm{cal}}
\def\SIkcal{\,\mathrm{kcal}}
\def\SImol{\,\mathrm{mol}}
\def\SIN{\,\mathrm{N}}
\def\SIHz{\,\mathrm{Hz}}
\def\SIm{\,\mathrm{m}}
\def\SIcm{\,\mathrm{cm}}
\def\SIfm{\,\mathrm{fm}}
\def\SImm{\,\mathrm{mm}}
\def\SInm{\,\mathrm{nm}}
\def\SImum{\,\mathrm{\mu m}}
\def\SIJ{\,\mathrm{J}}
\def\SIW{\,\mathrm{W}}
\def\SIkJ{\,\mathrm{kJ}}
\def\SIs{\,\mathrm{s}}
\def\SIkg{\,\mathrm{kg}}
\def\SIg{\,\mathrm{g}}
\def\SIK{\,\mathrm{K}}
\def\SImmHg{\,\mathrm{mmHg}}
\def\SIPa{\,\mathrm{Pa}}

\def\courseurl{https://github.com/zqhuang/SYSU\_TD}

\def\tpage#1#2{
\begin{frame}
\begin{center}
\begin{Large}
\bch
热学 \\
第#1讲 #2

{\vskip 0.3in}

黄志琦

\ech
\end{Large}
\end{center}

\vskip 0.2in

\bch
教材:《热学》第二版,赵凯华,罗蔚茵,高等教育出版社
\ech

\bch
课件下载
\ech
\courseurl
\end{frame}
}

\def\bfr#1{
\begin{frame}
\chtitle{#1} 
\bch
}

\def\efr{
\ech 
\end{frame}
}

  \date{}
  \begin{document}
  \bch


\tpage{33}{三维无界空间中的波动问题}

\begin{frame}
  \frametitle{本讲内容}
  
\tableofcontents

\end{frame}

\section{Review and Practices}

\thinka{把 $\ln\cosh x$ 在 $x=0$ 附近展开到$x^4$.}

\thinkb{一根长为 $L$ 的孤立均匀导热棒,其材质的导热系数为 $\lambda$,热传导方程参数为 $a$。一开始导热棒处于热平衡。从 $t=0$ 时刻开始,在导热棒的一端($x=0$)处注入热流密度为 $j_1$ 的恒定热流,并在另一端注入热流密度为 $j_2$的恒定热流( $j_2 < j_1$)。经过时间 $t \gg \frac{L^2}{a}$ 之后,导热棒上两端的温度差是多少?}

\thinkc{已知某种保温材料的比热很小可以忽略不计。测试员用该保温材料做成一个内半径为 $0.1\mathrm{m}$,外半径为 $0.11\mathrm{m}$ 的球形保温容器。把该容器装满 $100^\circ\mathrm{C}$ 的开水后,置于温度为 $20^\circ\mathrm{C}$ 的房间里,一天后发现容器内水温下降了 $40^\circ\mathrm{C}$,保温效果很不理想。测试员于是加大保温容器的厚度,使外半径达到 $0.2\mathrm{m}$,然后装满 $100^\circ\mathrm{C}$ 的开水,置于温度为 $20^\circ\mathrm{C}$ 的房间里,问:一天后保温容器内的水温下降多少度?}

\thinkd{从前有一个神秘的星球叫浮士德星,这个星球的每年有一万天。星球上的人出生时都会和魔鬼签署了一份协议:每个人每天起床都要投掷 $600$ 次骰子。如果 $1,2,3,4,5,6$ 每个数字均恰好出现 $100$ 次,魔鬼就要带走这个人的灵魂,否则魔鬼会保护这个人一天平安无事。  有一天赌王来到了浮士德星,传授给了浮士德星人一个秘术——666大法。使用秘术之后,每次掷骰子出现 $6$ 的概率会稍稍增加,但增加的幅度非常小以致于魔鬼无法察觉。从此浮士德村人的平均寿命延长了5年。请由此推断: 666秘术能使单次掷骰子出现 $6$ 的概率增加多少?}

\section{Green's Function: Wave Equation in 2D}

\secpage{三维空间波动问题的格林函数}{$$ \frac{\delta(at-|\mathbf{x}-\mathbf{x}_0|)}{4\pi a|\mathbf{x}-\mathbf{x}_0|}. $$}

\begin{frame}
  \frametitle{三维空间的波动问题}
  在三维无边界空间中,设有波动方程
  $$\frac{\partial^2u}{\partial t^2}-a^2\nabla^2u = 0.$$
  和初始条件
  $$ \left.u\right\vert_{t=0} = 0,\ \ \left.\frac{\partial u}{\partial t}\right\vert_{t=0} = \delta(\mathbf{x}-\mathbf{x}_0).$$

  这是个典型的格林函数问题。根据我们在一维和二维求解的经验,只要求出了初始速度脉冲的格林函数,对时间求一下偏导就能得到初始位移脉冲的格林函数。(请自行证明这个结论。)
\end{frame}


\begin{frame}  
  可以取 $\mathbf{x}_0$ 所在位置为原点建立球坐标系,直接写出格林函数为:
  $$G_v=  \int_0^\infty j_0(kr) c(k)\sin(akt) dk.$$
  {\scriptsize 注意利用初始条件的球对称性,我们仅保留了$l=m=0$的项。并且由于无边界,$k$可以连续取到一切非负实数值。这里的 $c(k)dk$ (函数$c$待定) 是连续情况下的 “展开系数”。}
\end{frame}

\begin{frame}
  \frametitle{初始条件}
  这样初始条件就是
  $$\frac{\delta(r-\epsilon)}{4\pi a r^2} = \int_0^\infty j_0(kr) c(k) kdk .$$
  这里$\epsilon\rightarrow 0^+$。(请自行思考为何 $\frac{\delta(r-\epsilon)}{4\pi r^2}$ 是在原点的三维 $\delta$ 函数。)

  利用贝塞尔函数的正交关系,
  $$\frac{\pi\delta(r-\epsilon)}{2r^2} = \int_0^\infty k^2j_0(kr)j_0(k\epsilon)dk.$$
  直接看出 $c(k) = \frac{kj_0(k\epsilon)}{2\pi^2 a} $。令 $\epsilon\rightarrow 0^+$ 即得 $c(k) = \frac{k}{2\pi^2a}$.
\end{frame}

\begin{frame}
  于是
  $$G_v=  \frac{1}{2\pi^2a} \int_0^\infty \frac{\sin(kr)}{kr} \sin(akt) k dk.$$
  利用 $j_0$ 的正交关系,有
  $$G_v=  \frac{1}{4\pi ar} \delta(at-r) = \frac{\delta(at-|\mathbf{x}-\mathbf{x}_0|)}{4\pi a|\mathbf{x}-\mathbf{x}_0|}.$$
\end{frame}


\begin{frame}
  考虑初值问题
  $$\frac{\partial^2u}{\partial t^2}-a^2\nabla^2u = 0.$$
  $$ \left.u\right\vert_{t=0} = \phi(\mathbf{x}),\ \ \left.\frac{\partial u}{\partial t}\right\vert_{t=0} = \psi(\mathbf{x}).$$
\end{frame}

\begin{frame}
  其解为
  \bea
  u(\mathbf{x},t) &=& \frac{1}{4\pi a} \left[\iiint \phi(\mathbf{x}_0) \frac{\delta(at-|\mathbf{x}-\mathbf{x}_0|)}{|\mathbf{x}-\mathbf{x}_0|} d^3\mathbf{x}_0\right. \newl
    &+& \left.\frac{\partial}{\partial t}\iiint \psi(\mathbf{x}_0) \frac{\delta(at-|\mathbf{x}-\mathbf{x}_0|)}{|\mathbf{x}-\mathbf{x}_0|} d^3\mathbf{x}_0\right].\nonumber
  \eea
\end{frame}


\ech
\end{document}
