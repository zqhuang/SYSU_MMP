\documentclass[12pt,CJK]{article}
\usepackage{geometry}
\input{reduced_macros.tex}
\geometry{tmargin=0.3in, bmargin=0.5in, lmargin=0.5in, rmargin=0.9in, nohead, nofoot}
\def\mark#1{{\color{blue} (#1分)}}
\renewcommand{\thepage}{}
\begin{document}
\bch
{\large 数理方法 课堂小测III 诸神黄昏版}

{\vskip 0.2in}

特殊函数定义列表:
\bea
J_\nu(x) &=& \sum_{k=0}^\infty \frac{(-1)^k}{k!(k+\nu)!}\left(\frac{x}{2}\right)^{2k+\nu}; \newl
Y_\nu(x) &=& \lim_{\mu\rightarrow \nu}\frac{J_\mu(x)\cos\mu\pi - J_{-\mu}(x)}{\sin (\mu\pi)}; \newl
j_\ell(x) &=& \sqrt{\frac{\pi}{2x}} J_{\ell+1/2}(x),\ \ \ell\in Z; \newl
y_\ell(x) &=& \sqrt{\frac{\pi}{2x}} Y_{\ell+1/2}(x), \ \ \ell\in Z; \newl
P_{\ell}(x) &=& \sum_{k=0}^\ell \frac{(\ell + k)!}{(k!)^2(\ell-k)!}\left(\frac{x-1}{2}\right)^k,\ \ \ell\in Z, \ell\ge 0; \newl
Y_{\ell, m}(\theta,\phi) &=&\frac{1}{2^\ell \ell !}\sqrt{\frac{(2\ell+1)}{4\pi} \frac{(\ell-m)!}{(\ell+m)!}}\left[\sin^m\theta \left(\frac{1}{\sin\theta}\frac{d}{d\theta}\right)^{\ell+m}\sin^{2\ell}\theta \right] e^{\ii m\phi},\ \ \ell,m\in Z, \ell\ge|m|.
\eea

\bitem
\item[(一)]{选择题,每题10分,共30分。

  \bitem

\item[(1)]{ $\sin\theta \,\frac{\partial Y_{7,3}(\theta,\phi)}{\partial \theta} $ 可以表示成哪些球谐函数的线性组合? \bropt
  
  \foptlist{$Y_{6,3}(\theta,\phi)$ 和 $Y_{7,3}(\theta,\phi)$}{$Y_{7,3}(\theta,\phi)$ 和 $Y_{8,3}(\theta,\phi)$}{$Y_{6,3}(\theta,\phi)$ 和 $Y_{8,3}(\theta,\phi)$}{$Y_{6,3}(\theta,\phi)$, $Y_{7,3}(\theta,\phi)$ 和 $Y_{8,3}(\theta,\phi)$}}
  
\item[(2)]{ 一个单位体积的比热为 $ 10^5 \SIJ \SIK^{-1}\SIm^{-3}$, 导热系数为$100\mathrm{W}\SIm^{-1}\SIK^{-1}$,半径为$R=0.1\SIm$的均匀材质孤立不良导体球。一开始上半球温度为$350\SIK$,下半球温度为$250\SIK$。估算至少经过多长时间后,球各处的温度都在$299.9\SIK$和$300.1\SIK$之间 (请选择数量级最接近的答案)。 \bropt

  \optlist{$0.003\SIs$}{$ 0.1\SIs$}{$3\SIs$}{$90\SIs$} }
  

\item[(3)]{把$e^k$的小数部分记为$r_k$,例如$e^1= 2.718\ldots$, $r_1 = 0.718\ldots$; $e^2=7.389\ldots$, $r_2=0.389\ldots$。
  把第$k$个质数记为$m_k$,例如$m_1=2, m_2=3, m_3=5, \ldots$。记$\ell = m_{1001}$,试估计
  $$\sum_{k=1}^{1000} \left[ Y_{\ell, m_k }\left(\arccos (2r_k-1) , 0\right)\right] ^2$$
  和下列哪一个数量级最接近? \bropt

  \optlist{$1$}{$10$}{$100$}{$1000$}}

  \eitem  
}
\item[(二)]{填空题(每题10分,共20分)
  \bitem
  \item[(1)]{计算不定积分: $\int \frac{dx}{x\left[\left(J_3(x)\right)^2+\left(Y_3(x)\right)^2\right]} =  $ \uline{2} 。}
\item[(2)]{请估算球面谐函数: $Y_{10000,2}\left(\frac{\pi}{50}, 0\right)\approx $ \uline{1} 。}
  \eitem
}
  
\item[(三)]{


  内半径为$R$,外半径为$2R$的均匀不良导体空心球,导热系数为 $\lambda$,单位质量比热为 $c$,质量密度为 $\rho$,一开始温度为 $T_0$。在 $t=0$ 时刻把空心球投入温度为 $2T_0$ 的热库,计算此后空心球内各点的温度变化。(25分)

    {\vskip 2in}

}

\item[(四)]{设在某个空间区域 $\Omega$ 内的每一点都有个对应的“势能”,即可以写出势能函数 $V(\vecx),\ \vecx\in \Omega$。考虑满足微分方程
  $$\left[V(\vecx)-\frac{1}{2}\nabla^2\right] u = E u, $$
  且在 $\Omega$ 的边界上满足一般零边界条件( $u$ 和 $\nabla u$ 的法向分量的某个固定线性组合为零,线性组合的系数允许在边界各个点不同)的解。其中 $E$ 为待定的“本征值”。在量子力学里, $u$是波函数,算符$-\frac{1}{2}\nabla^2$具有动能的含义,本征值$E$则代表了总能量。
  \bitem
\item[(1)]{证明任何两个不同的 $E$ 对应的两个解在 $\Omega$ 内正交(乘积的积分为零)。(10分)}
\item[(2)]{设 $\Omega$ 为整个三维空间;$V(\vecx) = -\frac{1}{|\vecx|}$;是否存在$E<0$ (对应量子力学里的束缚态) 使方程有处处有限且在无穷远处趋向于零的解?如果存在,$E$要取怎样的值? (15分)}  
  \eitem
}

\eitem  





\ech
\end{document}
