\documentclass[12pt,CJK]{article}
\usepackage{geometry}
\input{reduced_macros.tex}
\geometry{tmargin=0.3in, bmargin=0.5in, lmargin=0.8in, rmargin=0.8in, nohead, nofoot}
\def\mark#1{{\color{blue} (#1分)}}
\renewcommand{\thepage}{}
\begin{document}
\bch
{\large 大学生天文竞赛样题}

%{\vskip 0.3in}

姓名 ....................... {\hskip 0.3in}    报名号 .......................{\hskip 0.3in}  分数 ...................


\bitem
\item[(一)]{ 选择题,每题4分,共10题。

  \bitem
\item[(1)]{月食的原理是 \bropt
  
  \optlist{月亮被当作狗粮吃了}{地球磁场活动暂时屏蔽了月光}{月地日一线,地球挡住了太阳光}{月日地一线,太阳挡住了月光}
}
\item[(2)]{北斗七星属于哪个星座? \bropt

  \optlist{天琴座}{仙女座}{小熊座}{大熊座}
}  
  
\item[(3)]{距离地球最近的恒星是 \bropt
  
  \optlist{太阳}{天狼星}{织女星}{比邻星}
}
\item[(4)]{银河系大约有多少恒星? \bropt
  
  \optlist{一亿亿}{一千亿}{一千万}{一万}}

\item[(5)]{根据牛顿万有引力定律,太阳对地球上一个成年人的引力大约为 \bropt

  \optlist{$30\SIN$}{$3\SIN$}{$0.3\SIN$}{$0.03\SIN$}}
\item[(6)]{根据上题计算结果回答,如果在正午(太阳在头顶方向)和午夜(太阳在脚底方向)分别称体重,那么太阳引力造成的影响 \bropt

  \optlist{严格为零}{只有高阶的潮汐力影响,远小于$0.03\SIN$}{不小于$0.03\SIN$,和季节无关}{不小于$0.03\SIN$,具体大小和季节有关}}
  
  
\item[(7)]{宋朝的《宋会要》中记载了1054年的一个奇异天文景象:“初,至和元年五月,晨出东方,守天关。昼见如太白,芒角四出,色赤白,凡见二十三日。“这描述的是什么天文现象? \bropt

  \optlist{超新星爆发}{Gamma射线暴}{黑洞碰撞}{宇宙大爆炸}
}
\item[(8)]{一般的恒星演化次序为: \bropt
  
\optlist{超新星-主序星-致密星-红巨星}{主序星-超新星-红巨星-致密星}{主序星-致密星-红巨星-超新星}{主序星-红巨星-超新星-致密星}
}


\item[(9)]{一般认为恒星由大团的冷分子云引力塌缩而来。现在考虑一团温度为$10\SIK$,尺度大约为$10^{18}\SIm$的冷分子云,其主要成分为氢气。如果要求分子云不发生引力塌缩,那么分子云的密度的上限大约为 \bropt

  \optlist{$10^{-21}\SIkg/\SIm^3$}{$10^{-15}\SIkg/\SIm^3$}{$10^{-9}\SIkg/\SIm^3$}{$10^{-3}\SIkg/\SIm^3$}
}
  
\item[(10)]{根据大爆炸宇宙学标准模型,宇宙现在的年龄约为137亿年。试由此估算可观测宇宙的尺度约为: \bropt
  
  \optlist{$10^{30}\SIm$}{$10^{26}\SIm$}{$10^{22}\SIm$}{$10^{18}\SIm$}}
  
  \eitem
  }

  
\item[(二)]{假设在距离地球一亿$\SIAU$的$\alpha$行星上住着外星人。$\alpha$行星绕着一颗与太阳质量非常接近的恒星做近似圆周运动。$\alpha$行星上的``一年''(即绕恒星一周的时间)大概是地球上的27年。
  \bitem
\item[(1)]{$\alpha$行星离它围绕的恒星有多远?(5分)}
\item[(2)]{$\alpha$行星上的居民想用一个可见光波段的望远镜观测我们的太阳,并用三角视差法确定他们离我们的距离。这个望远镜的口径至少要多大? (10分)}
  \eitem

  \vspace{1in}
  }

  
\item[(三)]{《春秋》记载“秋七月,有星孛入于北斗”,描述的是哈雷彗星。哈雷彗星76年回归一次,离太阳最近时为$0.586\SIAU$,最远时为$35.1\SIAU$。假想一个场景:哈雷彗星在近日点受到小天体从侧面撞击,运动方向向外偏离了$5^\circ$(即和太阳连线的夹角从$90^\circ$变为$95^\circ$),运动速率保持不变。问这个情况下彗星的回归周期变为多少年? (15分)
    \addfig{3}{collision.png}

  \vspace{2in}    
    
  }


  
\item[(四)]{
  假设内径为$5\SIm$的封闭球形飞船绕中子星做每秒一周的匀速圆周运动。飞船的质心在球心,且自转和公转同步(即保持同一面对着中子星)。飞船内有温度为$285\SIK$的氧气。问:氧气的压强是均匀的吗?如果不均匀,最小压强和最大压强之比为多少? (15分)

  \vspace{2in}    
}
 

 

\item[(五)]{宇宙微波背景辐射是什么?(5分) 按照现在测量的宇宙膨胀速率,即哈勃常数$H_0 \approx 70 \SIkm/\SIs/\SIMpc$ (距离单位$\SIMpc$为百万秒差距,大概是$3.26$百万光年),计算多久以后宇宙微波背景辐射的平均温度会下降百万分之一。(10分)
}
\eitem

 


\ech
\end{document}
