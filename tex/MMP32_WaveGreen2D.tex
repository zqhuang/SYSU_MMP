\documentclass[CJK]{beamer}
\usepackage{CJKutf8}
\usepackage{beamerthemesplit}
\usetheme{Malmoe}
\useoutertheme[footline=authortitle]{miniframes}
\usepackage{amsmath}
\usepackage{amssymb}
\usepackage{graphicx}
\usepackage{eufrak}
\usepackage{color}
\usepackage{slashed}
\usepackage{simplewick}
\usepackage{tikz}
\usepackage{tcolorbox}
\graphicspath{{../figures/}}
%%figures
\def\lfig#1#2{\includegraphics[width=#1 in]{#2}}
\def\addfig#1#2{\begin{center}\includegraphics[width=#1 in]{#2}\end{center}}
\def\wulian{\includegraphics[width=0.18in]{emoji_wulian.jpg}}
\def\bigwulian{\includegraphics[width=0.35in]{emoji_wulian.jpg}}
\def\bye{\includegraphics[width=0.18in]{emoji_bye.jpg}}
\def\bigbye{\includegraphics[width=0.35in]{emoji_bye.jpg}}
\def\huaixiao{\includegraphics[width=0.18in]{emoji_huaixiao.jpg}}
\def\bighuaixiao{\includegraphics[width=0.35in]{emoji_huaixiao.jpg}}
\def\jianxiao{\includegraphics[width=0.18in]{emoji_jianxiao.jpg}}
\def\bigjianxiao{\includegraphics[width=0.35in]{emoji_jianxiao.jpg}}
%% colors
\def\blacktext#1{{\color{black}#1}}
\def\bluetext#1{{\color{blue}#1}}
\def\redtext#1{{\color{red}#1}}
\def\darkbluetext#1{{\color[rgb]{0,0.2,0.6}#1}}
\def\skybluetext#1{{\color[rgb]{0.2,0.7,1.}#1}}
\def\cyantext#1{{\color[rgb]{0.,0.5,0.5}#1}}
\def\greentext#1{{\color[rgb]{0,0.7,0.1}#1}}
\def\darkgray{\color[rgb]{0.2,0.2,0.2}}
\def\lightgray{\color[rgb]{0.6,0.6,0.6}}
\def\gray{\color[rgb]{0.4,0.4,0.4}}
\def\blue{\color{blue}}
\def\red{\color{red}}
\def\green{\color{green}}
\def\darkgreen{\color[rgb]{0,0.4,0.1}}
\def\darkblue{\color[rgb]{0,0.2,0.6}}
\def\skyblue{\color[rgb]{0.2,0.7,1.}}
%%control
\def\be{\begin{equation}}
\def\ee{\nonumber\end{equation}}
\def\bea{\begin{eqnarray}}
\def\eea{\nonumber\end{eqnarray}}
\def\bch{\begin{CJK}{UTF8}{gbsn}}
\def\ech{\end{CJK}}
\def\bitem{\begin{itemize}}
\def\eitem{\end{itemize}}
\def\bcenter{\begin{center}}
\def\ecenter{\end{center}}
\def\bex{\begin{minipage}{0.2\textwidth}\includegraphics[width=0.6in]{jugelizi.png}\end{minipage}\begin{minipage}{0.76\textwidth}}
\def\eex{\end{minipage}}
\def\chtitle#1{\frametitle{\bch#1\ech}}
\def\bmat#1{\left(\begin{array}{#1}}
\def\emat{\end{array}\right)}
\def\bcase#1{\left\{\begin{array}{#1}}
\def\ecase{\end{array}\right.}
\def\bmini#1{\begin{minipage}{#1\textwidth}}
\def\emini{\end{minipage}}
\def\tbox#1{\begin{tcolorbox}#1\end{tcolorbox}}
\def\pfrac#1#2#3{\left(\frac{\partial #1}{\partial #2}\right)_{#3}}
%%symbols
\def\bropt{\,(\ \ \ )}
\def\sone{$\star$}
\def\stwo{$\star\star$}
\def\sthree{$\star\star\star$}
\def\sfour{$\star\star\star\star$}
\def\sfive{$\star\star\star\star\star$}
\def\rint{{\int_\leftrightarrow}}
\def\roint{{\oint_\leftrightarrow}}
\def\stdHf{{\textit{\r H}_f}}
\def\deltaH{{\Delta \textit{\r H}}}
\def\ii{{\dot{\imath}}}
\def\skipline{{\vskip0.1in}}
\def\skiplines{{\vskip0.2in}}
\def\lagr{{\mathcal{L}}}
\def\hamil{{\mathcal{H}}}
\def\vecv{{\mathbf{v}}}
\def\vecx{{\mathbf{x}}}
\def\vecy{{\mathbf{y}}}
\def\veck{{\mathbf{k}}}
\def\vecp{{\mathbf{p}}}
\def\vecn{{\mathbf{n}}}
\def\vecA{{\mathbf{A}}}
\def\vecP{{\mathbf{P}}}
\def\vecsigma{{\mathbf{\sigma}}}
\def\hatJn{{\hat{J_\vecn}}}
\def\hatJx{{\hat{J_x}}}
\def\hatJy{{\hat{J_y}}}
\def\hatJz{{\hat{J_z}}}
\def\hatj#1{\hat{J_{#1}}}
\def\hatphi{{\hat{\phi}}}
\def\hatq{{\hat{q}}}
\def\hatpi{{\hat{\pi}}}
\def\vel{\upsilon}
\def\Dint{{\mathcal{D}}}
\def\adag{{\hat{a}^\dagger}}
\def\bdag{{\hat{b}^\dagger}}
\def\cdag{{\hat{c}^\dagger}}
\def\ddag{{\hat{d}^\dagger}}
\def\hata{{\hat{a}}}
\def\hatb{{\hat{b}}}
\def\hatc{{\hat{c}}}
\def\hatd{{\hat{d}}}
\def\hatN{{\hat{N}}}
\def\hatH{{\hat{H}}}
\def\hatp{{\hat{p}}}
\def\Fup{{F^{\mu\nu}}}
\def\Fdown{{F_{\mu\nu}}}
\def\newl{\nonumber \\}
\def\vece{\mathrm{e}}
\def\calM{{\mathcal{M}}}
\def\calT{{\mathcal{T}}}
\def\calR{{\mathcal{R}}}
\def\barpsi{\bar{\psi}}
\def\baru{\bar{u}}
\def\barv{\bar{\upsilon}}
\def\qeq{\stackrel{?}{=}}
\def\torder#1{\mathcal{T}\left(#1\right)}
\def\rorder#1{\mathcal{R}\left(#1\right)}
\def\contr#1#2{\contraction{}{#1}{}{#2}#1#2}
\def\trof#1{\mathrm{Tr}\left(#1\right)}
\def\trace{\mathrm{Tr}}
\def\comm#1{\ \ \ \left(\mathrm{used}\ #1\right)}
\def\tcomm#1{\ \ \ (\text{#1})}
\def\slp{\slashed{p}}
\def\slk{\slashed{k}}
\def\calp{{\mathfrak{p}}}
\def\veccalp{\mathbf{\mathfrak{p}}}
\def\Tthree{T_{\tiny \textcircled{3}}}
\def\pthree{p_{\tiny \textcircled{3}}}
\def\dbar{{\,\mathchar'26\mkern-12mu d}}
\def\erf{\mathrm{erf}}
\def\const{\mathrm{constant}}
\def\pheat{\pfrac p{\ln T}V}
\def\vheat{\pfrac V{\ln T}p}
%%units
\def\fdeg{{^\circ \mathrm{F}}}
\def\cdeg{^\circ \mathrm{C}}
\def\atm{\,\mathrm{atm}}
\def\angstrom{\,\text{\AA}}
\def\SIL{\,\mathrm{L}}
\def\SIkm{\,\mathrm{km}}
\def\SIyr{\,\mathrm{yr}}
\def\SIGyr{\,\mathrm{Gyr}}
\def\SIV{\,\mathrm{V}}
\def\SImV{\,\mathrm{mV}}
\def\SIeV{\,\mathrm{eV}}
\def\SIkeV{\,\mathrm{keV}}
\def\SIMeV{\,\mathrm{MeV}}
\def\SIGeV{\,\mathrm{GeV}}
\def\SIcal{\,\mathrm{cal}}
\def\SIkcal{\,\mathrm{kcal}}
\def\SImol{\,\mathrm{mol}}
\def\SIN{\,\mathrm{N}}
\def\SIHz{\,\mathrm{Hz}}
\def\SIm{\,\mathrm{m}}
\def\SIcm{\,\mathrm{cm}}
\def\SIfm{\,\mathrm{fm}}
\def\SImm{\,\mathrm{mm}}
\def\SInm{\,\mathrm{nm}}
\def\SImum{\,\mathrm{\mu m}}
\def\SIJ{\,\mathrm{J}}
\def\SIW{\,\mathrm{W}}
\def\SIkJ{\,\mathrm{kJ}}
\def\SIs{\,\mathrm{s}}
\def\SIkg{\,\mathrm{kg}}
\def\SIg{\,\mathrm{g}}
\def\SIK{\,\mathrm{K}}
\def\SImmHg{\,\mathrm{mmHg}}
\def\SIPa{\,\mathrm{Pa}}

\def\courseurl{https://github.com/zqhuang/SYSU\_TD}

\def\tpage#1#2{
\begin{frame}
\begin{center}
\begin{Large}
\bch
热学 \\
第#1讲 #2

{\vskip 0.3in}

黄志琦

\ech
\end{Large}
\end{center}

\vskip 0.2in

\bch
教材:《热学》第二版,赵凯华,罗蔚茵,高等教育出版社
\ech

\bch
课件下载
\ech
\courseurl
\end{frame}
}

\def\bfr#1{
\begin{frame}
\chtitle{#1} 
\bch
}

\def\efr{
\ech 
\end{frame}
}

  \date{}
  \begin{document}
  \bch


\tpage{32}{二维无界空间中的波动问题}

\begin{frame}
  \frametitle{本讲内容}
  
\tableofcontents

\end{frame}

\section{Review and Practices}

\thinka{估算积分
  $$\int_{100\pi}^\infty \frac{\sin x}{x} dx.$$}

\thinkb{长度为$L$的均匀导热棒一端和温度为 $200\mathrm{K}$ 的热库接触,并在 $t=0$ 时刻和热库处于热平衡。从 $t=0$ 时刻开始,在导热棒的另一端注入恒定大小的热流。设已知导热棒的材料的热传导方程参数为 $a$,在 $t=\frac{L^2}{a}$ 时刻,棒的中点的温度为 $300\mathrm{K}$。
问:在 $t=\frac{2L^2}{a}$ 时刻,棒的中点的温度为多少 $\mathrm{K}$?
}

\thinkd{设 $k_1>k_2>0$ 证明:
  $$\int_0^\infty xJ_m(k_1x)J_m(k_2x)dx = \frac{\delta(k_1-k_2)}{k_1}.$$}

\section{Green's Function: Wave Equation in 2D}

\begin{frame}
  \frametitle{有限大小薄膜振动}
求解一个周边固定的半径为 $R$ 的圆形弹性轻薄膜的横向(即垂直于薄膜表面的)小振动$u(r,\theta, t)$ (这里 $r,\theta$是以薄膜中心为原点的极坐标, $0\le r\le R$)。设其初始位移为已知函数 $\phi(r,\theta)$,初始速度为零。波动方程的参数 $a$ 已知。
\end{frame}


\begin{frame}
  \frametitle{第一步:写出三要素}
  写出方程
  $$\frac{\partial^2u}{\partial t^2}-a^2\nabla^2u = 0.$$
  边界条件
  $$ \left.u\right\vert_{r=R} = 0.$$
  和初始条件
  $$ \left.u\right\vert_{t=0} = \phi(r,\theta),\ \ \left.\frac{\partial u}{\partial t}\right\vert_{t=0} = 0.$$  
\end{frame}


\begin{frame}
  \frametitle{第二步:写出分离变量形式解}
  考虑所给区域为圆盘,直接写出解的分离变量形式:
  $$u(r,\theta, t)= \sum_{m=0}^\infty \sum_{i=1}^\infty J_m(k_{m,i}r)\left[A_{m,i}\cos(m\theta)+B_{m,i}\sin(m\theta)\right]\cos(ak_{m,i}t)$$
  {\scriptsize 这里我们只写了 $\cos(akt)$ 而没有写 $\sin(akt)$的成分是利用了初始速度为零的条件;如果是初始位移为零,初始速度不为零,则要写 $\sin(akt)$;如果初始速度和位移均不为零,则 $\cos(akt)$和 $\sin(akt)$的成分均要保留,待定系数就多了一倍(当然非平凡的初始条件也多了一倍,因此还是ok)。  }

\end{frame}

\begin{frame}
  \frametitle{第三步:解释解里面的$k$如何取值。}
    利用边界条件,我们对 $k_{m,i}$ 的限制为
    $$J_m(k_{m,i}R) = 0.$$
    即$k_{m,i}$是 $J_m$的第 $i$个正式数零点和 $R$之比。
\end{frame}

\begin{frame}
  \frametitle{第四步:把初始条件按展开基投影,写出展开系数}
  由初始条件有:
  $$ \phi(r,\theta)=\sum_{m=0}^\infty \sum_{i=1}^\infty J_m(k_{m,i}r)\left[A_{m,i}\cos(m\theta)+B_{m,i}\sin(m\theta)\right].$$
  于是把方程左边按右边的正交展开基进行投影得到
  $$ A_{m,i} = \frac{\int_0^Rrdr\int_0^{2\pi}d\theta \left[\phi(r,\theta)J_m(k_{m,i}r)\cos(m\theta)\right]}{\int_0^Rrdr\int_0^{2\pi}d\theta \left[J_m(k_{m,i}r)\cos(m\theta)\right]^2}.$$

  $$ B_{m,i} = \frac{\int_0^Rrdr\int_0^{2\pi}d\theta \left[\phi(r,\theta)J_m(k_{m,i}r)\sin(m\theta)\right]}{\int_0^Rrdr\int_0^{2\pi}d\theta \left[J_m(k_{m,i}r)\sin(m\theta)\right]^2}.$$  
  
\end{frame}

\begin{frame}
  \frametitle{第五步:把你会的积分都积出来}

  \addfig{2}{laijiewo.jpg}
  
\end{frame}

\begin{frame}
  \frametitle{来点格林函数问题}
  如果初始位移 $\phi = \delta(\mathbf{x}-\mathbf{x}_0)$,那么就是典型的格林函数问题。我们考虑一个简单的也通常是格林函数非常有用的$R\rightarrow \infty$,也就是无边界的情形。

  \skiplines
  
  这时可以取 $\mathbf{x}_0$ 所在位置为原点建立极坐标系,直接写出格林函数为:
  $$u =  \int_0^\infty J_0(kr)\cos(akt) c(k) dk.$$
  {\scriptsize 注意利用初始条件的旋转对称性,我们直接丢掉了 $m>0$的项。并且由于无边界,$k$可以连续取到一切非负实数值。这里的 $c(k)dk$ (函数$c$待定) 是连续情况下的 “展开系数”。}
\end{frame}

\begin{frame}
  \frametitle{初始条件}
  这样初始条件就是
  $$\frac{\delta(r-\epsilon)}{2\pi r} = \int_0^\infty J_0(kr) c(k) dk.$$
  这里$\epsilon\rightarrow 0^+$。(请自行思考为何 $\frac{\delta(r-\epsilon)}{2\pi r}$ 是在原点的二维 $\delta$ 函数。)

  两边同乘以 $ rJ_0(k'r)dr$ (这里 $k'$是任取的正数),并对 $r$ 积分,得到
  \bea
  \frac{J_0(k'\epsilon)}{2\pi} &=& \int_0^\infty c(k)dk \int_0^\infty r J_0(kr)J_0(k'r)dr\newl
  &=&  \int_0^\infty c(k)dk \frac{\delta(k-k')}{k'}  \newl
  &=& \frac{c(k')}{k'}\nonumber
  \eea

\end{frame}

\begin{frame}
  \frametitle{最终解}
  令$\epsilon\rightarrow 0^+$,即得到
  $$c(k) = \frac{k}{2\pi}.$$
  即最终得到
  $$ u = \frac{1}{2\pi} \int_0^\infty  k J_0(kr)\cos(akt)  dk.$$
\end{frame}


\begin{frame}
  \frametitle{位移脉冲和速度脉冲对应的格林函数}
  这样,在 $\mathbf{x}_0$ 处对应的初始位移脉冲产生的响应(也就是格林函数)是
  $$ G_s(\mathbf{x},t;\mathbf{x}_0) = \frac{1}{2\pi} \int_0^\infty  k J_0(k|\mathbf{x}-\mathbf{x}_0|)\cos(akt)  dk.$$  
  类似地,可以求出初始速度脉冲的响应为
  $$ G_v(\mathbf{x},t;\mathbf{x}_0) = \frac{1}{2\pi a} \int_0^\infty   J_0(k|\mathbf{x}-\mathbf{x}_0|)\sin(akt)  dk.$$  
\end{frame}


\begin{frame}
  \frametitle{通过一系列操作……}
  附录进一步给出了积分的结果
  $$ G_s(\mathbf{x},t;\mathbf{x}_0) =\frac{1}{2\pi}\left[ \frac{\delta(at-|\mathbf{x}-\mathbf{x}_0|)}{\sqrt{a^2t^2-|\mathbf{x}-\mathbf{x}_0|^2}} - \frac{at\, \heaviside(at-|\mathbf{x}-\mathbf{x}_0|)}{\left(a^2t^2-|\mathbf{x}-\mathbf{x}_0|^2\right)^{3/2}}\right].$$
  $$ G_v(\mathbf{x},t;\mathbf{x}_0) = \frac{1}{2\pi a}\frac{\heaviside(at-|\mathbf{x}-\mathbf{x}_0|)}{\sqrt{a^2t^2-|\mathbf{x}-\mathbf{x}_0|^2}}.$$
\end{frame}


\begin{frame}
  \frametitle{一般解}
  对于一般的初始位移 $\phi(\mathbf{x})$ 和初始速度 $\psi(\mathbf{x})$,解就可以用格林函数搞定:
 $$u(\mathbf{x},t) = \iint d^2\mathbf{x}_0   \left[G_s(\mathbf{x},t;\mathbf{x}_0) \phi(\mathbf{x}_0) + G_v(\mathbf{x},t;\mathbf{x}_0) \psi(\mathbf{x}_0)\right].$$
\end{frame}



\section{Appendix}

\secpage{计算积分}{耍赖参数很有用}

\begin{frame}
  我们来计算积分
  $$I= \int_0^\infty   J_0(kr)\sin(akt)  dk.$$  
  利用 $J_0$的积分表示,可以得到
  $$ I = \frac{1}{4\pi i} \int_0^\infty  \int_{-\pi}^{\pi} \left(e^{ik[r(\sin\theta+i\epsilon)+at]}- e^{ik[r(\sin\theta+i\epsilon)-at]}\right)d\theta dk.  $$
  这里为了让积分收敛,加入了耍赖参数 $\epsilon\rightarrow 0^+$;先对$k$积分得到
  $$ I = \frac{1}{2\pi}\int_{-\pi}^{\pi} \frac{t}{a^2t^2-r^2(\sin\theta + i\epsilon)^2 } d\theta.$$
\end{frame}

\begin{frame}
  把积分转化为单位圆上的围道积分,
  $$ I = \frac{1}{2\pi r}\oint_{|z|=1}\ \left[\frac{1}{z^2 +2(i\lambda-\epsilon)z - 1}-\frac{1}{z^2 -2(i\lambda+\epsilon)z - 1 } \right]dz.$$
  这里的 $\lambda = at/r$。
\end{frame}


\begin{frame}
  如果$\lambda< 1$,则记 $$\alpha_{\pm} = \epsilon-i\lambda \pm \sqrt{1-\lambda^2-2i\lambda \epsilon},\ \beta_{\pm} = \epsilon+i\lambda \pm \sqrt{1-\lambda^2 + 2i\lambda\epsilon}$$
  $$ I = \frac{1}{2\pi r}\oint_{|z|=1}\ \left[\frac{1}{(z-\alpha_+)(z-\alpha_-)}-\frac{1}{(z-\beta_+)(z-\beta_-)}\right]dz.$$
  孤立奇点$\alpha_-$, $\beta_-$在单位圆 $|z|=1$ 内部(现在你知道为什么我一直保留$\epsilon$了),留数之和为
  $$\frac{1}{\alpha_--\alpha_+}-\frac{1}{\beta_--\beta_+} = 0. $$
  也就是说,在区域 $r> at$ 内, $u = 0$ —— 物理上来看这是显然的。
\end{frame}

\begin{frame}
  如果 $\lambda>1$,则记 $$\alpha_{\pm} = \left(-\lambda \pm \sqrt{\lambda^2-1}\right)i,\ \beta_{\pm} = \left(\lambda \pm \sqrt{\lambda^2-1}\right)i$$
  $$ I = \frac{1}{2\pi r}\oint_{|z|=1}\ \left[\frac{1}{(z-\alpha_+)(z-\alpha_-)}-\frac{1}{(z-\beta_+)(z-\beta_-)}\right]dz.$$
  孤立奇点$\alpha_+$, $\beta_-$在单位圆 $|z|=1$ 内部,由留数定理得
  $$ I = \frac{i}{r}\left(\frac{1}{\alpha_+-\alpha_-}-\frac{1}{\beta_--\beta_+}\right) = \frac{1}{\sqrt{a^2t^2-r^2}}. $$
  最后得到
  $$ I = \frac{\heaviside(at-r)}{\sqrt{a^2t^2-r^2}}.$$
  这里的 $\heaviside$ 是单位跃阶函数。
\end{frame}


\begin{frame}
  把结果
  
  $$ \int_0^\infty   J_0(kr)\sin(akt)  dk = \frac{\heaviside(at-r)}{\sqrt{a^2t^2-r^2}}.$$  
  两边对 $t$求偏导,就得到:
  $$ \int_0^\infty  k J_0(kr)\cos(akt)  dk = \frac{\delta(at-r)}{\sqrt{a^2t^2-r^2}} - \frac{at\, \heaviside(at-r)}{\left(a^2t^2-r^2\right)^{3/2}}.$$  
\end{frame}  

\ech
\end{document}
