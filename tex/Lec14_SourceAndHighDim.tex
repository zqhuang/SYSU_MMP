\documentclass[CJK]{beamer}
\usepackage{CJKutf8}
\usepackage{beamerthemesplit}
\usetheme{Malmoe}
\useoutertheme[footline=authortitle]{miniframes}
\usepackage{amsmath}
\usepackage{amssymb}
\usepackage{graphicx}
\usepackage{eufrak}
\usepackage{color}
\usepackage{slashed}
\usepackage{simplewick}
\usepackage{tikz}
\usepackage{tcolorbox}
\graphicspath{{../figures/}}
%%figures
\def\lfig#1#2{\includegraphics[width=#1 in]{#2}}
\def\addfig#1#2{\begin{center}\includegraphics[width=#1 in]{#2}\end{center}}
\def\wulian{\includegraphics[width=0.18in]{emoji_wulian.jpg}}
\def\bigwulian{\includegraphics[width=0.35in]{emoji_wulian.jpg}}
\def\bye{\includegraphics[width=0.18in]{emoji_bye.jpg}}
\def\bigbye{\includegraphics[width=0.35in]{emoji_bye.jpg}}
\def\huaixiao{\includegraphics[width=0.18in]{emoji_huaixiao.jpg}}
\def\bighuaixiao{\includegraphics[width=0.35in]{emoji_huaixiao.jpg}}
\def\jianxiao{\includegraphics[width=0.18in]{emoji_jianxiao.jpg}}
\def\bigjianxiao{\includegraphics[width=0.35in]{emoji_jianxiao.jpg}}
%% colors
\def\blacktext#1{{\color{black}#1}}
\def\bluetext#1{{\color{blue}#1}}
\def\redtext#1{{\color{red}#1}}
\def\darkbluetext#1{{\color[rgb]{0,0.2,0.6}#1}}
\def\skybluetext#1{{\color[rgb]{0.2,0.7,1.}#1}}
\def\cyantext#1{{\color[rgb]{0.,0.5,0.5}#1}}
\def\greentext#1{{\color[rgb]{0,0.7,0.1}#1}}
\def\darkgray{\color[rgb]{0.2,0.2,0.2}}
\def\lightgray{\color[rgb]{0.6,0.6,0.6}}
\def\gray{\color[rgb]{0.4,0.4,0.4}}
\def\blue{\color{blue}}
\def\red{\color{red}}
\def\green{\color{green}}
\def\darkgreen{\color[rgb]{0,0.4,0.1}}
\def\darkblue{\color[rgb]{0,0.2,0.6}}
\def\skyblue{\color[rgb]{0.2,0.7,1.}}
%%control
\def\be{\begin{equation}}
\def\ee{\nonumber\end{equation}}
\def\bea{\begin{eqnarray}}
\def\eea{\nonumber\end{eqnarray}}
\def\bch{\begin{CJK}{UTF8}{gbsn}}
\def\ech{\end{CJK}}
\def\bitem{\begin{itemize}}
\def\eitem{\end{itemize}}
\def\bcenter{\begin{center}}
\def\ecenter{\end{center}}
\def\bex{\begin{minipage}{0.2\textwidth}\includegraphics[width=0.6in]{jugelizi.png}\end{minipage}\begin{minipage}{0.76\textwidth}}
\def\eex{\end{minipage}}
\def\chtitle#1{\frametitle{\bch#1\ech}}
\def\bmat#1{\left(\begin{array}{#1}}
\def\emat{\end{array}\right)}
\def\bcase#1{\left\{\begin{array}{#1}}
\def\ecase{\end{array}\right.}
\def\bmini#1{\begin{minipage}{#1\textwidth}}
\def\emini{\end{minipage}}
\def\tbox#1{\begin{tcolorbox}#1\end{tcolorbox}}
\def\pfrac#1#2#3{\left(\frac{\partial #1}{\partial #2}\right)_{#3}}
%%symbols
\def\bropt{\,(\ \ \ )}
\def\sone{$\star$}
\def\stwo{$\star\star$}
\def\sthree{$\star\star\star$}
\def\sfour{$\star\star\star\star$}
\def\sfive{$\star\star\star\star\star$}
\def\rint{{\int_\leftrightarrow}}
\def\roint{{\oint_\leftrightarrow}}
\def\stdHf{{\textit{\r H}_f}}
\def\deltaH{{\Delta \textit{\r H}}}
\def\ii{{\dot{\imath}}}
\def\skipline{{\vskip0.1in}}
\def\skiplines{{\vskip0.2in}}
\def\lagr{{\mathcal{L}}}
\def\hamil{{\mathcal{H}}}
\def\vecv{{\mathbf{v}}}
\def\vecx{{\mathbf{x}}}
\def\vecy{{\mathbf{y}}}
\def\veck{{\mathbf{k}}}
\def\vecp{{\mathbf{p}}}
\def\vecn{{\mathbf{n}}}
\def\vecA{{\mathbf{A}}}
\def\vecP{{\mathbf{P}}}
\def\vecsigma{{\mathbf{\sigma}}}
\def\hatJn{{\hat{J_\vecn}}}
\def\hatJx{{\hat{J_x}}}
\def\hatJy{{\hat{J_y}}}
\def\hatJz{{\hat{J_z}}}
\def\hatj#1{\hat{J_{#1}}}
\def\hatphi{{\hat{\phi}}}
\def\hatq{{\hat{q}}}
\def\hatpi{{\hat{\pi}}}
\def\vel{\upsilon}
\def\Dint{{\mathcal{D}}}
\def\adag{{\hat{a}^\dagger}}
\def\bdag{{\hat{b}^\dagger}}
\def\cdag{{\hat{c}^\dagger}}
\def\ddag{{\hat{d}^\dagger}}
\def\hata{{\hat{a}}}
\def\hatb{{\hat{b}}}
\def\hatc{{\hat{c}}}
\def\hatd{{\hat{d}}}
\def\hatN{{\hat{N}}}
\def\hatH{{\hat{H}}}
\def\hatp{{\hat{p}}}
\def\Fup{{F^{\mu\nu}}}
\def\Fdown{{F_{\mu\nu}}}
\def\newl{\nonumber \\}
\def\vece{\mathrm{e}}
\def\calM{{\mathcal{M}}}
\def\calT{{\mathcal{T}}}
\def\calR{{\mathcal{R}}}
\def\barpsi{\bar{\psi}}
\def\baru{\bar{u}}
\def\barv{\bar{\upsilon}}
\def\qeq{\stackrel{?}{=}}
\def\torder#1{\mathcal{T}\left(#1\right)}
\def\rorder#1{\mathcal{R}\left(#1\right)}
\def\contr#1#2{\contraction{}{#1}{}{#2}#1#2}
\def\trof#1{\mathrm{Tr}\left(#1\right)}
\def\trace{\mathrm{Tr}}
\def\comm#1{\ \ \ \left(\mathrm{used}\ #1\right)}
\def\tcomm#1{\ \ \ (\text{#1})}
\def\slp{\slashed{p}}
\def\slk{\slashed{k}}
\def\calp{{\mathfrak{p}}}
\def\veccalp{\mathbf{\mathfrak{p}}}
\def\Tthree{T_{\tiny \textcircled{3}}}
\def\pthree{p_{\tiny \textcircled{3}}}
\def\dbar{{\,\mathchar'26\mkern-12mu d}}
\def\erf{\mathrm{erf}}
\def\const{\mathrm{constant}}
\def\pheat{\pfrac p{\ln T}V}
\def\vheat{\pfrac V{\ln T}p}
%%units
\def\fdeg{{^\circ \mathrm{F}}}
\def\cdeg{^\circ \mathrm{C}}
\def\atm{\,\mathrm{atm}}
\def\angstrom{\,\text{\AA}}
\def\SIL{\,\mathrm{L}}
\def\SIkm{\,\mathrm{km}}
\def\SIyr{\,\mathrm{yr}}
\def\SIGyr{\,\mathrm{Gyr}}
\def\SIV{\,\mathrm{V}}
\def\SImV{\,\mathrm{mV}}
\def\SIeV{\,\mathrm{eV}}
\def\SIkeV{\,\mathrm{keV}}
\def\SIMeV{\,\mathrm{MeV}}
\def\SIGeV{\,\mathrm{GeV}}
\def\SIcal{\,\mathrm{cal}}
\def\SIkcal{\,\mathrm{kcal}}
\def\SImol{\,\mathrm{mol}}
\def\SIN{\,\mathrm{N}}
\def\SIHz{\,\mathrm{Hz}}
\def\SIm{\,\mathrm{m}}
\def\SIcm{\,\mathrm{cm}}
\def\SIfm{\,\mathrm{fm}}
\def\SImm{\,\mathrm{mm}}
\def\SInm{\,\mathrm{nm}}
\def\SImum{\,\mathrm{\mu m}}
\def\SIJ{\,\mathrm{J}}
\def\SIW{\,\mathrm{W}}
\def\SIkJ{\,\mathrm{kJ}}
\def\SIs{\,\mathrm{s}}
\def\SIkg{\,\mathrm{kg}}
\def\SIg{\,\mathrm{g}}
\def\SIK{\,\mathrm{K}}
\def\SImmHg{\,\mathrm{mmHg}}
\def\SIPa{\,\mathrm{Pa}}

\def\courseurl{https://github.com/zqhuang/SYSU\_TD}

\def\tpage#1#2{
\begin{frame}
\begin{center}
\begin{Large}
\bch
热学 \\
第#1讲 #2

{\vskip 0.3in}

黄志琦

\ech
\end{Large}
\end{center}

\vskip 0.2in

\bch
教材:《热学》第二版,赵凯华,罗蔚茵,高等教育出版社
\ech

\bch
课件下载
\ech
\courseurl
\end{frame}
}

\def\bfr#1{
\begin{frame}
\chtitle{#1} 
\bch
}

\def\efr{
\ech 
\end{frame}
}

  \date{}
\begin{document}
\tpage{13}{Higher Dimension and Source Terms}

\begin{frame}
\chtitle{本讲内容}
\bch
\bitem
\item{回顾}
\item{有源的方程}    
\item{高维情况的分离变量法}
\eitem
\ech
\end{frame}


\section{Review}
\secpage{回顾}{实验报告是把杀猪刀,老身感觉什么都不记得了……}

\begin{frame}
\chtitle{目前学过的线性齐次偏微分方程解法}
\bch
\bitem
\item[1]{{\blue \bf 分离变量法} 是首选方案。一般的套路:
  \bitem
      \item[1]{ 假设分离变量形式的解$\Phi(x)\Psi(t)$并求出$\Phi$和$\Psi$允许的形式。}
      \item[2]{如果空间边界条件是非齐次的,求(猜)特解把空间边界条件化为齐次; }
      \item[3]{用齐次空间边界条件确定展开式中的函数如何选取;}
      \item[4]{用初始条件确定展开系数}
        \eitem}
\item[2]{{\blue \bf 积分变换方法} 是分离变量法在无穷长区间上的推广。区别是:积分变换的存在性本身默认了一些无穷远处的性质,无须对展开式中的函数进行选取。(分离变量法第3步可以省略)}
\item[3]{{\blue \bf 格林函数方法} 也需要把空间边界条件化为齐次。区别是:不用分离变量的乘积形式函数对解进行分解;而用格林函数(初始条件为$\delta(x-x_0)$的解,或者说对$x_0$处的单位脉冲的响应)对解进行分解。}
\eitem
\ech
\end{frame}


\begin{frame}
\chtitle{目前学过的两种方程}
\bch
热传导方程
\tbox{$$\frac{\partial u}{\partial t} - a\nabla^2 u = 0 $$}

波动方程
\tbox{$$\frac{\partial^2 u}{\partial t^2} - a^2\nabla^2 u = 0 $$}

\ech
\end{frame}

\section{Source Terms}
\secpage{有源的方程}{原则上可以暴力解决,实际上……心好累}



\section{Higher Dimension}
\secpage{高维情况的分离变量法}{寻找谐函数是关键}

\begin{frame}
\chtitle{关于分离变量法的一个疑惑}
\bch
\addfig{1}{think1.jpg}

分离变量第1步设$\Phi(x)\Psi(t)$……那如果变量超过2个怎么办?
\ech
\end{frame}


\begin{frame}
\chtitle{例题1}
\bch
边长为$L$的正方形弹性膜,四边都固定,求解膜的小振动问题:
\bea
\frac{\partial^2 u}{\partial t^2} - a^2\nabla^2 u &=& 0 , \newl
\left. u \right\vert_{x = 0} &=& 0 , \newl
\left. u \right\vert_{x = L} &=& 0 , \newl
\left. u \right\vert_{y = 0} &=& 0 , \newl
\left. u \right\vert_{y = L} &=& 0 , \newl
\left. u \right\vert_{t=0} &=& \phi(x,y) , \newl
\left. \frac{\partial u}{\partial t} \right\vert_{t=0} &=& \psi(x,y) ,
\eea
其中$ \nabla^2 = \frac{\partial^2 }{\partial x^2} + \frac{\partial^2 }{\partial y^2}. $
\ech
\end{frame}



\begin{frame}
\chtitle{解答}
\bch
先寻找分离变量型解,令$u = \Phi(x,y)\Psi(t)$,代入波动方程,得到
$$ \frac{1}{a^2} \frac{\Psi'' }{\Psi} = \frac{\nabla^2\Phi}{\Phi} $$
等式左边是$t$的函数,右边是$x,y $的函数,要两边恒等,只能是常数:
$$ \frac{1}{a^2} \frac{\Psi'' }{\Psi} = \frac{\nabla^2\Phi}{\Phi} = - k^2$$
{\scriptsize ($\frac{\nabla^2\Phi}{\Phi} = k^2$的情形对常见边界条件用处不大,被无情抛弃。)}

解出分离变量型解:
$$ u_1 = Q(x, y) e^{\pm iakt}$$
其中$Q(x,y)$是
{\blue $$\nabla^2Q = -k^2 Q$$
的解,称为 谐函数}。
\ech
\end{frame}



\begin{frame}
\chtitle{谐函数}
\bch
在边界为正方形的情况下,取直角坐标系是很漂亮的操作,那么
$$\nabla^2Q = -k^2 Q$$
的解是平面波
$$ Q = e^{\ii \veck \cdot \vecx} $$
这里我们用矢量写法$\vecx$来表示$(x, y)$,$\veck$来表示$(k_x, k_y)$ (满足$|\veck|=\sqrt{k_x^2+k_y^2} = k$)。


和一维的情况类似,对应一组$k_x>0, k_y>0$有四种满足条件的解(因为$k_x, k_y$分别可以取),可以重新写成正弦和余弦的实数表达形式。
\ech
\end{frame}


\begin{frame}
\chtitle{级数展开}
\bch
空间边界条件已经是齐次的了,直接把解进行分解:
\bea u &=& \sum_{m,n=0}^\infty c_{mn} \sin\frac{m\pi x}{L} \sin\frac{n\pi y}{L} \cos{(akt)} \newl
&& + \sum_{m,n=0}^\infty s_{mn} \sin\frac{m\pi x}{L} \sin\frac{n\pi y}{L} \sin{(akt)}.
\eea
利用两个初始条件分别可以求出系数$c_{mn}$和$s_{mn}$。
\ech
\end{frame}



\begin{frame}
\chtitle{例题2}
\bch
有孤立的,半径为$a$的均匀厚度圆形金属片。初态$t=0$时刻金属片中心温度为$T_0$,金属片上距离中心$r$处的温度为
$$T(r) =T_0\left(1+\frac{r^2}{a^2}\right).$$
已知金属片质量面密度为$\sigma$,单位质量比热为$c$。求之后圆盘上各点的温度变化。

\skiplines

{(\wulian 这题看着好眼熟)}
\ech
\end{frame}



\begin{frame}
  \chtitle{分离变量}
  \bch
  \ech
\end{frame}



\section{Homework}

\begin{frame}
\chtitle{课后作业 (题号 30-32)}
\bch
\bitem
\item[30]{}
\eitem
\ech
\end{frame}


\begin{frame}
\chtitle{课后作业 (题号 30-32)}
\bch
\bitem
\item[31]{
}
\eitem
\ech
\end{frame}


\begin{frame}
\chtitle{课后作业 (题号 30-32)}
\bch
\bitem
\item[32]{
}
\eitem
\ech
\end{frame}


\end{document}
